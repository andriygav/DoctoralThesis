\section{Задача оценки сложности моделей и данных}

Заданы выборка и мера сложности выборки:
$$\mathfrak{D} = \{D_i\}_{i=1}^{l}, \quad \mu_D(D_i) : \mathfrak{D} \to \mathbb{R}_+.$$

Заданы аппроксимирующие модели и мера сложности моделей:
$$\mathfrak{F} = \{f\}_{j=1}^{n}, \quad \mu_f(f_i) : \mathfrak{F} \to \mathbb{R}_+.$$

\begin{definition}
    Назовем модель $f\in\mathfrak{F}$ \textit{обучаемой} на выборке $D\in\mathfrak{D},$ если $\mu_f(f)\leq \mu_D(D)$.
\end{definition}

\section{Сложность данных, как объем обучающей выборки}

В рамках определения об обучаемости модели на выборке, введем определения \textit{условной сложности выборки}:
$$\mu_D(D_i|f) : \mathfrak{D} \to \mathbb{R}_+.$$

Частным случаем \textit{условной сложности выборки} является достаточный объем выборки~--- минимальный объем данных из выборки~$D$ необходимый для обучения модели~$f$.

\begin{definition}
    Размер выборки $m^*$ называется \textit{достаточным} согласно критерия $T$, если $T$ выполняется для всех $k \geqslant m^*$.
\end{definition}

\section{Оценка изменений функций потерь}

\begin{figure}[h!t]\center
    \centering
    \includegraphics[width=0.7\linewidth]{figures/chapter-3/losses_difference.pdf}
    \caption{Пример изменения функции потерь при добавлении нового объекта}
    \label{fig-chapter-3-losses-difference}
\end{figure}


Задана выборка:
\begin{equation}
    D = \left\{ (\mathbf{x}_i, \mathbf{y}_i) \right\}, \quad i = 1, \ldots, m, \quad \mathbf{x} \in \mathcal{X}, \ \mathbf{y} \in \mathcal{Y}.
\end{equation}

Условное распределение $p(\mathbf{y}|\mathbf{x})$ аппроксимируется $f_{\boldsymbol{\theta}}: \mathcal{X} \to \mathcal{Y}$, $\boldsymbol{\theta} \in \mathbb{R}^{P}$.

Функция потерь
\begin{equation}
    \mathcal{L}_m(\boldsymbol{\theta}) = \dfrac{1}{m} \sum\limits_{i=1}^{m} \ell(f_{\boldsymbol{\theta}}(\mathbf{x}_i), \mathbf{y}_i) \approx \mathbb{E}_{(\mathbf{x}, \mathbf{y}) \sim p(\mathbf{x}, \mathbf{y})} \left[ \ell(f_{\boldsymbol{\theta}}(\mathbf{x}), \mathbf{y}) \right]
\end{equation}


Изменение значения при добавлении одного объекта
\begin{equation}
    \mathcal{L}_{k+1}(\boldsymbol{\theta}) - \mathcal{L}_k(\boldsymbol{\theta}) = \dfrac{1}{k+1} \left( \ell(f_{\boldsymbol{\theta}}(\mathbf{x}_{k+1}), \mathbf{y}_{k+1}) - \mathcal{L}_{k}(\boldsymbol{\theta}) \right)
\end{equation}


Абсолютное изменение функции потерь
\begin{equation}
    \left| \mathcal{L}_{k+1}(\boldsymbol{\theta}) - \mathcal{L}_k(\boldsymbol{\theta}) \right| \leqslant M_l+ \dfrac{1}{k+1} \left\|\boldsymbol{\theta} - \boldsymbol{\theta}^*\right\|_2^2 \left\| \mathbf{H}_{k+1}(\boldsymbol{\theta}^*) - \dfrac{1}{k} \sum\limits_{i=1}^{k} \mathbf{H}_{i}(\boldsymbol{\theta}^*) \right\|_2
\end{equation}