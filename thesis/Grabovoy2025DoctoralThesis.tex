\documentclass{dissert}
\usepackage[T2A]{fontenc}
\usepackage[utf8]{inputenc}
\usepackage[russian]{babel}

\usepackage{fullpage}
\usepackage{lastpage}
\usepackage{enumerate}
\usepackage{svg}

\usepackage[all]{xy}
% colors
\usepackage{xcolor}
\definecolor{darkgreen}{rgb}{0.0, 0.2, 0.13}
\definecolor{darkcyan}{rgb}{0.0, 0.55, 0.55}

\usepackage{geometry}
\geometry{left=2.5cm}
\geometry{right=1.0cm}
\geometry{top=2.0cm}
\geometry{bottom=2.0cm}
\renewcommand{\baselinestretch}{1.5}

\newcommand{\paragraph}[1]{\noindent\textbf{#1}\quad}


%https://tex.stackexchange.com/questions/163451/total-number-of-citations
\usepackage{totcount}
\newtotcounter{citnum} %From the package documentation
\def\oldbibitem{} \let\oldbibitem=\bibitem
\def\bibitem{\stepcounter{citnum}\oldbibitem}

\makeatletter
\long\def\@makecaption#1#2{%
  \vskip\abovecaptionskip
  \sbox\@tempboxa{#1.~#2}%
  \ifdim \wd\@tempboxa >\hsize
    #1.~#2\par
  \else
    \global \@minipagefalse
    \hb@xt@\hsize{\hfil\box\@tempboxa\hfil}%
  \fi
  \vskip\belowcaptionskip}
\makeatother

\reversemarginpar

\renewcommand{\contentsname}{Содержание}
\renewcommand{\contentsdesc}{Стр.}
\renewcommand{\chaptername}{Глава}

%%% Библиография %%%
\makeatletter
\bibliographystyle{utf8gost71u}     % Оформляем библиографию по ГОСТ 7.1 (ГОСТ Р 7.0.11-2011, 5.6.7)
\makeatother


% Нужные мне пакеты
\usepackage[bookmarksnumbered]{hyperref}
\usepackage{diagbox}
\usepackage{amsthm}
\usepackage{graphicx}
\usepackage{amsmath,amssymb,amsfonts}
\usepackage{graphicx}
\usepackage{subfig}
\usepackage{caption}
\usepackage{color}
\usepackage{bm}
\usepackage{tabularx}
\usepackage{url}
\usepackage{multirow}

\newtheorem{theorem}{Теорема}
\newtheorem{lemma}[theorem]{Лемма}
\newtheorem{definition}{Определение}
\newtheorem{assumption}{Предположение}
\newtheorem{property}{Свойство}
\newtheorem{corollary}{Следствие}
\newtheorem{remark}{Замечание}

\usepackage{comment}
\usepackage{rotating}

\usepackage{autonum}

\begin{document}

% Титульный лист
\thispagestyle{empty}


\begin{titlepage}
\begin{center}
\end{center}
\vspace{1.5cm}
\begin{flushright}
{На правах рукописи}
\end{flushright}
\vspace{1.5cm}
\begin{center}
{Грабовой Андрей Валериевич}
\par
\vspace{2cm}
\textsc{О сложности модели и данных \\ в современных моделях глубокого обучения}\\
\par
\vspace{2cm}
{1.2.1~--- Искусственный интеллект и машинное обучение}
\par
\vspace{2cm}
{Диссертация на соискание ученой степени\\
доктора физико-математических наук}
\end{center}
\vspace{2cm}
\hfill\parbox{8,4cm}{Научный консультант:
\\д.ф.-м.н. К.\,В.\,Воронцов}
\par
\vspace{3.5cm}
\begin{center}
{Москва~--- 2025}
\end{center}
\end{titlepage}
% Нумерация должна начинаться со второй страницы
\setcounter{page}{2}

% Оглавление
\newpage
\tableofcontents

% Введение
\newpage{}
\chapter*{Введение}
\addcontentsline{toc}{section}{Введение}
\paragraph{Актуальность темы.}
Ключевым фактором развития нейросетевых моделей с начала XXI века является прогресс в вычислительной технике, характеризующийся экспоненциальным ростом производительности суперкомпьютеров, измеряемой в операциях с плавающей запятой~(англ. FLOPS): от значений на уровне терафлопсов в начале столетия до достижения экзафлопсов в настоящий момент.
Параллельно наблюдается сопоставимый рост сложности моделей глубокого обучения, выраженный в увеличении количества обучаемых параметров на несколько порядков~--- от тысяч в начале нулевых годов до миллиардов и сотен миллиардов в двадцатых, с прогнозируемым переходом к триллионам параметров в ближайшем десятилетии.
Современные исследования, посвященные анализу сложности таких моделей, в значительной степени опираются на эмпирические корреляции, связывающие их эффективность с количественными метриками~--- числом параметров и вычислительными затратами на обучение либо применение.
Отсутствие же строгой теоретической основы для предсказания поведения моделей при масштабировании делает этот процесс экономически и энергетически нерациональным и зачастую приводит к получению необоснованных и противоречивых результатов.

Развитие больших языковых моделей~(англ. LLM) сопряжено с высокими затратами на их обучение, выражающимися в большом потреблении вычислительных, энергетических и финансовых ресурсов.
Существенной проблемой в этом процессе является присущая ему непредсказуемость, для нивелирования которой на предварительной стадии обучения~(англ. pretrain) эмпирически подбираются оптимальные соотношения между размером модели в параметрах и объемом обучающих данных в токенах при заданной допустимой вычислительном ресурсе в операциях с плавающей запятой.
Однако данные, полученные для одной архитектуры, не являются переносимыми на другие модели, что делает подобные оценки зачастую несостоятельными и приводит к непредвиденным результатам при полномасштабном обучении.
В этой связи разработка теоретических оценок сложности моделей представляется критически важной, поскольку она позволила бы получать асимптотические оценки сложности моделей, которые связаны со сложностью выборки еще на этапе проектирования архитектуры, чувствительной к любым модификациям своей структуры.

Оценка сложности моделей машинного обучения является глубоко изученной областью для классических методов и находит ограниченное применение при анализе моделей глубокого обучения.
Фундаментальный вклад в теорию оценивания сложности моделей и в теорию машинного обучения в целом внесли работы Владимира Наумовича Вапника и Алексея Яковлевича Червоненкиса, заложившие основы статистической теории обучения~\cite{vapnik1974StatTheory}.
Альтернативный математический аппарат для анализа алгоритмов обучения был разработан Лесли Вэлиантом, предложившим приближенно правильного обучения~(англ. PAC-Learning)~\cite{valiant1984LearnableTheory}.
Современный подход к определению сложности основан на Радемахеровской сложности, предложенная Владимиром Кольчинским и Дмитрием Панченко~\cite{koltchinskii2000RadamakherTheory}. Современные исследования в области теории машинного обучения редко рассматривают современные нейросетевые модели в виду их сложности и невозможности получения ``адекватных'' оценок на сложность таких моделей. Значимая часть современных исследованиях на ведущих конференция посвящена классическим методам машинного анализа и улучшения оценок для известных методов, так как текущие оценки сложности даже для классических моделей машинного обучения являются сильно завышенными.

В работе же проводится теоретический анализ нейросетевых моделей, опирающийся на анализ их матриц Гессе.
Матрица Гессе используется как для анализа важности параметров в задачах прореживания нейросетевых моделей, так и для анализа ландшафта функции потерь, который используется для анализа сложности модели.

\paragraph{Степень разработанности темы диссертационного исследования.}
Современное состояние анализа сложности моделей машинного обучения описаны в~\cite{macKay2003}. С другой стороны сложность нейросетевых моделей является слабо изученной темой на текущий момент. В свою очередь рассмотрения сложности выборки сводится только к анализу размера выборки для обучения~\cite{motrenko2022numerical613055643, demidenko2007, joseph1997, joseph1995, kloek1975, lindley1997, motrenko2014, qumsiyeh2013, rubin1998, self1988, self1992, shieh2000, shieh2005, wang2002}, что является не совсем корректным, так как сложность выборки также обусловлено и сложностью каждого объекта. Современные методы оценки сложность объектов выборки опираются исследования сложности многообразий, которые аппроксимируют данный элемент выборки. Одним из способов оценки объема выборки для обучения больших языковых моделей являются эмперические оценки вместе с методологией их получения для новых моделей~\cite{hoffmann2022Chinchila,kaplan2020ScalingLaws} полученные Джаредом Капланом и Джорданом Хофманом в рамках исследования законов масштабирования нейросетевых моделей~\cite{rae2022scalinglanguagemodelsmethods}. Что касается задачи снижения сложности моделей, на текущий момент существует несколько направлений. Первое направление направлено на квантизацию параметров моделей, второе же, более теоретическое, направлено на дистиляцию и привилегированное обучение~\cite{hinton2015, lopez2016, grabovoi2021bayesian609999432}.

Однако, на текущий момент, в исследованиях не существует теоретического апарата, для описания сложности моделей и данных, чтобы получить асимптотические оценки сложности моделей и данных. Вместе с тем полученный математический аппарат позволит проводить сравнительный анализ различных нейросетевых архитектур, для выбора лучшего решения для заданной задачи, которая описывается выборкой заданной сложности.

В данной работе проведен комплексный подход к оценке сложности моделей и данных. Предложен альтернативный математический аппарат для анализа сложности на основе анализа ландшафта функции потерь нейросетевых моделей. Получены теоретические оценки для матриц Гессе различных моделей глубокого обучения, которые требуются для оценки сложности модели. Отдельной частью данной работы является анализ связи объема и сложности выборки со сложностью модели. Вводятся частные случае общей теории сложности, которые позволяют получать практические оценки для прикладных задач.

\paragraph{Объектом исследования} в работе являются параметрические семейства функций, которые представимы в виде суперпозиции линейных и нелинейных преобразований; данные, которые используются для настройки параметров параметрических семейств.

\paragraph{Предмет исследования:} разработка моделей, методов и теоретического аппарата работы со сложностью моделей глубокого обучения и данными, которые используются для обучения.

\paragraph{Цель и задачи исследования.}
Целью исследования является получения оценок сложности моделей глубокого обучения, а также получения оценок сложности данных для обучения моделей.

Для достижения цели были поставлены и решены следующие задачи:
\begin{itemize}
    \item[--] разработка методов оценки сложности моделей глубокого обучения;~\textcolor{red}{Теоретическое введение теории сложности, с понятием условной сложности.}
    \item[--] вычисление оценок матриц Гессе для различных архитектур моделей глубокого обучения;~\textcolor{red}{Все работы связанные с оценкой матриц Гессе: Киселев/Мешков/Петров.}
    \item[--] вычисление оценок размера выборки и их связь со сложностью моделей и данных;~\textcolor{red}{Все работы связанные с оценкой размера выборки: Киселев.}
    \item[--] вычисление оценок важности параметров глубоких нейросетевых моделей при обучении и предсказании;~\textcolor{red}{Работы с прореживанием нейросетевых моделей, дистиляция моделей. Требуюется работа связана с использованием новых оценок матрицы Гессе для вычисления важности параметров.}
    \item[--] оценка влияния регуляризаторов параметров моделей глубокого обучения на оценку сложности моделей глубокого обучения;~\textcolor{red}{Требуются дополнительные, новые исследования: Мешков.}
    \item[--] практическое применение анализа сложности моделей и данных при обучении моделей глубокого обучения;~\textcolor{red}{Тут все эксперименты, которые есть: Киселев/Мешков.}
    \item[--] неявная регуляризация сложности модели при мультизадачном обучении.~\textcolor{red}{Все работы связанные с детекцией машингена в мультизадачном подходе: Грицай/Ремизова.}
\end{itemize}

\paragraph{Методы исследования.}
Для решения поставленных задач в диссертации используются методы: машинного обучения, статистического анализа данных, статистической теории машинного обучения, линейной алгебры, дискретной математики, теории вероятности, глубокого обучения.

\paragraph{Научная новизна.}
Научной новизной проведенного исследования являются теоретические оценки сложности моделей глубокого обучения и их связи со сложностью данных для обучения.
Диссертация представляет новый подход к рассмотрению сложности моделей глубокого обучения на стыке строго терроризированных оценок и их практической применимостью в реальных задачах.
В диссертации предлагается новый подход к оценке сложности модели на основе анализа ландшафта оптимизационной задачи.
В рамках диссертационной работы получены оценки матриц Гессе для некоторых моделей глубокого обучения.
На основе полученных оценок матриц Гессе в рамках диссертационной работы рассмотрены новые подходы для построения методов оценки важности параметров нейросетевых моделей, которые ранее были слабо исследованы из-за невычислимости данной матрицы для больших нейросетевых моделей.


\paragraph{Теоретическая значимость работы.}
Диссертационная работа в значительной степени представляет из себя именно теоретический результат. В диссертационной работе рассматриваются фундаментальные вопросы о сложности моделей машинного обучения. Полученные оценки на матрицы Гессе открывают большой спектр задач в теории выбора моделей машинного обучения. 

\paragraph{Практическая значимость работы.}
Предложенные теоретические оценки в работе также рассматривались и из практической стороны. В диссертационной работе предложены адаптации различных полученных оценок, для их применимости в различных прикладных задачах. Отдельно проведены работы с сравнением теоретических оценок и эмпирических оценок закона масштабирования моделей глубокого обучения.

\paragraph{Положения, выносимые на защиту:}
\begin{enumerate}
    \item Оценки сложности моделей глубокого обучения.
    \item Оценки сложности данных.
    \item Ландшафтная мера сложности моделей глубокого обучения и ее связь со сложностью данных.
    \item Оценки матриц Гессе для некоторых классов нейросетевых архитектур и их связь с ландшафтной мерой сложности моделей глубокого обучения.
    \item Оценки достаточного размера выборки и их связь со сложностью моделей и данных.
    \item Методы снижения размерности пространства параметров моделей глубокого обучения на основе анализа матриц Гессе.
\end{enumerate}

\paragraph{Степень достоверности результатов.}
Достоверность научных результатов работы подтверждается непротиворечивостью и согласованностью с известными фактами и исследованиями в рассматриваемой области, высокой степенью сходимости теоретических результатов с данными экспериментов и определяется применением теоретических и методологических основ разработок ведущих ученых в области обработки естественного языка, корректным и обоснованным использованием математического аппарата, экспериментальными исследованиями разработанных моделей и методов.

\paragraph{Соответствие диссертации паспорту специальности.}
Тема и основные результаты диссертации соответствуют следующим областям исследований паспорта специальности 1.2.1 --- Искусственный интеллект и машинное обучение.

2 Исследования в области оценки качества и эффективности алгоритмических и программных решений для систем искусственного интеллекта и машинного обучения. Методики сравнения и выбора алгоритмических и программных решений при многих критериях.

4 Разработка методов, алгоритмов и создание систем искусственного интеллекта и машинного обучения для обработки и анализа текстов на естественном языке, для изображений, речи, биомедицины и других специальных видов данных.

16 Исследования в области специальных методов оптимизации, проблем сложность и элиминации перебора, снижения размерности.

17 Исследования в области многослойных алгоритмических конструкций, в том числе – многослойных нейросетей.

\paragraph{Апробация результатов диссертации.}
Основные результаты работы докладывались и обсуждались на Всероссийской конференции с международным участием «Математические методы распознавания образов»~(Москва, 2019, Москва, 2021, Муром, 2025), Международной конференции «Интеллектуализация обработки информации»~(Гаэта, 2018, Москва, 2020, Москва, 2022), Всероссийской научной конференции МФТИ~(Москва, 2018, 2019, 2020, 2021, 2023, 2024, 2025), Ivannikov Ispras Open Conference~(Москва, 2021, 2022, 2023, 2024), Ivannikov Memorial Workshop~(Казань, 2022), Iberian Languages Evaluation Forum co-located with the Conference of the Spanish Society for Natural Language Processing~(Андалусия, 2023, 2024), 35th Conference of Open Innovations Association~(Тампере, 2024), Fourth Workshop on Scholarly Document Processing~(Бангкок, 2024), 1st Workshop on GenAI Content Detection (GenAIDetect)~(Абу-Даби 2025), 19th International Workshop on Semantic Evaluation~(Вена, 2025).

\paragraph{Публикации.}
По теме диссертации опубликовано 56 научных работ, из
которых 17 статей в научно-технических журналах, входящих в перечень ВАК, 32 – в изданиях, входящих в международные наукометрические базы Scopus и Web of Science. В трудах российских и международных конференций опубликовано 39 работ.

\paragraph{Личный вклад соискателя.}
Все выносимые на защиту результаты и положения, составляющие основное содержание диссертационного исследования, разработаны и получены лично автором или при его непосредственном участии вместе с учениками. В работах, опубликованных в соавторстве, соискателю принадлежит определяющая роль в построении теоретических методов и направлении. В работе ...

\paragraph{Структура и объем работы.}
Диссертация состоит из оглавления, введения, шести разделов, заключения, списка иллюстраций, списка таблиц, перечня основных обозначений и списка литературы из~\total{citnum} наименований. Основной текст занимает~\pageref{LastPage} страницы.

% Первая глава
\clearpage
\chapter{Сложность модели и данных}
Современные модели глубокого обучения демонстрируют исключительную эффективность в решении сложных задач, однако их успешное применение требует понимания фундаментального взаимодействия между сложностью модели и характеристиками данных.
В главе~\ref{chapter:gesian} проводится анализ матриц Гессе для различных архитектур нейронных сетей, что позволяет получить количественные оценки кривизны функции потерь и сложности оптимизационного ландшафта.
Эти результаты создают теоретическую основу для формального определения и измерения сложности как моделей, так и данных.

Ключевой идеей настоящей главы является установление формального соотношения между мерой сложности модели $\mu_f(f)$ и мерой сложности данных $\mu_D(D)$, определяемого через условие обучаемости:
\begin{equation}
    \mu_f(f) \leq \mu_D(D),
\end{equation}
а также получения частных случаев, которые имеют более подробный практический и теоретический анализ.

В рамках данного подхода основным является анализ изменения функции потерь при непрерывном изменении выборки. В разделе~\ref{chapter:complexity:loss} описывается то как абсолютное изменение функции потерь оценивается через спектральную норму матрицы Гессе:
\begin{equation}
    \left| \mathcal{L}_{k+1}(\boldsymbol{\theta}) - \mathcal{L}_k(\boldsymbol{\theta}) \right| \leqslant M_l + \dfrac{1}{k+1} \left\|\boldsymbol{\theta} - \boldsymbol{\theta}^*\right\|_2^2 \left\| \mathbf{H}_{k+1}(\boldsymbol{\theta}^*) - \dfrac{1}{k} \sum\limits_{i=1}^{k} \mathbf{H}_{i}(\boldsymbol{\theta}^*) \right\|_2
\end{equation}

Этот результат, основанный на теоретических выкладках из главы~\ref{chapter:gesian}, позволяет формализовать понятие~\textit{условной сложности выборки}~$\mu_D(D_i|f)$ и установить критерии достаточности объема данных для обучения конкретной модели.

Предлагаемый формализм не только углубляет теоретическое понимание процессов обучения глубоких нейронных сетей, но и имеет практическую значимость для разработки эффективных стратегий обучения, выбора архитектур моделей и планирования экспериментов.
Результаты данной главы создают мост между теоретическим анализом оптимизационных свойств моделей и практическими аспектами их применения к реальным данным, открывая новые возможности для систематического подхода к проектированию и обучению сложных нейросетевых архитектур.

\section{Оценка сложности моделей и данных}

В современной теории глубокого обучения фундаментальной проблемой является установление соответствия между сложностью модели и характеристиками данных.
В рамках данного раздела произведем формализацию данного определения.

\begin{definition}\label{chapter:complex:def-gamma}
    Генеральной совокупностью данных~$\Gamma$ назовем произвольное множество объектов, которые исследуются в рамках той либо иной задаче. В общем случае нет никаких ограничение на счетность множества генеральной совокупности.
\end{definition}

Определение~\ref{chapter:complex:def-gamma} позволяет работать как с однородными, так и с многородной генеральными совокупностями.

\begin{definition}\label{chapter:complex:def-gamma-modality}
    Генеральную совокупность~$\Gamma$ назовем однородной, если все объекты генеральной совокупности порождаются из одного распределения. В противном случае выборку назовем~$k$-родной, где~$k$ является числом распределений на основе которой была сгенерирована генеральная совокупность;
\end{definition}

В определении~\ref{chapter:complex:def-gamma-modality} примером двуродной генеральной совокупности выступает выборка состоящая из текстов и из изображений в качестве объекта исследования. Например, современные большие языковые модели одновременно работают как с текстами, так и с изображениями и называются многомодальными моделями.

Пусть задана генеральная совокупность данных~$\Gamma,$ где задано множество всех подмножеств объектов образующих кольцо выборок:
\[
    \mathfrak{D} = \{D_\Gamma^i\}, \quad D_\Gamma^i \subset \Gamma.
\]

\begin{definition}[Мера сложности выборки]\label{chapter:complex:def-data-complexity}
    Мерой сложности выборки назовем отображение~$\mu_D,$ такое, что: 
    \[
        \mu_D(D_i) : \mathfrak{D}_\Gamma \to \mathbb{R}_+,
    \]
    удовлетворяющее свойству:
    \begin{align}
        \mu_D(D_i\cup D_j) \leq \mu_D(D_i)+\mu_D(D_j),
    \end{align}
    где равенство достигается при условии~$D_i\cap D_j=\emptyset.$
\end{definition}

Определение~\ref{chapter:complex:def-data-complexity} является классическим определением из теории меры, удовлетворяющее свойству конечной-адитивности. Конечной-адитивности нам достаточно, так как в рамках исследований предполагается конечное число объектов при обучении моделей глубокого обучения.
Отдельно стоит оговорить, что мы предполагаем, что мы производим сравнение выборок только из одной генеральной совокупности, но заметим, что мы никак не ограничиваем саму генеральную совокупность и в рамках определения возможны мультимодальные генеральные совокупности.


Пусть задано множество параметрических аппроксимирующих моделей
\[
    \mathfrak{F} = \left\{f_i\right\},
\]
где каждое~$f_i$ является некоторым множеством параметрическим функций. В определении~\ref{chapter:complex:def-model} вводиться общее определение характеристики параметрического семейства функций~$f,$ которые мы в дальнейшем будем рассматривать в качестве моделей моделей глубокого обучения.
\begin{definition}\label{chapter:complex:def-model}
    Мерой сложности модели~$f$ назовем отображение~$\mu_f(f_i)$:
    \[
        \mu_f(f_i) : \mathfrak{F} \to \mathbb{R}_+.
    \]
\end{definition}

Заметим, что определение меры сложности модели~$f$ не является определением меры в общем случае, так как множество~$\mathfrak{F}$ не является кольцом, поэтому данная мера является некоторым отображением, которая является некоторой характеристикой сложности.
К примеру число параметров модели удовлетворяет определению~\ref{chapter:complex:def-model}.

Введя определения меры сложности как для выборки, так и для модели, перейдем к определению обучаемости модели на выборке6 которое сформулировано в определении~\ref{chapter:complex:def-model-bound-data}.

\begin{definition}\label{chapter:complex:def-model-bound-data}
    Назовем модель $f\in\mathfrak{F}$ \textit{обучаемой} на выборке $D\in\mathfrak{D},$ если 
    \[
        \mu_f(f)\leq \mu_D(D).
    \]
\end{definition}

В рамках определения~\ref{chapter:complex:def-model-bound-data} не вводится никакого ограничения на качество аппроксимации модели после обучения, более подробно это будет определено для частных случаев мер в следующих разделах. Также, определение~\ref{chapter:complex:def-model-bound-data} имеет эмпирическую интерпретацию: сложность модели не должна превышать сложности данных, на которых она обучается, так как в противном случае возникает проблема переобучения, когда модель запоминает шум в данных вместо выявления значимых закономерностей.

В рамках диссертационной работы предполагается исследовать частный случай меры сложности модели на основе оценок ландшафта оптимизационных задач используя матрицы Гессе для различных нейросетевых архитектур, которые описаны в главе~\ref{chapter:gesian}. Подробнее о частном случае меры сложности описано в разделе~\ref{chapter:complexity:loss}.

\begin{theorem}\label{chapter:complex:theorem-finetunning}
    Если для исходной выборки~$D\in\mathfrak{D}$ выполняется условие~$\mu_f(f) \leq \mu_D(D)$, тогда для новой выборки~$D'\in\mathfrak{D}$ модель может быть успешно дообучена при условии:
    \[
        \mu_f(f) - \mu_D(D) \leq \mu_D(D').
    \]
\end{theorem}
\begin{proof}
Доказательство основано на свойствах мер сложности и условии обучаемости модели. 

По определению обучаемости модели на выборке $D$ имеем:
\[
    \mu_f(f) \leq \mu_D(D).
\]
При добавлении новых данных $D'$ к исходной выборке $D$ сложность объединенной выборки не убывает:
\[
    \mu_D(D) \leq \mu_D(D\cup D')
\]
Из свойства адитивности меры сложности выборки, получаем:
\[
    \mu_D(D\cup D') \leq \mu_D(D)+\mu_D(D'),
\]
тогда, объединяя эти три неравенства, получаем цепочку:
\[
    \mu_f(f) \leq \mu_D(D) \leq \mu_D(D\cup D') \leq \mu_D(D)+\mu_D(D'),
\]
тогда перенося $\mu_D(D)$ в левую часть, получаем окончательное неравенство:
\[
    \mu_f(f) - \mu_D(D) \leq \mu_D(D').
\]
\end{proof}

Это неравенство показывает, что ``оставшаяся емкость'' модели, а именно, что разность между сложностью модели и сложностью исходных данных не превосходит сложности новых данных~$D'$, что является необходимым условием для успешного дообучения модели на новых данных.

Теорема~\ref{chapter:complex:theorem-finetunning} предполагает, что мера сложности данных обладает свойством монотонности и субаддитивности. В практических приложениях эти свойства должны быть проверены для конкретных выбранных мер сложности.

Таким образом, введение формальных мер сложности моделей и данных создает теоретическую основу для решения практических задач проектирования архитектур нейронных сетей, планирования экспериментов и оптимизации процессов обучения.


\section{Достаточный объем выборки, как мера сложности данных}

В рамках введенного определения об обучаемости модели на выборке, ключевым понятием становится \textit{условная сложность выборки}:
\begin{equation}\label{chapter:complex:eq-data-submessure}
    \mu_D(D|f) : \mathfrak{D} \to \mathbb{R}_+,
\end{equation}
которая характеризует сложность данных $D\in\mathfrak{D}$ относительно заданной параметрической модели~$f$.
Эта мера отражает, насколько ``трудной'' является выборка~$D$ для обучения модели $f$.

Мотивация введения условной сложности выборки проистекает из практического опыта обучения нейронных сетей.
Одна и та же выборка данных может представлять различную сложность для разных архитектур моделей.

Таким образом мера сложности модели~$\mu_f(f)$ индуцирует меру сложности выборки следующим образом:
\begin{equation}\label{chapter:complex:eq-data-submessure-infinum}
    \mu_D(D|f) = \inf \{ \mu_D(D') : D' \subseteq D, \quad \mu_f(f) \leq \mu_D(D') \},
\end{equation}
то есть условная сложность выборки может быть задана как минимальная сложность данных, при которой модель $f$ остается обучаемой.

\begin{definition}
    Условной сложностью выборки~$D$ относительно заданной параметрической модели~$f$ назовем отображение~\eqref{chapter:complex:eq-data-submessure} определяющиеся выражением~\eqref{chapter:complex:eq-data-submessure-infinum}.
\end{definition}

Рассмотрим частный случай меры сложности данных~$\mu_D$ заданной из определения достаточного объема выборки.
Предположим, что генеральная совокупность~$\Gamma_C$ состоит из объектов одинаковой сложности~$C$, то есть для каждого объекта $\gamma \in \Gamma_C$ выполняется:
\[
    \mu_D(\gamma) = C,
\]
где~$C\in\mathbb{R}_+$ некоторая агрегирована сложность одного объекта выборки. Заметим, что данное предположение является сильным ограничением и может не выполняться на практике, поскольку в реальных задачах различные объекты могут обладать разной сложностью для модели.

\begin{remark}
Константа $C$ представляет собой ``стоимость'' одного объекта выборки в единицах сложности.
На практике $C$ может зависеть от характеристик генеральной совокупности~$\Gamma$ и должна калиброваться экспериментально.
\end{remark}

\begin{definition}
    Однородную генеральную совокупность~$\Gamma_C$ назовем простой, если она состоит из объектов одинаковой сложности~$C.$
\end{definition}

\begin{theorem}
    Для простой генеральной совокупности, мера сложности любой выборки $D \subset \Gamma$ равна ее объему:
    \[
        \mu_D(D) = C\cdot|D|.
    \]
\end{theorem}
\begin{proof}
Докажем, что функция $\mu_D(D) = C \cdot |D|$ удовлетворяет определению меры сложности выборки~\ref{chapter:complex:def-data-complexity}.

Для начала докажем, что функция~$\mu_D$ является неотрицательной.
Поскольку $|D| \geq 0$ для любой выборки $D \subset \Gamma$, и константа $C\in\mathbb{R}_+$, то $\mu_D(D) = C|D| \geq 0$.

Докажем монотоность функции~$\mu_D$.
Пусть $D_1 \subseteq D_2 \subset \Gamma$.
Тогда $|D_1| \leq |D_2|$, следовательно:
\[
    \mu_D(D_1) = C|D_1| \leq C|D_2| = \mu_D(D_2)
\]

Докажем субадитивность функции~$\mu_D$.
Для любых непересекающихся выборок $D_1, D_2 \subset \Gamma$ выполняется:
\[
    \mu_D(D_1 \cup D_2) = C|D_1 \cup D_2| = C(|D_1| + |D_2|) = \mu_D(D_1) + \mu_D(D_2)
\]

Таким образом, функция $\mu_D(D) = C\cdot|D|$ удовлетворяет всем требованиям меры сложности данных в рамках сделанных предположений.
\end{proof}

Частным случаем~\textit{условной сложности выборки} является достаточный объем выборки~--- минимальный объем данных из выборки~$D$ необходимый для обучения модели~$f$.

\begin{definition}
    Размер выборки $m^*$ называется \textit{достаточным} согласно критерию $T$, если $T$ выполняется для всех $k \geqslant m^*$.
\end{definition}

Таким образом исследования достаточного объема выборки является частным случаем предложенного определения меры сложности данных.
Подробный методов оценки достаточного объема выборки рассматривается в главе~\ref{chapter:samplesize}.


\section{Сходимость ландшафта оптимизационной задачи, как мера сложности модели}\label{chapter:complexity:loss}

\begin{figure}[h!t]\center
    \centering
    \includegraphics[width=0.7\linewidth]{figures/chapter-3/losses_difference.pdf}
    \caption{Пример изменения функции потерь при добавлении нового объекта}
    \label{fig-chapter-3-losses-difference}
\end{figure}


Рассмотрим выборку из простой генеральной совокупности~$\Gamma_C$:
\begin{equation}
    D = \left\{ (\mathbf{x}_i, \mathbf{y}_i) \right\}, \quad i = 1, \ldots, m, \quad \mathbf{x} \in \mathcal{X}, \ \mathbf{y} \in \mathcal{Y}, \quad D\subset \Gamma_C.
\end{equation}

Рассмотрим некоторое параметрическое отображение~$f_{\boldsymbol{\theta}}: \mathcal{X} \to \mathcal{Y},$ которое аппроксимирует условное распределение целевой переменной для заданного признакового описания объекта~$p(\mathbf{y}|\mathbf{x}).$ Параметры~$\boldsymbol{\theta}$ функции~$f_{\boldsymbol{\theta}}$ принадлежат пространству~$\mathbb{R}^{P},$ где~$P$ описывает число параметров отображения~$f_{\boldsymbol{\theta}}$.

Пусть, для выбора оптимального вектора параметров~$\hat{\boldsymbol{\theta}}$ используется подход минимизации эмпирического риска:
\begin{equation}
    \hat{\boldsymbol{\theta}} = \arg\min_{\boldsymbol{\theta}} \mathcal{L}_m(\boldsymbol{\theta}),
\end{equation}
где функция эмпирического риска для выборки размера~$|D|=m$ задается в следующем виде: 
\begin{equation}
    \mathcal{L}_m(\boldsymbol{\theta}) = \dfrac{1}{m} \sum\limits_{i=1}^{m} \ell(f_{\boldsymbol{\theta}}(\mathbf{x}_i), \mathbf{y}_i) \approx \mathbb{E}_{(\mathbf{x}, \mathbf{y}) \sim p(\mathbf{x}, \mathbf{y})} \left[ \ell(f_{\boldsymbol{\theta}}(\mathbf{x}), \mathbf{y}) \right],
\end{equation}
где функция~$\ell\left(\mathbf{z}, \mathbf{y}\right)$ описывает ошибку на одном объекте. Далее в качестве функции~$\ell$ будут рассматриваться либо кросс-энтропийная функция ошибка либо средняя квадратическая ошибка, в зависимости от рассматриваемой задачи и архитектуры модели.

Заметим, что функция эмпирического риска~$\mathcal{L}_m(\boldsymbol{\theta})$ задает некоторую поверхностью в пространстве размерности~$P.$
Изменение значения при добавлении одного объекта
\begin{align}\label{chapter:complex:equation-difference}
    \mathcal{L}_{k+1}(\boldsymbol{\theta}) - \mathcal{L}_k(\boldsymbol{\theta}) &= \frac{1}{k+1}\sum_{i=1}^k\ell(f_{\boldsymbol{\theta}}(\mathbf{x}_{i}), \mathbf{y}_{i}) - \frac{1}{k+1}\sum_{i=1}^k\ell(f_{\boldsymbol{\theta}}(\mathbf{x}_{i}), \mathbf{y}_{i}) = \\
    &= \frac{1}{k+1}\ell(f_{\boldsymbol{\theta}}(\mathbf{x}_{k+1}), \mathbf{y}_{k+1})-\sum_{i=1}^{k}\frac{1}{k(k+1)}\ell(f_{\boldsymbol{\theta}}(\mathbf{x}_{i}), \mathbf{y}_{i}) = \\
    &= \dfrac{1}{k+1} \left(\ell(f_{\boldsymbol{\theta}}(\mathbf{x}_{k+1}), \mathbf{y}_{k+1}) - \mathcal{L}_{k}(\boldsymbol{\theta})\right).
\end{align}

Дальнейшее исследования ландшафта нацелено на исследования данной разницы, причем особенно особенно интересуют предельные свойства при стремлении размера выборки к бесконечности. Для дальнейших оценок данной разности вводиться предположение~\ref{chapter:complex:assumption-local-optima-not-change}, которое в целом подтверждается на практике, но в свою очередь является достаточно сильным, что упрощает дальнейшие выкладки.

\begin{assumption}\label{chapter:complex:assumption-local-optima-not-change}
    Пусть $\boldsymbol{\theta}^*$ является локальным минимумом обеих эмпирических функций потерь $\mathcal{L}_{k}(\boldsymbol{\theta})$ и $\mathcal{L}_{k+1}(\boldsymbol{\theta})$, т.е.
    \[
        \nabla \mathcal{L}_{k}(\boldsymbol{\theta}^*) = \nabla \mathcal{L}_{k+1}(\boldsymbol{\theta}^*) = \mathbf{0}.
    \]
\end{assumption}
Содержательно, предположение \ref{chapter:complex:assumption-local-optima-not-change} имеет простую эвристическую интерпретацию, что новый объект данных является \textit{репрезентативным} для уже обученной модели~--- он не приносит ``новой информации'', а лишь уточняет ее.
В целом при асимптотически большом объеме выборки, данное свойство не противоречит эмпирическим результатам.

Воспользуемся квадратичным приближением Тейлора для упомянутых выше функций потерь в окрестности точки $\boldsymbol{\theta}^*$. Предполагаем, что разложение до второго порядка будет достаточным для изучения локального поведения. Член первого порядка обращается в ноль, поскольку градиенты $\nabla \mathcal{L}_{k}(\boldsymbol{\theta}^*)$ и $\nabla \mathcal{L}_{k+1}(\boldsymbol{\theta}^*)$ равны нулю:
\begin{equation}\label{chapter:complex:equation-approx}
    \mathcal{L}_{k}(\boldsymbol{\theta}) \approx \mathcal{L}_{k}(\boldsymbol{\theta}^*) + \dfrac{1}{2} (\boldsymbol{\theta} - \boldsymbol{\theta}^*)^\top \mathbf{H}^{(k)}(\boldsymbol{\theta}^*) (\boldsymbol{\theta} - \boldsymbol{\theta}^*),
\end{equation}
где мы обозначили гессиан функции $\mathcal{L}_{k}(\boldsymbol{\theta})$ по параметрам $\boldsymbol{\theta}$ в точке $\boldsymbol{\theta}^*$ как $\mathbf{H}^{(k)}(\boldsymbol{\theta}^*) \in \mathbb{R}^{P \times P}$. Более того, полный гессиан может быть записан как среднее значение гессианов отдельных членов эмпирической функции потерь:
\[
    \mathbf{H}^{(k)}(\boldsymbol{\theta}) = \nabla^2_{\boldsymbol{\theta}} \mathcal{L}_{k}(\boldsymbol{\theta}) = \dfrac{1}{k} \sum\limits_{i=1}^{k} \nabla^2_{\boldsymbol{\theta}} \ell(f_{\boldsymbol{\theta}}(\mathbf{x}_{i}), \mathbf{y}_{i}) = \dfrac{1}{k} \sum\limits_{i=1}^{k} \mathbf{H}_{i}(\boldsymbol{\theta}).
\]

Следовательно, используя полученное квадратичное приближение~\eqref{chapter:complex:equation-approx}, формула для разности потерь~\eqref{chapter:complex:equation-difference} принимает вид:
\begin{align}
    \mathcal{L}_{k+1}(\boldsymbol{\theta}) - \mathcal{L}_k(\boldsymbol{\theta}) &= \dfrac{1}{k+1} \left( \ell(f_{\boldsymbol{\theta}^*}(\mathbf{x}_{k+1}), \mathbf{y}_{k+1}) - \dfrac{1}{k} \sum\limits_{i=1}^{k} \ell(f_{\boldsymbol{\theta}^*}(\mathbf{x}_{i}), \mathbf{y}_{i}) \right) +\\
    &\quad+ \dfrac{1}{k+1} (\boldsymbol{\theta} - \boldsymbol{\theta}^*)^\top \left( \mathbf{H}_{k+1}(\boldsymbol{\theta}^*) - \dfrac{1}{k} \sum\limits_{i=1}^{k} \mathbf{H}_{i}(\boldsymbol{\theta}^*) \right) (\boldsymbol{\theta} - \boldsymbol{\theta}^*),
\end{align}
причем, используя неравенство треугольника, мы можем вывести следующую оценку:
\begin{align}
    \left| \mathcal{L}_{k+1}(\boldsymbol{\theta}) - \mathcal{L}_k(\boldsymbol{\theta}) \right| &\leqslant \dfrac{1}{k+1} \left| \ell(f_{\boldsymbol{\theta}^*}(\mathbf{x}_{k+1}), \mathbf{y}_{k+1}) - \dfrac{1}{k} \sum\limits_{i=1}^{k} \ell(f_{\boldsymbol{\theta}^*}(\mathbf{x}_{i}), \mathbf{y}_{i}) \right| +\\
    &\quad+ \dfrac{1}{k+1} \left\|\boldsymbol{\theta} - \boldsymbol{\theta}^*\right\|_2^2 \left\| \mathbf{H}_{k+1}(\boldsymbol{\theta}^*) - \dfrac{1}{k} \sum\limits_{i=1}^{k} \mathbf{H}_{i}(\boldsymbol{\theta}^*) \right\|_2.
\end{align}

Заметим, что первое слагаемое может быть легко ограничено константой, поскольку сама функция потерь принимает ограниченные значения.
Однако выражение с гессианами не так просто оценить.
Подробный анализ матриц Гессе для различных типов параметрических моделей глубокого обучения представлен в главе~\ref{chapter:gesian}.
Таким образом, мы анализируем локальную сходимость ландшафта функции, ее матрицу Гессе.

Итого, получаем следующее интересующее нас выражение для анализа, описывающее поведение ландшафта функции потерь:
\begin{equation}\label{chapter:complex:equation-mod-difference}
    \left| \mathcal{L}_{k+1}(\boldsymbol{\theta}) - \mathcal{L}_k(\boldsymbol{\theta}) \right| \leqslant M_l+ \dfrac{1}{k+1} \left\|\boldsymbol{\theta} - \boldsymbol{\theta}^*\right\|_2^2 \left\| \mathbf{H}_{k+1}(\boldsymbol{\theta}^*) - \dfrac{1}{k} \sum\limits_{i=1}^{k} \mathbf{H}_{i}(\boldsymbol{\theta}^*) \right\|_2.
\end{equation}

% Вторая глава
\clearpage
\chapter{Аппроксимация матриц Гессе в нейросетевых архитектурах}
В современных задачах оптимизации, к которым редуцируется процесс обучения моделей глубокого обучения, фундаментальную роль играет анализ свойств целевой функции потерь $\mathcal{L}(\boldsymbol{\theta})$, заданной на многомерном пространстве параметров $\boldsymbol{\theta} \in \mathbb{R}^n$.
Глубокие нейронные сети, обладающие способностью к аппроксимации сложных нелинейных зависимостей, порождают высокосложные не выпуклые функции потерь с многочисленными локальными минимумами, седловыми точками и сложным ландшафтом оптимизационной задачи.
Если градиент $\nabla \mathcal{L}(\boldsymbol{\theta})$ характеризует скорость и направление наискорейшего спуска в параметрическом пространстве, то матрица Гессе $\mathbf{H}(\mathcal{L})$~--- симметричная матрица вторых частных производных функции потерь~--- предоставляет информацию о ее локальной кривизне, описывающая геометрические свойства ландшафта.
Матрица Гессе содержит информацию о локальном поведении функции в окрестности заданной точки, позволяя не только предсказывать траекторию оптимизации, но и анализировать устойчивость найденных решений.

Формальное определение матрицы Гессе для функции $\mathcal{L}(\boldsymbol{\theta})$ от $n$ параметров задается следующим выражением:
\[
\mathbf{H}(\mathcal{L})_{ij} = \frac{\partial^2 \mathcal{L}}{\partial \theta_i \partial \theta_j},
\]
где индексы $i, j = 1, \dots, n$ соответствуют компонентам вектора параметров $\boldsymbol{\theta}$.
Для функций, обладающих непрерывными вторыми производными, матрица Гессе симметрична в силу теоремы Шварца-Клеро-Янга о равенстве смешанных производных.
Эта симметричность обеспечивает вещественность всех собственных значений и ортогональность соответствующих собственных векторов, что имеет фундаментальное значение для спектрального анализа в дальнейшем.

Применения матрицы Гессе в контексте глубокого обучения проявляется в решении разнообразных теоретических и практических задач.
В аспекте оптимизации, на основе матрицы Гессе строятся методы второго порядка, такие как метод Ньютона, который использует обратную матрицу Гессе~$\mathbf{H}^{-1}$ для вычисления адаптивных направлений обновления параметров, учитывающих локальную кривизну поверхности потерь.
Это свойство обеспечивает существенное ускорение сходимости в окрестности локального минимума по сравнению с методами первого порядка, основанными исключительно на градиентной информации.
Однако прямая реализация методов второго порядка сопряжена с существенными вычислительными сложностями, что стимулировало развитие квази-ньютоновских методов и методов приближенного вычисления обратной матрицы Гессе.

В контексте теоретического анализа моделей машинного обучения, матрица Гессе вносит существенный вклад в оценку сложности и обобщающей способности моделей. 
Собственные значения матрицы Гессе, вычисленные в стационарной точке, содержат информацию о геометрии ландшафта функции потерь.
Спектральный анализ матрицы Гессе позволяет количественно охарактеризовать локальную кривизну функции потерь вдоль различных направлений в параметрическом пространстве, выявить наличие седловых точек и оценить устойчивость найденного решения:
\begin{itemize}
    \item[--] Малые собственные значения соответствуют направлениям с незначительной кривизной~--- так называемым ``плоским'' регионам, где параметры могут варьироваться без существенного роста ошибки.
    Эти направления часто ассоциируются с параметрами, оказывающими незначительное влияние на выход модели, или с симметриями в архитектуре сети.
    В противоположность этому, большие собственные значения указывают на ``острые'' минимумы с выраженной кривизной.
    Эмпирические и теоретические исследования подтверждают, что плоские минимумы демонстрируют улучшенную обобщающую способность, обусловленную их пониженной чувствительностью к малым возмущениям в данных и параметрах модели.
    Этот феномен, известный как "острота" минимума, активно изучаится в современной теории глубокого обучения и оптимизации.
    \item[--] След матрицы Гессе, равный сумме ее собственных значений, является интегральной характеристикой общей кривизны функции потерь. 
    Детальный анализ спектрального состава матрицы Гессе, в частности оценка кратности собственных значений вблизи нуля, позволяет количественно оценить эффективную размерность пространства параметров, влияющих на выход модели, и идентифицировать структурную избыточность в архитектуре нейронной сети.
    Кроме того, распределение собственных значений матрицы Гессе тесно связано с устойчивостью модели к шуму, способность к интерполяции данных и обобщающая способность на тестовых выборках.
\end{itemize}

Таким образом, матрица Гессе является не только эффективным инструментов для ускорения процесса оптимизации, но и является аналитическим аппаратом для анализа внутренних свойств модели: от устойчивости найденного решения до прогнозирования его способности к обобщению на новые данные.
Однако, использование матрицы Гессе в задачах глубокого обучения с миллионами и миллиардами параметров сталкивается с вычислительными ограничениями, поскольку требования к памяти и вычислительным ресурсам для хранения и обращения плотной матрицы размерности~$n \times n$ становятся непрактичными для реальных приложений.
Это методологическое ограничение обуславливает актуальность разработки эффективных методов аппроксимации матрицы Гессе и ее ключевых спектральных характеристик, чему и посвящена настоящая глава.
Современные подходы к решению этой проблемы включают методы случайного проектирования, разложения Кронекера, диагональные и блочно-диагональные аппроксимации, а также методы, основанные на теории случайных матриц.

\section{Полносвязная нейросетевая модель глубокого обучения}

Рассмотрим формальную постановку задачи $K$--классовой классификации с использованием функции потерь кросс-энтропии.
В данной постановке входные данные представляются вектором $\mathbf{x} \in \mathbb{R}^{l}$, а выходные данные~--- вектором $\mathbf{y} \in \mathbb{R}^{K}$, имеющим структуру one-hot кодирования, где все компоненты равны нулю, за исключением позиции $y_k = 1$, соответствующей истинной метке класса для входного образца $\mathbf{x}$.
Такое представление данных является стандартным для задач классификации и позволяет естественным образом использовать категориальное распределение для моделирования неопределенности предсказаний.

Рассматривается $L$-слойная полносвязная нейронная сеть $f_{\boldsymbol{\theta}}(\cdot)$ с функцией активации ReLU, применяемой после каждого линейного преобразования.
Выбор функции активации ReLU обусловлен ее вычислительной эффективностью и свойством устранять проблему затухающих градиентов, а также ее удобство для анализа в теоретическом анализе нейросетевых архитектур.
Для функции активации ReLU, определяемой как $\sigma(\mathbf{x}) = \left[ \mathbf{x} \geqslant\mathbf{0} \right] \mathbf{x},$ где $[\cdot]$ обозначает поэлементную индикаторную функцию, выход сети представляет собой вектор логитов $\mathbf{z} \in \mathbb{R}^{K}$.
Вычисление логитов осуществляется посредством последовательного применения следующих рекуррентных соотношений:
\begin{align}
    \mathbf{z}^{(p)} &= \mathbf{W}^{(p)} \mathbf{x}^{(p)} + \mathbf{b}^{(p)}, \\
    \mathbf{x}^{(p+1)} &= \sigma(\mathbf{z}^{(p)}).
\end{align}
Здесь $\mathbf{x}^{(p)}$ и $\mathbf{z}^{(p)}$ обозначают вход и выход $p$-го слоя соответственно, при этом полагается $\mathbf{x}^{(1)} = \mathbf{x}$ и $\mathbf{z} = f_{\boldsymbol{\theta}}(\mathbf{x}) = \mathbf{z}^{(L)}$. 
Совокупность всех параметров модели обозначается как $\boldsymbol{\theta} = \mathrm{col}(\mathbf{w}^{(1)}, \mathbf{b}^{(1)}, \ldots, \mathbf{w}^{(L)}, \mathbf{b}^{(L)}) \in \mathbb{R}^{n}$. Для $p$-го слоя $\mathbf{w}^{(p)}$ представляет собой векторизованную матрицу весов $\mathbf{W}^{(p)}$, а $\mathbf{b}^{(p)}$~--- соответствующий вектор смещений.
Общее число параметров $n$ в таких моделях может достигать миллионов и даже миллиардов, что и создает вычислительные трудности для точного вычисления матрицы Гессе.

Выходы модели определяется как $\mathbf{p} = \mathrm{softmax}(\mathbf{z}) \in \mathbb{R}^{K}$, где каждая компонента вычисляется по формуле:
\[
    p_i = \mathrm{softmax}(\mathbf{z})_i = \dfrac{\exp{(z_i)}}{\sum_{j=1}^{K} \exp{(z_j)}} \in (0; 1).
\] 
Функция потерь представляет собой стандартную кросс-энтропийную функцию ошибки:
\[
    \ell(\mathbf{z}, \mathbf{y}) = \mathrm{CE}(\mathbf{p}, \mathbf{y}) = - \sum_{k=1}^{K} y_k \log p_k \in \mathbb{R}^{+}.
\]
Эта функция является выпуклой по логитам $\mathbf{z}$, но невыпуклой по параметрам сети $\boldsymbol{\theta}$ из-за сложной композиционной структуры нейронной сети.

Согласно установленным результатам в литературе \cite{sagun2018empiricalanalysishessianoverparametrized}, применение цепного правила для матриц второго порядка \cite{skorski2019chainruleshessianhigher} позволяет декомпозировать матрицу Гессе на сумму двух структурно различных компонент:
\[
    \mathbf{H}_{i}(\boldsymbol{\theta}) = \underbrace{\nabla_{\boldsymbol{\theta}} \mathbf{z}_i \dfrac{\partial^2 \ell(\mathbf{z}_i, \mathbf{y}_i)}{\partial \mathbf{z}_{i}^2} \nabla_{\boldsymbol{\theta}} \mathbf{z}_i^{\text{T}} }_{\text{G-компонента}} + \underbrace{\sum\limits_{k=1}^{K} \dfrac{\partial \ell(\mathbf{z}_i, \mathbf{y}_i)}{\partial z_{ik}} \nabla^2_{\boldsymbol{\theta}} z_{ik}}_{\text{H-компонента}},
\]
где $\nabla_{\boldsymbol{\theta}} \mathbf{z}_i \in \mathbb{R}^{P \times K}$ представляет собой матрицу Якоби функции нейронной сети по параметрам, а $\dfrac{\partial^2 \ell(\mathbf{z}_i, \mathbf{y}_i)}{\partial \mathbf{z}_{i}^2}$~--- матрицу Гессе функции потерь относительно выходных логитов для $i$-го наблюдения. Первое слагаемое (G-компонента) отражает влияние кривизны функции потерь, в то время как второе слагаемое (H-компонента)~--- кривизну самой нейронной сети.

Эмпирические исследования \cite{pmlr-v97-ghorbani19b,sagun2018empiricalanalysishessianoverparametrized,papyan2019spectrumdeepnethessiansscale} демонстрируют, что спектральное распределение матрицы Гессе характеризуется наличием основной массы собственных значений, сосредоточенной вблизи нуля (обусловленной H-компонентой), и выбросов, распределенных в области ненулевых значений (обусловленных G-компонентой). Это бимодальное распределение собственных значений является характерной чертой гессиинов глубоких нейронных сетей и отражает фундаментальные свойства параметрического пространства таких моделей. Вследствие данной спектральной структуры, для практического анализа наиболее релевантной является G-компонента, что обосновывает использование следующей аппроксимации:
\[
    \mathbf{H}_{i}(\boldsymbol{\theta}) \approx \nabla_{\boldsymbol{\theta}} \mathbf{z}_i \dfrac{\partial^2 \ell(\mathbf{z}_i, \mathbf{y}_i)}{\partial \mathbf{z}_{i}^2} \nabla_{\boldsymbol{\theta}} \mathbf{z}_i^{\text{T}}.
\]

Дополнительное теоретическое обоснование данной аппроксимации предоставляется в рамках теории ядра нейронного касательного пространства \cite{NEURIPS2018_5a4be1fa,Lee_2020}, где предполагается линейная зависимость логитов $\mathbf{z}$ от параметров $\boldsymbol{\theta}$ в окрестности точки оптимума. Данное предположение имплицирует исчезающую кривизну логитов $\nabla^2_{\boldsymbol{\theta}} z_{ik}$, что влечет тождественное обращение в ноль H-компоненты.

На основе работ \cite{wu2022dissecting}, предлагающих аналитическую аппроксимацию G-компоненты для полносвязных нейронных сетей, принимается следующая параметризация: $\mathbf{H}_{i}(\boldsymbol{\theta}) \approx \mathbf{F}_i^{\text{T}} \mathbf{A}_i \mathbf{F}_i$.
Эта факторизация позволяет эффективно вычислять приближения гессиана без явного построения полной матрицы, используя разложения в матрицы меньшей размерности.
Введем систему обозначений (для упрощения записи индекс $i$ опущен):

\begin{itemize}
    \item Матричное представление функции активации ReLU:
    \[
        \mathbf{D}^{(p)} = \mathrm{diag}([\mathbf{z}^{(p)} \geqslant \mathbf{0}]),
    \]
    Эта диагональная матрица кодирует паттерн активации нейронов на $p$-м слое и играет основную роль в определении функциональной структуры сети.
    
    \item Матрица прямого распространения от $p$-го слоя к выходу:
    \[
        \mathbf{G}^{(p)} = \dfrac{\partial \mathbf{z}}{\partial \mathbf{z}^{(p)}} = \mathbf{W}^{(L)} \mathbf{D}^{(L-1)} \mathbf{W}^{(L-1)} \mathbf{D}^{(L-2)} \cdot \ldots \cdot \mathbf{D}^{(p)}, 
    \]
    Эта матрица описывает, как изменения в активациях на $p$-м слое через последующие слои к выходу сети.
    
    \item Блочная матрица всех производных логитов по параметрам:
    \[
        \mathbf{F}^{\text{T}} =
        \begin{pmatrix}
            (\mathbf{G}^{(1)})^{\text{T}} \otimes \mathbf{x}^{(1)} \\
            (\mathbf{G}^{(1)})^{\text{T}} \\ 
            \vdots \\
            (\mathbf{G}^{(L)})^{\text{T}} \otimes \mathbf{x}^{(L)} \\
            (\mathbf{G}^{(L)})^{\text{T}} \\ 
        \end{pmatrix}, 
    \]
    где $\otimes$ обозначает произведение Кронекера. Эта блочная структура естественным образом отражает слоистую архитектуру сети и позволяет эффективное вычисление.
    
    \item Гессиан функции потерь относительно логитов, имеющий структуру ковариационной матрицы \cite{singla2019understanding}:
    \[
        \mathbf{A} = \nabla^2_\mathbf{z} \ell(\mathbf{z}, \mathbf{y}) = \mathrm{diag}(\mathbf{p}) - \mathbf{p} \mathbf{p}^{\text{T}}.
    \]
    Эта матрица является положительно полуопределенной и вырожденной, что отражает инвариантность функции softmax к сдвигам в пространстве логитов.
\end{itemize}

На основе предложенной параметризации получена верхняя оценка спектральной нормы матрицы Гессе в полносвязной нейронной сети, формулируемая в теореме~\eqref{thm:hess}.

\section{Матричные модели глубокого обучения}

В данном разделе рассматривается общий класс матричных моделей глубокого обучения, которые представляют собой композицию последовательных линейных преобразований и нелинейных функций активации.
Частным случаем такого представления являются сверточные нейронные сети (англ. CNN), а также другие архитектуры, где каждый слой может быть представлен в виде линейного оператора.

Пусть $f_{\boldsymbol{\theta}}(\mathbf{x})$ является суперпозицией $L+1$ слоев с активациями ReLU, что формально записывается как:
\[
    f_{\boldsymbol{\theta}}(\mathbf{x}) = \mathbf{T}^{(L+1)} \circ \sigma \circ \dots \circ \sigma \circ \mathbf{T}^{(1)}(\mathbf{x}).
\]

В этом представлении каждый $\mathbf{T}^{(p+1)}$ представляет собой линейный оператор или его матричное представление, а $\sigma$ обозначает функцию активации ReLU, применяемую поэлементно.
Такая композиционная структура позволяет описывать глубокие нейронные сети как последовательность преобразований, где каждый слой осуществляет линейное отображение с последующей нелинейной активацией.

Промежуточные результаты вычисления функции сети могут быть представлены в виде системы уравнений:
\begin{align}
  \begin{cases}
    \mathbf{z}^{(p+1)} &= \mathbf{T}^{(p+1)}\mathbf{x}^{(p)}, \\
    \mathbf{x}^{(p+1)} &= \sigma(\mathbf{z}^{(p+1)})
  \end{cases}
\end{align}
где выход сети определяется как $f_{\boldsymbol{\theta}}(\mathbf{x}) = \mathbf{z} := \mathbf{z}^{(L+1)}$, а входные данные задаются как $\mathbf{x}^{(0)} := \mathbf{x}$.
Здесь $\mathbf{z}^{(p+1)}$ представляет собой выход линейного оператора на $(p+1)$-м слое до применения активации, а $\mathbf{x}^{(p+1)}$~--- результат после применения функции активации ReLU, который является входом для следующего слоя.

Рассмотрим матрицу~$\boldsymbol{\Lambda}^{(p+1)} := \mathrm{diag}(\mathbf{x}^{(p+1)} > 0)$, зависящую от входных данных, которая кодирует паттерн активации нейронов на $(p+1)$-м слое.
Элементы этой матрицы равны 1 для нейронов с положительной активацией и 0 в противном случае, что отражает свойство функции ReLU "отсекать" отрицательные значения и представляет функцию активации в виде линейного оператора.
Используя эти диагональные матрицы, всю функцию нейронной сети можно представить в виде произведения матриц, а имменно супперпозиций линейных операторов:
\begin{equation}\label{eq::m-net:repr}
    f_{\boldsymbol{\theta}}(\mathbf{x}) = \mathbf{T}^{(L+1)}\boldsymbol{\Lambda}^{(L)}\dots \boldsymbol{\Lambda}^{(1)}\mathbf{T}^{(1)}\mathbf{x}.
\end{equation}

Данное представление удобно, так как позволяет рассматривать глубокую нейронную сеть с активациями ReLU как кусочно-линейную функцию, где нелинейность возникает исключительно за счет бинарных переключений в матрицах $\boldsymbol{\Lambda}^{(p)}$, зависящих от входных данных.

Вектор параметров модели объединяет все обучаемые параметры сети: $\boldsymbol{\theta} = \mathrm{col}(\mathbf{W}^{(L+1)}, \dots, \mathbf{W}^{(1)})$, где каждый линейный оператор $\mathbf{T}^{(p)}$ дифференцируемо параметризуется соответствующей частью вектора параметров $\mathbf{W}^{(p)}$. Для анализа модели вводится производная слоя по его параметрам:
\[
    \mathbf{Q}^{(p)} := \frac{\partial \mathbf{T}^{(p)}}{\partial \mathbf{W}^{(p)}},
\]
а затем строится блочно-диагональная матрица, объединяющая эти производные по всем слоям:
\[
    \mathbf{Q}:= \mathrm{diag}({\mathbf{Q}}^{(1)}, \dots, {\mathbf{Q}}^{(L+1)}).
\]

Матрица $\mathbf{Q}^{(p)}$ полностью описывает расположение параметров в $p$-м слое и их влияние на линейное преобразование этого слоя.

Для дальнейшего анализа вводятся дополнительные обозначения, которые описывают предшествующие и последующие преобразования относительно $p$-го слоя. Преобразования, которые последуют после $p$-го слоя:
\begin{align}
    \mathbf{G}^{(p)} &:= \mathbf{T}^{(L+1)}\boldsymbol{\Lambda}^{(L)} \dots \mathbf{T}^{(p+1)}\boldsymbol{\Lambda}^{(p)}; \\
    \mathbf{G}^{(L+1)} &:= \mathbf{I};
\end{align}
и преобразования, которые предшествовали $p$-му слою:
\begin{align}
    \mathbf{R}^{(p)} &:= \boldsymbol{\Lambda}^{(p)}\mathbf{T}^{(p)} \dots \boldsymbol{\Lambda}^{(1)}\mathbf{T}^{(1)};\;\;p = \overline{1,L}; \\ 
    \mathbf{R}^{(0)} &:= \mathbf{I}.
\end{align}
Матрица $\mathbf{G}^{(p)}$ описывает линейные преобразования от $p$-го слоя к выходу сети, в то время как $\mathbf{R}^{(p)}$ описывает линейные преобразование входа до $p$-го слоя.

Используя введенные обозначения, запишем следующие выражения:
\begin{align}
    \mathbf{z} &= \mathbf{G}^{(p)}\mathbf{z}^{(p)},\\
    \mathbf{x}^{(p)} &= \mathbf{R}^{(p)}\mathbf{x}, \\
    \mathbf{z} &= f_{\boldsymbol{\theta}}(\mathbf{x}) = \mathbf{G}^{(p)}\mathbf{T}^{(p)}\mathbf{R}^{(p-1)}\mathbf{x}.
\end{align}
Первое уравнение выражает выход сети через промежуточные значения на $p$-м слое, второе показывает преобразование входного сигнала до $p$-го слоя, а третье дает полное представление выхода сети через параметры $p$-го слоя и преобразования до и после него.

Объединяя матрицы $\mathbf{G}^{(p)}$ и $\mathbf{R}^{(p)}$ в единый оператор, получаем расширенную матрицу:
\begin{align}
\mathbf{F}^T := 
    \begin{pmatrix}
        \mathbf{G}^{(1)^T} \otimes \mathbf{R}^{(0)}\mathbf{x} \\
        \vdots \\
        \mathbf{G}^{(k)^T} \otimes \mathbf{R}^{(k-1)}\mathbf{x} \\
        \vdots \\
        \mathbf{G}^{(L+1)^T} \otimes \mathbf{R}^{(L)}\mathbf{x}
    \end{pmatrix}.
\end{align}
Эта блочная матрица содержит в себе всю информацию о том, как изменения параметров в различных слоях влияют на выход сети.
Каждый блок этой матрицы соответствует определенному слою и содержит информацию как о линейных преобразованиях от этого слоя к выходу ($\mathbf{G}^{(k)^T}$), так и о преобразовании входа до этого слоя ($\mathbf{R}^{(k-1)}\mathbf{x}$).

В случае использования функции потерь кросс-энтропии (CE) для задач классификации, гессиан функции потерь относительно логитов имеет структуру:
\[
    \mathbf{A} := \nabla^2_{\mathbf{z}} \ell = \mathrm{diag}(\mathbf{p}) - \mathbf{p}\mathbf{p}^T,
\]
где $\mathbf{p} := \mathrm{softmax}(\mathbf{z})$ представляет вектор вероятностей, предсказанных моделью.
Матрица $\mathbf{A}$ является ковариационной матрицей многомерного распределения и обладает свойствами положительной полуопределенности и вырожденности, что отражает инвариантность функции softmax к сдвигам в пространстве логитов.

\subsection{Матричная факторизация матрицы Гессе}

\begin{theorem}\label{theorem:m-net:hesstruct}
    Пусть функция нейронной сети $f_{\boldsymbol{\theta}}(\mathbf{x})$ представима в виде \eqref{eq::m-net:repr}, тогда матрица Гессе функции потерь относительно параметров модели может быть представлен в факторизованной форме:
    $\mathbf{H}_O(\boldsymbol{\theta}) = \mathbf{Q}^{T}\mathbf{F}^{T}\mathbf{A}\mathbf{F}\mathbf{Q}$, где $\mathbf{H}_O$ описывает G-компоненту матрицы Гессе.
\end{theorem}
\begin{proof}
Доказательство теоремы основано на последовательном применении цепного правила матричного дифференцирования и использовании свойств произведения Кронекера.
Исходное представление выхода матричной нейросетевой модели:
\[
    \mathbf{z} = f_{\boldsymbol{\theta}}(\mathbf{x}) = \mathbf{T}^{(L+1)}\boldsymbol{\Lambda}^{(L)}\mathbf{T}^{(L)}....\boldsymbol{\Lambda}^{(1)}\mathbf{T}^{(1)}\mathbf{x}.
\]

Производные выхода сети по параметрам модели вычисляются с использованием цепного правила:
\begin{align}
 \frac{\partial\mathbf{z}}{\partial\mathbf{W}^{(p)}} = \frac{\partial\mathbf{z}}{\partial\mathbf{z}^{(p)}} \frac{\partial\mathbf{z}^{(p)}}{\partial\mathbf{T}^{(p)}} \frac{\partial\mathbf{T}^{(p)}}{\partial\mathbf{W}^{(p)}}.
\end{align}

Для вычисления $\frac{\partial\mathbf{z}^{(p)}}{\partial\mathbf{T}^{(p)}}$ используется тождество для векторизации матричных произведений: $vec(\mathbf{B}\mathbf{V}\mathbf{A}^T) = (\mathbf{A} \otimes \mathbf{B})vec(\mathbf{V})$.
Применяя это тождество с $\mathbf{A} = \mathbf{I}$ и векторизуя $\mathbf{z}^{(p)} = \mathbf{T}^{(p)}\mathbf{x}^{(p-1)}$, получаем:
\[
    vec(\mathbf{z}^{(p)}) = vec(\mathbf{T}^{(p)}\mathbf{x}^{(p-1)}) = (\mathbf{I} \otimes \mathbf{x}^{(p-1)})vec(\mathbf{T}^{(p)}).
\]

Отсюда следует, что:
\[
    \frac{\partial \mathbf{z}^{(p)}}{\partial \mathbf{T}^{(p)}} = \mathbf{I} \otimes \mathbf{x}^{(p-1)^T}.
\]

Используя выражение $\mathbf{z} = \mathbf{G}^{(p)}\mathbf{z}^{(p)}$, получаем производную выхода сети по промежуточному значению:
\[
    \frac{\partial \mathbf{z}}{\partial \mathbf{z}^{(p)}} = \mathbf{G}^{(p)}.
\]

По определению $\mathbf{Q}^{(p)}$ имеем:
\[
    \frac{\partial\mathbf{T}^{(p)}}{\partial\mathbf{W}^{(p)}} = {\mathbf{Q}}^{(p)}.
\]

Для объединения этих выражений используется свойство произведения Кронекера: если $\mathbf{A}_i \in \mathbb{R}^{m_i \times n_i}$, то $\mathbf{A}_1 \otimes \mathbf{A}_2 = (\mathbf{A}_1 \otimes \mathbf{I}_{m_2})(\mathbf{I}_{m_1} \otimes \mathbf{A}_2)$.
Применяя это свойство с $m_2 = 1$, получаем:
\[
    \mathbf{G}^{(p)}\big(\mathbf{I} \otimes \mathbf{x}^{(p-1)^T}\big) = \big(\mathbf{G}^{(p)} \otimes \mathbf{I}_1\big)\big(\mathbf{I} \otimes \mathbf{x}^{(p-1)^T}\big) = \mathbf{G}^{(p)} \otimes \mathbf{x}^{(p-1)^T}.
\]

Подставляя все компоненты в исходную формулу для производной, получаем окончательное выражение:
\[
    \frac{\partial\mathbf{z}}{\partial\mathbf{W}^{(p)}} = (\mathbf{G}^{(p)} \otimes \mathbf{I}_1)(\mathbf{I} \otimes \mathbf{x}^{(p-1)^T})\mathbf{Q}^{(p)} = (\mathbf{G}^{(p)} \otimes \mathbf{x}^{(p-1)^T})\mathbf{Q}^{(p)}.
\]

Используя результаты работ по анализу гессиана в нейронных сетях \cite{singh2023hessianperspectivenatureconvolutional}, получаем выражение для блоков матрицы Гессе:
\begin{align}
    \mathbf{H}_O^{(kl)} &= J(\boldsymbol{\theta})^T \mathbf{A} J(\boldsymbol{\theta}) = \\
    &= \mathbf{Q}^{(k)^T}(\mathbf{G}^{(k)^T} \otimes \mathbf{R}^{(k-1)}\mathbf{x})A(\mathbf{G}^{(l)} \otimes \mathbf{x}^T\mathbf{R}^{(l-1)^T})\mathbf{Q}^{(l)}.
\end{align}

Объединяя все блоки в единую матрицу, получаем итоговое выражение для матрицы Гессе:
\[
    \mathbf{H}_O = \mathbf{Q}^\mathbf{T}\mathbf{F}\mathbf{A}\mathbf{F}^\mathbf{T}\mathbf{Q}.
\]
\end{proof}

Данная теорема устанавливает фундаментальный результат о структуре гессиана в матричных моделях глубокого обучения.
Предложенная факторизация позволяет эффективно анализировать и вычислять гессиан без необходимости явного построения полной матрицы вторых производных, что особенно важно для моделей с большим количеством параметров.
Структура $\mathbf{H}_O = \mathbf{Q}^\mathbf{T}\mathbf{F}\mathbf{A}\mathbf{F}^\mathbf{T}\mathbf{Q}$ подчеркивает, что гессиан может быть представлен как преобразование "внутреннего" гессиана $\mathbf{A}$ (зависящего только от логитов и функции потерь) с помощью матриц $\mathbf{F}$ и $\mathbf{Q}$, которые capture архитектурные свойства сети и параметризацию слоев соответственно.

\section{Матрица Гессе для трансформерной модели глубокого обучения}
Пусть $f_{\mathbf{w}}(\cdot)$ обозначает нейронную сеть, в данном случае слой самовнимания (англ. self-attention) или полный блок трансформера (англ. Transformer), с параметрами $\mathbf{w} \in \Omega$.
При наличии дважды дифференцируемых потерь $l(\cdot, \cdot)$ потери на выборку равны $l_i(\mathbf{w}) := l(f_{\mathbf{w}}(\mathbf{x}_i), \mathbf{y}_i)$.
Эмпирические потери для выборок $L = k$ равны $\mathcal{L}_k(\mathbf{w}) = \frac{1}{k} \sum_{i=1}^k l_i(\mathbf{w})$, с гессианом $\mathbf{H}^{(k)}(\mathbf{w}) = \frac{1}{k} \sum_{i=1}^k \nabla^2_{\mathbf{w}} l_i(\mathbf{w})$.


Пусть заданы входные вектора эмбедингов (англ. embeddings) $\mathbf{X} \in \mathbb{R}^{L \times d_V}$.
Выход слоя одной головы (англ. single-head) слоя самовнимания задается в виде: 
\begin{equation} \label{eq:self_attention}
    \mathbf{F}(\mathbf{X}) = \mathbf{A}(\mathbf{X}) \mathbf{X} \mathbf{W}_V,
\end{equation}
где $\mathbf{A}(\mathbf{X}) = \mathrm{softmax}\left( \frac{\mathbf{X} \mathbf{W}_Q \mathbf{W}_K^\top \mathbf{X}^\top}{\sqrt{d_K}} \right)$, а $\mathbf{W}_Q, \mathbf{W}_K \in \mathbb{R}^{d_V \times d_K}$, $\mathbf{W}_V \in \mathbb{R}^{d_V \times d_V}$.

Используя \eqref{eq:self_attention}, полный блок трансформера выглядит в следующим образом:
\begin{align}
    \text{LayerNorm}\Big(\mathbf{\text{LayerNorm}(\mathbf{X} + \mathbf{F}(\mathbf{X}))} + \mathrm{FFN}(\mathbf{\text{LayerNorm}(\mathbf{X} + \mathbf{F}(\mathbf{X}))})\Big)
\end{align}
где $\mathrm{FFN}(\cdot)$ является блоком полносвязной сети с некоторой нелинейностью.
Слой LayerNorm для входной матрицы $\mathbf{U} \in \mathbb{R}^{m \times n}$ описывается выражением:
\[
    \text{LayerNorm}(\mathbf{U})_{i,j} = \gamma_j \frac{\mathbf{U}_{i,j} - \mu_i}{\sqrt{\sigma_i^2}} + \mathbf{\beta}_j,
\]
где $\mu_i = \frac{1}{m} \sum_{j=1}^m \mathbf{U}_{i,j}, \quad \sigma_i^2 = \frac{1}{m} \sum_{j=1}^m (\mathbf{U}_{i,j} - \mu_i)^2$.

\begin{assumption}\label{assumption:LayerNorm}
Для входной матрицы слоя LayerNorm: $\mathbf{X} + \mathbf{F}(\mathbf{X})$, $\mathbf{Y} + \mathrm{FFN}(\mathbf{Y})$, построчная дисперсия удовлетворяет условию $\min_i \sigma_i^2 > 0$.
\end{assumption}

Предположение~\ref{assumption:LayerNorm} является техническим и требуется для доказательства ряда теорем. Выполнения данного свойства можно добиться, добавив к знаменателю положительную константу, но это усложнит вычисления.

Для оценки матрицы Гессе рассматривается среднеквадратичная функция ошибки:
\[
    l(\cdot, \textbf{Target}) = \frac{1}{L d_V} \|\cdot - \textbf{Target}\|_F^2.
\]
В дальнейшем для доказательств будет использоваться разложения Гаусса-Ньютона матрицы Гессе $\mathcal{L}_k \circ f_{\mathbf{w}}$:
\begin{equation}\label{eq:gauss_decomposition}
    \frac{\partial^2 (\mathcal{L}_k \circ f_{\mathbf{w}})}{\partial \mathbf{W}_i \partial \mathbf{W}_j} = \frac{\partial f_{\mathbf{w}}}{\partial \mathbf{W}_i} (\cdot)^\top \frac{\partial^2 \mathcal{L}_k}{\partial f_{\mathbf{w}}^2} (f_{\mathbf{w}}(\cdot)) \frac{\partial f_{\mathbf{w}}}{\partial \mathbf{W}_j}(\cdot) + \left( \frac{\partial \mathcal{L}_k}{\partial f_{\mathbf{w}}} (f_{\mathbf{w}}(\cdot)) \otimes \mathbf{I}_{p_i q_i} \right) \frac{\partial^2 f_{\mathbf{w}}}{\partial \mathbf{W}_i \partial \mathbf{W}_j}(\cdot)
\end{equation}

Для начала вычислим обобщённые выражения матрицы Гессе для слоя самовнимания и расширяем их до полного блока модели трансформер. Подход основан на теоретической базе \cite{ormaniec2024attentionhessian}, адаптируя и обобщая её результаты.

Матрица Гессе функции ошибки $\mathcal{L}_k$ относительно параметров модели $\mathbf{w}$:
\[
    \mathbf{H}^{(k)}(\mathbf{w}) = \nabla^2_{\mathbf{w}} \mathcal{L}_k(\mathbf{w}) = \frac{1}{k} \sum_{i=1}^k \nabla^2_{\mathbf{w}} l_i(\mathbf{w}) = \frac{1}{k} \sum_{i=1}^k \mathbf{H}_i(\mathbf{w})
\]
где $\mathbf{H}_k(\mathbf{w})$ является матрицей Гессе блока самовнимания для параметров $\mathbf{w}$ относящиеся к матрицам $\{ \mathbf{W}_Q, \mathbf{W}_K, \mathbf{W}_V\}$. Используя разложения Гаусса-Ньютона \eqref{eq:gauss_decomposition}:
\[
    \mathbf{H}_k(\mathbf{W}_i, \mathbf{W_j}) = \frac{\partial^2 l}{\partial \mathbf{W}_i \partial \mathbf{W}_j} = \mathbf{H}_o(\mathbf{W}_i, \mathbf{W}_j) + \mathbf{H}_f(\mathbf{W}_i, \mathbf{W}_j),
\]
где $\mathbf{H}_o$ является outer-product частью матрицы Гессе, а $\mathbf{H}_f$ является матрицей Гессе функции самовнимания.
Результаты этого разложения можно вычислить согласно теоремам 3.1-3.2 в работе~\cite{ormaniec2024attentionhessian}.

\subsection{Матрица Гессе для слоя самовнимания}
Проведем оценку нормы матрицы Гессе для одного слоя самовнимания. Результат данной оценки показан в теореме~\ref{thm:self_attention_hessian_estimation}.

\begin{theorem}\label{thm:self_attention_hessian_estimation}

Пусть $\|\cdot\|_2$ является спектральной нормой, тогда для слоя самовнимания получаем:
\[
    \|\mathbf{H}_i(\mathbf{w}^*)\|_2 \leq M,
\]
где
\begin{align}
M &= 3\cdot\max \Bigg(\frac{2L}{d_V} \| \mathbf{X}\|^2_2, \\
    &\frac{8}{L^3 d_V d_K} \| \mathbf{W}_K\|_2^2 \| \mathbf{W}_V\|^2_2 \| \mathbf{X}\|^6_2 + \\
    &\quad+\frac{12}{d_V d_K} \sqrt{\min(L, d_V)} (L \|\mathbf{X}\|_2 \|\mathbf{W}_V \|_2 + \|\textbf{Target}\|_2) \| \mathbf{W}_V \|_2 \| \mathbf{W}_K\|^2_2 \| \mathbf{X}\|^5_2, \\
    &\frac{4}{L d_V \sqrt{d_K}} \| \mathbf{W}_V\|_2 \| \mathbf{W}_K \|_2 \| \mathbf{X}\|^4_2 +\\
    &\quad+\frac{4\sqrt{\min(L, d_V)}}{L^2\sqrt{d_K}} (L \|\mathbf{X}\|_2 \|\mathbf{W}_V \|_2 + \|\textbf{Target}\|_2) \|\mathbf{W}_K\|_2 \|\mathbf{X}\|^3_2,\\
    &\frac{8}{L^3 d_V d_K} \|\mathbf{W}_K\|_2 \|\mathbf{W}_Q\|_2 \| \mathbf{W}_V\|^2_2 \|\mathbf{X} \|^6_2 + \\
    &\quad+\frac{4\sqrt{\min(L, d_V)} (L \|\mathbf{X}\|_2 \|\mathbf{W}_V \|_2 + \|\textbf{Target}\|_2)}{L d_V \sqrt{d_K}} \|\mathbf{W}_V\|_2 \cdot\\
    &\quad\quad\cdot\Big(3L \|\mathbf{W}_K\|_2 \|\mathbf{W}_Q\|_2 \| \mathbf{X}\|^5_2 + \frac{d_V}{L} \|\mathbf{X}\|^3_2\Big)\Bigg).
\end{align}
\end{theorem}
\begin{proof}

Используя результаты Леммы A.3 из работы \cite{noci2022signalpropagationtransformerstheoretical}, а также свойство~\ref{prop:matrix_product_norm} и свойство~\ref{prop:kronecker_product_norm} получаем:
\begin{align}
    \| \frac{\partial\mathbf{A}}{\partial\mathbf{T}}\|_2 = \frac{1}{L} \| \mathbf{I}_L\|_2 \| \mathbf{I}_L - \frac{1}{L}\mathbf{1}_{L \times L}\|_2 \leq \frac{1}{L}
\end{align}
Данное неравенство верно в силу того, что $\frac{1}{L}\mathbf{1}_{L \times L}$ является матрицей проекции, поэтому $\mathbf{I}_L - \frac{1}{L}\mathbf{1}_{L \times L}$ также матрица проекции и следовательно норма $\| \mathbf{I}_L - \frac{1}{L}\mathbf{1}_{L \times L}\|_2 \leq 1$.

Далее для аппроксимации нормы матрицы $\mathbf{Z}_1$ используем те же свойства~\ref{prop:matrix_product_norm} и \ref{prop:kronecker_product_norm}:
\begin{align}\label{chapter-2:theorem:proof_Znorm}
    \|\mathbf{Z}_1\|_2 &\leq \| \mathbf{I}_L \otimes \mathbf{X}^{\top}\|_2 \left\| \frac{\partial \mathbf{A}}{\partial \mathbf{T}}\right\|_2 \| \mathbf{X} \otimes \mathbf{X}\|_2 \leq\\
    & \leq \| \mathbf{X}\|_2 \frac{1}{L} \|\mathbf{X}\|^2_2 = \frac{1}{L} \|\mathbf{X}\|^3_2
\end{align}
где дополнительно было использовано свойство \ref{prop:transposed_matrix_norm} for $\| \mathbf{X}\|_2 = \| \mathbf{X}^\top\|_2$.

Оценим норму матрицы $\| \mathbf{A}\|_2,$ которая является матрицей, где каждая строка является результатом применения функции~$\text{softmax},$ а следовательно, каждый элемент матрицы~$\mathbf{A}_{i,j} \leq 1$.
Далее используя свойства \ref{prop:matrix_norm_inequalities} получаем $ \|\mathbf{A}\|_{\max} \leq \|\mathbf{A}\|_2 \leq \sqrt{LL} \|\mathbf{A}\|_{\max} = L\| \mathbf{A}\|_{max} \leq L$.
Также получаем, что:
\[
    \|\mathbf{M}_1\|_2 = \|\mathbf{A}\mathbf{X}\|_2 \leq L \|\mathbf{X}\|_2.
\]

Итого, легко получаем outer-product матрицы Гессе $\|\mathbf{H}_o (\mathbf{W}_i, \mathbf{W}_j) \|_2$  для разных матриц.
В случае матрицы~$\mathbf{W}_V$ и матрицы~$\mathbf{W}_V$:
\begin{align}
    \| \mathbf{H}_o(\mathbf{W}_V, \mathbf{W}_V)\|_2 &\leq \frac{2}{L d_V} \| \mathbf{M}_1\|^2_2 1 \leq \frac{2}{L d_V}\| \mathbf{A}\|^2_2 \| \mathbf{X}\|^2_2 \leq\\
    & \leq \frac{2}{L d_V} L^2 \| \mathbf{X}\|^2_2 = \frac{2L}{d_V}\| \mathbf{X}\|^2_2.
\end{align}
Для матриц~$\mathbf{W}_Q$ и~$\mathbf{W}_Q$ получаем:
\begin{align}
    \| \mathbf{H}_o(\mathbf{W}_Q, \mathbf{W}_Q)\|_2 &\leq \| \frac{2}{L d_V d_K} (\mathbf{I}_{d_V} \otimes \mathbf{W}^\top_K) \mathbf{Z}^\top_1 (\mathbf{I}_{L} \otimes \mathbf{W}_V\mathbf{W}^\top_V)\ \mathbf{Z}_1 (\mathbf{I}_{d_V} \otimes \mathbf{W}_K)\|_2 \leq\\
    &\leq \frac{2}{L d_V d_K} \| \mathbf{W}_K\|_2^2 \|\mathbf{Z}_1 \|^2_2 \| \mathbf{W}_V\|^2_2 \leq\\
    &\leq \frac{2}{L d_V d_K}\| \mathbf{W}_K\|_2^2\| \mathbf{W}_V\|^2_2 \frac{1}{L^2} \| \mathbf{X}\|^6_2 =\\
    &= \frac{2}{L^3 d_V d_K} \| \mathbf{W}_K\|_2^2\| \mathbf{W}_V\|^2_2 \mathbf{X}\|^6_2.
\end{align}
Между матрицей~$\mathbf{W}_V$ и матрицей~$\mathbf{W}_Q$:
\begin{align}
    \| \mathbf{H}_o(\mathbf{W}_V, \mathbf{W}_Q)\|_2 &\leq \frac{2}{L d_V \sqrt{d_K}} \|\mathbf{M}_1^\top \otimes \mathbf{W}_V^\top \|_2 \| \mathbf{Z}_1\|_2 \| \mathbf{I}_{d_V} \otimes \mathbf{W}_K \|_2 \leq\\
    &\leq \frac{2}{L d_V \sqrt{d_K}} L \| \mathbf{X}\|_2 \| \mathbf{W}_V\|_2 \frac{1}{L} \| \mathbf{X}\|^3_2 \| \mathbf{W}_K \|_2 =\\
    &= \frac{2}{L d_V \sqrt{d_K}} \| \mathbf{W}_V\|_2 \| \mathbf{W}_K \|_2 \| \mathbf{X}\|^4_2
\end{align}
Между матрицей~$\mathbf{W}_Q$ и матрицей~$\mathbf{W}_K$:
\begin{align}
    &\left\|\mathbf{H}_o(\mathbf{W}_Q, \mathbf{W}_K)\right\|_2 \leq \\
    &\leq \frac{2}{L d_V d_K} \big\|(\mathbf{I}_{d_V} \otimes \mathbf{W}^\top_K) \mathbf{Z}_1^\top(\mathbf{I}_L \otimes \mathbf{W}_V \mathbf{W}^\top_V)\mathbf{Z}_1 (\mathbf{W}_Q \otimes \mathbf{I}_{d_V}) \mathbf{K}{d_K, d_V}\big\|_2 \leq \\
    &\leq \frac{2}{L^3 d_V d_K} \big\|\mathbf{W}_K\|_2 \|\mathbf{W}_Q\|_2 \| \mathbf{W}_V\|_2^2 \|\mathbf{X} \big\|^6_2.
\end{align}
Для всех оценок использовались свойства~\ref{prop:matrix_product_norm}, \ref{prop:kronecker_product_norm}, а также свойства $\|\mathbf{K}_{d_V d_K} \|_2=1$, потому что $\mathbf{K}_{m,n}$ является коммутативной матрицей, описанной в определении~\ref{def:commutation_matrix}. 

Далее проведем анализ матрицы~$\mathbf{H}_f$.
Для этого начнем анализ с матрицы~$\mathbf{R}_m = \mathrm{vec}_r (\mathbf{F} (\mathbf{X}) - \textbf{Target})^T \otimes \mathbf{I}_m,$ который описан в рамках теоремы 3.2 в работе~\cite{ormaniec2024attentionhessian}.
Так, как~$\mathrm{vec}_r(\cdot)$ является функцией векторизации: 
\begin{align}
    \|\mathrm{vec}_r(\mathbf{F}(\mathbf{X}) - \textbf{Target})\|_2 &= \| \mathbf{F}(\mathbf{X}) - \textbf{Target} \|_F \leq\\
    &\leq\sqrt{\mathrm{rank}(\mathbf{F}(\mathbf{X}) - \textbf{Target} )} \| \mathbf{F}(\mathbf{X}) - \textbf{Target} \|_2,
\end{align}
тогда согласно свойству \ref{prop:matrix_norm_inequalities} получаем: 
\begin{align}
    \| \mathbf{R}_m\| &\leq \sqrt{\mathrm{rank}(\mathbf{F}(\mathbf{X}) - \textbf{Target})} \|\mathbf{F}(\mathbf{X}) - \textbf{Target}\|_2 \leq\\
    &\leq\sqrt{\mathrm{rank}(\mathbf{F}(\mathbf{X}) - \textbf{Target})} (\| \mathbf{A} \|_2 \|\mathbf{X}\|_2 \|\mathbf{W}_V \|_2 + \|\textbf{Target}\|_2) \leq\\
    &\leq \sqrt{\mathrm{rank}(\mathbf{F}(\mathbf{X}) - \textbf{Target})} (L \|\mathbf{X}\|_2 \|\mathbf{W}_V \|_2 + \|\textbf{Target}\|_2)
\end{align}
где для получения оценок были использованы свойства матриц \ref{prop:matrix_product_norm} и \ref{prop:matrix_sum_norm}. В свою очередь норма матрицы перемешивания (англ. shuffling matrix) оценивается следующим образом:
\begin{align}
    \| \mathbf{S}\|_2 &= \|(\mathbf{I}_{d_V} \otimes \mathbf{K}_{d_V,d_V}) (\mathrm{vec}_r (\mathbf{I}_{d_V}) \otimes \mathbf{I}_{d_V})\|_2 \leq \\
    &\leq \| \mathrm{vec}_r (\mathbf{I}_{d_V}) \|_2 = \| \mathbf{I}_{d_V}\|_{F} = \sqrt{d_V}.
\end{align}

Для верхней оценки нормы матрицы $\| \frac{\partial^2 \mathbf{A}}{\partial \mathbf{T}^2} \|_2$ воспользуемся леммой C1 с работы \cite{ormaniec2024attentionhessian}, где указано, что:
\begin{equation}\label{chapter-2:theorem:proof_partialA}
    \frac{\partial^2 \mathbf{A}_{i,j}}{\partial \mathbf{T}_{i,:} \partial \mathbf{T}_{i,:}} = \mathbf{A}_{i,j} \left(2\mathbf{A}_{i,:}\mathbf{A}_{i,:}^{\top} + \mathbf{E}_{j,j}^{L,L} - \text{diag}(\mathbf{A}_{i,:}) - \mathbf{e}_j \mathbf{A}_{i,:}^{\top} - \mathbf{A}_{i,:}\mathbf{e}_j^{\top}\right) \in \mathbb{R}^{L \times L},
\end{equation}
где 
\[
    \mathbf{E}_{j,j}^{L, L} = \mathbf{e}_j \mathbf{e}_j^{\top} \in \mathbb{R}^{L \times L},
\]
поэтому он содержит только один ненулевой элемент, который равен 1 в позиции $(j, j)$.
Кроме того, вторая производная softmax имеет блочно-диагональную структуру, а следовательно используя свойство~\ref{prop:block_matrix_norm} нормы блочно диагональной матрицы получаем:
\[
    \left\|\frac{\partial^2 \mathbf{A}}{\partial \mathbf{T}^2}\right\|_2 = \max_{i,j} \left\|\frac{\partial^2 \mathbf{A}_{i,j}}{\partial \mathbf{T}_{i,:} \partial \mathbf{T}_{i,:}}\right\|_2.
\]
Приходим к тому, что требуется оценить следующую норму:
\[
    \left\|\frac{\partial^2 \mathbf{A}_{i,j}}{\partial \mathbf{T}_{i,:} \partial \mathbf{T}_{i,:}} \right\|_2.
\]

Как было указано ранее $\mathbf{A}_{i,j} \leq 1,$ а следовательно можем оценить матрицу $\| \mathbf{A}_{i,:}\mathbf{A}_{i,:}^{\top} \|_2$, так как $\mathbf{A}_{i,:}$ является строкой $\text{softmax}$-матрицы, то значение суммы строки равняются $1$.
Таким образом, мы можем использовать векторно-матричные неравенства для получения выражение:
\begin{equation}\label{chapter-2:theorem:proof_Arow}
    \| \mathbf{A}_{i,:}\mathbf{A}_{i,:}^{\top} \|_2 \leq \|\mathbf{A}_{i,:}\|^2_2 \leq \|\mathbf{A}_{i,:}\|_1^2 = 1.
\end{equation}

Аналогично заметим, что 
\begin{equation}\label{chapter-2:theorem:proof_Enorm}
    \|\mathbf{E}_{j,j}^{m,n}\|_2 = \| \mathbf{e}_j \mathbf{e}_j^{\top}\|_2 \leq 1.
\end{equation}

Перейдем к оценке нормы диагональной матрицы $\|diag(\mathbf{A}_{i,:})\|_2$.
Заметим, что для диагональной матрицы верно следующее выражение:
\begin{equation}\label{chapter-2:theorem:proof_Adiag}
    \|diag(\mathbf{A}_{i,:})\|_2 = \max \limits_j \mathbf{A}_{i,j} \leq 1.
\end{equation}

Используя оценки~\eqref{chapter-2:theorem:proof_Arow} и~\eqref{chapter-2:theorem:proof_Adiag} и~\eqref{chapter-2:theorem:proof_Enorm} оценим нормы $\mathbf{e}_j \mathbf{A}_{i,:}^{\top}$ и $\mathbf{A}_{i,:}\mathbf{e}_j^{\top}.$
Матрицы $\mathbf{e}_j \mathbf{A}_{i,:}^{\top}$ and $\mathbf{A}_{i,:}\mathbf{e}_j^{\top}$ являются матрицами ранга~$1,$ причем только с одной не нулевой строкой и колонкой соотвественно с элементами матрицы~$\mathbf{A}_{i,:}.$
Следовательно их спектральные нормы оценивается сверху нормой матрицы~$\|\mathbf{A}_{i,:}\|_2 \leq 1$.

Получаем, что все слагаемые в выражении~\eqref{chapter-2:theorem:proof_partialA} имеют верхнюю оценку $1,$ а следовательно:
\begin{align}
    \left\| \frac{\partial^2 \mathbf{A}}{\partial \mathbf{T}^2} \right\|_2 \leq 6 
\end{align}

Возвращаясь к выражению~\eqref{chapter-2:theorem:proof_Znorm} получаем оценку матрицы $\| \mathbf{Z}_2 \|_2$:
\begin{align}
    \| \mathbf{Z}_2 \|_2 &= \| \left(\mathbf{I}_L \otimes \mathbf{X}^\top \otimes \mathbf{X}^\top \otimes \mathbf{X}^\top\right) \left(\partial^2\mathbf{A}/\partial\mathbf{T}^2\right) \left(\mathbf{X} \otimes \mathbf{X}\right) \|_2 \leq \\
    &\leq \| \mathbf{X}\|^5_2 \left\| \frac{\partial^2\mathbf{A}}{\partial\mathbf{T}^2} \right\|_2 \leq 6 \| \mathbf{X}\|^5_2
\end{align}
Оцениваем часть~$\mathbf{H}_\mathrm{f}.$ Для нормы между матрицами~$\mathbf{W}_V$ и~$\mathbf{W}_V$:
\begin{align}
    \|\mathbf{H}_\mathrm{f}(\mathbf{W}_V, \mathbf{W}_V)\|_2 = 0
\end{align}
Для нормы между матрицами~$\mathbf{W}_Q$ и~$\mathbf{W}_Q$:
\begin{align}
    \|\mathbf{H}_\mathrm{f}(\mathbf{W}_Q, \mathbf{W}_Q)\|_2 &= \frac{2}{Ld_V d_K} \|\mathbf{R}_{d_V d_K} \left(\mathbf{I}_L \otimes \mathbf{W}_V^\top \otimes \mathbf{I}_{d_V} \otimes \mathbf{W}_K^\top\right) \mathbf{Z}_2 \left(\mathbf{I}_{d_V} \otimes \mathbf{W}_K\right)\|_2, \\
    &\leq \frac{2}{Ld_V d_K} \| \mathbf{R}_{d_V d_K} \|_2 \| \mathbf{W}_V \|_2 \| \mathbf{W}_K\|_2 \|\mathbf{Z}_2 \|_2 \| \mathbf{W}_K\|_2 \\
    &\leq 6\frac{2}{Ld_V d_K}\sqrt{\mathrm{rank}(\mathbf{F}(\mathbf{X}) - \textbf{Target})} \Big(L \|\mathbf{X}\|_2 \|\mathbf{W}_V \|_2 +\\
    &\quad +\|\textbf{Target}\|_2\Big)\| \mathbf{W}_V \|_2 \| \mathbf{W}_K\|^2_2\| \mathbf{X}\|^5_2 =\\
    &= \frac{12}{d_V d_K}\sqrt{\mathrm{rank}(\mathbf{F}(\mathbf{X}) - \textbf{Target})} \Big(L \|\mathbf{X}\|_2 \|\mathbf{W}_V \|_2 + \\
    &\quad+\|\textbf{Target}\|_2\Big)\| \mathbf{W}_V \|_2 \| \mathbf{W}_K\|^2_2\| \mathbf{X}\|^5_2
\end{align}
Для нормы между матрицами~$\mathbf{W}_V$ и~$\mathbf{W}_Q$:
\begin{align}
    \|\mathbf{H}_\mathrm{f}(\mathbf{W}_V, \mathbf{W}_Q)\|_2 &= \frac{2}{Ld_V\sqrt{d_K}} \|\mathbf{R}_{d_V^2} \left(\mathbf{I}_L \otimes \mathbf{S}\right) \mathbf{Z}_1 \left(\mathbf{I}_{d_V} \otimes \mathbf{W}_K\right)\|_2 \leq \\
    &\leq \frac{2}{Ld_V\sqrt{d_K}} \| \mathbf{R}_{d_V^2}\|_2 \| \mathbf{S} \|_2 \|\mathbf{Z}_1 \|_2 \| \mathbf{W}_K\|_2 \leq\\
    &\leq \frac{2}{Ld_V\sqrt{d_K}} \sqrt{\mathrm{rank}(\mathbf{F}(\mathbf{X}) - \textbf{Target})} \Big(L \|\mathbf{X}\|_2 \|\mathbf{W}_V \|_2 +\\
    &\quad +\|\textbf{Target}\|_2\Big) \sqrt{d_V} \frac{1}{L} \|\mathbf{X}\|^3_2\|\mathbf{W}_K\|_2 = \\
    &= \frac{2\sqrt{\mathrm{rank}(\mathbf{F}(\mathbf{X}) - \textbf{Target})}}{L^2\sqrt{d_Vd_K}}\Big(L \|\mathbf{X}\|_2 \|\mathbf{W}_V \|_2 + \\
    &\quad+\|\textbf{Target}\|_2\Big)\|\mathbf{W}_K\|_2 \|\mathbf{X}\|^3_2
\end{align}
Для нормы между матрицами~$\mathbf{W}_Q$ и~$\mathbf{W}_K$:
\begin{align}
    &\|\mathbf{H}_\mathrm{f}(\mathbf{W}_Q, \mathbf{W}_K)\| \leq \\
    &\leq\frac{2}{Ld_V d_K}\| \mathbf{R}_{d_V d_K} \left(\mathbf{I}_L \otimes \mathbf{W}_V^\top \otimes \mathbf{I}_{d_V} \otimes \mathbf{W}_K^\top\right) \mathbf{Z}_2 \left(\mathbf{W}_Q \otimes \mathbf{I}_{d_V}\right) \mathbf{K}_{d_K, d_V}\|_2 + \\
    &\quad + \frac{2}{Ld_V\sqrt{d_K}} \|\mathbf{R}_{d_V} \left(\mathbf{I}_L \otimes \mathbf{W}_V^\top \otimes \mathbf{I}_{d_V}\right) \left(\mathbf{Z}_1 \otimes \mathbf{I}_{d_V}\right) \mathbf{S} \otimes \mathbf{I}_{d_K}\|_2 \leq\\
    &\leq  \frac{2}{Ld_V d_K}\sqrt{\mathrm{rank}(\mathbf{F}(\mathbf{X}) - \textbf{Target})} \Big(L \|\mathbf{X}\|_2 \|\mathbf{W}_V \|_2 +\\
    &\quad + \|\textbf{Target}\|_2\Big) \|\mathbf{W}_V\|_2 \|\mathbf{W}_K\|_2 \|\mathbf{W}_Q\|_26 \| \mathbf{X}\|^5_2 + \\
    &\quad+\frac{2}{Ld_V\sqrt{d_K}}\sqrt{\mathrm{rank}(\mathbf{F}(\mathbf{X}) - \textbf{Target})} \Big(L \|\mathbf{X}\|_2 \|\mathbf{W}_V \|_2 + \\
    &\quad +\|\textbf{Target}\|_2\Big) \|\mathbf{W}_V \|_2 \frac{1}{L} \|\mathbf{X}\|^3_2 \sqrt{d_V} =\\
    &=\frac{2\sqrt{\mathrm{rank}(\mathbf{F}(\mathbf{X}) - \textbf{Target})} (L \|\mathbf{X}\|_2 \|\mathbf{W}_V \|_2 + \|\textbf{Target}\|_2)}{Ld_V\sqrt{d_Vd_K}} \|\mathbf{W}_V\|_2 \cdot \\
    & \quad \cdot \Big(3L\|\mathbf{W}_K\|_2 \|\mathbf{W}_Q\|_2 \| \mathbf{X}\|^5_2 + \frac{d_V}{L} \|\mathbf{X}\|^3_2\Big).
\end{align}

Собирая все оценки вместе, используя матричное свойство \ref{prop:matrix_norm_inequalities} для всех блоков $\{K, Q, V\}$:
\begin{align}
    &\| \mathbf{H} (\mathbf{W}_i, \mathbf{W}_j)\|_2 \leq 3\max \limits_{i,j \in \{Q, K, V\}} \Big(\|\mathbf{H}_o(\mathbf{W}_i, \mathbf{W}_j)\|_2 + \|\mathbf{H}_f(\mathbf{W}_i, \mathbf{W}_j)\|_2\Big)
\end{align}
Подставляя оценки получаем следующую оценку на матрицу Гессе:
\begin{align}
    &\| \mathbf{H} (\mathbf{W}_i, \mathbf{W}_j)\|_2 \leq\\
    & \leq 3 \max \Bigg(\frac{2L}{d_V} \| \mathbf{X}\|^2_2, \\
    &\frac{2}{L^3 d_V d_K} \| \mathbf{W}_K\|_2^2 \| \mathbf{W}_V\|^2_2 \| \mathbf{X}\|^6_2 +\\
    &\quad+ \frac{12}{d_V d_K} \sqrt{\mathrm{rank}(\mathbf{F}(\mathbf{X}) - \textbf{Target})} \Big(L \|\mathbf{X}\|_2 \|\mathbf{W}_V \|_2 +\\
    &\qquad+\|\textbf{Target}\|_2\Big) \| \mathbf{W}_V \|_2 \| \mathbf{W}_K\|^2_2 \| \mathbf{X}\|^5_2, \\
    &\frac{2}{L d_V \sqrt{d_K}} \| \mathbf{W}_V\|_2 \| \mathbf{W}_K \|_2 \| \mathbf{X}\|^4_2 +\\
    &\quad+\frac{2\sqrt{\mathrm{rank}(\mathbf{F}(\mathbf{X}) - \textbf{Target})}}{L^2\sqrt{d_V d_K}} (L \|\mathbf{X}\|_2 \|\mathbf{W}_V \|_2 + \|\textbf{Target}\|_2) \|\mathbf{W}_K\|_2 \|\mathbf{X}\|^3_2,\\
    &\frac{2}{L^3 d_V d_K} \|\mathbf{W}_K\|_2 \|\mathbf{W}_Q\|_2 \| \mathbf{W}_V\|^2_2 \|\mathbf{X} \|^6_2 + \\
    &\quad+\frac{2\sqrt{\mathrm{rank}(\mathbf{F}(\mathbf{X}) - \textbf{Target})} \Big(L \|\mathbf{X}\|_2 \|\mathbf{W}_V \|_2 + \|\textbf{Target}\|_2\Big)}{L d_V \sqrt{d_V d_K}} \cdot\\
    &\qquad\cdot\|\mathbf{W}_V\|_2 \Big(3L \|\mathbf{W}_K\|_2 \|\mathbf{W}_Q\|_2 \| \mathbf{X}\|^5_2 + \frac{d_V}{L} \|\mathbf{X}\|^3_2\Big)\Bigg).
\end{align}

Полученное выражение почти полностью соответствует выражению~$M,$ где для полного соотвествия требуется воспользоваться неравенством~$\mathrm{rank}(\mathbf{F}(\mathbf{X}) - \textbf{Target}) \le \min(L, d_V)$. 
\end{proof}

Теорема~\ref{thm:self_attention_hessian_estimation} оценивает только один слой самовнивания. Теперь перейдем к оценке полного блока трансформера. Полный трансформер слой содержит слой самовнимания, блок полносвязной сети (англ. FFN), и слоя нормализации выходов (англ. LayerNorm). Весь блок описывается следующими выражениями:
\begin{align}\label{eq:transformer}
    \mathbf{Y} &= \text{LayerNorm}(\mathbf{X} + \mathbf{F}(\mathbf{X})) \\ 
    \mathbf{Z} &= \text{LayerNorm}(\mathbf{Y} + \text{FFN}(\mathbf{Y})), 
\end{align} 
где
\[
    \text{FFN}(\mathbf{Y}) = \sigma(\mathbf{Y} \mathbf{W}_1 + \mathbf{b}_1) \mathbf{W}_2 + \mathbf{b}_2,
\]
с матрицами параметров $\mathbf{W}_1 \in \mathbb{R}^{d_V \times d_{\text{ff}}} $, матрицами$\mathbf{W}_2 \in \mathbb{R}^{d_{\text{ff}} \times d_V}$, векторами~$b_1 \in \mathbb{R}^{d_{\text{ff}}}$, $b_2 \in \mathbb{R}^{d_V}$, а также функцией активации~$\sigma$.
Функция $\text{LayerNorm}(\mathbf{X})$ определяется для входной матрицы~$\mathbf{X} \in \mathbb{R}^{L \times d_V}$ следующим образом:
\begin{align}
    \text{LayerNorm}(\mathbf{X})_{i,j} = \mathbf{\gamma}_j \cdot \frac{\mathbf{X}_{i,j} - \mu_i}{\sqrt{\sigma_i^2}} + \mathbf{\beta}_j,
\end{align}
где параметры~$\mu_i, \sigma_i$ определяются следующим образом:
\[
    \mu_i = \frac{1}{d_V} \sum_{j=1}^{d_V} \mathbf{X}_{i,j}, \quad \sigma_i^2 = \frac{1}{d_V} \sum_{j=1}^{d_V} (\mathbf{X}_{i,j} - \mu_i)^2,
\]
а параметры~$\gamma_j, \beta_j$ являются настраиваемым в процессе оптимизации.

Итого, получаем полный список параметров полного слоя трансформера:
\[
    \mathbf{w} = \{\mathbf{W}_Q, \mathbf{W}_K, \mathbf{W}_V, \mathbf{W}_1, \mathbf{W}_2, \mathbf{b}_1, \mathbf{b}_2, \mathbf{\gamma}, \mathbf{\beta}\}
\], где $\mathbf{\gamma},\mathbf{\beta}$ являются параметрами LaterNorm, для простоты вычисления в некоторых случаях введем предположения, что параметры~$\mathbf{\gamma},\mathbf{\beta}$ являются постоянными и не меняются в процессе оптимизации.

\subsection{Матрица Гессе для LayerNorm слоя}

Для начала вычислим для функции~$\mathrm{LayerNorm}$ матрицу Якоби относительно параметров модели, для этого докажем теорему~\ref{thm:layernorm_derivative}.
\begin{theorem}\label{thm:layernorm_derivative}
    Пусть задана матрица $\mathbf{X} \in \mathbb{R}^{L \times d_V}$. Определим функцию~$\mathbf{M}(\mathbf{X})$ следующим образом:
    \begin{align}
        \mathbf{M}(\mathbf{X}) &= \mathbf{X} - \tfrac{1}{d_V}\mathbf{X}\mathbf{1}_{d_V}\mathbf{1}_{d_V}^\top, \\
        \sigma(\mathbf{X}) &= \tfrac{1}{\sqrt{d_V}}\big(\mathbf{M}(\mathbf{X})^{\circ 2}\mathbf{1}_{d_V}\big)^{\circ 1/2},\\
        \mathbf{P}(\mathbf{X}) &= \mathrm{diag}^{-1}(\sigma(\mathbf{X})).
    \end{align}
    Тогда для функции LayerNorm :
    \[
        \text{LayerNorm}(\mathbf{X}) = \mathbf{P}(\mathbf{X}) \mathbf{M}(\mathbf{X}),
    \]
    матрица Якоби относительно переменной~$\mathbf{X}$ определяется следующим образом:
    \begin{align}
        \frac{\partial \text{LayerNorm}(\mathbf{X})}{\partial \mathbf{X}} &= (\mathbf{P}(\mathbf{X}) \otimes \mathbf{I}_{d_V}) \left(\mathbf{I}_{Ld_V} - \tfrac{1}{d_V}(\mathbf{I}_L \otimes \mathbf{1}_{d_V \times d_V})\right)+\\
    &\quad+\left(\mathbf{I}_L \otimes \mathbf{M}(\mathbf{X})^\top\right)\frac{\partial\mathbf{P}(\mathbf{X})}{\partial\mathbf{X}},
    \end{align}
    где
    \begin{align}
    \frac{\partial \mathbf{P}}{\partial \mathbf{X}}&=\frac{1}{\sqrt{d_V}}\Big( -\mathbf{D}^{-1} \otimes \mathbf{D}^{-\top} \Big)\cdot\\
    &\quad\cdot\big( \mathbf{e}_1 \otimes \mathbf{e}_1, \dots, \mathbf{e}_L \otimes \mathbf{e}_L \big)\cdot\\
    &\quad\cdot\Big(\mathrm{diag}^{-1}\!\big(\mathrm{vec}_r^{1/2}(\mathbf{M}^{\circ 2}\mathbf{1}_{d_V})\big)(\mathbf{I}_L \otimes \mathbf{1}_{d_V}^\top)\mathrm{diag}\left(\mathrm{vec}_r(\mathbf{M})\right)\frac{\partial \mathbf{M}}{\partial \mathbf{X}}\Big),
    \end{align}
    где $\mathbf{D} = \mathrm{diag}(\sigma(\mathbf{X})).$
\end{theorem}
\begin{proof}
Представим функцию LayerNorm в матричном виде:
\begin{equation}
    \text{LayerNorm}(\mathbf{X}) = \mathbf{P}(\mathbf{X})\mathbf{M}(\mathbf{X}),
\end{equation}
где матрица~$\mathbf{P}(\mathbf{X}) = \mathbf{D}^{-1}$, а матрица~$\mathbf{D} = \textit{diag}(\sigma(\mathbf{X})),$ в свою очередь согласно свойству~\ref{prop:elem_wise_division} матрица~$\mathbf{M}(\mathbf{X}) = (\mathbf{X} - \mu(\mathbf{X})\mathbf{1}^\top_{d_V}).$

Используя лемму~\ref{lemma:matrix_funcs_product_derivative} получаем выражение для произведения матричнозначных функций:
\begin{equation}
    \frac{\partial\text{LayerNorm}(\mathbf{X})}{\partial \mathbf{X}} = ( \mathbf{P}(\mathbf{X}) \otimes \mathbf{I}_{d_V}) \frac{\partial \mathbf{M}}{\partial \mathbf{X}} + (\mathbf{I}_L\otimes \mathbf{M}^\top)\frac{\partial \mathbf{P}}{\partial \mathbf{X}}
\end{equation}

Вычислим значение производной~$\frac{\partial \mathbf{M}}{\partial \mathbf{X}},$ используя матричные вычисления~$\mathbf{M}(\mathbf{X}) = (\mathbf{X} - \mu(\mathbf{X}) \mathbf{1}^{\top}_{d_V}) = (\mathbf{X} - \frac{1}{d_V}\mathbf{X} \mathbf{1}_{d_V} \mathbf{1}^{\top}_{d_V}) = (\mathbf{X} - \frac{1}{d_V}\mathbf{X} \mathbf{1}_{d_V \times d_V})$. Получаем:
\begin{equation}
   \frac{\partial \mathbf{M}}{\partial \mathbf{X}} = \frac{\partial  (\mathbf{X} - \frac{1}{d_V}\mathbf{X} \mathbf{1}_{d_V \times d_V})}{\partial \mathbf{X}} = (\mathbf{I}_L \otimes \mathbf{I}_{d_V}) - \frac{1}{d_V}(\mathbf{I}_L \otimes \mathbf{1}_{d_V \times d_V})
\end{equation}

Далее вычислим значение производной~$\frac{\partial \mathbf{P}}{\partial \mathbf{X}}.$
Для начала получим выражение для нелинейного преобразования~$\sigma(\mathbf{X}).$
Данное выражение в матричном виде принимает вид:
\[
    \sigma(\mathbf{X}) =  \left(\frac{1}{d_V} (\mathbf{X} - \mu(X)\mathbf{1}_{d_V}^\top)^{\circ 2} \mathbf{1}_{d_V}\right)^{\circ \frac{1}{2}} = \frac{1}{\sqrt{d_V}} \left(\mathbf{M}(\mathbf{X})^{\circ 2} \mathbf{1}_{d_V}\right)^{\circ{\frac{1}{2}}},
\]
где~$\circ \alpha$ операция поэлементного взятия степени $\alpha$ описанного в определении~\ref{def:vec_elem_ops}.
Далее, применив цепное правило получаем:
\begin{align}
    \frac{\partial \mathbf{P}}{\partial \mathbf{X}} &= \frac{\partial \mathbf{D}^{-1}}{\partial \mathbf{D}} \frac{\partial\textit{diag}(\sigma(\mathbf{X}))}{\partial \sigma(\mathbf{X})} \frac{\partial \sigma(\mathbf{X})}{\partial \mathbf{X}},
\end{align}
причем используя свойства~\ref{lemma:hadamard_square_derivative}, \ref{lemma:hadamard_root_derivative} и~\ref{prop:matrix_product_derivative} получаем выражение для~$\frac{\partial \sigma(\mathbf{X})}{\partial \mathbf{X}}:$
\begin{equation}
    \frac{\partial \sigma(\mathbf{X})}{\partial \mathbf{X}} = \frac{1}{\sqrt{d_V}} \frac{\partial \tau^{\circ \frac{1}{2}}}{\partial \tau} \frac{\partial \tau}{\partial \mathbf{Q}} \frac{\partial \mathbf{Q}}{\partial {\mathbf{X}}},
\end{equation}
где~$\tau = \mathbf{Q}\cdot\mathbf{1}_L, \mathbf{Q} = \mathbf{M}^{\circ{2}},$ а следовательно подставляя получаем:
\begin{align}
    \frac{\partial \sigma(\mathbf{X})}{\partial \mathbf{X}} &= \frac{1}{\sqrt{d_V}} \frac{\partial \tau^{\circ \frac{1}{2}}}{\partial \tau} \frac{\partial \mathbf{Q}\cdot\mathbf{1}_{d_V}}{\partial \mathbf{Q}} \frac{\partial \mathbf{M}^{\circ{2}}}{\partial \mathbf{M}} \frac{\partial \mathbf{M}}{\partial \mathbf{X}} = \\
    \\ &= \frac{1}{\sqrt{d_V}} \frac{1}{2} \textit{diag}^{-1}(\mathrm{vec}_r^{\circ \frac{1}{2}}(\tau)) (\mathbf{I}_L \otimes \mathbf{1}^T_{d_V}) 2\cdot \textit{diag}(\mathrm{vec}_r (\mathbf{M}))\frac{\partial \mathbf{M}}{\partial \mathbf{X}} = \\
    &= \frac{1}{\sqrt{d_V}}\textit{diag}^{-1}(\mathrm{vec}_r^{\circ \frac{1}{2}}(\mathbf{M}^{\circ{2}}\cdot\mathbf{1}_{d_V}))\cdot (\mathbf{I}_L \otimes \mathbf{1}^T_{d_V})\cdot \textit{diag}(\mathrm{vec}_r (\mathbf{M})) \frac{\partial \mathbf{M}}{\partial \mathbf{X}}.
\end{align}

Используя леммы~\ref{lemma:invert_derivative} и \ref{lemma:diag_derivative} получаем:
\begin{align}
    \frac{\partial \mathbf{P}}{\partial \mathbf{X}} &= \frac{1}{\sqrt{d_V}}\left(-\mathbf{D}^{-1} \otimes \mathbf{D}^{-\top} \right) \Big(\mathbf{e}_1 \otimes \mathbf{e}_1 \quad \dots  \quad \mathbf{e}_L \otimes \mathbf{e}_L\Big) \cdot \\
    &\quad\cdot\left(\textit{diag}^{-1}(\mathrm{vec}_r^{\circ \frac{1}{2}}(\mathbf{M}^{\circ{2}}\cdot\mathbf{1}_{d_V}))\cdot (\mathbf{I}_L \otimes \mathbf{1}^T_{d_V})\cdot \textit{diag}(\mathrm{vec}_r (\mathbf{M}))\frac{\partial \mathbf{M}}{\partial \mathbf{X}}\right).
\end{align}

И того мы получили все составляющие для вычисления матрицы Якобы для LayerNorm оператора:
\begin{align}
    &\frac{\partial\text{LayerNorm}(\mathbf{X})}{\partial \mathbf{X}} = ( \mathbf{P}(\mathbf{X}) \otimes \mathbf{I}_{d_V}) \frac{\partial \mathbf{M}}{\partial \mathbf{X}} + (\mathbf{I}_L\otimes \mathbf{M}^\top)\frac{\partial \mathbf{P}}{\partial \mathbf{X}} = \\
    & = ( \mathbf{P}(\mathbf{X}) \otimes \mathbf{I}_{d_V}) \frac{\partial \mathbf{M}}{\partial \mathbf{X}} +\\
    &\quad+ (\mathbf{I}_L\otimes \mathbf{M}^\top)\frac{1}{\sqrt{d_V}}\left(-\mathbf{D}^{-1} \otimes \mathbf{D}^{-\top} \right) \Big(\mathbf{e}_1 \otimes \mathbf{e}_1 \quad \dots  \quad \mathbf{e}_L \otimes \mathbf{e}_L\Big) \cdot \\
    &\qquad\cdot \left(\textit{diag}^{-1}(\mathrm{vec}_r^{\circ \frac{1}{2}}(\mathbf{M}^{\circ{2}}\cdot\mathbf{1}_{d_V}))\cdot (\mathbf{I}_L \otimes \mathbf{1}^T_{d_V})\cdot \textit{diag}(\mathrm{vec}_r (\mathbf{M}))\frac{\partial \mathbf{M}}{\partial \mathbf{X}}\right),
\end{align}
где
\begin{align}
    \mathbf{M}(\mathbf{X}) &= (\mathbf{X} - \frac{1}{d_V}\mathbf{X} \mathbf{1}_{d_V \times d_V})\\
    \mathbf{P}(\mathbf{X}) &= \textit{diag}^{-1}(\sigma(\mathbf{X}))\\
    \frac{\partial \mathbf{M}}{\partial \mathbf{X}} &= (\mathbf{I}_L \otimes \mathbf{I}_{d_V}) - \frac{1}{d_V}(\mathbf{I}_L \otimes \mathbf{1}_{d_V \times d_V}.
\end{align}
\end{proof}

Теперь же вычислим для функции матрицу Гессе относительно параметров модели, для этого докажем теорему~\ref{thm:layernorm_second_derivative}.
\begin{theorem}\label{thm:layernorm_second_derivative}
    Пусть задан оператор~$LayerNorm$ в виде аналогичном теореме~\ref{thm:layernorm_derivative}:
    \[
        \text{LayerNorm}(\mathbf{X}) = \mathbf{P}(\mathbf{X}) \mathbf{M}(\mathbf{X}),
    \]
    с матрицей Якоби, полученную в теореме~\ref{thm:layernorm_derivative}:
    \[
        \frac{\partial \text{LayerNorm}}{\partial \mathbf{X}} = (\mathbf{P} \otimes \mathbf{I}_{d_V}) \mathbf{G} + (\mathbf{I}_L \otimes \mathbf{M}^\top) \mathbf{H},
    \]
    где дополнительно введены обозначения константы:
    \[
        \mathbf{G} = \left(\mathbf{I}_{Ld_V} - \tfrac{1}{d_V}(\mathbf{I}_L \otimes \mathbf{1}_{d_V \times d_V})\right),
    \] а также оператор, аналогичный оператору в теореме~\ref{thm:layernorm_derivative}:
    \[
        \mathbf{H} = \frac{\partial \mathbf{P}}{\partial \mathbf{X}}.
    \]
    Тогда для функции~$\text{LayerNorm}$ матрица Гессе относительно параметров~$\mathbf{X}$ имеет вид:
    \begin{align}\label{chapter-2:theorem:layernorm_second_derivative:eqstemant}
        \frac{\partial^2 \text{LayerNorm}}{\partial \mathbf{X}^2} &= \left( (\mathbf{P}(\mathbf{X}) \otimes \mathbf{I}_{d_V}) \otimes \mathbf{I}_{L d_V} \right)\frac{\partial^2 \mathbf{M}}{\partial \mathbf{X}^2} +\\
        &\quad+\left( \mathbf{I}_{L d_V} \otimes \mathbf{G}^\top \right) \frac{\partial (\mathbf{P}(\mathbf{X}) \otimes \mathbf{I}_{d_V})}{\partial \mathbf{X}} + \\
        &\quad+ \left( (\mathbf{I}_L \otimes \mathbf{M}^\top ) \otimes \mathbf{I}_{L d_V} \right) \frac{\partial^2 \mathbf{P}}{\partial \mathbf{X}^2} + \left( \mathbf{I}_{L d_V} \otimes \mathbf{H}^\top \right) \frac{\partial (\mathbf{I}_L \otimes \mathbf{M}^\top )}{\partial\mathbf{X}}.
    \end{align}
    причем все матрицы явно вычислимые и задаются формулами, которые указаны ниже в доказательстве.
\end{theorem}
\begin{proof}
Используя свойство матричного произведения~\ref{prop:matrix_product_derivative} получаем следующее выражение матрицы Гессе для оператора~\text{LayerNorm}:
\begin{align}
    \frac{\partial^2\text{LayerNorm}}{\partial\mathbf{X}^2} &= \left( (\mathbf{P}(\mathbf{X})  \otimes  \mathbf{I}_{d_V}) \otimes \mathbf{I}_{Ld_V} \right)\frac{\partial^2 \mathbf{M}}{\partial \mathbf{X}^2} +\\
    &\quad+\left( \mathbf{I}_{Ld_V} \otimes  \left(\frac{\partial \mathbf{M}}{\partial \mathbf{X}}\right)^\top \right) \frac{\partial (\mathbf{P}(\mathbf{X}) \otimes \mathbf{I}_{d_V})}{\partial \mathbf{X}} + \\
    &\quad+ \left( (\mathbf{I}_L \otimes \mathbf{M}^\top ) \otimes \mathbf{I}_{L d_V}  \right) \frac{\partial^2 \mathbf{P}}{\partial \mathbf{X}^2} +\\
    &\quad+\left( \mathbf{I}_{L d_V} \otimes \left(\frac{\partial \mathbf{P}}{\partial \mathbf{X}}\right)^\top \right) \frac{\partial (\mathbf{I}_L \otimes \mathbf{M}^\top )}{\partial\mathbf{X}},
\end{align}
причем заметим, что~$\mathbf{P} \in \mathbb{R}^{L \times L}$, $\mathbf{M} \in \mathbb{R}^{L \times d_V}$, $\frac{\partial \mathbf{M}}{\partial \mathbf{X}} \in \mathbb{R}^{Ld_V \times Ld_V}$, $\frac{\partial \mathbf{P}}{\partial \mathbf{X}} \in \mathbb{R}^{L^2 \times Ld_V}.$ Используя свойства~\ref{prop:kronecker_product_derivative} и леммы~\ref{lemma:transposed_matrix_derivative} получаем следующие выражения первых и вторых производных: 
\begin{align}
    \frac{\partial^2 \mathbf{M}}{\partial \mathbf{X}^2} &= 0, \\
    \frac{\partial (\mathbf{P}(\mathbf{X}) \otimes \mathbf{I}_{d_V} )}{\partial \mathbf{X}} &= \frac{\partial (\mathbf{P} \otimes \mathbf{I}_L)}{\partial \mathbf{P}} \frac{\partial \mathbf{P}}{\partial\mathbf{X}} = \left(\mathbf{I}_L \otimes \mathbf{K}_{L, L} \otimes \mathbf{I}_L \right) \left(\mathbf{I}_{L^2} \otimes  \mathrm{vec}_r(\mathbf{I}_L)  \right) \frac{\partial \mathbf{P}}{\partial \mathbf{X}}, \\
    \frac{\partial (\mathbf{I}_L \otimes \mathbf{M}^\top )}{\partial\mathbf{X}} &= \frac{\partial ( \mathbf{I}_L \otimes \mathbf{M}^\top)}{\partial \mathbf{M}^\top} \frac{\partial \mathbf{M}^\top}{\partial \mathbf{M}} \frac{\partial \mathbf{M}}{\partial \mathbf{X}} =\\
    &=\left(\mathbf{I}_{L} \otimes \mathbf{K}_{d_V,L} \otimes \mathbf{I}_L \right) \left(\mathrm{vec}_r(\mathbf{I}_L) \otimes  \mathbf{I}_{L d_V} \right) \mathbf{K}_{d_V, L} \frac{\partial \mathbf{M}}{\partial \mathbf{X}}.
\end{align}

Перейдем к оценке вторых производных матрицы~$\mathbf{P}.$
Рассмотрим каждое слагаемое в матрице подробнее.
Матрица~$\mathbf{D}$ является диагональной матрицей~$\textit{diag}(\sigma(\mathbf{X})),$ причем размер вектор~$\sigma(\mathbf{X})$ имеет размерность  $L\times 1$, тогда матрица~$\mathbf{D} \in \mathbb{R}^{L\times L},$ а следовательно слагаемое~$\left(-\mathbf{D}^{-1} \otimes \mathbf{D}^{-\top} \right) \in \mathbb{R}^{L^2 \times L^2}$.
Каждый базисный вектор~$\mathbf{e}_i$ имеет размерность~$L\times1$, а следовательно~$\mathbf{e}_i \otimes \mathbf{e}_i \in \mathbb{R}^{L^2 \times 1},$ и тогда $ \Big(\mathbf{e}_1 \otimes \mathbf{e}_1 \quad \dots  \quad \mathbf{e}_L \otimes \mathbf{e}_L\Big) \in \mathbb{R}^{L^2 \times L}$.
Ранее было доказано, что~$\mathbf{M}(\mathbf{X}) \in \mathbb{R}^{L \times d_V}$, тогда $M\cdot\mathbf{1}_{d_V} \in \mathbb{R}^{L \times 1}$, и следовательно слагаемое~$\textit{diag}^{-1}(\mathrm{vec}_r^{\circ \frac{1}{2}}(\mathbf{M}^{\circ{2}}\cdot\mathbf{1}_{d_V}))$ имеет размерность $L \times L$.
Следующие слагаемые~$(\mathbf{I}_L \otimes \mathbf{1}^T_{d_V}) \in \mathbb{R}^{L \times Ld_V}$ и $\textit{diag}(\mathrm{vec}_r (M)) \in \mathbb{R}^{Ld_V \times Ld_V}$.
Последнее слагаемое~$\frac{\partial \mathbf{M}}{\partial \mathbf{X}}$ уже было посчитано ранее, причем его размерность~$Ld_V \times Ld_V$. Итого матрица~$\frac{\partial \mathbf{P}}{\partial \mathbf{X}}$ принадлежит пространству~$\mathbb{R}^{L^2 \times Ld_V}$.
 
Для удобства введем обозначение:
\[
    \frac{\partial \mathbf{P}}{\partial \mathbf{X}} = \frac{1}{\sqrt{d_V}} \mathbf{A}_1(\mathbf{X})\cdot \mathbf{B}_1(\mathbf{X}),
\]
где $\mathbf{A}_1 = \left(-\mathbf{D}^{-1} \otimes \mathbf{D}^{-\top} \right),$ а $\mathbf{B}_1$ все остальное, обе матрицы были вычислены ранее.
Используя свойство произведения матричнозначных функций~\ref{lemma:matrix_funcs_product_derivative} вторая производная принимает вид:
\begin{align}
    \frac{\partial^2 \mathbf{P}}{\partial \mathbf{X}^2} = \frac{1}{\sqrt{d_V}} \frac{\partial \mathbf{A}_1(\mathbf{X})\cdot \mathbf{B}_1(\mathbf{X})}{\partial \mathbf{X}} = \frac{1}{\sqrt{d_V}} \left(\mathbf{A}_1 \otimes  \mathbf{I}_{Ld_V}  \right) \frac{\partial \mathbf{B}_1}{\partial \mathbf{X}} + \left( \mathbf{I}_{L^2} \otimes  \mathbf{B}_1^\top \right) \frac{\partial \mathbf{A}_1}{\partial \mathbf{X}}.
\end{align}
Разберем вторую производную по частям. Сначала вычислим~$\frac{\partial \mathbf{A}_1}{\partial \mathbf{X}}.$ Используя лемму~\ref{lemma:matrix_funcs_kronecker_product_derivative} получаем следующее выражение:
\begin{align}
    \frac{\partial \mathbf{A}_1}{\partial \mathbf{X}} &= \frac{\partial \left(-\mathbf{D}^{-1} \otimes \mathbf{D}^{-\top} \right)}{\partial \mathbf{X}} =\\
    &=\left(\mathbf{I}_L \otimes \mathbf{K}_{L, L} \otimes \mathbf{I}_L \right) \Big((\mathbf{I}_{L^2} \otimes \mathrm{vec}_r(\mathbf{\mathbf{D}^{-\top}})) \cdot \frac{\partial -\mathbf{\mathbf{D}^{-1}}}{\partial \mathbf{X}}+ \\
    &\quad+ (\mathrm{vec}_r(-\mathbf{\mathbf{D}^{-1}}) \otimes \mathbf{I}_{L^2}) \cdot \frac{\partial \mathbf{\mathbf{D}^{-\top}}}{\partial \mathbf{X}} \Big).
\end{align}
Далее используя леммы~\ref{lemma:transposed_matrix_derivative}, \ref{lemma:invert_derivative} получаем выражение на:
\begin{align}
    \frac{\partial -\mathbf{\mathbf{D}^{-1}}}{\partial \mathbf{X}} &= \frac{\partial -\mathbf{\mathbf{D}^{-1}}}{\partial \mathbf{D}}\frac{\partial \mathbf{D}}{\partial \mathbf{X}} = \left( \mathbf{D}^{-1} \otimes \mathbf{D}^{-\top}\right) \frac{\partial \mathbf{D}}{\partial \mathbf{X}},\\
    \frac{\partial \mathbf{\mathbf{D}^{-\top}}}{\partial \mathbf{X}} &= \frac{\partial \mathbf{\mathbf{D}^{-\top}}}{\partial \mathbf{D}^{-1}}\frac{\partial \mathbf{\mathbf{D}^{-1}}}{\partial \mathbf{D}} \frac{\partial \mathbf{\mathbf{D}}}{\partial \mathbf{X}} = \mathbf{K}_{L, L} \left(-\mathbf{D}^{-1} \otimes \mathbf{D}^{-\top} \right) \frac{\partial \mathbf{\mathbf{D}}}{\partial \mathbf{X}},
\end{align}
где $\frac{\partial \mathbf{\mathbf{D}}}{\partial \mathbf{X}}$ вычисляется аналогично тому, как в теореме~\ref{thm:layernorm_derivative}:
\begin{align}
    \frac{\partial \mathbf{\mathbf{D}}}{\partial \mathbf{X}} &= \Big(\mathbf{e}_1 \otimes \mathbf{e}_1 \dots   \mathbf{e}_L \otimes \mathbf{e}_L\Big)\cdot\\
    &\quad\cdot \left(\textit{diag}^{-1}(\mathrm{vec}_r^{\circ \frac{1}{2}}(\mathbf{M}^{\circ{2}}\cdot\mathbf{1}_{d_V}))\cdot (\mathbf{I}_L \otimes \mathbf{1}^T_{d_V})\cdot \textit{diag}(\mathrm{vec}_r (\mathbf{M}))\frac{\partial \mathbf{M}}{\partial \mathbf{X}}\right),
\end{align}
заканчивая вывод оценки~$\frac{\partial \mathbf{A}_1}{\partial \mathbf{X}}.$

Перейдем к оценке~$\frac{\partial \mathbf{B}_1}{\partial \mathbf{X}}.$
Для начала снова представим матрицу~$\mathbf{B}_1$ в виде произведения матриц:
\begin{align}
    \mathbf{B}_1 = \mathbf{E} \mathbf{A}_2 \mathbf{B}_2,
\end{align}
где введены следующие обозначения матриц:
\begin{align}
    \mathbf{A}_2 &= \textit{diag}^{-1}(\mathrm{vec}_r^{\circ \frac{1}{2}}(\mathbf{M}^{\circ{2}}\cdot\mathbf{1}_{d_V}))\\
    \mathbf{B}_2 &= (\mathbf{I}_L \otimes \mathbf{1}^T_{d_V})\cdot \textit{diag}(\mathrm{vec}_r (\mathbf{M}))\frac{\partial \mathbf{M}}{\partial \mathbf{X}}\\
    \mathbf{E} &= \Big(\mathbf{e}_1 \otimes \mathbf{e}_1 \quad \dots  \quad \mathbf{e}_L \otimes \mathbf{e}_L\Big).
\end{align}
Для начала, заметим, что $\mathbf{E}$ является константной матрицей относительно матрицы~$\mathbf{X},$ а следовательно используя результат леммы~\ref{lemma:matrix_funcs_product_derivative} получаем
\begin{align}
    \frac{\partial\mathbf{B}_1}{\partial\mathbf{X}} &= \frac{\partial \mathbf{E} \mathbf{A}_2 \mathbf{B}_2}{\partial (\mathbf{A}_2 \mathbf{B}_2)} \frac{\partial \mathbf{A}_2 \mathbf{B}_2}{\partial \mathbf{X}} = \left( \mathbf{E} \otimes \mathbf{I}_{L d_V}\right)\frac{\partial \mathbf{A}_2 \mathbf{B}_2}{\partial \mathbf{X}} \\
    &= \left( \mathbf{E} \otimes \mathbf{I}_{L d_V}\right) \left( (\mathbf{A}_2\otimes \mathbf{I}_{L d_V} )\frac{\partial \mathbf{B}_2}{\partial \mathbf{X}} + (\mathbf{I}_L \otimes \mathbf{B}_2^\top) \frac{\partial \mathbf{A}_2}{\partial \mathbf{X}}\right).
\end{align}

Далее осталось оценить матрицы~$\frac{\partial\mathbf{A}_2}{\partial \mathbf{X}}$ и~$\frac{\partial\mathbf{B}_2}{\partial \mathbf{X}}.$
Для оценки~$\frac{\partial\mathbf{B}_2}{\partial \mathbf{X}}$ разобьем на части:
\begin{align}
    \mathbf{B}_2 = \mathbf{J} \mathbf{A}_3 \mathbf{B}_3,
\end{align}
где~$\mathbf{J} = (\mathbf{I}_L \otimes \mathbf{1}^T_{d_V})$, $\mathbf{A}_3 = \textit{diag}(\mathrm{vec}_r (\mathbf{M})), \mathbf{B}_3 = \frac{\partial \mathbf{M}}{\partial \mathbf{X}}$. Аналогично, используя лемму~\ref{lemma:matrix_funcs_product_derivative} получаем:
\begin{align}
    \frac{\partial \mathbf{B}_2}{\partial \mathbf{X}} &= \frac{\partial \mathbf{J} \mathbf{A}_3 \mathbf{B}_3}{\partial (\mathbf{A}_3 \mathbf{B}_3)} \frac{\partial \mathbf{A}_3 \mathbf{B}_3}{\partial \mathbf{X}} = \left( \mathbf{J} \otimes \mathbf{I}_{L d_V}\right)\frac{\partial \mathbf{A}_3 \mathbf{B}_3}{\partial \mathbf{X}} \\
    &= \left( \mathbf{J} \otimes \mathbf{I}_{L d_V}\right) \left((\mathbf{A}_3 \otimes \mathbf{I}_{Ld_V} ) \frac{\partial \mathbf{B}_3}{\partial \mathbf{X}} + ( \mathbf{I}_{Ld_V} \otimes \mathbf{B}_3^\top)\frac{\partial\mathbf{A}_3}{\partial \mathbf{X}}\right),
\end{align}
где
\begin{align}
    \frac{\partial\mathbf{A}_3}{\partial \mathbf{X}} &= \frac{\partial \textit{diag}(\mathrm{vec}_r(\mathbf{M}))}{\partial \mathbf{X}} = \frac{\partial \textit{diag}(\mathbf{v})}{\partial (\mathbf{v})} \frac{\partial \mathrm{vec}_r(\mathbf{M})}{\partial \mathbf{M}} \frac{\partial \mathbf{M}}{\partial \mathbf{X}},\\
    \frac{\partial \mathbf{B}_3}{\partial \mathbf{X}} &= \frac{\partial^2 \mathbf{M}}{\partial\mathbf{X}^2} = 0,
\end{align}
причем, используя лемму~\ref{lemma:diag_derivative} получаем, что~$\frac{\partial \textit{diag}(\mathbf{v})}{\partial (\mathbf{v})} = \Big(\mathbf{e}_1 \otimes \mathbf{e}_1 \quad \dots  \quad \mathbf{e}_L \otimes \mathbf{e}_L\Big),$ где $\mathbf{e}_i \in \mathbb{R}^{Ld_V \times 1}$, а также $\frac{\partial \mathrm{vec}_r(\mathbf{M})}{\partial \mathbf{M}}=\mathbf{I}_{L d_V}.$
Для вычисления матрицы~$\frac{\partial\mathbf{A}_2}{\partial \mathbf{X}}$ воспользуемся леммами~\ref{lemma:invert_derivative}, \ref{lemma:diag_derivative},\ref{lemma:hadamard_root_derivative},\ref{lemma:identification_theorem_vec_r} получаем выражение:
\begin{align}
    \textcolor{red}{\frac{\partial\mathbf{A}_2}{\partial\mathbf{X}} = \frac{\partial\textit{diag}^{-1}(\mathrm{vec}_r^{\circ \frac{1}{2}}(\mathbf{M}^{\circ{2}}\cdot\mathbf{1}_{d_V}))}{\partial \mathbf{X}}=}
\end{align}

Собирая все полученные выражения воедино, получаем выражение для~$\frac{\partial^2 \mathbf{P}}{\partial \mathbf{X}^2}$:
\begin{align}
    \frac{\partial^2 \mathbf{P}}{\partial \mathbf{X}^2} &= \frac{1}{\sqrt{d_V}} \left(\mathbf{A}_1 \otimes  \mathbf{I}_{Ld_V}  \right) \frac{\partial \mathbf{B}_1}{\partial \mathbf{X}} + \left( \mathbf{I}_{L^2} \otimes  \mathbf{B}_1^\top \right) \frac{\partial \mathbf{A}_1}{\partial \mathbf{X}},
\end{align}
где~$\frac{\partial \mathbf{B}_1}{\partial \mathbf{X}},\frac{\partial \mathbf{A}_1}{\partial \mathbf{X}}, \mathbf{B}_1, \mathbf{A}_1$ определены и получены выше.

Итого, все матрицы выражения~\eqref{chapter-2:theorem:layernorm_second_derivative:eqstemant} вычислены, что заканчивает доказательство.
\end{proof}

\subsection{Матрица Гессе для нелинейности ReLU}

\begin{theorem}\label{theorem:relu_derivative_hessian}
Пусть задана матрица~$\mathbf{X} \in \mathbb{R}^{m \times n},$ тогда для оператора~$\mathrm{ReLU}$ почти всюду верно следующее выражение:
\begin{align}
    \frac{\partial \mathrm{ReLU}(\mathbf{X})}{\partial \mathbf{X}} &= \mathrm{diag}\!\big(\mathrm{vec}_r(\mathbf{1}_{\{\mathbf{X}>0\}})\big), \\
    \frac{\partial^2 \mathrm{ReLU}(\mathbf{X})}{\partial \mathbf{X}^2} &= \mathbf{0}.
\end{align}
\end{theorem}
\begin{proof}
Оператор~$\mathrm{ReLU}$ принимает следующий вид:
\[
    \mathrm{ReLU}(x) = \max(0, x),
\]
то есть, для каждого элемента~$x_{ij}$ в матрице~$\mathbf{X} \in \mathbb{R}^{m \times n}$ получаем:
\[
    \frac{\partial \mathrm{ReLU}(x_{ij})}{\partial x_{ij}} =
    \begin{cases}
    1 & \text{если}~x_{ij} > 0, \\
    0 & \text{если}~x_{ij} < 0, \\
    \text{неопределенно (субградиент }\in[0,1]\text{)} & \text{если}~x_{ij} = 0.
    \end{cases}
\]
В случае скалярной величины~$x \in \mathbb{R},$ множеством с неопределенным градиентом является множество~$\{0\},$ которое является множеством меры 0.
Рассматривая же матрицу~$\mathbf{X} \in \mathbb{R}^{m \times n}$ как точку в~$\mathbb{R}^{m\times n},$ дифференцируемым множеством является множество:
\[
    \mathcal{N} = \bigcup_{i,j} \left\{ \mathbf{X} \in \mathbb{R}^{m \times n} : x_{ij} = 0 \right\}.
\]
Заметим, что каждое множество~$\{x_{ij} = 0\}$ является гиперплоскостью коразмерности~$1$ в пространстве~$\mathbb{R}^{m\times n},$ а следовательно является множеством меры~$0$.
Так как, множество~$\mathcal{N}$ является конечным объединением множеств меры~$0,$ то и множество~$\mathcal{N}$ также имеет меру~$0.$ Получили, что оператор~$\mathrm{ReLU}$ является почти всюду дифференцируем в пространстве~$\mathbb{R}^{m \times n}.$

Для каждой дифференцируемой точки~$\mathbf{X} \notin \mathcal{N},$ применим построчкую векторизацию и лемму~\ref{lemma:identification_theorem_vec_r}:
\begin{align}
    \mathrm{vec}_r(d\mathrm{ReLU}(\mathbf{X}))
    = \mathrm{diag}(\mathrm{vec}_r(\mathbf{1}_{\{\mathbf{X}>0\}}))  \mathrm{vec}_r(d\mathbf{X}),
\end{align}
причем, используя свойство~\ref{prop:vec_r_hadamard_product} и лемму~\ref{lemma:diag_derivative} для диагональной матрицы получаем:
\begin{align}
    \frac{\partial \mathrm{ReLU}(\mathbf{X})}{\partial \mathbf{X}}
= \mathrm{diag}\!\big(\mathrm{vec}_r(\mathbf{1}_{\{\mathbf{X}>0\}})\big).
\end{align}

В силу того, что матрица Якоби является кусочно-постоянной, то ее дифференциал равен нулю почти всюду:
\[
    d\left(\frac{\partial \mathrm{ReLU}(\mathbf{X})}{\partial \mathbf{X}}\right) = \mathbf{0}, \quad \mathbf{X} \notin \mathcal{N},
\]
а следовательно и матрица Гессе почти всюду равна нулевой матрице:
\[
    \frac{\partial^2 \mathrm{ReLU}(\mathbf{X})}{\partial \mathbf{X}^2} = \mathbf{0}, \quad \mathbf{X} \notin \mathcal{N}.
\]
\end{proof}

\subsection{Матрица Гессе для трансформера}

\begin{theorem}\label{thm:transformer_derivative}
Для модели глубокого обучения архитектуры трансформер~\ref{eq:transformer} матрица Якоби~$\frac{\partial\mathbf{Z}}{\partial \mathbf{W}_i}$ вычисляется в следующем виде:
\begin{align}
    \frac{\partial\mathbf{Z}}{\partial \mathbf{W}_i} &= \frac{\partial \text{LayerNorm}(\text{FFN}(\mathbf{Y}) + \mathbf{Y})}{\partial (\text{FFN}(\mathbf{Y}) + \mathbf{Y})} \frac{\partial (\text{FFN}(\mathbf{Y}) + \mathbf{Y})}{\partial \mathbf{W}_i},\qquad i \in \{1,2\},
\end{align}
где
\begin{equation}
    \frac{\partial (\text{FFN}(\mathbf{Y}) + \mathbf{Y})}{\partial \mathbf{W}_i} = \begin{cases}
        \left(\mathbf{I}_L \otimes \mathbf{W}_2^\top \right) \mathrm{diag}\!\big(\mathrm{vec}_r(\mathbf{1}_{\{\mathbf{X}>0\}})\big) \left( \mathbf{Y} \otimes \mathbf{I}_{d_{ff}}\right), & \text{for } i = 1 \\
        \sigma(\mathbf{Y} \mathbf{W}_1) \otimes \mathbf{I}_{d_V}, & \text{for } i = 2
    \end{cases},
\end{equation}
причем~$\frac{\partial \text{LayerNorm}(\text{FFN}(\mathbf{Y}) + \mathbf{Y})}{\partial (\text{FFN}(\mathbf{Y}) + \mathbf{Y})}$ вычисляется согласно теоремы~\ref{thm:layernorm_derivative}.
\begin{align}
    \frac{\partial\mathbf{Z}}{\partial \mathbf{W}_i} &= \frac{\partial \text{LayerNorm}(\text{FFN}(\mathbf{Y}) + \mathbf{Y})}{\partial (\text{FFN}(\mathbf{Y}) + \mathbf{Y})} \frac{\partial (\text{FFN}(\mathbf{Y}) + \mathbf{Y})}{\partial \mathbf{Y}} \frac{\partial \mathbf{Y}}{\partial \mathbf{W}_i}, \qquad i \in \{ K, Q, V\},
\end{align}
где
\begin{align}
    \frac{\partial (\text{FFN}(\mathbf{Y}) + \mathbf{Y})}{\partial \mathbf{Y}} &= \left( \mathbf{I}_L \otimes \mathbf{W}_2^\top\right) \mathrm{diag}\!\big(\mathrm{vec}_r(\mathbf{1}_{\{\mathbf{X}>0\}})\big) \left( \mathbf{I}_L \otimes \mathbf{W}_1^\top \right) + \left( \mathbf{I}_L \otimes \mathbf{I}_{d_V}\right),
\end{align}
причем~$\frac{\partial \mathbf{Y}}{\partial\mathbf{W}_i} = \frac{\partial \text{LayerNorm}(\mathbf{F} (\mathbf{X}) + \mathbf{X})}{\partial (\mathbf{F} (\mathbf{X}) + \mathbf{X})} \frac{\partial \mathbf{F}(\mathbf{X})}{\partial \mathbf{W}_i},$ где $\frac{\partial \mathbf{F}(\mathbf{X})}{\partial \mathbf{W}_i}$ вычисляется при помощи леммы~A.2 в работе~\cite{noci2022signalpropagationtransformerstheoretical}, а матрица~$\frac{\partial \text{LayerNorm}(\mathbf{F} (\mathbf{X}) + \mathbf{X})}{\partial (\mathbf{F} (\mathbf{X}) + \mathbf{X})}$ вычисляется согласно теоремы~\ref{thm:layernorm_derivative}.
\end{theorem}
\begin{proof}
В дальнейшем при доказательстве вводим следующие обозначения и предположения~$\mathbf{X} \in R^{L \times d_V}, \mathbf{Y} \in R^{L\times d_V}, \mathbf{W}_1 \in R^{d_V \times d_{ff}}, \text{ReLU}(\mathbf{Y\mathbf{W}_1}) \in R^{L \times d_{ff}}, \mathbf{W}_2 \in R^{d_{ff} \times d_V}.$
Трансформер блок определен в выражении~\eqref{eq:transformer}, а именно:
\begin{align}
    \mathbf{Y} &= \text{LayerNorm}(\mathbf{F}(\mathbf{X}) + \mathbf{X}),\\
    \mathbf{Z} &= \text{LayerNorm}(\text{FFN}(\mathbf{Y}) + \mathbf{Y}).
\end{align}

Начнем вычисления матрицы Гессе для полного трансформера с матриц~$\frac{\partial\mathbf{Z}}{\partial \mathbf{W}_i},$ тогда для~$i \in \{1,2\}$ получаем:
\begin{equation}
    \frac{\partial\mathbf{Z}}{\partial \mathbf{W}_i} = \frac{\partial \text{LayerNorm}(\text{FFN}(\mathbf{Y}) + \mathbf{Y})}{\partial (\text{FFN}(\mathbf{Y}) + \mathbf{Y})} \frac{\partial (\text{FFN}(\mathbf{Y}) + \mathbf{Y})}{\partial \mathbf{W}_i},
\end{equation}
где
\begin{equation}
    \frac{\partial (\text{FFN}(\mathbf{Y}) + \mathbf{Y})}{\partial \mathbf{W}_i} = \frac{\partial (\text{FFN}(\mathbf{Y}))}{\partial \mathbf{W}_i} = \frac{\partial \mathbf{I}_L \sigma(\mathbf{Y}\mathbf{W}_1)\mathbf{W}_2 \mathbf{I}_{d_V}}{\partial \mathbf{W}_i},
\end{equation}
причем используя свойство~\ref{prop:matrix_product_derivative} об производной произведения матриц получаем:
\begin{align}
    \frac{\partial \mathbf{I}_L \sigma(\mathbf{Y}\mathbf{W}_1)\mathbf{W}_2 \mathbf{I}_{d_V}}{\partial \mathbf{W}_2} &= \sigma(\mathbf{Y} \mathbf{W}_1) \otimes \mathbf{I}_{d_V}\\
    \frac{\partial \mathbf{I}_L \sigma(\mathbf{Y}\mathbf{W}_1)\mathbf{W}_2 \mathbf{I}_{d_V}}{\partial \mathbf{W}_1} &= \frac{\partial \sigma(\mathbf{Y}\mathbf{W}_1) \mathbf{W}_2}{\partial \sigma(\mathbf{Y}\mathbf{W}_1)} \frac{\partial \sigma(\mathbf{Y}\mathbf{W}_1)}{\partial \mathbf{Y}\mathbf{W}_1} \frac{\partial \mathbf{Y}\mathbf{W}_1}{\partial \mathbf{W}_1} =\\
    &= \left(\mathbf{I}_L \otimes \mathbf{W}_2^\top \right) \frac{\partial \sigma(\mathbf{Y}\mathbf{W}_1)}{\partial \mathbf{Y}\mathbf{W}_1} \left( \mathbf{I}_L \otimes \mathbf{W}_1^\top\right).
\end{align}
Используя результаты теоремы~\ref{theorem:relu_derivative_hessian} для производной оператора ReLU для матрицы~$\frac{\partial \sigma(\mathbf{Y}\mathbf{W}_1)}{\partial \mathbf{Y}\mathbf{W}_1}$ получаем следующее выражение:
\begin{align}
    \frac{\partial \mathbf{I}_L \sigma(\mathbf{Y}\mathbf{W}_1)\mathbf{W}_2 \mathbf{I}_{d_V}}{\partial \mathbf{W}_i} = \left(\mathbf{I}_L \otimes \mathbf{W}_2^\top \right) \mathrm{diag}\!\big(\mathrm{vec}_r(\mathbf{1}_{\{\mathbf{X}>0\}})\big) \left( \mathbf{Y} \otimes \mathbf{I}_{d_{ff}}\right).
\end{align}
Тогда в общем виде для~$i \in \{1, 2\}$ получаем следующее выражение: 
\begin{equation}
    \frac{\partial (\text{FFN}(\mathbf{Y}) + \mathbf{Y})}{\partial \mathbf{W}_i} = \begin{cases}
        \left(\mathbf{I}_L \otimes \mathbf{W}_2^\top \right) \mathrm{diag}\!\big(\mathrm{vec}_r(\mathbf{1}_{\{\mathbf{X}>0\}})\big) \left( \mathbf{Y} \otimes \mathbf{I}_{d_{ff}}\right), \text{если}~i = 1 \\
        \sigma(\mathbf{Y} \mathbf{W}_1) \otimes \mathbf{I}_{d_V}, \text{если}~i = 2
    \end{cases}.
\end{equation}
Тогда весь блок тронсформера имеет следующую производную:
\begin{equation}
    \frac{\partial\mathbf{Z}}{\partial \mathbf{W}_i} = 
    \begin{cases}
        \frac{\partial \text{LayerNorm}(\text{FFN}(\mathbf{Y}) + \mathbf{Y})}{\partial (\text{FFN}(\mathbf{Y}) + \mathbf{Y})}\left(\mathbf{I}_L \otimes \mathbf{W}_2^\top \right) \mathrm{diag}\!\big(\mathrm{vec}_r(\mathbf{1}_{\{\mathbf{X}>0\}})\big) \left( \mathbf{Y} \otimes \mathbf{I}_{d_{ff}}\right), i = 1 \\
        \frac{\partial \text{LayerNorm}(\text{FFN}(\mathbf{Y}) + \mathbf{Y})}{\partial (\text{FFN}(\mathbf{Y}) + \mathbf{Y})}\sigma(\mathbf{Y} \mathbf{W}_1) \otimes \mathbf{I}_{d_V}, i = 2
    \end{cases}
\end{equation}
причем, согласно теореме~\ref{thm:layernorm_derivative} об производной LayerNorm получаем следующее выражение в нашем случае:
\begin{align}
    &\frac{\partial\text{LayerNorm}(\text{FFN}(\mathbf{Y}) + \mathbf{Y})}{\partial (\text{FFN}(\mathbf{Y}) + \mathbf{Y})} =\\
    &=( \mathbf{P}(\text{FFN}(\mathbf{Y}) + \mathbf{Y}) \otimes \mathbf{I}_{d_V}) \frac{\partial \mathbf{M}}{\partial (\text{FFN}(\mathbf{Y}) + \mathbf{Y})} +\\
    &\quad+ (\mathbf{I}_L\otimes \mathbf{M}^\top)\frac{1}{\sqrt{d_V}}\left(-\mathbf{D}^{-1} \otimes \mathbf{D}^{-\top} \right) \Big(\mathbf{e}_1 \otimes \mathbf{e}_1 \quad \dots  \quad \mathbf{e}_L \otimes \mathbf{e}_L\Big) \cdot\\
    &\qquad\cdot \left(\textit{diag}^{-1}(\mathrm{vec}_r^{\circ \frac{1}{2}}(\mathbf{M}^{\circ{2}}\cdot\mathbf{1}_{d_V}))\cdot (\mathbf{I}_L \otimes \mathbf{1}^T_{d_V})\cdot \textit{diag}(\mathrm{vec}_r (\mathbf{M}))\frac{\partial \mathbf{M}}{\partial (\text{FFN}(\mathbf{Y}) + \mathbf{Y})}\right),
\end{align}
где
\begin{align}
    \mathbf{M}(\text{FFN}(\mathbf{Y}) + \mathbf{Y}) &= ((\text{FFN}(\mathbf{Y}) + \mathbf{Y}) - \frac{1}{d_V}(\text{FFN}(\mathbf{Y}) + \mathbf{Y}) \mathbf{1}_{d_V \times d_V}),\\
    \mathbf{P}((\text{FFN}(\mathbf{Y}) + \mathbf{Y})) &= \textit{diag}^{-1}(\sigma(\text{FFN}(\mathbf{Y}) + \mathbf{Y}),\\
    \frac{\partial \mathbf{M}}{\partial(\text{FFN}(\mathbf{Y}) + \mathbf{Y})} &= (\mathbf{I}_L \otimes \mathbf{I}_{d_V}) - \frac{1}{d_V}(\mathbf{I}_L \otimes \mathbf{1}_{d_V \times d_V}),
\end{align}
где~$\sigma$ вычисляется согласно определению оператору LayerNorm.

Далее, перейдем к вычислению матриц Гессе~$\frac{\partial\mathbf{Z}}{\partial \mathbf{W}_i}$ для~$i \in \{ K, Q, V\},$ где получаем:
\begin{align}
    \frac{\partial\mathbf{Z}}{\partial \mathbf{W}_i} = \frac{\partial \text{LayerNorm}(\text{FFN}(\mathbf{Y}) + \mathbf{Y})}{\partial (\text{FFN}(\mathbf{Y}) + \mathbf{Y})} \frac{\partial (\text{FFN}(\mathbf{Y}) + \mathbf{Y})}{\partial \mathbf{Y}} \frac{\partial \mathbf{Y}}{\partial \mathbf{W}_i},
\end{align}
где используя свойство~\ref{prop:matrix_product_derivative} и результат теоремы~\ref{theorem:relu_derivative_hessian} получаем:
\begin{align}
    \frac{\partial (\text{FFN}(\mathbf{Y}) + \mathbf{Y})}{\partial \mathbf{Y}} &= \frac{\partial \text{FFN}(\mathbf{Y}) }{\partial \mathbf{Y}} + \frac{\partial \mathbf{Y}}{\partial \mathbf{Y}} =\\
    &= \frac{\partial \text{FFN}(\mathbf{Y}) }{\partial \mathbf{Y}} + \left( \mathbf{I}_L \otimes \mathbf{I}_{d_V}\right) =\\
    &=\frac{\partial \sigma(\mathbf{Y} \mathbf{W}_1) \mathbf{W}_2}{\partial \mathbf{Y}} +\left( \mathbf{I}_L \otimes \mathbf{I}_{d_V}\right)=\\ 
    &= \left( \mathbf{I}_L \otimes \mathbf{W}_2^\top\right) \frac{\partial \sigma (\mathbf{Y} \mathbf{W}_1)}{\partial \mathbf{Y}\mathbf{W}_1} \frac{\partial \mathbf{Y}\mathbf{W}_1}{\partial \mathbf{Y}} + \left( \mathbf{I}_L \otimes \mathbf{I}_{d_V}\right) = \\
    & = \left( \mathbf{I}_L \otimes \mathbf{W}_2^\top\right) \mathrm{diag}\!\big(\mathrm{vec}_r(\mathbf{1}_{\{\mathbf{X}>0\}})\big) \left( \mathbf{I}_L \otimes \mathbf{W}_1^\top \right) + \left( \mathbf{I}_L \otimes \mathbf{I}_{d_V}\right).
\end{align}
Для вычисления матрицы~$\frac{\partial \mathbf{Y}}{\partial \mathbf{W}_i}$ используем результат леммы~A.2 с работы~\cite{noci2022signalpropagationtransformerstheoretical}:
\begin{align}
    \frac{\partial \mathbf{F}}{\partial \mathbf{W}_V} &= \text{softmax}\left(\frac{\mathbf{X}\mathbf{W}_Q\mathbf{W}_{K}^{\top}\mathbf{X}^\top}{\sqrt{d_K}}\right) \mathbf{X} \otimes \mathbf{I}_{d_V}\\
    \frac{\partial \mathbf{F}}{\partial \mathbf{W}_Q} &= \left(\mathbf{I}_L \otimes \mathbf{W}_{V}^{\top}\mathbf{X}^\top\right) \frac{\partial \mathbf{A}}{\partial \mathbf{M}} \left(\frac{\mathbf{X} \otimes \mathbf{X}\mathbf{W}_K}{\sqrt{d_K}}\right),
\end{align}
где
\begin{equation}
    \frac{\partial \mathbf{A}}{\partial \mathbf{M}} = \text{blockdiag}\left(\frac{\partial \mathbf{A}_i}{\partial \mathbf{M}_i^\top}\right),
\end{equation}
причем данное выражение сильно упрощается используя свойства матрицы~$\mathbf{A}$:
\begin{align}
    \frac{\partial \mathbf{A}_i}{\partial \mathbf{M}_i^\top} = \text{diag}(\mathbf{A}_i) - \mathbf{A}_i\mathbf{A}_i^\top,
\end{align}
где~$\mathbf{A}_i$ является $i$-й строкой матрицы~$\mathbf{A}$ в формате вектора. Итого в условия равномерного внимания (англ. uniform-attention) данное выражение упрощается до:
\begin{equation}
    \frac{\partial \mathbf{A}}{\partial \mathbf{M}} = \frac{1}{n}\mathbf{I}_L \otimes \left(\mathbf{I}_L - \frac{1}{L}\mathbf{1}_{L \times L}\right)
\end{equation}
Аналогично, используя лемму~\ref{lemma:transposed_matrix_derivative} вычисляем производну относительно матрицы~$\mathbf{W}_K$:  
\begin{align}
    \frac{\partial \mathbf{F}}{\partial \mathbf{W}_K} &= \left(\mathbf{I}_L \otimes \mathbf{W}_{V}^{\top}\mathbf{X}^\top\right) \frac{\partial \mathbf{A}}{\partial \mathbf{M}} \left(\frac{(\mathbf{X} \mathbf{W}_Q \otimes \mathbf{X})\mathbf{K}_{d_V d_K}}{\sqrt{d_k}}\right).
\end{align}
Получаем, что матрица~$\frac{\partial \mathbf{Y}}{\partial\mathbf{W}_i}$ для~$i \in \{ K, Q, V\}$ вычисляется следующим образом:
\begin{align}
    \frac{\partial\mathbf{Y}}{\partial\mathbf{W}_i} &= \frac{\partial \text{LayerNorm}(\mathbf{F} (\mathbf{X}) + \mathbf{X})}{\partial \mathbf{W}_i} =\\
    &=\frac{\partial\text{LayerNorm}(\mathbf{F} (\mathbf{X}) + \mathbf{X})}{\partial (\mathbf{F} (\mathbf{X}) + \mathbf{X})} \frac{\partial \mathbf{F}(\mathbf{X})}{\partial \mathbf{W}_i},
\end{align}
где~$\frac{\partial \mathbf{F}(\mathbf{X})}{\partial \mathbf{W}_i}$ вычисляется согласно леммы~A.2 с работы~\cite{noci2022signalpropagationtransformerstheoretical}, а матрица~$\frac{\partial \text{LayerNorm}(\mathbf{F} (\mathbf{X}) + \mathbf{X})}{\partial (\mathbf{F} (\mathbf{X}) + \mathbf{X})}$ вычисляется согласно теоремы~\ref{thm:layernorm_derivative}.
\end{proof}

В теореме~\ref{thm:transformer_derivative} получен вид матрицы Якоби для полного блока трансформера, теперь можно перейти к вычислению матрицы Гессе для, который получен в виде теоремы~\ref{thm:transformer_hessian}.

\begin{theorem}\label{thm:transformer_hessian}
Пусть заданы матрицы параметров модели трансформера~$\mathbf{X} \in \mathbb{R}^{L \times d_V}$, $\mathbf{Y} \in \mathbb{R}^{L \times d_V}$, $\mathbf{W}_1 \in \mathbb{R}^{d_V \times d_{ff}}$, $\mathbf{W}_2 \in \mathbb{R}^{d_{ff} \times d_V}$, $\mathbf{W}_Q, \mathbf{W}_K \in \mathbb{R}^{d_V \times d_K}$, $\mathbf{W}_V \in \mathbb{R}^{d_V \times d_V},$
где блок трансформатор описан в виде следующих матричнозначных функций:
\[
    \mathbf{S}(\mathbf{Y},\mathbf{W}_1,\mathbf{W}_2) = \sigma(\mathbf{Y}\mathbf{W}_1)\mathbf{W}_2 + \mathbf{Y} \in \mathbb{R}^{L \times d_V}, \qquad\mathbf{Z} = \mathrm{LayerNorm}(\mathbf{S}) \in \mathbb{R}^{L \times d_V},
\]
для которых в условиях теорем~\ref{thm:layernorm_derivative} и \ref{thm:layernorm_second_derivative} вычислимые матрицы Якобы и Гессе вида:
\[
    \mathbf{J}_Z := \frac{\partial\mathrm{LayerNorm}(\mathbf{S})}{\partial \mathbf{S}} \in \mathbb{R}^{L d_V \times L d_V}, \quad\mathbf{H}_Z := \frac{\partial^2\mathrm{LayerNorm}(\mathbf{S})}{\partial \mathbf{S}^2} \in \mathbb{R}^{(L d_V)^2 \times L d_V}.
\]
Также в условиях теорем~\ref{thm:layernorm_derivative}, \ref{thm:layernorm_second_derivative} и \ref{theorem:relu_derivative_hessian} введем следующее:
\begin{align}
    \mathbf{D}_\sigma := \mathrm{diag}\big(\mathrm{vec}_r(\mathbf{1}_{\{\mathbf{Y}\mathbf{W}_1>0\}})\big) \in \mathbb{R}^{L d_{ff} \times L d_{ff}},\\
    \mathbf{J}_{SY} := \frac{\partial \mathbf{S}}{\partial \mathbf{Y}} = (\mathbf{I}_L \otimes \mathbf{W}_2^\top)\mathbf{D}_\sigma (\mathbf{I}_L \otimes \mathbf{W}_1^\top) + (\mathbf{I}_L \otimes \mathbf{I}_{d_V}) \in \mathbb{R}^{L d_V \times L d_V},
\end{align}
где для матрицы~$\mathbf{Y} = \mathrm{LayerNorm}(\mathbf{F}(\mathbf{X}) + \mathbf{X})$ в условиях теорем~~\ref{thm:layernorm_derivative}, \ref{thm:layernorm_second_derivative} определено:
\begin{align}
    \mathbf{J}_Y &:= \frac{\partial \mathrm{LayerNorm}(\mathbf{F}(\mathbf{X})+\mathbf{X})}{\partial (\mathbf{F}(\mathbf{X})+\mathbf{X})} \in \mathbb{R}^{L d_V \times L d_V}, \\
    \mathbf{H}_Y &:= \frac{\partial^2 \mathrm{LayerNorm}(\mathbf{F}(\mathbf{X})+\mathbf{X})}{\partial (\mathbf{F}(\mathbf{X})+\mathbf{X})^2} \in \mathbb{R}^{(L d_V)^2 \times L d_V},
\end{align}
где для удобства введем следующие обозначения:~ $n_1 = d_V d_{ff}, n_2 = d_{ff} d_V, n_Q = n_K = d_V d_K, n_V = d_V^2.$
Пусть матрицы Якоби вычислимы в условиях теоремы~\ref{thm:transformer_derivative} в следующем виде:
\begin{align}
     \mathbf{G}_V &:= \frac{\partial \mathbf{F}}{\partial \mathbf{W}_V} \in \mathbb{R}^{L d_V \times n_V},\\
     \mathbf{G}_Q &:= \frac{\partial \mathbf{F}}{\partial \mathbf{W}_Q} \in \mathbb{R}^{L d_V \times n_Q},\\
     \mathbf{G}_K &:= \frac{\partial \mathbf{F}}{\partial \mathbf{W}_K} \in \mathbb{R}^{L d_V \times n_K},\\
     \mathbf{B}_1 &:= \frac{\partial \mathbf{S}}{\partial \mathbf{W}_1} = (\mathbf{I}_L \otimes \mathbf{W}_2^\top) \mathbf{D}_\sigma (\mathbf{Y} \otimes \mathbf{I}_{d_{ff}}) \in \mathbb{R}^{L d_V \times n_1},\\
     \mathbf{B}_2 &:= \frac{\partial \mathbf{S}}{\partial \mathbf{W}_2} = \sigma(\mathbf{Y}\mathbf{W}_1) \otimes \mathbf{I}_{d_V} \in \mathbb{R}^{L d_V \times n_2},\\
     \mathbf{B}_k &:= \frac{\partial \mathbf{S}}{\partial \mathbf{W}_k} = \mathbf{J}_{SY} \mathbf{J}_Y \mathbf{G}_k \in \mathbb{R}^{L d_V \times n_k}, \quad k \in \{K,Q,V\}.
\end{align}

Тогда матрицы Гессе трансформера~$\mathbf{Z}$ по параметрам модели~$(\mathbf{W}_i,\mathbf{W}_j)$ задается в виде:
\begin{equation}\label{eq:block_hessian_transformer}
    \;\mathbf{H}_{\mathrm{tr}}^{(i,j)} := \frac{\partial^2 \mathbf{Z}}{\partial \mathbf{W}_i \partial \mathbf{W}_j}
    = \left( \mathbf{J}_Z \otimes \mathbf{I}_{n_i} \right) \boldsymbol{\xi}_{ij}
      + \left( \mathbf{I}_{L d_V} \otimes \mathbf{B}_i^\top \right) \mathbf{H}_Z \mathbf{B}_j,
\end{equation}
где размерность матрицы Гессе~$\mathbf{H}_{\mathrm{tr}}^{(i,j)} \in \mathbb{R}^{(L d_V \cdot n_i) \times n_j},$ также введены дополнительные матрицы для удобства:
\[
    \boldsymbol{\xi}_{ij} := \frac{\partial}{\partial \mathbf{W}_j} \left( \frac{\partial \mathbf{S}}{\partial \mathbf{W}_i} \right) \in \mathbb{R}^{(L d_V \cdot n_i) \times n_j}.
\]
Матрицы $\boldsymbol{\xi}_{ij}$ вычисляются для всех пар~$(i,j)$ почти всюду.

Для пар FFN:
\begin{align}
    \boldsymbol{\xi}_{11} &= \mathbf{0}_{(L d_V \cdot n_1) \times n_1}, \\
    \boldsymbol{\xi}_{22} &= \mathbf{0}_{(L d_V \cdot n_2) \times n_2}, \\
    \boldsymbol{\xi}_{12} &= \left( \mathbf{I}_L \otimes \mathbf{K}_{d_V, d_{ff}} \otimes \mathbf{I}_{d_V} \right) \left( \mathbf{I}_{L d_{ff}} \otimes \mathrm{vec}_r(\mathbf{I}_{d_V}) \right)\left( \mathbf{D}_\sigma  (\mathbf{Y} \otimes \mathbf{I}_{d_{ff}}) \right),\\
    \boldsymbol{\xi}_{21} &= \left( \mathbf{I}_{L d_V} \otimes \left( (\mathbf{Y} \otimes \mathbf{I}_{d_{ff}})^\top \mathbf{D}_\sigma^\top \right) \right)\left( \mathbf{I}_L \otimes \mathbf{K}_{d_V, L} \otimes \mathbf{I}_{d_{ff}} \right) \left( \mathrm{vec}_r(\mathbf{I}_L) \otimes \mathbf{I}_{d_V d_{ff}} \right) \mathbf{K}_{d_{ff}, d_V},
\end{align}
где матрицы~$\boldsymbol{\xi}_{12},\boldsymbol{\xi}_{21}$ имеют размерности $(L d_V \cdot n_1) \times n_2$ и $(L d_V \cdot n_2) \times n_1$ соответственно.

Для пар FFN с параметрами слоев внимания для всех~$k \in \{K,Q,V\}$:
\begin{align}
    \boldsymbol{\xi}_{1k} &= \left( (\mathbf{I}_L \otimes \mathbf{W}_2^\top) \mathbf{D}_\sigma \otimes \mathbf{I}_{n_k}\right)\left( \mathbf{I}_L \otimes \mathbf{K}_{d_{ff}, d_V} \otimes \mathbf{I}_{d_{ff}} \right)\left( \mathbf{I}_{L d_V} \otimes \mathrm{vec}_r(\mathbf{I}_{d_{ff}}) \right) \left( \mathbf{J}_Y \mathbf{G}_k \right),\\
    \boldsymbol{\xi}_{2k} &= \left( \mathbf{I}_L \otimes \mathbf{K}_{d_V, d_{ff}} \otimes \mathbf{I}_{d_V} \right)\left( \mathbf{I}_{L d_{ff}} \otimes \mathrm{vec}_r(\mathbf{I}_{d_V}) \right)\left( \mathbf{D}_\sigma (\mathbf{I}_L \otimes \mathbf{W}_1^\top)  \mathbf{J}_Y \mathbf{G}_k \right),
\end{align}
где размерности матрицы~$\boldsymbol{\xi}_{1k} \in \mathbb{R}^{(L d_V \cdot n_1) \times n_k}$ и матрицы~$\boldsymbol{\xi}_{2k} \in \mathbb{R}^{(L d_V \cdot n_2) \times n_k}$.

Для пар слоев внимания~$k,\ell \in \{K,Q,V\}$:
\begin{align}
    \boldsymbol{\xi}_{k\ell} 
    = \left( \mathbf{J}_{SY} \otimes \mathbf{I}_{n_k} \right)
    \left[\left( \mathbf{I}_{L d_V} \otimes \mathbf{G}_k^\top \right) \left( \mathbf{H}_Y \mathbf{G}_\ell \right)+ \left( \mathbf{J}_Y \otimes \mathbf{I}_{n_k} \right) \boldsymbol{\Phi}_{k\ell}\right],
\end{align}
где~$\boldsymbol{\Phi}_{k\ell} := \frac{\partial \mathbf{G}_k}{\partial \mathbf{W}_\ell} \in \mathbb{R}^{(L d_V \cdot n_k) \times n_\ell}$ является второй производной слоя внимания~$\mathbf{F}$ по ее параметрам, которые вычислены в рамках леммы~\ref{lemma:attention_phi_from_functional_hessian}. Все матрицы имеют следующие размерности~$\boldsymbol{\xi}_{k\ell} \in \mathbb{R}^{(L d_V \cdot n_k) \times n_\ell}$.

Также матрица Гессе удовлетворяет следующим свойствам почти везде:
\[
    \mathbf{H}_{\mathrm{tr}}^{(i,j)} = \mathbf{H}_{\mathrm{tr}}^{(j,i)},
\]
так как, во первых единственные нелинейности с потенциально ненулевым вторым дифференциалом является оператор LayerNorm, для которого получены матрицы~$\mathbf{H}_Z,\mathbf{H}_Y$ в рамках теоремы~\ref{thm:layernorm_second_derivative} и которые являются симметричными по построению и оператор ReLU, для которого матрица Гессе является нулевой согласно теоремы~\ref{theorem:relu_derivative_hessian}, а во вторых все другие отображения являются линейными, а следовательно согласно леммы~\ref{lemma:matrix_funcs_product_derivative} и свойства производной произведения Кронекера их частные производные являются коммутативными почти всюду.
\end{theorem}
\begin{proof}
Вычислим производной матрицы Якоби с теоремы~\ref{thm:transformer_derivative} используя лемму~\ref{lemma:matrix_funcs_product_derivative} об производной матричного умножения, также свойство производной произведения Кронекера~\ref{prop:kronecker_product_derivative}, также лемму~\ref{lemma:transposed_matrix_derivative} об производной транспонированной матрицы, лемму~\ref{lemma:identification_theorem_vec_r} и теорему~\ref{theorem:relu_derivative_hessian}. Для удобства доказательства разделим его на 4 шага.

На 1-м шаге для всех $i \in \{1,2,K,Q,V\}$ получаем:
\begin{align}
    \frac{\partial \mathbf{Z}}{\partial \mathbf{W}_i} \;=\; \mathbf{J}_Z  \mathbf{B}_i,\qquad\mathbf{J}_Z \in \mathbb{R}^{L d_V \times L d_V},
\end{align}
где $\mathbf{B}_i := \frac{\partial \mathbf{S}}{\partial \mathbf{W}_i}$ задается следующим образом:
\begin{align}
    \mathbf{B}_1 &= (\mathbf{I}_L \otimes \mathbf{W}_2^\top) \mathbf{D}_\sigma (\mathbf{Y} \otimes \mathbf{I}_{d_{ff}}) \in \mathbb{R}^{L d_V \times n_1}, \\
    \mathbf{B}_2 &= \sigma(\mathbf{Y}\mathbf{W}_1) \otimes \mathbf{I}_{d_V} \in \mathbb{R}^{L d_V \times n_2},\\
    \mathbf{B}_k &= \mathbf{J}_{SY} \mathbf{J}_Y \mathbf{G}_k \in \mathbb{R}^{L d_V \times n_k},\qquad k \in \{K,Q,V\},
\end{align}
где матрица~$\mathbf{J}_{SY}$ вычисляется следующим образом:
\begin{align}
    \mathbf{J}_{SY} = \frac{\partial \mathbf{S}}{\partial \mathbf{Y}} = (\mathbf{I}_L \otimes \mathbf{W}_2^\top)\mathbf{D}_\sigma(\mathbf{I}_L \otimes \mathbf{W}_1^\top) + (\mathbf{I}_L \otimes \mathbf{I}_{d_V}) \in \mathbb{R}^{L d_V \times L d_V},
\end{align}
матрица имеет следующую размерность~$\mathbf{J}_Y \in \mathbb{R}^{L d_V \times L d_V},$ а матрица~$\mathbf{G}_k$ описана в теореме~\ref{thm:transformer_derivative}.
Используя лемму~\ref{lemma:matrix_funcs_product_derivative} и теорему~\ref{thm:layernorm_second_derivative} получаем выражение для блока матрицы Гессе:
\begin{align}
     \frac{\partial^2 \mathbf{Z}}{\partial \mathbf{W}_i \partial \mathbf{W}_j} &= \left( \mathbf{J}_Z \otimes \mathbf{I}_{n_i} \right) \boldsymbol{\xi}_{ij}  + \left( \mathbf{I}_{L d_V} \otimes \mathbf{B}_i^\top \right) \mathbf{H}_Z \mathbf{B}_j,\\
     \boldsymbol{\xi}_{ij} &:= \frac{\partial \mathbf{B}_i}{\partial \mathbf{W}_j} \in \mathbb{R}^{(L d_V \cdot n_i) \times n_j}.
\end{align}


На 2-м шаге вычисляем размерности и вид матриц ~$\mathbf{B}_i.$ Используя результаты теорем~\ref{thm:transformer_derivative} и~\ref{theorem:relu_derivative_hessian} получаем следующие выражения:
\begin{align}
    \mathbf{B}_1 &= (\mathbf{I}_L \otimes \mathbf{W}_2^\top) \mathbf{D}_\sigma (\mathbf{Y} \otimes \mathbf{I}_{d_{ff}}) \in \mathbb{R}^{L d_V \times n_1}, \\
    \mathbf{B}_2 &= \sigma(\mathbf{Y}\mathbf{W}_1) \otimes \mathbf{I}_{d_V} \in \mathbb{R}^{L d_V \times n_2},
\end{align}
где матрица~$\mathbf{D}_\sigma \in \mathbb{R}^{L d_{ff} \times L d_{ff}}$, матрица~$(\mathbf{Y} \otimes \mathbf{I}_{d_{ff}}) \in \mathbb{R}^{L d_{ff} \times d_V d_{ff}}$.
Тогда для всех матриц~$\mathbf{B}_k,$ где~$k \in \{K,Q,V\}$ получаем:
\begin{align}
    \mathbf{B}_k = \mathbf{J}_{SY} \mathbf{J}_Y  \mathbf{G}_k \in \mathbb{R}^{L d_V \times n_k}.
\end{align}

На 3-м шаге вычисляем вычисляем матрицы~$\boldsymbol{\xi}_{ij}$ для всех пар~$(i, j)$.

Начнем вычисления с пар FFN. Заметим, что матрица~$\mathbf{B}_1$ не зависит от матрицы~$\mathbf{W}_1,$ а следовательно~$\boldsymbol{\xi}_{11} = \mathbf{0}.$ Аналогично матрица~$\mathbf{B}_2$ не зависит от матрицы~$\mathbf{W}_2,$ а следовательно~$\boldsymbol{\xi}_{22} = \mathbf{0}.$ Вычислим~$\frac{\partial \mathbf{B}_2}{\partial \mathbf{W}_1}$ используя свойство~\ref{prop:kronecker_product_derivative} производной произведения Кронекера для $\frac{\partial (\mathbf{X} \otimes \mathbf{Y})}{\partial \mathbf{X}},$ где $\mathbf{X}=\sigma(\mathbf{Y}\mathbf{W}_1)$ и $\mathbf{Y}=\mathbf{I}_{d_V}$:
\begin{align}
    \frac{\partial \mathbf{B}_2}{\partial \mathbf{W}_1}=\left( \mathbf{I}_L \otimes \mathbf{K}_{d_V, d_{ff}} \otimes \mathbf{I}_{d_V} \right)\left( \mathbf{I}_{L d_{ff}} \otimes \mathrm{vec}_r(\mathbf{I}_{d_V}) \right)\frac{\partial \mathrm{vec}_r(\sigma(\mathbf{Y}\mathbf{W}_1))}{\partial \mathbf{W}_1},
\end{align}
далее используя то, что $\frac{\partial\mathrm{vec}_r(\sigma(\mathbf{Y}\mathbf{W}_1))}{\partial \mathbf{W}_1} = \mathbf{D}_\sigma(\mathbf{Y} \otimes \mathbf{I}_{d_{ff}})$ получаем оценку на~$\boldsymbol{\xi}_{12}:$
\begin{align}
    \boldsymbol{\xi}_{12}= \left( \mathbf{I}_L \otimes \mathbf{K}_{d_V, d_{ff}} \otimes \mathbf{I}_{d_V} \right) \left( \mathbf{I}_{L d_{ff}} \otimes \mathrm{vec}_r(\mathbf{I}_{d_V}) \right)\left( \mathbf{D}_\sigma  (\mathbf{Y} \otimes \mathbf{I}_{d_{ff}}) \right).
\end{align}
Вычислим~$\frac{\partial \mathbf{B}_1}{\partial \mathbf{W}_2}$ используя лемму~\ref{lemma:matrix_funcs_product_derivative}, где в качестве левого множителя выступает~$(\mathbf{I}_L \otimes \mathbf{W}_2^\top)$:
\begin{align}
    \frac{\partial\mathbf{B}_1}{\partial \mathbf{W}_2}= \left( \mathbf{I}_{L d_V} \otimes \left( (\mathbf{Y} \otimes \mathbf{I}_{d_{ff}})^\top \mathbf{D}_\sigma^\top \right) \right) \frac{\partial(\mathbf{I}_L \otimes \mathbf{W}_2^\top)}{\partial \mathbf{W}_2},
\end{align}
далее используя свойство~\ref{prop:kronecker_product_derivative} об производной произведения Кронекера и лемму~\ref{lemma:transposed_matrix_derivative} об производной транспонированной матрицы получем:
\begin{align}
    \frac{\partial(\mathbf{I}_L \otimes \mathbf{W}_2^\top)}{\partial \mathbf{W}_2}= \left( \mathbf{I}_L \otimes \mathbf{K}_{d_V, L} \otimes \mathbf{I}_{d_{ff}} \right) \left( \mathrm{vec}_r(\mathbf{I}_L) \otimes \mathbf{I}_{d_V d_{ff}} \right) \mathbf{K}_{d_{ff}, d_V}.
\end{align}
Далее собирая все полученные матрицы, получаем оценку на~$\boldsymbol{\xi}_{21}:$
\begin{align}
    \boldsymbol{\xi}_{21} = \left( \mathbf{I}_{L d_V} \otimes \left( (\mathbf{Y} \otimes \mathbf{I}_{d_{ff}})^\top \mathbf{D}_\sigma^\top \right) \right)\left( \mathbf{I}_L \otimes \mathbf{K}_{d_V, L} \otimes \mathbf{I}_{d_{ff}} \right) \left( \mathrm{vec}_r(\mathbf{I}_L) \otimes \mathbf{I}_{d_V d_{ff}} \right) \mathbf{K}_{d_{ff}, d_V}.
\end{align}

Перейдем к оценке пар FFN с параметрами слоев внимания для всех $k \in \{K, Q, V\}.$ Для матрицы~$\mathbf{B}_1 = (\mathbf{I}_L \otimes \mathbf{W}_2^\top) \mathbf{D}_\sigma (\mathbf{Y} \otimes \mathbf{I}_{d_{ff}}),$ заметим, что почти всюду матрица~$\frac{\partial \mathbf{D}_\sigma}{\partial \mathbf{Y}}=\mathbf{0}$ равняется нулю согласно теоремы~\ref{theorem:relu_derivative_hessian}, а следовательно только последний множитель зависит от матрицы~$\mathbf{W}_k.$ Используя лемму~\ref{lemma:matrix_funcs_product_derivative}, где первый множитель является константой, а также цепное правило относительно переменной~$\mathbf{Y}$ получаем:
\begin{align}
    \frac{\partial(\mathbf{Y} \otimes \mathbf{I}_{d_{ff}})}{\partial \mathbf{W}_k} = \left( \frac{\partial (\mathbf{Y} \otimes \mathbf{I}_{d_{ff}})}{\partial \mathbf{Y}} \right) \frac{\partial\mathbf{Y}}{\partial \mathbf{W}_k},
\end{align}
причем согласно свойства~\ref{prop:kronecker_product_derivative} об производной произведения Кронекера с матрицей~$\mathbf{X} = \mathbf{Y}$ и матрицей~$\mathbf{Y}=\mathbf{I}_{d_{ff}}$ получаем:
\begin{align}
    \frac{\partial (\mathbf{Y} \otimes \mathbf{I}_{d_{ff}})}{\partial \mathbf{Y}} = \left( \mathbf{I}_L \otimes \mathbf{K}_{d_{ff}, d_V} \otimes \mathbf{I}_{d_{ff}} \right)\left( \mathbf{I}_{L d_V} \otimes \mathrm{vec}_r(\mathbf{I}_{d_{ff}}) \right),
\end{align}
а в свою очередь согласно теоремы~\ref{thm:transformer_derivative} матрица~$\frac{\partial\mathbf{Y}}{\partial \mathbf{W}_k} = \mathbf{J}_Y \mathbf{G}_k.$ Тогда получаем оценку на матрицу~$\boldsymbol{\xi}_{1k}$ следующего вида:
\begin{align}
    \boldsymbol{\xi}_{1k} = \left( (\mathbf{I}_L \otimes \mathbf{W}_2^\top) \mathbf{D}_\sigma \otimes \mathbf{I}_{n_k}\right)\left( \mathbf{I}_L \otimes \mathbf{K}_{d_{ff}, d_V} \otimes \mathbf{I}_{d_{ff}} \right)\left( \mathbf{I}_{L d_V} \otimes \mathrm{vec}_r(\mathbf{I}_{d_{ff}}) \right) \left( \mathbf{J}_Y \mathbf{G}_k \right).
\end{align}
Перейдем к матрице~$\mathbf{B}_2 = \sigma(\mathbf{Y}\mathbf{W}_1) \otimes \mathbf{I}_{d_V},$ заметим, что только первый множитель произведения Кронекера зависит от матриц~$\mathbf{W}_k,$ а следовательно используя свойство~\ref{prop:kronecker_product_derivative} производной произведения Кронекера и цепное правило получим следующее выражение:
\[
    \boldsymbol{\xi}_{2k} = \left( \mathbf{I}_L \otimes \mathbf{K}_{d_V, d_{ff}} \otimes \mathbf{I}_{d_V} \right)\left( \mathbf{I}_{L d_{ff}} \otimes \mathrm{vec}_r(\mathbf{I}_{d_V}) \right)\left( \mathbf{D}_\sigma (\mathbf{I}_L \otimes \mathbf{W}_1^\top)  \mathbf{J}_Y \mathbf{G}_k \right),
\]
где использовано свойство~\ref{prop:matrix_product_derivative} для преобразования~$\frac{\partial (\mathbf{Y}\mathbf{W}_1)}{\partial \mathbf{Y}} = \mathbf{I}_L \otimes \mathbf{W}_1^\top,$ а также теорема~\ref{theorem:relu_derivative_hessian} для выражения $\frac{\partial \sigma(\cdot)}{\partial (\cdot)} = \mathbf{D}_\sigma,$ а также согласно теореме~\ref{thm:transformer_derivative} матрица~$\frac{\partial\mathbf{Y}}{\partial \mathbf{W}_k} = \mathbf{J}_Y \mathbf{G}_k.$

Перейдем к оценке пар слоев внимания~$(k,\ell)$ with $k,\ell \in \{K,Q,V\}$.
Рассмотрим матрицу~$\mathbf{B}_k = \mathbf{J}_{SY} \mathbf{J}_Y  \mathbf{G}_k,$ заметим, что почти всюду~$\frac{\partial \mathbf{J}_{SY}}{\partial \mathbf{Y}} = \mathbf{0},$ так как матричнозначная функция~$\mathbf{D}_\sigma$ является кусочно-постоянной, согласно теоремы~\ref{theorem:relu_derivative_hessian}, а следовательно используя лемму~\ref{lemma:matrix_funcs_product_derivative} с матрицей~$\mathbf{A}(\cdot)=\mathbf{J}_Y,$ и с матрицей~$\mathbf{B}(\cdot)=\mathbf{G}_k$ получаем:
\begin{align}
    \frac{\partial\mathbf{B}_k}{\partial \mathbf{W}_\ell}= (\mathbf{J}_{SY} \otimes \mathbf{I}_{n_k}) \frac{\partial(\mathbf{J}_Y \mathbf{G}_k)}{\partial \mathbf{W}_\ell},
\end{align}
далее вычислим матрицу~$\frac{\partial(\mathbf{J}_Y \mathbf{G}_k)}{\partial \mathbf{W}_\ell}$ используя свойство производной матричного произведения:
\begin{align}
    \frac{\partial(\mathbf{J}_Y \mathbf{G}_k)}{\partial \mathbf{W}_\ell} = (\mathbf{J}_Y \otimes \mathbf{I}_{n_k}) \boldsymbol{\Phi}_{k\ell}+ \left( \mathbf{I}_{L d_V} \otimes \mathbf{G}_k^\top \right) \frac{\partial\mathbf{J}_Y}{\partial \mathbf{W}_\ell}.
\end{align}
В условия теоремы~\ref{thm:layernorm_second_derivative} легко получить следующее выражение~$\frac{\partial\mathbf{J}_Y}{\partial \mathbf{W}_\ell} = \mathbf{H}_Y \mathbf{G}_\ell,$ а следовательно получаем оценки:
\begin{align}
    \boldsymbol{\xi}_{k\ell} = \left( \mathbf{J}_{SY} \otimes \mathbf{I}_{n_k} \right)\left[\left( \mathbf{I}_{L d_V} \otimes \mathbf{G}_k^\top \right) \left( \mathbf{H}_Y \mathbf{G}_\ell \right)+ \left( \mathbf{J}_Y \otimes \mathbf{I}_{n_k} \right) \boldsymbol{\Phi}_{k\ell}\right].
\end{align}


На 4-м шаге проверим симметричность полученных выражений.
Во первых единственные нелинейности с потенциально ненулевым вторым дифференциалом является оператор LayerNorm, для которого получены матрицы~$\mathbf{H}_Z,\mathbf{H}_Y$ в рамках теоремы~\ref{thm:layernorm_second_derivative} и которые являются симметричными по построению и оператор ReLU, для которого матрица Гессе является нулевой согласно теоремы~\ref{theorem:relu_derivative_hessian}, а во вторых все другие отображения являются линейными, а следовательно согласно леммы~\ref{lemma:matrix_funcs_product_derivative} и свойства производной произведения Кронекера их частные производные являются коммутативными почти всюду.
\end{proof}

\subsection{Спектральные оценки матрицы Гессе для трансформера}

\begin{theorem}\label{thm:transformer_hessian_estimate}
Пусть матрица Гессе~$\mathbf{H}_{\mathrm{tr}}^{(i,j)}$ описывает матрицу Гессе между~$(i,j)$-м блоком трансформер модели~\eqref{eq:block_hessian_transformer}, где~$i,j\in\{1,2,K,Q,V\}, n_i=\dim(\mathbf{W}_i).$
Тогда для каждой пары~$(i,j)$ получаем оценку нормы:
\begin{equation}\label{eq:block_bound_transformer}
    \big\|\mathbf{H}_{\mathrm{tr}}^{(i,j)}\big\|_2 \le \|\mathbf{J}_Z\|_2 \|\boldsymbol{\xi}_{ij}\|_2 + \|\mathbf{B}_i\|_2 \|\mathbf{H}_Z\|_2 \|\mathbf{B}_j\|_2,
\end{equation}
где~$\boldsymbol{\xi}_{ij}=\frac{\partial}{\partial \mathbf{W}_j}\!\left(\frac{\partial \mathbf{S}}{\partial \mathbf{W}_i}\right), \mathbf{B}_i=\frac{\partial \mathbf{S}}{\partial \mathbf{W}_i}$. 

Пусть матрица~$\mathbf{H}_{\mathrm{tr}}$ является полной матрицей Гессе размера~$m_b \times n_b$ состоящей из блоков матрицы~$\mathbf{H}_{\mathrm{tr}}^{(i,j)},$ где $m_b=n_b=5,i\in\{1,2,K,Q,V\}, j\in\{1,2,K,Q,V\}$).
Тогда
\begin{equation}\label{eq:full_bound_transformer}
    \|\mathbf{H}_{\mathrm{tr}}\|_2 \le
    \sqrt{m_b n_b} \max \limits_{i,j} \left(\frac{2}{L d_V}\|\frac{\partial \mathbf{Z}}{\partial \mathbf{W}_i}\|_2 \|\frac{\partial \mathbf{Z}}{\partial \mathbf{W}_j}\|_2 + \|\mathbf{R}^{\text{tr}}_m\|_2 \|\mathbf{H}_{\text{tr}}^{(i,j)}\|_2 \right).
\end{equation}
\end{theorem}
\begin{proof}
Рассмотрим блоки матрицы Гессе~\eqref{eq:block_hessian_transformer}:
\begin{align}
    \mathbf{H}_{\mathrm{tr}}^{(i,j)} = \big( \mathbf{J}_Z \otimes \mathbf{I}_{n_i} \big) \boldsymbol{\xi}_{ij} + \big( \mathbf{I}_{L d_V} \otimes \mathbf{B}_i^\top \big) \mathbf{H}_Z  \mathbf{B}_j.
\end{align}
Используя свойства норм матриц~\ref{prop:matrix_sum_norm},~\ref{prop:matrix_product_norm} и~\ref{prop:kronecker_product_norm} получаем оценку
\begin{align}
    \big\|\mathbf{H}_{\mathrm{tr}}^{(i,j)}\big\|_2 &\le\big\|\mathbf{J}_Z \otimes \mathbf{I}_{n_i}\big\|_2 \|\boldsymbol{\xi}_{ij}\|_2+\big\|\mathbf{I}_{L d_V} \otimes \mathbf{B}_i^\top\big\|_2 \|\mathbf{H}_Z\|_2 \|\mathbf{B}_j\|_2 = \\
    &= \|\mathbf{J}_Z\|_2 \|\boldsymbol{\xi}_{ij}\|_2+\|\mathbf{B}_i\|_2 \|\mathbf{H}_Z\|_2 \|\mathbf{B}_j\|_2,
\end{align}
описанной в условиях теоремы~\eqref{eq:block_bound_transformer}.

Оценим все слагаемые операторных норм в полученной оценке, а именно норму~$\|\mathbf{B}_i\|_2,$ и норму~$\|\boldsymbol{\xi}_{ij}\|_2,$ которые используются в формуле~\eqref{eq:block_bound_transformer}.
Используя свойствы матричных норм~\ref{prop:matrix_product_norm}, \ref{prop:kronecker_product_norm}, \ref{prop:matrix_sum_norm}, \ref{prop:matrix_norm_inequalities}, \ref{prop:transposed_matrix_norm}, а также определение~\ref{def:commutation_matrix} коммутативных матрицы, получаем, что $\|\mathbf{K}_{m,n}\|_2=1,$ а также согласно свойству~\ref{prop:matrix_norm_inequalities} получаем нормы~$\|\mathrm{vec}_r(\mathbf{I}_{d})\|_2=\|\mathbf{I}_{d}\|_F=\sqrt{d}$ и~$\|\mathbf{I}_p\|_2=1$.
В доказательстве теоремы~\ref{thm:self_attention_hessian_estimation} было доказано, что :
\begin{align}
    \Big\|\frac{\partial \mathbf{A}}{\partial \mathbf{T}}\Big\|_2 &\le \frac{1}{L},\\
    \|\mathbf{Z}_1\|_2 &= \|(\mathbf{I}_L \otimes \mathbf{X}^\top)(\partial \mathbf{A}/\partial \mathbf{T})(\mathbf{X}\otimes \mathbf{X})\|_2\le \|\mathbf{X}\|_2 \frac{1}{L} \|\mathbf{X}\|_2^2= \frac{1}{L}\|\mathbf{X}\|_2^3, \\
    \Big\|\frac{\partial^2 \mathbf{A}}{\partial \mathbf{T}^2}\Big\|_2 &\le 6, \\
    \|\mathbf{Z}_2\|_2 &\le \|\mathbf{X}\|_2^5 \Big\|\frac{\partial^2 \mathbf{A}}{\partial \mathbf{T}^2}\Big\|_2 \le 6 \|\mathbf{X}\|_2^5,\\
    \|\mathbf{A}\|_2 &\le \sqrt{L L}\|\mathbf{A}\|_{\max} = L,
\end{align}
а следовательно согласно свойству~\ref{prop:matrix_product_norm} получаем оценку $\|\mathbf{A}\mathbf{X}\|_2 \le \|\mathbf{A}\|_2 \|\mathbf{X}\|_2 \le L \|\mathbf{X}\|_2.$
Оценим матрицы~$\boldsymbol{\Phi}_{k\ell}$ полученные в рамках леммы~\ref{lemma:attention_phi_from_functional_hessian}.
Используя свойства матричных норм~\ref{prop:matrix_product_norm}, \ref{prop:kronecker_product_norm}, а также верхние оценки на матрицы~$\|\mathbf{Z}_1\|_2$, $\|\mathbf{Z}_2\|_2$ получаем:
\begin{align}
    \|\boldsymbol{\Phi}_{VV}\|_2 &= 0,\\
    \|\boldsymbol{\Phi}_{QQ}\|_2 &\le \frac{2}{L d_V d_K}\|\mathbf{W}_V\|_2 \|\mathbf{W}_K\|_2 \|\mathbf{Z}_2\|_2 \|\mathbf{W}_K\|_2 \le\\
    &\le\frac{12}{L d_V d_K} \|\mathbf{W}_V\|_2 \|\mathbf{W}_K\|_2^2 \|\mathbf{X}\|_2^5,\\
    \|\boldsymbol{\Phi}_{VQ}\|_2 &\le \frac{2}{L d_V \sqrt{d_K}}\|\mathbf{I}_L \otimes \mathbf{S}\|_2  \|\mathbf{Z}_1\|_2  \|\mathbf{I}_{d_V} \otimes \mathbf{W}_K\|_2 \le\\
    &\le\frac{2}{L^2 \sqrt{d_V d_K}} \|\mathbf{W}_K\|_2 \|\mathbf{X}\|_2^3,\\
    \|\boldsymbol{\Phi}_{QK}\|_2 &\le \frac{2}{L d_V d_K} \|\mathbf{W}_V\|_2 \|\mathbf{W}_K\|_2 \|\mathbf{Z}_2\|_2 \|\mathbf{W}_Q\|_2
    + \frac{2}{L d_V \sqrt{d_K}} \|\mathbf{W}_V\|_2  \|\mathbf{Z}_1\|_2  \|\mathbf{S}\|_2 \le \\
    &\le \frac{12}{L d_V d_K} \|\mathbf{W}_V\|_2 \|\mathbf{W}_K\|_2 \|\mathbf{W}_Q\|_2 \|\mathbf{X}\|_2^5
    + \frac{2}{L^2 \sqrt{d_V d_K}} \|\mathbf{W}_V\|_2 \|\mathbf{X}\|_2^3.
\end{align}
Для оценки матричных норм~$\|\mathbf{B}_i\|_2$ рассмотрим чему они равны при разных~$i$ из определения в теоремах~\ref{thm:transformer_hessian}, \ref{theorem:relu_derivative_hessian}. Для матрицы~$\mathbf{B}_1 = (\mathbf{I}_L \otimes \mathbf{W}_2^\top) \mathbf{D}_\sigma (\mathbf{Y} \otimes \mathbf{I}_{d_{ff}}),$ тогда используя свойства матричных норм~\ref{prop:kronecker_product_norm}, \ref{prop:matrix_product_norm}, \ref{prop:transposed_matrix_norm}, а также оценку~$\|\mathbf{D}_\sigma\|_2 \le 1$ получаем:
\begin{equation}\label{eq:B1_norm}
    \|\mathbf{B}_1\|_2 \le \|\mathbf{I}_L \otimes \mathbf{W}_2^\top\|_2 \|\mathbf{D}_\sigma\|_2 \|\mathbf{Y} \otimes \mathbf{I}_{d_{ff}}\|_2
    = \|\mathbf{W}_2\|_2 \|\mathbf{Y}\|_2.
\end{equation}
Для матрицы~$\mathbf{B}_2 = \sigma(\mathbf{Y}\mathbf{W}_1) \otimes \mathbf{I}_{d_V}$ воспользовавшись свойством~\ref{prop:kronecker_product_norm} получаем:
\begin{equation}\label{eq:B2_norm}
    \|\mathbf{B}_2\|_2 = \|\sigma(\mathbf{Y}\mathbf{W}_1)\|_2.
\end{equation}
Для матриц~$\mathbf{B}_k = \mathbf{J}_{SY}\mathbf{J}_Y\mathbf{G}_k,$ где~$k\in\{K,Q,V\},$ используя свойство~\ref{prop:matrix_product_norm}:
\begin{equation}\label{eq:Bk_norm}
    \|\mathbf{B}_k\|_2 \le \|\mathbf{J}_{SY}\|_2  \|\mathbf{J}_Y\|_2  \|\mathbf{G}_k\|_2.
\end{equation}
Для матрицы~$\mathbf{J}_{SY} = (\mathbf{I}_L \otimes \mathbf{W}_2^\top)\mathbf{D}_\sigma(\mathbf{I}_L \otimes \mathbf{W}_1^\top) + (\mathbf{I}_L \otimes \mathbf{I}_{d_V})$ используя свойства~\ref{prop:matrix_sum_norm}, \ref{prop:matrix_product_norm}, \ref{prop:kronecker_product_norm}, \ref{prop:transposed_matrix_norm}, а также оценку~$\|\mathbf{D}_\sigma\|_2 \le 1$ получаем оценку матричной нормы:
\begin{align}\label{eq:JSY_norm}
    \|\mathbf{J}_{SY}\|_2 &\le \|\mathbf{I}_L \otimes \mathbf{W}_2^\top\|_2 \|\mathbf{D}_\sigma\|_2 \|\mathbf{I}_L \otimes \mathbf{W}_1^\top\|_2 + \|\mathbf{I}_L \otimes \mathbf{I}_{d_V}\|_2 =\\
    &=\|\mathbf{W}_2\|_2 \|\mathbf{W}_1\|_2 + 1.
\end{align}
Для матриц~$\|\mathbf{G}_V\|_2,\|\mathbf{G}_Q\|_2,\|\mathbf{G}_K\|_2,$ используя свойства~\ref{prop:matrix_product_norm}, \ref{prop:kronecker_product_norm} получаем оценки на нормы:
\begin{align}\label{eq:Gk_bounds}
    \|\mathbf{G}_V\|_2 &\le L \|\mathbf{X}\|_2,\\
    \|\mathbf{G}_Q\|_2 &\le \frac{1}{L\sqrt{d_K}} \|\mathbf{W}_V\|_2 \|\mathbf{W}_K\|_2 \|\mathbf{X}\|_2^3, \\
    \|\mathbf{G}_K\|_2 &\le \frac{1}{L\sqrt{d_K}} \|\mathbf{W}_V\|_2 \|\mathbf{W}_Q\|_2 \|\mathbf{X}\|_2^3.
\end{align}
Для оценки матричных норм~$\|\boldsymbol{\xi}_{ij}\|_2,$ рассмотрим чему они равны из определения в теореме~\ref{thm:transformer_hessian}.
В случае пар FFN для матриц~$\|\boldsymbol{\xi}_{11}\|_2,\|\boldsymbol{\xi}_{12}\|_2,\|\boldsymbol{\xi}_{21}\|_2,\|\boldsymbol{\xi}_{22}\|_2$ используя свойства~\ref{prop:kronecker_product_norm}, \ref{prop:matrix_product_norm}, \ref{prop:matrix_norm_inequalities} матричных норм, а также свойство коммутативных матриц~$\|\mathbf{K}_{m,n}\|_2=1$ получаем оценки:
\begin{align}\label{eq:xiffn}
    \|\boldsymbol{\xi}_{11}\|_2 &= 0,\\
    \|\boldsymbol{\xi}_{22}\|_2 &= 0,\\
    \|\boldsymbol{\xi}_{12}\|_2 &\le \|\mathbf{I}_L \otimes \mathbf{K}_{d_V, d_{ff}} \otimes \mathbf{I}_{d_V}\|_2  \|\mathbf{I}_{L d_{ff}} \otimes \mathrm{vec}_r(\mathbf{I}_{d_V})\|_2  \|\mathbf{D}_\sigma\|_2  \|\mathbf{Y}\otimes \mathbf{I}_{d_{ff}}\|_2 \nonumber\\
    &= 1 \cdot \|\mathrm{vec}_r(\mathbf{I}_{d_V})\|_2 \cdot 1 \cdot \|\mathbf{Y}\|_2
    = \sqrt{d_V} \|\mathbf{Y}\|_2, \\
    \|\boldsymbol{\xi}_{21}\|_2 &\le \|\mathbf{I}_L \otimes \mathbf{W}_2^\top\|_2  \|\mathbf{D}_\sigma\|_2  \|\mathbf{I}_L \otimes \mathbf{K}_{d_{ff}, d_V} \otimes \mathbf{I}_{d_{ff}}\|_2  \|\mathbf{I}_{L d_V} \otimes \mathrm{vec}_r(\mathbf{I}_{d_{ff}})\|_2 \nonumber\\
    &= \|\mathbf{W}_2\|_2 \cdot 1 \cdot 1 \cdot \|\mathrm{vec}_r(\mathbf{I}_{d_{ff}})\|_2 = \sqrt{d_{ff}} \|\mathbf{W}_2\|_2.
\end{align}
В случае пар FFN с параметрами слоев внимания для всех~$k\in\{K,Q,V\}$ получаем оценки:
\begin{align}\label{eq:xiffnk}
    \|\boldsymbol{\xi}_{1k}\|_2 &\le \|(\mathbf{I}_L \otimes \mathbf{W}_2^\top)\mathbf{D}_\sigma \otimes \mathbf{I}_{n_k}\|_2 \|\mathbf{I}_L \otimes \mathbf{K}_{d_{ff}, d_V} \otimes \mathbf{I}_{d_{ff}}\|_2 \cdot\\
    &\quad\cdot\|\mathbf{I}_{L d_V} \otimes \mathrm{vec}_r(\mathbf{I}_{d_{ff}})\|_2 \|\mathbf{J}_Y\|_2 \|\mathbf{G}_k\|_2 \nonumber\\
    &\le \|\mathbf{W}_2\|_2 \cdot 1 \cdot 1 \cdot \sqrt{d_{ff}} \cdot\|\mathbf{J}_Y\|_2 \|\mathbf{G}_k\|_2 =\\
    &= \sqrt{d_{ff}} \|\mathbf{W}_2\|_2 \|\mathbf{J}_Y\|_2 \|\mathbf{G}_k\|_2, \\
    \|\boldsymbol{\xi}_{2k}\|_2 &\le \|\mathbf{I}_L \otimes \mathbf{K}_{d_V, d_{ff}} \otimes \mathbf{I}_{d_V}\|_2 \|\mathbf{I}_{L d_{ff}} \otimes \mathrm{vec}_r(\mathbf{I}_{d_V})\|_2 \|\mathbf{D}_\sigma\|_2  \|\mathbf{I}_L \otimes \mathbf{W}_1^\top\|_2 \cdot\\
    &\quad\cdot \|\mathbf{J}_Y\|_2 \|\mathbf{G}_k\|_2 \nonumber\\
    &\le 1 \cdot \sqrt{d_V} \cdot 1 \cdot \|\mathbf{W}_1\|_2 \cdot \|\mathbf{J}_Y\|_2 \cdot \|\mathbf{G}_k\|_2 =\\
    &= \sqrt{d_V} \|\mathbf{W}_1\|_2 \|\mathbf{J}_Y\|_2 \|\mathbf{G}_k\|_2.
\end{align}
Для пар слоев внимания~$k,\ell\in\{K,Q,V\}$
\begin{align}
    \boldsymbol{\xi}_{k\ell}
    = \big(\mathbf{J}_{SY} \otimes \mathbf{I}_{n_k}\big)
    \Big[ \big(\mathbf{I}_{L d_V} \otimes \mathbf{G}_k^\top\big) (\mathbf{H}_Y \mathbf{G}_\ell)+ \big(\mathbf{J}_Y \otimes \mathbf{I}_{n_k}\big) \boldsymbol{\Phi}_{k\ell}\Big],
\end{align}
используя свойства~\ref{prop:matrix_product_norm}, \ref{prop:kronecker_product_norm} получаем следующие оценки:
\begin{align}\label{eq:xikell}
    \|\boldsymbol{\xi}_{k\ell}\|_2 &\le \|\mathbf{J}_{SY}\|_2 \Big( \|\mathbf{I}_{L d_V} \otimes \mathbf{G}_k^\top\|_2 \|\mathbf{H}_Y\|_2 \|\mathbf{G}_\ell\|_2
    + \|\mathbf{J}_Y\|_2 \|\boldsymbol{\Phi}_{k\ell}\|_2 \Big) =\\
    &=\|\mathbf{J}_{SY}\|_2 \Big( \|\mathbf{G}_k\|_2 \|\mathbf{H}_Y\|_2 \|\mathbf{G}_\ell\|_2
    + \|\mathbf{J}_Y\|_2 \|\boldsymbol{\Phi}_{k\ell}\|_2 \Big).
\end{align}
Итого собирая все части выражения~\eqref{eq:block_bound_transformer}, используя для каждой пары~$(i,j)$ оценки норм матриц~$\|\boldsymbol{\xi}_{ij}\|_2$ с выражений~\eqref{eq:xiffn},\eqref{eq:xiffnk},\eqref{eq:xikell}, а также оценки норм матриц~$\|\mathbf{B}_i\|_2$ с выражений \eqref{eq:B1_norm},\eqref{eq:Bk_norm},\eqref{eq:JSY_norm},\eqref{eq:Gk_bounds} и подставляя в выражение~\eqref{eq:block_bound_transformer} получаем следующие оценки норм на все блоки матрицы Гесее:
\begin{align}
    \big\|\mathbf{H}_{\mathrm{tr}}^{(1,1)}\big\|_2
    &\le \|\mathbf{J}_Z\|_2 \cdot 0 + \|\mathbf{B}_1\|_2^2 \|\mathbf{H}_Z\|_2 \le\\
    &\le \|\mathbf{H}_Z\|_2 (\|\mathbf{W}_2\|_2 \|\mathbf{Y}\|_2)^2,\\
    \big\|\mathbf{H}_{\mathrm{tr}}^{(1,2)}\big\|_2 &\le \|\mathbf{J}_Z\|_2  \sqrt{d_V}\|\mathbf{Y}\|_2 + \|\mathbf{H}_Z\|_2  (\|\mathbf{W}_2\|_2 \|\mathbf{Y}\|_2) \|\sigma(\mathbf{Y}\mathbf{W}_1)\|_2,\\
    \big\|\mathbf{H}_{\mathrm{tr}}^{(1,k)}\big\|_2 &\le \|\mathbf{J}_Z\|_2 \sqrt{d_{ff}}\|\mathbf{W}_2\|_2 \|\mathbf{J}_Y\|_2 \|\mathbf{G}_k\|_2 +\\
    &\quad+\|\mathbf{H}_Z\|_2  (\|\mathbf{W}_2\|_2 \|\mathbf{Y}\|_2)  (\|\mathbf{J}_{SY}\|_2 \|\mathbf{J}_Y\|_2 \|\mathbf{G}_k\|_2),\\
    \big\|\mathbf{H}_{\mathrm{tr}}^{(k,\ell)}\big\|_2
    &\le \|\mathbf{J}_Z\|_2 \|\mathbf{J}_{SY}\|_2 \Big( \|\mathbf{G}_k\|_2 \|\mathbf{H}_Y\|_2 \|\mathbf{G}_\ell\|_2 + \|\mathbf{J}_Y\|_2 \|\boldsymbol{\Phi}_{k\ell}\|_2 \Big) +\\
    &\quad + \|\mathbf{H}_Z\|_2  (\|\mathbf{J}_{SY}\|_2 \|\mathbf{J}_Y\|_2 \|\mathbf{G}_k\|_2)  (\|\mathbf{J}_{SY}\|_2 \|\mathbf{J}_Y\|_2 \|\mathbf{G}_\ell\|_2),
\end{align}

В оценке матричной норм~$\|\mathbf{Y} \|_2$ and $\| \mathbf{S}\|_2$ были использованы результаты леммы~\ref{lemma:Y_S_norm_bounds}. Нормы матриц~$\mathbf{H}_Z,\mathbf{H}_Y$ оцениваются в рамках леммы~\ref{lemma:layernorm_deriv_hessian_norm}.
\end{proof}

% Третья глава
\clearpage
\chapter{Оценка достаточно объема выборки}
В главе~\ref{chapter:complexity} был разработан единый теоретический аппарат для формализации соотношения между сложностью модели и сложностью данных. Ключевым результатом этой главы является введение формальных определений меры сложности выборки $\mu_D(D)$ и меры сложности модели $\mu_f(f)$, а также установление критерия обучаемости модели на выборке: $\mu_f(f) \leq \mu_D(D)$. 

В разделе~\ref{chapter:complexity:sample-size} главы~\ref{chapter:complexity} было показано, что частным случаем условной меры сложности данных $\mu_D(D|f)$ является достаточный размер выборки $m^*$~--- минимальный объем данных из выборки $D$, необходимый для обучения модели $f$. Для простой генеральной совокупности $\Gamma_C$, состоящей из объектов одинаковой сложности $C$, мера сложности выборки принимает вид $\mu_D(D) = C \cdot |D|$, что устанавливает прямую связь между размером выборки и ее сложностью.

В разделе~\ref{chapter:complexity:loss} главы~\ref{chapter:complexity} была введена ландшафтная мера сложности модели $\mu_f(f|D)$, определяемая через спектральные свойства матриц Гессе функции потерь. Установлено, что анализ сходимости ландшафта оптимизационной задачи при увеличении объема выборки сводится к анализу спектральной нормы матрицы Гессе, что позволяет количественно оценить влияние добавления новых объектов данных на локальную геометрию функции потерь в окрестности оптимума.

В главе~\ref{chapter:gesian} были получены конкретные оценки спектральных норм матриц Гессе для различных архитектур нейронных сетей. Теорема~\ref{theorem:hess-kiselev-theorem} устанавливает оценку для полносвязных сетей: $\|\mathbf{H}_i(\boldsymbol{\theta})\|_2 \propto L(hM)^{2L}$, теоремы~\ref{thm:1Dconv} и~\ref{thm:2Dconv}~--- для сверточных сетей, а теорема~\ref{thm:transformer_hessian_estimate}~--- для трансформерных моделей. Эти оценки обеспечивают вычислимо осуществимые методы оценки ландшафтной меры сложности модели $\mu_f(f|D)$, что создает теоретическую основу для практического применения формализма, разработанного в главе~\ref{chapter:complexity}.

Однако для практического использования введенного теоретического аппарата необходимо решить следующую задачу: как на основе имеющихся данных и выбранной модели определить конкретное численное значение достаточного размера выборки $m^*$? Теоретические результаты предыдущих глав устанавливают формальные связи между сложностью модели и сложностью данных, но не предоставляют алгоритмических процедур для вычисления $m^*$ в конкретных прикладных задачах. Более того, для моделей глубокого обучения прямое вычисление матриц Гессе и оценка ландшафтной меры сложности остаются вычислительно дорогостоящими, что ограничивает практическую применимость теоретических результатов.

В рамках настоящей главы рассматриваются различные методы определения достаточного размера выборки для различных моделей, от линейных до моделей глубокого обучения. Предлагаемые методы опираются на теоретический аппарат, разработанный в предыдущих главах, и обеспечивают практические инструменты для оценки необходимого объема данных при планировании экспериментов. В отличие от теоретических оценок, основанных на анализе матриц Гессе, предлагаемые методы используют наблюдаемые характеристики процесса обучения для определения момента, когда добавление новых объектов данных перестает существенно влиять на свойства модели.

Планирование эксперимента требует оценки минимального размера выборки: числа выполненных измерений набора характеристик, необходимых для построения сформулированных условий. Выбор метода оценки размера выборки зависит от решаемой задачи, которая определяет формулировку статистической гипотезы и статистики для ее проверки. 

В целом существуют различные подходы к определению достаточного размера выборки, такие как статистические, байесовские и эвристические методы. Каждый из этих подходов имеет свои преимущества и ограничения, которые определяют область их применимости.

Статистические методы предполагают, что выборка соответствует некоторым предварительным условиям, сформулированным ранее. Эти условия сформулированы как статистический критерий~\cite{self1988,self1992,shieh2000,demidenko2007}. Метод оценки размера выборки, связанный с этим критерием, гарантирует достижение фиксированной статистической мощности~$1-\beta$ со степенью ошибки первого рода, не превышающей установленное значение~$\alpha$. Такой размер выборки называется достаточным.

Однако практическое применение методов оценки размера выборки предполагает, что модель соответствует измеренным данным~\cite{kloek1975}. Эти модели выбираются в соответствии с постановкой задачи регрессии или классификации. В настоящей главе рассматриваются обобщенные линейные модели.

В работе~\cite{self1992} предложен подход к оценке мощности и размера выборки на основе теста отношения максимального правдоподобия. Этот подход оказался более точным для ряда независимых переменных. В работе~\cite{shieh2005} предложен метод оценки мощности для статистики Вальда. В работе~\cite{motrenko2014} в случае логистической регрессии предлагается использовать метод, использующий кривую ROC-AUC и концепцию сдвига.

Классические методы~\cite{self1988,self1992,shieh2000,shieh2005,demidenko2007} имеют ряд ограничений, связанных с практическим применением. Чтобы оценить размер выборки, необходимо знать дисперсию оценки параметра или, в более общем случае, иметь оценку параметра нецентральности в распределении статистики, используемой при альтернативной гипотезе. Указанные методы не предоставляют алгоритмических процедур для получения этих значений. Кроме того, дисперсия оценки и параметр нецентральности оцениваются с неопределенностью, влияние которой на результат оценки размера выборки не учитывается.

Статистические методы позволяют оценить размер выборки на основе предположений о распределении данных и информации о соответствии между наблюдаемыми значениями и предположениями нулевой гипотезы.

Когда размер исследуемой выборки является достаточным или чрезмерным, возможно применение методов, основанных на наблюдении изменения определенной характеристики процедуры построения модели при увеличении размера выборки. В частности, наблюдая за соотношением качества прогнозирования с контрольной выборкой и обучающей выборкой~\cite{motrenko2014}, определяется достаточный размер выборки, который соответствует началу переобучения.

В работе~\cite{qumsiyeh2013} для оценки достаточного размера выборки используется процедура бутстрапа. Превышение текущего размера выборки проверяется на основе анализа доверительных интервалов оцениваемого параметра. Ширина доверительного интервала с разными значениями объема выборки оценивается с помощью метода бутстрапа. Для этого выборки меньшего размера отбираются заданное число раз и вычисляется доверительный интервал ошибки при оценке параметра модели. Размер выборки считается достаточным, если ширина доверительного интервала не превышает заранее установленного значения.

Перечисленные выше ограничения статистических методов оценки размера выборки подробно исследуются в байесовской процедуре~\cite{lindley1997,rubin1998,wang2002}. В рамках данного подхода оценка размера выборки определяется на основе максимизации ожидаемого значения некоторой функции качества~\cite{lindley1997}. Функция качества может включать в себя явные функции распределения параметров и штрафы за увеличение размера выборки.

Альтернативой подходам~\cite{wang2002}, основанным на функции качества, является выбор размера выборки путем установления ограничений на определенный критерий качества оценки параметров модели. Примеры критериев: критерий средней апостериорной дисперсии (AVPC), критерий средней длины (ALC), критерий среднего покрытия (ACC). Для каждого перечисленного критерия оценка размера выборки определяется как минимальное значение размера выборки, для которого ожидаемое значение выбранного критерия не превышает какого-либо фиксированного порога.

В работе~\cite{motrenko2014} предлагается считать размер выборки достаточным, если расстояние Кульбака-Лейблера между распределениями, оцененными на основе подвыборок такого размера, достаточно мало. Такой подход не требует дальнейшего обобщения в случае нескольких переменных. Кроме того, оценка может производиться как при наличии предположений о распределении данных, так и при их отсутствии. Недостаток этого подхода заключается в том, что количественная оценка может быть получена только при чрезмерно большом размере выборки.

\section{Статистические методы определения достаточного размера выборки}

В настоящем разделе рассматриваются статистические методы определения достаточного размера выборки для обобщенных линейных моделей. Основой данных методов является использование информационной матрицы Фишера и статистических критериев для проверки гипотез о параметрах модели.

Рассмотрим выборку размера~$m$:
\[
\label{eq:ps:1}
\begin{aligned}
	\mathfrak{D}_{m} = \{\textbf{x}_i, y_i\}_{i = 1}^{m},
\end{aligned}
\]
где~$\textbf{x}_i\in \mathbb{R}^{n}$~--- вектор признаков,~$y_i\in \mathbb{Y}$~--- целевая переменная. Вектор признаков~$\textbf{x} = [\textbf{u}, \textbf{v}]$ объединяет~$\textbf{u}_i\in \mathbb{R}^{k}$ и~$\textbf{v}_i\in \mathbb{R}^{n-k}$.
Выборка~$\mathfrak{D}_{m}$ случайным образом разделяется на обучающую и тестовую части:
\[
\label{eq:ps:2}
\begin{aligned}
	\mathfrak{D}_{\mathcal{T}_{m}} = \{\textbf{x}_i, \textbf{y}_i\}_{i \in \mathcal{T}_{m}}, \quad \mathfrak{D}_{\mathcal{L}_{m}} = \{\textbf{x}_i, \textbf{y}_i\}_{i \in \mathcal{L}_{m}}, \quad  \mathcal{T}_{m}\sqcup\mathcal{L}_{m} = \{1, ..., m\}.
\end{aligned}
\]
Определим параметрическое семейство функций для аппроксимации неизвестного распределения~$p(y|\textbf{x}, \mathfrak{D}_{\mathcal{L}_{m}})$:
\[
\label{eq:ps:3}
\begin{aligned}
	\mathfrak{F} = \left\{f\left(y,\textbf{x}, \textbf{w}\right)|\textbf{w}\in\mathbb{W}, \int_{y\in \mathbb{Y}, \textbf{x}\in\mathbb{R}^{n}}f\left(y, \textbf{x}, \textbf{w}\right)dyd\textbf{x}=1\right\}.
\end{aligned}
\]

Для модели~$f$ с вектором параметров~$\textbf{w}$ определим функцию правдоподобия и логарифм функции правдоподобия выборки~$\mathfrak{D}$:
\[
\label{eq:ps:4}
\begin{aligned}
	L\left(\mathfrak{D}, \textbf{w}\right) = \prod f\left(y,\textbf{x}, \textbf{w}\right),\quad l\left(\mathfrak{D}, \textbf{w}\right) = \sum \log f\left(y,\textbf{x}, \textbf{w}\right),
\end{aligned}
\]
где~$f(y,\textbf{x}, \textbf{w})$ является оценкой плотности вероятности для объекта $(y, \textbf{x})$ при заданном векторе параметров~$\textbf{w}$.

Используя принцип максимального правдоподобия для оценки параметров~$\textbf{w}$:
\[
\label{eq:ps:5}
\begin{aligned}
	\hat{\textbf{w}} = \arg\max_{\textbf{w}\in\mathbb{W}}L\left(\mathfrak{D}_{\mathcal{L}}, \textbf{w}\right).
\end{aligned}
\]

Информационная матрица Фишера (англ. Fisher Information Matrix) имеет вид:
\[
\label{eq:ps:6}
\begin{aligned}
	\textbf{I}\left(\mathfrak{D}, \textbf{w}\right) = -\nabla\nabla^{\mathsf{T}}l\left(\mathfrak{D}, \textbf{w}\right), \quad  \textbf{V} = \textbf{I}^{-1}\left(\mathfrak{D}, \textbf{m}\right),
\end{aligned}
\]
где~$\textbf{V}$~--- ковариационная матрица оценок параметров. Статистические методы и байесовские методы используют информационную матрицу Фишера для оценки размера выборки.

Основным преимуществом методов, основанных на статистике, является их способность оценивать достаточный размер выборки при недостаточном наборе данных. Они позволяют прогнозировать необходимое число объектов на ранней стадии эксперимента.

Рассмотрим обобщенную линейную модель, в которой плотность распределения целевой переменной задается выражением
\[
\label{eq:sb:1}
\begin{aligned}
	p(y|\textbf{u},\textbf{v},\textbf{w}_{u},\textbf{w}_{v}) = \exp\bigl(y\theta- b(\theta) + c\left(y\right)\bigr),
\end{aligned}
\]
где~$\theta$ является каноническим параметром распределения, получаемым с помощью функции связи~$\theta=\theta\bigr(\textbf{u},\textbf{v},\textbf{w}_{u},\textbf{w}_{v}\bigr)$, а функции~$b(\theta)$ и~$c(y)$ определяют конкретный тип распределения.

Тестируемая гипотеза
\[
\label{eq:sb:2}
\begin{aligned}
	H_0: \textbf{m}_{u} = \textbf{m}^0_{u}, \quad H_1: \textbf{m}_{u} \not= \textbf{m}^0_{u}.
\end{aligned}
\]

Пусть статистики~$S_{m,u}\left(\textbf{w}_{u}, \textbf{w}_{v}\right)$ и~$S_{m,v}\left(\textbf{w}_{u}, \textbf{w}_{v}\right)$ представляют собой производные логарифма правдоподобия выборки~$\mathfrak{D}_{m}$ по параметрам~$\textbf{w}_{u}$ и~$\textbf{w}_{v}$ соответственно.
Рассмотрим~$\textbf{s}_{m} = S_{m,u}\left(\textbf{m}^{0}_{u}, \hat{\textbf{w}}^{0}_{v}\right)$, где~$\hat{\textbf{w}}^{0}_{v}$ получается из уравнения
\[
\label{eq:sb:3}
\begin{aligned}
	S_{m,v}\left(\textbf{m}^{0}_{u}, \textbf{w}_{v}\right) = 0.
\end{aligned}
\]
Статистика множителей Лагранжа (англ. Lagrange Multiplier) определяется как
\[
\label{eq:sb:4}
\begin{aligned}
	LM = \textbf{s}^{\mathsf{T}}_{m}\textbf{Q}_{m}^{-1}\textbf{s}_{m},
\end{aligned}
\]
где~$\textbf{Q}_{m}$~--- ковариационная матрица вектора~$\textbf{s}_{m}$.
	
В случае истинности гипотезы~$H_0$ статистика~$LM$ асимптотически имеет центральное распределение~$\chi^2(k)$. В~\cite{self1988} показано, что при альтернативной гипотезе~$H_1$ статистика~$LM$ асимптотически имеет нецентральное распределение~$\chi^2(k,\gamma)$, где~$\gamma$ является параметром нецентральности
\[
\label{eq:sb:5}
\begin{aligned}
	\gamma = \bm{\xi}_{m}^{\mathsf{T}}\bm{\Sigma}^{-1}_{m}\bm{\xi}_{m} = m\bm{\xi}^{\mathsf{T}}\bm{\Sigma}^{-1}\bm{\xi}= m\gamma^0,
\end{aligned}
\]
где~$\bm{\xi}_{m}$ и~$\bm{\Sigma}_{m}$~--- соответственно вектор математического ожидания и матрица ковариации~$\textbf{s}_{m}$. Обозначим~$\bm{\xi}_1 = \bm{\xi}$, ~$\bm{\Sigma}_1 = \bm{\Sigma}$. 
	
Альтернативный метод получения~$\gamma$ включает условия на уровне значимости~$\alpha$ и вероятность ошибки II рода~$\beta$:
\[
\label{eq:sb:6}
\begin{aligned}
	\gamma^*:\chi^2_{k, 1-\alpha} = \chi^2_{k, \beta}\left(\gamma\right).
\end{aligned}
\]
Используя соотношения~\eqref{eq:sb:5} и~\eqref{eq:sb:6}, получаем
\[
\label{eq:sb:7}
\begin{aligned}
	m^* = \frac{\gamma^*}{\gamma^0}.
\end{aligned}
\]
Полученное значение~$m^*$ представляет собой достаточный минимальный размер выборки, необходимый для различения вектора~$\textbf{m}_{u}$ от~$\textbf{m}^0_{u}$ с заданными уровнями значимости~$\alpha$ и мощности~$1-\beta$.

Рассмотрим случай, когда правдоподобие выборки задается выражением
\[
\label{eq:sb:8}
\begin{aligned}
	p(y|\textbf{u},\textbf{v},\textbf{w}_{u},\textbf{w}_{v}) = \exp\left(\frac{y\theta- b(\theta)}{a(\phi)} + c\left(y, \phi\right)\right),
\end{aligned}
\]
где~$\theta$ является параметром распределения, который вычисляется с помощью функции связи~$\theta=\theta\bigr(\textbf{u},\textbf{v},\textbf{w}_{u},\textbf{w}_{v}\bigr)$.

Тестируемая гипотеза
\[
\label{eq:sb:9}
\begin{aligned}
	H_0: \textbf{m}_{u} = \textbf{m}^0_{u}, \quad H_1: \textbf{m}_{u} \not= \textbf{m}^0_{u}.
\end{aligned}
\]
Определим логарифм статистики отношения правдоподобий:
\[
\label{eq:sb:10}
\begin{aligned}
	LR = 2\Big(l\left(\mathfrak{D}, \hat{\textbf{w}}\right) - l\left(\mathfrak{D}, \hat{\textbf{w}}^0\right)\Big),
\end{aligned}
\]
где~$\hat{\textbf{w}} = [\hat{\textbf{w}}_{u},\hat{\textbf{w}}_{v}]$~--- вектор параметров, максимизирующий правдоподобие \eqref{eq:sb:8}, а~$\hat{\textbf{w}}^{0} = [\textbf{m}^{0}_{u},\hat{\textbf{w}}^{0}_{v}]$~--- вектор параметров, максимизирующий правдоподобие \eqref{eq:sb:8} при фиксированном подвекторе параметров~$\textbf{m}^{0}_{u}$.

В случае истинности гипотезы~$H_0$ статистика~$LR$ асимптотически имеет центральное распределение~$\chi^2(k)$. В~\cite{shieh2000} показано, что при альтернативной гипотезе~$H_1$ статистика~$LR$ асимптотически имеет нецентральное распределение~$\chi^2(k,\gamma)$, где~$\gamma$ является параметром нецентральности
\[
\label{eq:sb:11}
\begin{aligned}
	\gamma = m\Delta^*, \quad \Delta^* = \mathsf{E}\left[2a^{-1}(\phi)\left\{\left(\theta - \theta^*\right)\nabla b(\theta) - b(\theta) + b(\theta^*)\right\}\right], 
\end{aligned}
\]
где параметры~$\theta$ и~$\theta^*$ рассчитываются с использованием параметров~$\textbf{w} = [\textbf{w}_{u}, \textbf{w}_{v}]$ и~$\textbf{w}^* = [\textbf{w}^{0}_{u}, \textbf{w}^{*}_{v}]$. Параметры~$\textbf{w}^{*}_{v}$ вычисляются на основе решения уравнения
\[
\label{eq:sb:12}
\begin{aligned}
	\lim_{m\to\infty}m^{-1}\mathsf{E}\left(\frac{\partial l\left(\mathfrak{D}, \left[\textbf{m}^{0}_{u}, \textbf{w}_{v}\right]\right)}{\partial \textbf{w}_{v}}\right) = 0.
\end{aligned}
\]
	
Тогда с учетом~$\alpha$ и~$\beta$ достаточный размер выборки~$m^*$ вычисляется
\[
\label{eq:sb:13}
\begin{aligned}
	m^* = \frac{\gamma^*}{\Delta^*}, \quad \gamma^*:\chi^2_{k, 1-\alpha} = \chi^2_{k, \beta}\left(\gamma\right), 
\end{aligned}
\]
где~$\chi^2_{k, 1-\alpha}$ и~$\chi^2_{k, \beta}\left(\gamma^*\right)$~--- квантили распределений~$\chi^{2}_k$ и~$\chi^2_{k}\left(\gamma^*\right)$ соответственно.
Правдоподобие выборки:
\[
\label{eq:sb:14}
\begin{aligned}
	p(y|\textbf{u},\textbf{v},\textbf{w}_{u},\textbf{w}_{v}) = \exp\left(\frac{y\theta- b(\theta)}{a(\phi)} + c\left(y, \phi\right)\right),
\end{aligned}
\]
где~$\theta$ является параметром распределения, который вычисляется с помощью функции связи~$\theta=\theta\bigr(\textbf{u},\textbf{v},\textbf{w}_{u},\textbf{w}_{v}\bigr)$.

Тестируемая гипотеза:
\[
\label{eq:sb:15}
\begin{aligned}
	H_0: \textbf{m}_{u} = \textbf{m}_{u}^{0}, \quad H_1: \textbf{m}_{u} \not=\textbf{m}_{u}^{0}.
\end{aligned}
\]
Тест Вальда для гипотезы:
\[
\label{eq:sb:16}
\begin{aligned}
	W = \left(\hat{\textbf{w}}_{u} - \textbf{m}_{u}^{0}\right)^{\mathsf{T}}\hat{\textbf{V}}_{u}^{-1}\left(\hat{\textbf{w}}_{u} - \textbf{m}_{u}^{0}\right),
\end{aligned}
\]
где~$\hat{\textbf{w}} = [\hat{\textbf{w}}_{u},\hat{\textbf{w}}_{v}]$ вектор параметров, который максимизирует правдоподобие выборки \eqref{eq:sb:14}, где матрица~$\hat{\textbf{V}}_u$ задается в выражении \eqref{eq:ps:6}.

В случае истинности гипотезы~$H_0$ статистика Вальда~$W$ асимптотически имеет центральное распределение~$\chi^2(k)$. В~\cite{shieh2005} показано, что в случае истинности альтернативной гипотезы~$H_1$ статистика Вальда~$W$ асимптотически имеет нецентральное распределение~$\chi^2(k,\gamma)$ с параметром нецентральности~$\gamma$:
\[
\label{eq:sb:17}
\begin{aligned}
	\gamma = m\delta, \quad \delta = \left(\hat{\textbf{w}}_{u} - \textbf{m}_{u}^{0}\right)^{\mathsf{T}}\bm{\Sigma}^{-1}_u\left(\hat{\textbf{w}}_{u} - \textbf{m}_{u}^{0}\right), \quad \bm{\Sigma}_u = m\hat{\textbf{V}}_u.
\end{aligned}
\]

Используя заданный уровень значимости~$\alpha$ и заданную ошибку второго рода~$\beta$, определим оптимальный размер выборки:
\[
\label{eq:sb:18}
\begin{aligned}
	m^* = \frac{\gamma^*}{\delta}, \quad \gamma^*:\chi^2_{k, 1-\alpha^{*}} = \chi^2_{k, \beta}\left(\gamma\right),
\end{aligned}
\]
где~$\chi^2_{k, 1-\alpha^*}$ и~$\chi^2_{k, \beta}\left(\gamma^*\right)$~--- квантили соответствующих распределений, а параметр~$\alpha^*$ представляет собой поправку на уровень значимости:
\[
\label{eq:sb:19}
\begin{aligned}
	\alpha^* = P\left(\bm{\xi}^{\mathsf{T}}\bm{\Sigma}^{*-1} \bm{\xi} > \chi^2_{k,1 - \alpha}\right), \quad \Sigma^* = \textbf{I}^{-1}\left(\mathfrak{D}, \textbf{w}^*\right),
\end{aligned}
\]
где~$\textbf{w}^{*} = \left[\textbf{m}_{u}^{0}, \textbf{w}^{*}_v\right]$ представляет собой решение уравнения
\[
\label{eq:sb:20}
\begin{aligned}
	\lim_{m\to\infty}m^{-1}\mathsf{E}\left(\frac{\partial l\left(\mathfrak{D}, \left[ \textbf{m}_{u}^{0}, \textbf{w}_{v}\right]\right)}{\partial \textbf{w}_{v}}\right) = 0.
\end{aligned}
\]

Статистические методы, рассмотренные выше, требуют знания дисперсии оценки параметра или параметра нецентральности, что ограничивает их практическое применение. В следующем разделе рассматриваются эвристические методы, которые не требуют таких предположений и могут применяться в более широком классе задач.

\section{Эвристические методы определения достаточного размера выборки}

В настоящем разделе рассматриваются эвристические методы определения достаточного размера выборки, основанные на популярных статистических эвристиках, таких как бутстрап, перекрестная проверка и задание функции полезности. В отличие от статистических методов, эвристические подходы не требуют строгих предположений о распределении данных и могут применяться в ситуациях, когда теоретические гарантии недоступны.

Определим набор индексов~$\mathcal {A}~$ для параметров логистической регрессии~$\textbf {w}~$. Тестируется гипотеза
\[
\label{eq:hb:1}
\begin{aligned}
	H_0: j \not\in\mathcal{A} \left(\text{w}_{j} = 0\right), \quad H_1: j \in \mathcal{A}^* \left(\text{w}_{j} \not= 0\right),
\end{aligned}
\]
где~$\text{w}_{j}$ является~$j$-м элементом вектора~$\textbf{w}$.
Установим параметр отступа~$ c_0~$ для задачи логистической регрессии:
\[
\label{eq:hb:2}
\begin{aligned}
	H_0: 1-c_0 = p_0, \quad H_1: 1-c_0 = p_1,
\end{aligned}
\]
где~$c_0$ оптимальное решение, когда исключен~$j$-й элемент вектора.
Используя статистику
\[
\label{eq:hb:3}
\begin{aligned}
	Z = \frac{\hat{p}-p_0}{\sqrt{p_0(1-p_0)}}\sqrt{m}, \quad \hat{p} = \frac{1}{m}\sum_{i=1}^{m}y_i.
\end{aligned}
\]
В случае истинности нулевой гипотезы~$H_0$ статистика~$Z$ асимптотически имеет распределение~$\mathcal{N}\left(0, 1\right)$. В случае истинности альтернативной гипотезы~$H_1$ статистика~$Z$  асимптотически имеет распределение~$ \mathcal{N}\left(p_1-p_0, \sqrt{\frac{p_1(1-p_1)}{p_0(1-p_0)}}\right)$.
      
Достаточный объем выборки задается выражением
\[
\label{eq:hb:4}
\begin{aligned}
	m^* = \frac{p_0(1-p_0)\left(Z_{1-\alpha/2} + Z_{1-\beta}\sqrt{\frac{p_1(1-p_1)}{p_0(1-p_0)}}\right)^2}{(p_1-p_0)^2},
\end{aligned}
\]
где~$Z_{1-\alpha/2}$ и~$Z_{1-\beta}$~--- квантили стандартного нормального распределения~$\mathcal{N}\left(0, 1\right)$.
    
Данный метод не рассматривается далее, поскольку его можно использовать только в задаче логистической регрессии.

Рассмотрим метод на основе кросс-валидации (англ. cross-validation). Определим критерий переобучения как
\[
\label{eq:hb:5}
\begin{aligned}
	RS(m) = \ln\frac{L(\mathfrak{D}_{\mathcal{L}(m)}, \hat{\textbf{w}})}{L(\mathfrak{D}_{\mathcal{T}(m)}, \hat{\textbf{w}})}, \quad \frac{|\mathcal{T}(m)|}{|\mathcal{L}(m)|} = \text{const} \leq 0.5.
\end{aligned}
\]
Справедливо следующее предельное соотношение:
\[
\label{eq:hb:6}
\begin{aligned}
	\lim_{m\to \infty}RS(m) = 0.
\end{aligned}
\]

Достаточный размер выборки~$m^*$ определяется согласно условию:
\[
\label{eq:hb:7}
\begin{aligned}
	m^*: \forall m \geq m^* \mathsf{E}_{\mathfrak{D}_{m}}RS(m) \leq \varepsilon,
\end{aligned}
\]
где~$\varepsilon$ некоторый параметр, который задается экспертно.

Этот метод предполагает, что длины доверительных интервалов квантиля не превышают некоторого фиксированного значения~$l$. Для некоторого размера выборки~$m$ вычисляются квантильные доверительные интервалы~$\left (a^m_1, b^m_1\right), \left(a^m_2, b^m_2 \right), ..., \left(a^m_n, b^m_n \right)$ с уровнем значимости~$\alpha$ с использованием начальной загрузки для каждого параметра модели. Достаточный размер выборки задается выражением:
\[
\label{eq:hb:8}
\begin{aligned}
	m^*: \forall m\geq m^* \max_i\left(b^m_i - a^m_i\right) < l.
\end{aligned}
\]
    
Данный метод является покоординатным, и следовательно для повышения точности прогноза требуется значительное увеличение размера выборки.

Эвристические методы, рассмотренные выше, основаны на наблюдении за поведением модели при изменении размера выборки, но не предоставляют строгих теоретических гарантий. В следующем разделе рассматриваются байесовские методы, которые позволяют формализовать задачу определения достаточного размера выборки в рамках вероятностного подхода.

\section{Байесовские методы определения достаточного размера выборки}

В настоящем разделе рассматриваются байесовские методы оценки размера выборки, основанные на ограничении некоторых характеристик модели. Для анализа эффективности определяется функция размера выборки. Увеличение этой функции интерпретируется как снижение эффективности модели. Размер выборки~$m^*$ выбирается таким, чтобы исследуемая функция принимала значения меньше некоторого порогового значения~$\varepsilon$.

Размер выборки~$m^*$ определяется условием:
\[
\label{eq:bs:1}
\begin{aligned}
	\forall m \geq m^*    \mathsf{E}_{\mathfrak{D}_m}\mathsf{D}\left[\hat{\textbf{w}}|\mathfrak{D}_m\right] \leq l.
\end{aligned}
\]
где~$l$ некоторый заданный экспертно параметр, который количественно определяет неопределенность оценки параметра.

Обозначим через~$A\left(\mathfrak{D}\right) \subset \mathbb{R}^n$ некоторый набор параметров модели~$\textbf{w}$:
\[
\label{eq:bs:2}
\begin{aligned}
	A\left(\mathfrak{D}\right) = \left\{\textbf{w}:||\textbf{w} - \hat{\textbf{w}}||\leq l\right\},
\end{aligned}
\]
где~$l$~--- некоторый фиксированный радиус шара.
Размер выборки~$m^*$ определяется критерием среднего покрытия:
\[
\label{eq:bs:3}
\begin{aligned}
	\forall m \geq m^*    \mathsf{E}_{\mathfrak{D}_m}\mathsf{P}\left\{\textbf{w} \in A\left(\mathfrak{D}_m\right)\right\} \geq 1-\alpha,
\end{aligned}
\]
где~$\alpha$ некоторый параметр заданный экспертно.

Определим функцию~$A\left(\mathfrak{D}\right)$:
\[
\label{eq:bs:4}
\begin{aligned}
	\mathsf{P}\left(A\left(\mathfrak{D}\right)\right) =  1- \alpha.
\end{aligned}
\]
Оценка критерия средней длины~$m^*$, заданная в \eqref{eq:bs:3}:
	
\[
\label{eq:bs:5}
\begin{aligned}
	\forall m \geq m^*    \mathsf{E}_{\mathfrak{D}_m}r_m\leq l,
\end{aligned}
\]
где~$r_m$ является радиусом шара~$A\left(\mathfrak{D}_{m}\right)$.

Следующие методы максимизируют ожидание некоторой функции полезности~$u\left(\mathfrak{D}, \textbf{w}\right)$ по размеру выборки:
\[
\label{eq:bs:6}
\begin{aligned}
	m^* = \arg\max_{m} \mathsf{E}_{\mathfrak{D}_m}\int_{\textbf{w}}u\left(\mathfrak{D}_m, \textbf{w}\right)p(\textbf{w}|\mathfrak{D}_m)d\textbf{w},
\end{aligned}
\]
где функция полезности~$u\left(\mathfrak{D}, \textbf{w}\right)$ задается в виде:

\[
\label{eq:bs:7}
\begin{aligned}
	u\left(\mathfrak{D}_m, \textbf{w}\right) = l\left(\mathfrak{D}_m, \textbf{w}\right) - cm,
\end{aligned}
\]
где~$c$~--- коэффициент штрафа для каждого элемента в наборе выборки.
	 
Назовем индексы~$\mathcal{B}_1,\mathcal{B}_2 \subset \{1,...,m\}$ по соседству, если
\[
\label{eq:bs:8}
\begin{aligned}
	\left|\mathcal{B}_1 \Delta \mathcal{B}_2\right| = 1.
\end{aligned}
\]
Таким образом,~$\mathcal{B}_2~$ можно преобразовать в~$\mathcal{B}_1$ путем удаления, замены или добавления одного элемента. В~\cite{motrenko2014} показано, что если размер набора выборок~$\mathfrak {D}_{\mathcal {B}_1}$ достаточно велик, то параметры модели~$\hat{\textbf {w}}_1$, оптимизированные с помощью~$\mathfrak{D}_{\mathcal{B}_1}$, должны находиться в окрестности параметров модели~$\hat{\textbf{w}}_2~$, которые оптимизированы с помощью~$\mathfrak{D}_{\mathcal {B}_2}$.
	 
Используя дивергенцию Кульбака-Лейблера в качестве функции близости между распределениями параметров модели, оптимизированных с помощью~$\mathfrak{D}_{\mathcal{B}_1}$ и~$\mathfrak{D}_{\mathcal{B}_2}$:
\[
\label{eq:bs:9}
\begin{aligned}
	D_\text{KL}\left(p_1, p_2\right) = \int_{\textbf{w}\in\mathbb{W}}p_1(\textbf{w})\log\frac{p_1(\textbf{w})}{p_2(\textbf{w})}d\textbf{w},
\end{aligned}
\]
где~$p_1$ и~$p_2$~--- апостериорные распределения вектора параметров~$\textbf{w}$, рассчитанные на подвыборках~$\mathfrak{D}_{\mathcal{B}_1}$ и~$\mathfrak{D}_{\mathcal{B}_2}$ соответственно. Также предполагается, что~$\mathfrak{D}_{\mathcal{B}_1}$ и~$\mathfrak{D}_{\mathcal{B}_2}$ находятся по соседству.
Достаточный размер выборки~$m^*$ оценивается:
\[
\label{eq:bs:10}
\begin{aligned}
	\forall \mathfrak{D}_{\mathcal{B}_1}: \left|\mathfrak{D}_{\mathcal{B}_1}\right| \geq m^*    \mathsf{E}_{\mathfrak{D}_{\mathcal{B}_2}}D_{KL}\left(p_1, p_2\right) \leq \varepsilon.
\end{aligned}
\]

Рассмотренные выше классические методы определения достаточного размера выборки имеют существенные ограничения: статистические методы требуют знания дисперсии оценок параметров, байесовские методы~--- вычислительно сложны для моделей с большим числом параметров, а эвристические методы не предоставляют строгих теоретических гарантий. В следующих разделах предлагаются новые методы, основанные на анализе стабильности функции правдоподобия и близости апостериорных распределений, которые преодолевают указанные ограничения.

\section{Метод определения достаточного размера выборки на основе сэмплирования эмпирической функции ошибки}

В настоящем разделе рассматривается метод определения достаточного размера выборки, основанный на анализе стабильности функции правдоподобия при изменении объема данных. Предполагается, что выполняется условие~$m^* \leqslant m$, где~$m$~--- размер доступной выборки~$D$, а~$m^*$~--- искомый достаточный размер. Таким образом, требуется определить минимальный объем выборки, который следует считать достаточным для обучения модели, при условии наличия достаточного количества объектов в самой выборке~$D$.

Для определения достаточности используется функция правдоподобия. Когда доступно достаточное количество объектов, естественно ожидать, что полученная оценка параметров не будет существенно изменяться от одной реализации выборки к другой~\cite{joseph1997,joseph1995}. Аналогичное утверждение справедливо и для функции правдоподобия. Таким образом, формализуем критерии, позволяющие определить достаточный объем выборки.

Критерий определяется в определении~\ref{sufficient-variance}.

\begin{definition}[D-достаточный размер выборки]\label{sufficient-variance}
    Пусть задано некоторое~$\varepsilon > 0$.
    Размер выборки~$m^*$ назовем D-достаточным, если для всех~$k\geqslant m^*$ выполняется условие:
    \[
        D(k) = \mathbb{D}_{\hat{\mathbf{w}}_{k}} L(\mathfrak{D}_m, \hat{\mathbf{w}}_{k}) \leqslant \varepsilon.
    \]
\end{definition}

Определение~\ref{sufficient-variance} формализует идею о том, что при достаточном размере выборки дисперсия функции правдоподобия по различным реализациям подвыборок должна быть мала, что указывает на стабильность оценки параметров.

С другой стороны, при наличии достаточного количества объектов естественно ожидать, что при добавлении еще одного объекта к рассмотрению результирующая оценка параметра изменится незначительно; на основе данного свойства получаем определение~\ref{sufficient-difference}.

\begin{definition}[M-достаточный размер выборки]\label{sufficient-difference}
    Пусть задано некоторое~$\varepsilon > 0$.
    Размер выборки~$m^*$ назовем M-достаточным, если для всех~$k\geqslant m^*$ выполняется условие:
    \[
        M(k) = \left| \mathbb{E}_{\hat{\mathbf{w}}_{k+1}} L(\mathfrak{D}_m, \hat{\mathbf{w}}_{k+1}) - \mathbb{E}_{\hat{\mathbf{w}}_{k}} L(\mathfrak{D}_m, \hat{\mathbf{w}}_{k}) \right| \leqslant \varepsilon.
    \]
\end{definition}

Определение~\ref{sufficient-difference} формализует условие, при котором добавление нового объекта к выборке не приводит к существенному изменению математического ожидания функции правдоподобия, что указывает на достижение достаточного объема данных для стабильной оценки параметров модели.

В приведенных выше определениях вместо функции правдоподобия~$L(\mathfrak{D}_m, \hat{\mathbf{w}}_{k})$ рассматривается ее логарифм~$l(\mathfrak{D}_m, \hat{\mathbf{w}}_{k})$, что упрощает математический анализ за счет перехода от произведения к сумме.

Предположим, что~$\mathbb{W} = \mathbb{R}^n$, информационная матрица Фишера задана матрицей:
\[
    \left[\mathcal{I}(\mathbf{w})\right]_{ij} = - \mathbb{E}\left[ \frac{\partial^2 \log p(\mathbf{y} | \mathbf{x}, \mathbf{w})}{\partial w_i \partial w_j} \right],
\]
Известным результатом является асимптотическая нормальность оценки максимального правдоподобия:
\[
    \sqrt{k}\left(\hat{\mathbf{w}}_k -\mathbf{w}\right)\xrightarrow{d}\mathcal{N}\left(0, \mathcal{I}^{-1}(\mathbf{w})\right),
\]
где~$\xrightarrow{d}$ обозначает сходимость по распределению. Следует отметить, что сходимость по распределению, вообще говоря, не влечет сходимости моментов случайного вектора.
Однако если предположить сходимость моментов, то в некоторых моделях можно доказать корректность предложенного определения M-достаточного размера выборки.

Для удобства обозначим параметры распределения~$\hat{\mathbf{w}}_k$ следующим образом: математическое ожидание~$\mathbb{E}\hat{\mathbf{w}}_k=\mathbf{m}_k$ и матрица ковариаций~$\mathbb{D} \hat{\mathbf{w}}_k = \mathbf{\Sigma}_k$.
Тогда справедлива теорема~\ref{chapter:samplesize:theorem-kiselev-likelihood-bootstraping}, которая доказывает сходимость параметров.

\begin{theorem}[Корректность M-достаточного размера выборки]\label{chapter:samplesize:theorem-kiselev-likelihood-bootstraping}
    Пусть~$\|\mathbf{m}_{k+1} - \mathbf{m}_k\|_2 \to 0$ и~$\|\mathbf{\Sigma}_{k+1} - \mathbf{\Sigma}_k\|_{F}\to 0$ при~$k\to \infty$. 
    Тогда в модели линейной регрессии определение M-достаточного размера выборки корректно.
    А именно, для любого~$\varepsilon > 0$ существует такой~$m^*$, что для всех~$k\geqslant m^*$ выполняется~$M(k)\leqslant\varepsilon$.
\end{theorem}
\begin{proof}
Рассмотрим определение M-достаточного размера выборки в терминах логарифма функции правдоподобия. В модели линейной регрессии
\begin{align}
    L\left( \mathfrak{D}_m, \hat{\mathbf{w}}_k \right) &= p(\mathbf{y} | \mathbf{X}, \hat{\mathbf{w}}_k) = \prod_{i=1}^{m} p(y_i | \mathbf{x}_i, \hat{\mathbf{w}}_k) = \prod_{i=1}^{m} \mathcal{N}\left( y_i | \hat{\mathbf{w}}_k^{\top} \mathbf{x}_i, \sigma^2 \right) =\\
    &= \left(2\pi\sigma^2 \right)^{-m/2} \exp\left(-\frac{1}{2\sigma^2}\|\mathbf{y} -\mathbf{X} \hat{\mathbf{w}}_k\|_2^2 \right).
\end{align}
Возьмем логарифм:
\[
    l\left( \mathfrak{D}_m, \hat{\mathbf{w}}_k \right) = \log p(\mathbf{y} | \mathbf{X}, \hat{\mathbf{w}}_k) = -\frac{m}{2}\log\left( 2\pi\sigma^2 \right) - \frac{1}{2\sigma^2} \| \mathbf{y} - \mathbf{X} \hat{\mathbf{w}}_k \|_2^2.
\]
Возьмем математическое ожидание по~$\mathfrak{D}_k$, учитывая что~$\mathbb{E}_{\mathfrak{D}_k}\hat{\mathbf{w}}_k=\mathbf{m}_k$ и~$\text{cov}(\hat{\mathbf{w}}_k) = \mathbf{\Sigma}_k$:
\[
    \mathbb{E}_{\mathfrak{D}_k} l\left( \mathfrak{D}_m, \hat{\mathbf{w}}_k \right) = -\frac{m}{2}\log\left( 2\pi\sigma^2 \right) - \frac{1}{2\sigma^2} \Big( \| \mathbf{y} - \mathbf{X} \mathbf{m}_k \|_2^2 + \text{tr}\left( \mathbf{X}^{\top}\mathbf{X} \mathbf{\Sigma}_k \right) \Big).
\]
Запишем выражение для разности математических ожиданий:
\begin{align}
    &\mathbb{E}_{\mathfrak{D}_{k+1}} l(\mathfrak{D}_m, \hat{\mathbf{w}}_{k+1}) - \mathbb{E}_{\mathfrak{D}_k} l(\mathfrak{D}_m, \hat{\mathbf{w}}_{k}) = \\
    &\quad= \frac{1}{2\sigma^2} \Big( \| \mathbf{y} - \mathbf{X} \mathbf{m}_k \|_2^2 - \| \mathbf{y} - \mathbf{X} \mathbf{m}_{k+1} \|_2^2 \Big) + \frac{1}{2\sigma^2} \text{tr} \Big( \mathbf{X}^{\top}\mathbf{X} \left( \mathbf{\Sigma}_k - \mathbf{\Sigma}_{k+1} \Big) \right) = \\
    &\quad= \frac{1}{2\sigma^2} \Big( 2 \mathbf{y}^{\top} \mathbf{X} (\mathbf{m}_{k+1} - \mathbf{m}_k) + (\mathbf{m}_k - \mathbf{m}_{k+1})^{\top} \mathbf{X}^{\top}\mathbf{X} (\mathbf{m}_k + \mathbf{m}_{k+1}) \Big) + \\
    &\qquad+ \frac{1}{2\sigma^2} \text{tr} \Big( \mathbf{X}^{\top}\mathbf{X} \left( \mathbf{\Sigma}_k - \mathbf{\Sigma}_{k+1} \right) \Big).
\end{align}
Значение функции~$M(k)$ представляет собой модуль от приведенного выше выражения. Применим неравенство треугольника для модуля, а затем оценим каждое слагаемое.\\
Оценим первое слагаемое, используя неравенство Коши-Буняковского:
\[
    \big| \mathbf{y}^{\top}\mathbf{X}(\mathbf{m}_{k+1}-\mathbf{m}_k)\big| \leqslant \| \mathbf{X}^{\top}\mathbf{y} \|_2 \|\mathbf{m}_{k+1} - \mathbf{m}_k\|_2.
\]
Второе слагаемое оцениваем с помощью неравенства Коши-Буняковского, свойства согласованности спектральной нормы матрицы, а также ограниченности последовательности векторов~$\mathbf{m}_k$, что следует из приведенного условия сходимости:
\begin{align}
    \big| (\mathbf{m}_k - \mathbf{m}_{k+1})^{\top} \mathbf{X}^{\top}\mathbf{X} (\mathbf{m}_k + \mathbf{m}_{k+1}) \big| &\leqslant \| \mathbf{X} (\mathbf{m}_k - \mathbf{m}_{k+1}) \|_2 \| \mathbf{X} (\mathbf{m}_k + \mathbf{m}_{k+1}) \|_2 \leqslant \\
    &\leqslant \| \mathbf{X} \|_2^2 \| \mathbf{m}_k - \mathbf{m}_{k+1} \|_2 \| \mathbf{m}_k + \mathbf{m}_{k+1} \|_2 \leqslant \\
    &\leqslant C \| \mathbf{X} \|_2^2 \| \mathbf{m}_k - \mathbf{m}_{k+1} \|_2.
\end{align}
Последнее слагаемое оцениваем, используя неравенство Гельдера для нормы Фробениуса:
\[
    \Big| \text{tr} \Big( \mathbf{X}^{\top}\mathbf{X} \left( \mathbf{\Sigma}_k - \mathbf{\Sigma}_{k+1} \right) \Big) \Big| \leqslant \| \mathbf{X}^{\top}\mathbf{X} \|_F \| \mathbf{\Sigma}_k - \mathbf{\Sigma}_{k+1} \|_F.
\]
Наконец, поскольку~$\|\mathbf{m}_k - \mathbf{m}_{k+1} \|_2\to 0$ и~$\|\mathbf{\Sigma}_k - \mathbf{\Sigma}_{k+1}\|_{F}\to 0$ при~$k\to\infty$, то~$M(k)\to 0$ при~$k\to \infty$, что и доказывает теорему.
\end{proof}

Теорема~\ref{chapter:samplesize:theorem-kiselev-likelihood-bootstraping} устанавливает корректность определения M-достаточного размера выборки для модели линейной регрессии при выполнении условий сходимости моментов оценок параметров. Указанные условия являются естественными для асимптотически нормальных оценок максимального правдоподобия и выполняются при стандартных предположениях о регулярности модели.

\begin{corollary}[Корректность M-достаточного размера выборки при сходимости к истинным параметрам]\label{chapter:samplesize:corollary-kiselev-likelihood-bootstraping}
    Пусть~$\|\mathbf{m}_k - \mathbf{w}\|_2\to 0$ и~$\|\mathbf{\Sigma}_k - \left[k\mathcal{I}(\mathbf{w})\right]^{-1}\|_{F}\to 0$ при~$k \to \infty$. 
    
    Тогда в модели линейной регрессии определение M-достаточного размера выборки корректно.
\end{corollary}

По условию, задана только одна выборка, а следовательно, в эксперименте невозможно вычислить математическое ожидание и дисперсию, указанные в определениях~\ref{sufficient-variance} и~\ref{sufficient-difference}.
Поэтому для их оценки используется метод бутстрэпирования (англ. bootstrap): из заданной выборки~$\mathfrak{D}_m$ генерируется некоторое число~$B$ подвыборок размера~$k$ с возвращением.
Для каждой подвыборки получается оценка параметров~$\hat{\mathbf{w}}_{k}$ и вычисляется значение~$L(\mathfrak{D}_m, \hat{\mathbf{w}}_{k})$.
Для оценки математического ожидания и дисперсии используются выборочное среднее и несмещенная выборочная дисперсия соответственно. Количество подвыборок~$B$ выбирается достаточно большим (обычно~$B \geqslant 1000$) для обеспечения точности оценок.

Предложенные выше определения также могут быть применены в тех задачах, где минимизируется произвольная функция потерь, а не максимизируется функция правдоподобия. В этом случае вместо функции правдоподобия~$L(\mathfrak{D}_m, \hat{\mathbf{w}}_{k})$ используется функция потерь~$\mathcal{L}(\mathfrak{D}_m, \hat{\mathbf{w}}_{k})$, а критерии достаточности формулируются аналогичным образом.
Строгое теоретическое обоснование данного обобщения отсутствует, однако эмпирические результаты демонстрируют применимость таких методов на практике.

Метод, рассмотренный выше, основан на анализе стабильности функции правдоподобия при изменении объема данных. Альтернативный подход заключается в анализе близости апостериорных распределений параметров модели на близких подвыборках, что составляет содержание следующего раздела.

\section{Метод определения достаточного размера выборки на основе близости апостериорных распределений}

В настоящем разделе рассматривается метод определения достаточного размера выборки, основанный на анализе близости апостериорных распределений параметров модели. В работе~\cite{motrenko2014} предлагается использовать расхождение Кульбака-Лейблера для оценки достаточного размера выборки в задаче бинарной классификации.

Идея метода основана на том, что если две подвыборки отличаются друг от друга одним объектом, то полученные по ним апостериорные распределения должны быть близки. Эта близость определяется расхождением Кульбака-Лейблера.

\begin{figure}[h!t]\center
    \includesvg[width=0.7\textwidth]{figures/chapter-2/posterior_ru}
    \caption{Визуализация сдвига апостериорных распределений параметров модели при последовательном добавлении объектов в выборку. Иллюстрация демонстрирует концепцию близости распределений, используемую в методах KL- и S-достаточности для определения достаточного размера выборки.}
    \label{fig-chapter-2-posterior-ru}
\end{figure}

В рамках данного раздела предлагается использовать не только расхождение Кульбака-Лейблера, но и функцию схожести s-score из работы~\cite{aduenko2017}, для этого рассмотрим две подвыборки~$\mathfrak{D}^1\subseteq\mathfrak{D}_m$ и~$\mathfrak{D}^2\subseteq\mathfrak{D}_m$.
Пусть~$\mathcal{I}_1 \subseteq \mathcal{I} = \{1, \ldots, m\}$ и~$\mathcal{I}_2 \subseteq \mathcal{I} =\{1, \ldots,m\}$~--- соответствующие им подмножества индексов.

\begin{definition}[Близкие подвыборки]
    Подвыборки~$\mathfrak{D}^1$ и~$\mathfrak{D}^2$ назовем близкими, если~$\mathcal{I}_2$ может быть получено из~$\mathcal{I}_1$ путем удаления, замены или добавления одного элемента. Формально это условие записывается как
    \[
        \left| \mathcal{I}_1 \triangle \mathcal{I}_2 \right| = \left| \left( \mathcal{I}_1 \setminus \mathcal{I}_2 \right) \cup \left( \mathcal{I}_2 \setminus \mathcal{I}_1 \right) \right| = 1,
    \]
    где~$\triangle$ обозначает симметрическую разность множеств.
\end{definition}

Рассмотрим две близкие подвыборки~$\mathfrak{D}_k = (\mathbf{X}_k,\mathbf{y}_k)$ и~$\mathfrak{D}_{k+1} = (\mathbf{X}_{k+1}, \mathbf{y}_{k+1})$ размеров~$k$ и~$k+1$ соответственно, где выборка~$\mathfrak{D}_{k+1}$ получена путем добавления одного элемента к выборке~$\mathfrak{D}_k$.

Вычислим апостериорное распределение параметров модели по каждой из этих подвыборок:
\[
    p_j(\mathbf{w}) = p(\mathbf{w} | \mathfrak{D}_j) = \frac{p(\mathfrak{D}_j | \mathbf{w}) p(\mathbf{w})}{p(\mathfrak{D}_j)} \propto p(\mathfrak{D}_j | \mathbf{w}) p(\mathbf{w}), \quad j = k, k+1.
\]

\begin{definition}[KL-достаточный размер выборки]
    Пусть задано некоторое~$\varepsilon > 0$.
    Размер выборки~$m^*$ называется KL-достаточным, если для всех~$k\geqslant m^*$
    \[
        KL(k) = D_{KL}(p_k \| p_{k+1}) = \int p_k(\mathbf{w}) \log{\frac{p_k(\mathbf{w})}{p_{k+1}(\mathbf{w})}} d\mathbf{w} \leqslant \varepsilon.
    \]
\end{definition}

Для пары нормальных распределений расхождение Кульбака-Лейблера имеет аналитическое выражение. Предположив, что апостериорное распределение является нормальным,~$p_k(\mathbf{w}) = \mathcal{N}\left(\mathbf{w}|\mathbf{m}_k, \mathbf{\Sigma}_k\right)$, где~$\mathbf{m}_k$~--- вектор математического ожидания, а~$\mathbf{\Sigma}_k$~--- ковариационная матрица, получаем формулировку теоремы~\ref{chapter:samplesize:theorem-kiselev-posterior-similiarity}

\begin{theorem}[Корректность KL-достаточного размера выборки]\label{chapter:samplesize:theorem-kiselev-posterior-similiarity}
    Пусть~$\|\mathbf{m}_{k+1} - \mathbf{m}_k\|_2 \to 0$ и~$\|\mathbf{\Sigma}_{k+1} - \mathbf{\Sigma}_k\|_{F}\to 0$ при~$k\to \infty$.
    
    Тогда в модели с нормальным апостериорным распределением параметров определение KL-достаточного размера выборки корректно.
    А именно, для любого~$\varepsilon > 0$ существует такой~$m^*$, что для всех~$k\geqslant m^*$ выполняется~$KL(k)\leqslant\varepsilon$.
\end{theorem}
\begin{proof}
Рассмотрим выражение для расхождения Кульбака-Лейблера между двумя нормальными апостериорными распределениями~$p_k = \mathcal{N}(\mathbf{m}_k, \mathbf{\Sigma}_k)$ и~$p_{k+1} = \mathcal{N}(\mathbf{m}_{k+1}, \mathbf{\Sigma}_{k+1})$.
Для двух многомерных нормальных распределений данная метрика имеет аналитическое выражение:
\begin{align}
    &D_{\text{KL}}\left( p_k \| p_{k+1} \right) =\\
    &=\frac{1}{2} \left( \mathrm{tr}\left( \mathbf{\Sigma}_{k+1}^{-1} \mathbf{\Sigma}_k \right) + (\mathbf{m}_{k+1} - \mathbf{m}_k)^{\top} \mathbf{\Sigma}_{k+1}^{-1} (\mathbf{m}_{k+1} - \mathbf{m}_k) - n + \log{\left( \frac{\det \mathbf{\Sigma}_{k+1}}{\det \mathbf{\Sigma}_{k}} \right)} \right).
\end{align}
Для анализа поведения каждого слагаемого при~$k \to \infty$ введем обозначение для разности ковариационных матриц:~$\mathbf{\Sigma}_{k+1} = \mathbf{\Sigma}_k + \Delta\mathbf{\Sigma}$, где по условию теоремы~$\|\Delta \mathbf{\Sigma}\|_F = \|\mathbf{\Sigma}_{k+1} - \mathbf{\Sigma}_k\|_F \to 0$. Первое слагаемое представляет собой след произведения матриц:
\begin{align}
    \mathrm{tr}\left( \mathbf{\Sigma}_{k+1}^{-1} \mathbf{\Sigma}_k \right) &= \mathrm{tr}\left(\left(\mathbf{\Sigma}_k + \Delta \mathbf{\Sigma} \right)^{-1} \mathbf{\Sigma}_k \right).
\end{align}
Используя разложение в ряд для обратной матрицы при малых~$\Delta\mathbf{\Sigma}$, получаем:
\[
    (\mathbf{\Sigma}_k + \Delta \mathbf{\Sigma})^{-1} = \mathbf{\Sigma}_k^{-1} - \mathbf{\Sigma}_k^{-1} \Delta \mathbf{\Sigma} \mathbf{\Sigma}_k^{-1} + O(\|\Delta \mathbf{\Sigma}\|_F^2).
\]
Тогда:
\[
    (\mathbf{\Sigma}_k + \Delta \mathbf{\Sigma})^{-1} \mathbf{\Sigma}_k = \mathbf{I}_n - \mathbf{\Sigma}_k^{-1} \Delta \mathbf{\Sigma} + O(\|\Delta \mathbf{\Sigma}\|_F^2).
\]
Взяв след от этого выражения и учитывая, что~$\mathrm{tr}(\mathbf{I}_n) = n$, а~$\|\mathbf{\Sigma}_k^{-1} \Delta \mathbf{\Sigma}\|_F \to 0$ при~$\|\Delta \mathbf{\Sigma}\|_F \to 0$, получаем:
\[
    \mathrm{tr}\left( \mathbf{\Sigma}_{k+1}^{-1} \mathbf{\Sigma}_k \right) \to n \quad \text{при} \quad \| \Delta \mathbf{\Sigma} \|_F \to 0.
\]
Второе слагаемое представляет собой квадратичную форму:
\begin{align}
    (\mathbf{m}_{k+1} - \mathbf{m}_k)^{\top} \mathbf{\Sigma}_{k+1}^{-1} (\mathbf{m}_{k+1} - \mathbf{m}_k).
\end{align}
По условию теоремы~$\|\mathbf{m}_{k+1} - \mathbf{m}_k\|_2 \to 0$. Квадратичная форма оценивается сверху следующим образом:
\[
    \left| (\mathbf{m}_{k+1} - \mathbf{m}_k)^{\top} \mathbf{\Sigma}_{k+1}^{-1} (\mathbf{m}_{k+1} - \mathbf{m}_k) \right| \leqslant \| \mathbf{m}_{k+1} -\mathbf{m}_k\|_2^2 \cdot \|\mathbf{\Sigma}_{k+1}^{-1} \|_2.
\]
Поскольку ковариационная матрица~$\mathbf{\Sigma}_{k+1}$ является положительно определенной и сходится к некоторой предельной матрице, ее спектральная норма~$\|\mathbf{\Sigma}_{k+1}^{-1} \|_2$ ограничена. Следовательно, при~$\| \mathbf{m}_{k+1} - \mathbf{m}_k\|_2 \to 0$ данное слагаемое стремится к нулю.
Третье и четвертое слагаемые составляют:
\[
    -n + \log{\left( \frac{\det \mathbf{\Sigma}_{k+1}}{\det \mathbf{\Sigma}_{k}} \right)} = \log{\left( \frac{\det \mathbf{\Sigma}_{k+1}}{\det \mathbf{\Sigma}_{k}} \right)} - n.
\]
Преобразуем отношение определителей:
\[
    \frac{\det \mathbf{\Sigma}_{k+1}}{\det \mathbf{\Sigma}_{k}} = \frac{\det (\mathbf{\Sigma}_k + \Delta \mathbf{\Sigma})}{\det \mathbf{\Sigma}_{k}} = \det (\mathbf{I}_n + \mathbf{\Sigma}_k^{-1} \Delta \mathbf{\Sigma}).
\]
Для малых~$\Delta \mathbf{\Sigma}$ используем приближение:
\[
    \det (\mathbf{I}_n + \mathbf{\Sigma}_k^{-1} \Delta \mathbf{\Sigma}) = 1 + \mathrm{tr}(\mathbf{\Sigma}_k^{-1} \Delta \mathbf{\Sigma}) + O(\|\Delta \mathbf{\Sigma}\|_F^2).
\]
Тогда:
\[
    \log{\left( \frac{\det \mathbf{\Sigma}_{k+1}}{\det \mathbf{\Sigma}_{k}} \right)} = \log \det (\mathbf{I}_n + \mathbf{\Sigma}_k^{-1} \Delta \mathbf{\Sigma}) = \mathrm{tr}(\mathbf{\Sigma}_k^{-1} \Delta \mathbf{\Sigma}) + O(\|\Delta \mathbf{\Sigma}\|_F^2).
\]
Поскольку~$\|\Delta \mathbf{\Sigma}\|_F \to 0$, то~$\mathrm{tr}(\mathbf{\Sigma}_k^{-1} \Delta \mathbf{\Sigma}) \to 0$, и следовательно:
\[
    \log{\left( \frac{\det \mathbf{\Sigma}_{k+1}}{\det \mathbf{\Sigma}_{k}} \right)} \to 0.
\]

Таким образом, все четыре слагаемых в выражении для~$D_{\text{KL}}$ сходятся к своим пределам при~$k \to \infty$: первое слагаемое стремится к~$n$, второе~--- к 0, третье равно~$-n$, четвертое~--- к 0.
Сумма этих пределов равна~$n + 0 - n + 0 = 0$, что доказывает, что~$D_{\text{KL}}(p_k \| p_{k+1}) \to 0$ при~$k \to \infty$.
Следовательно, для любого~$\varepsilon > 0$ существует такой~$m^*$, что для всех~$k \geqslant m^*$ выполняется~$KL(k) \leqslant \varepsilon$, что и требовалось доказать.
\end{proof}

Теорема~\ref{chapter:samplesize:theorem-kiselev-posterior-similiarity} устанавливает, что расхождение Кульбака-Лейблера между двумя нормальными апостериорными распределениями стремится к нулю по мере сходимости их векторов математических ожиданий и ковариационных матриц. Это позволяет использовать KL-дивергенцию в качестве критерия достаточности размера выборки, анализируя аналитические выражения для моментов апостериорных распределений.

Рассмотрим функцию схожести s-score из работы~\cite{aduenko2017} в качестве меры близости распределений по аналогии, как это было с KL-дивергенцией:
\[
    \text{s-score}(g_1, g_2) = \frac{\int_{\mathbf{w}} g_1(\mathbf{w}) g_2(\mathbf{w}) d\mathbf{w}}{\max_{\mathbf{b}} \int_{\mathbf{w}} g_1(\mathbf{w} - \mathbf{b}) g_2(\mathbf{w}) d\mathbf{w}}.
\]
\begin{definition}[S-достаточный размер выборки]
    Пусть задано некоторое~$\varepsilon > 0$.
    Размер выборки~$m^*$ называется S-достаточным, если для всех~$k\geqslant m^*$
    \[
        S(k) = \text{s-score}(p_k, p_{k+1}) \geqslant 1-\varepsilon.
    \]
\end{definition}
Как и в случае с KL-достаточным размером выборки, в модели с нормальным апостериорным распределением можно записать выражение для используемого критерия, который записан в виде теоремы~\ref{chapter:samplesize:theorem-kiselev-posterior-similiarity-s-score}

\begin{theorem}[Корректность S-достаточного размера выборки]\label{chapter:samplesize:theorem-kiselev-posterior-similiarity-s-score}
    Пусть~$\|\mathbf{m}_{k+1} - \mathbf{m}_k\|_2\to 0$ при~$k \to \infty$.
    
    Тогда в модели с нормальным апостериорным распределением параметров определение S-достаточного размера выборки корректно.
    А именно, для любого~$\varepsilon > 0$ существует такой~$m^*$, что для всех~$k\geqslant m^*$ выполняется~$S(k)\geqslant 1-\varepsilon$.
\end{theorem}
\begin{proof}
Используем выражение для s-score пары нормальных апостериорных распределений из работы~\cite{aduenko2017}:
\[
    \text{s-score}(p_k, p_{k+1}) = \exp{\left( -\frac{1}{2} (\mathbf{m}_{k+1} - \mathbf{m}_k)^{\top} \left( \mathbf{\Sigma}_k + \mathbf{\Sigma}_{k+1} \right)^{-1} (\mathbf{m}_{k+1} - \mathbf{m}_k) \right)}.
\]
Оценим квадратичную форму в показателе экспоненты:
\[
    \left| (\mathbf{m}_{k+1} - \mathbf{m}_k)^{\top} \left( \mathbf{\Sigma}_k + \mathbf{\Sigma}_{k+1} \right)^{-1} (\mathbf{m}_{k+1} - \mathbf{m}_k) \right| \leqslant \| \mathbf{m}_{k+1} - \mathbf{m}_k \|_2^2 \| \left( \mathbf{\Sigma}_k + \mathbf{\Sigma}_{k+1} \right)^{-1} \|_2.
\]
Поскольку ковариационные матрицы~$\mathbf{\Sigma}_k$ и~$\mathbf{\Sigma}_{k+1}$ являются положительно определенными и сходятся к некоторой предельной матрице, спектральная норма~$\| \left( \mathbf{\Sigma}_k + \mathbf{\Sigma}_{k+1} \right)^{-1} \|_2$ ограничена. При условии~$\|\mathbf{m}_{k+1} - \mathbf{m}_k\|_2\to 0$ значение квадратичной формы в показателе экспоненты стремится к нулю.
Следовательно,~$\text{s-score}(p_k, p_{k+1}) \to 1$ при~$\|\mathbf{m}_{k+1} - \mathbf{m}_k\|_2\to 0$, что и требовалось доказать.
\end{proof}

Теорема~\ref{chapter:samplesize:theorem-kiselev-posterior-similiarity-s-score} устанавливает корректность определения S-достаточного размера выборки при более слабых условиях, чем теорема~\ref{chapter:samplesize:theorem-kiselev-posterior-similiarity}. В отличие от KL-дивергенции, для сходимости s-score к единице требуется только сходимость математических ожиданий апостериорных распределений, что делает данный критерий менее консервативным и более применимым на практике.

Пусть в модели линейной регрессии задано нормальное априорное распределение параметров. В силу свойства сопряженности априорного распределения и правдоподобия, апостериорное распределение также будет нормальным.
Таким образом, приходим к одному из простейших примеров модели, для которой справедливы приведенные выше теоремы.
Фактически, для линейной регрессии могут быть сформулированы более простые утверждения.

\begin{theorem}[Сходимость апостериорных распределений в линейной регрессии]\label{chapter:samplesize:theorem-kiselev-posterior-similiarity-linear-regression}
    Пусть множества значений признаков и целевой переменной ограничены, то есть существует константа~$M\in \mathbb{R}$ такая, что~$\|\mathbf{x}\|_2\leqslant M$ и~$|y|\leqslant M$ для всех объектов выборки.
    Если~$\lambda_{\min}\left(\mathbf{X}^{\top}_k \mathbf{X}_k \right) = \omega(\sqrt{k})$ при~$k\to \infty$, то в модели линейной регрессии с нормальным априорным распределением параметров~$\|\mathbf{m}_{k+1} - \mathbf{m}_k\|_2\to 0$ и~$\|\mathbf{\Sigma}_{k+1} - \mathbf{\Sigma}_k\|_{F}\to 0$ при~$k\to \infty$.
\end{theorem}
\begin{proof}
Рассмотрим линейную регрессионную модель с нормальным априорным распределением параметров:~$p(\mathbf{w})=\mathcal{N}\left(\mathbf{w}|\mathbf{0}, \alpha^{-1}\mathbf{I}\right)$.
Данное априорное распределение является сопряженным для нормального правдоподобия, что существенно упрощает анализ.
Нормальное правдоподобие задается в виде:
\[
    p(\mathbf{y} | \mathbf{X}, \mathbf{w}) = \mathcal{N}\left(\mathbf{y} |\mathbf{X}\mathbf{w}, \sigma^2\mathbf{I}\right) =\left( 2\pi\sigma^2\right)^{-m/2} \exp\left( -\frac{1}{2\sigma^2} \|\mathbf{y} - \mathbf{X}\mathbf{w}\|_2^2\right).
\]

Благодаря свойству сопряженности нормального априорного распределения и нормального правдоподобия, апостериорное распределение также является нормальным:
\[
    p(\mathbf{w} | \mathbf{X}, \mathbf{y}) = \mathcal{N}\left(\mathbf{w} | \mathbf{m}, \mathbf{\Sigma} \right),
\]
где параметры распределения имеют аналитическое выражение:
\begin{align}
    \mathbf{\Sigma} &= \left( \alpha \mathbf{I} + \frac{1}{\sigma^2} \mathbf{X}^{\top} \mathbf{X} \right)^{-1}, \\
    \mathbf{m} &= \frac{1}{\sigma^2} \mathbf{\Sigma} \mathbf{X}^{\top} \mathbf{y} = \left( \mathbf{X}^{\top} \mathbf{X} + \alpha \sigma^2 \mathbf{I} \right)^{-1} \mathbf{X}^{\top} \mathbf{y}.
\end{align}

Перейдем к анализу сходимости ковариационных матриц.
Рассмотрим подвыборки размера~$k$ и~$k+1$, полученные из исходных данных. Нас интересует поведение разности~$\|\mathbf{\Sigma}_{k+1} - \mathbf{\Sigma}_k\|_2$ при~$k \to \infty$.
Введем обозначение~$\mathbf{A}_k = \frac{1}{\sigma^2}\mathbf{X}_k^{\top}\mathbf{X}_k$ для нормированной матрицы ковариации признаков. Тогда разность обратных матриц можно преобразовать, используя матричное тождество:
\begin{align}
    \| \mathbf{\Sigma}_{k+1} - \mathbf{\Sigma}_k \|_2 &= \left\| \left( \alpha \mathbf{I} + \mathbf{A}_{k+1} \right)^{-1} - \left( \alpha \mathbf{I} + \mathbf{A}_k \right)^{-1} \right\|_2.
\end{align}
Применяя матричное тождество для разности обратных матриц, получаем:
\begin{align}
    \left( \alpha \mathbf{I} + \mathbf{A}_{k+1} \right)^{-1} - \left( \alpha \mathbf{I} + \mathbf{A}_k \right)^{-1} = 
    \left( \alpha \mathbf{I} + \mathbf{A}_{k+1} \right)^{-1} \left( \mathbf{A}_k - \mathbf{A}_{k+1} \right) \left( \alpha \mathbf{I} + \mathbf{A}_k \right)^{-1}.
\end{align}
Используя субмультипликативное свойство спектральной нормы, оцениваем:
\begin{align}
    \| \mathbf{\Sigma}_{k+1} - \mathbf{\Sigma}_k \|_2 &\leqslant \left\| \left( \alpha \mathbf{I} + \mathbf{A}_{k+1} \right)^{-1} \right\|_2 \left\| \left( \alpha \mathbf{I} + \mathbf{A}_k \right)^{-1} \right\|_2 \left\| \mathbf{A}_{k+1} - \mathbf{A}_k \right\|_2.
\end{align}
Проанализируем каждый из множителей в полученной оценке.
Спектральная норма обратной матрицы выражается через минимальное собственное значение исходной матрицы:
\[
    \left\| \left( \alpha \mathbf{I} + \mathbf{A} \right)^{-1} \right\|_2 = \frac{1}{\lambda_{\min}(\alpha \mathbf{I} + \mathbf{A})}.
\]
Поскольку~$\alpha > 0$ и матрица~$\mathbf{A}$ положительно полуопределена, имеем~$\lambda_{\min}(\alpha \mathbf{I} + \mathbf{A}) \geq \alpha + \lambda_{\min}(\mathbf{A})$.
Однако для получения точной асимптотики используем более слабую оценку:
\[
\left\| \left( \alpha \mathbf{I} + \mathbf{A} \right)^{-1} \right\|_2 \leq \frac{1}{\lambda_{\min}(\mathbf{A})}.
\]
Объединяя полученные оценки, получаем цепочку неравенств:
\begin{align}
    \| \mathbf{\Sigma}_{k+1} - \mathbf{\Sigma}_k \|_2 &\leqslant \frac{1}{\lambda_{\min}\left( \mathbf{A}_{k+1} \right)} \frac{1}{\lambda_{\min}\left( \mathbf{A}_k \right)} \left\| \mathbf{A}_{k+1} - \mathbf{A}_k \right\|_2 = \\
    &= \sigma^2  \frac{1}{\lambda_{\min}\left( \mathbf{X}_{k+1}^{\top} \mathbf{X}_{k+1} \right)} \frac{1}{\lambda_{\min}\left( \mathbf{X}_k^{\top} \mathbf{X}_k \right)} \left\| \mathbf{X}_{k+1}^{\top} \mathbf{X}_{k+1} - \mathbf{X}_k^{\top} \mathbf{X}_k \right\|_2.
\end{align}
Поскольку выборка~$\mathfrak{D}_{k+1}$ получается из~$\mathfrak{D}_k$ добавлением одного наблюдения, имеем:
\begin{align}
    \left\| \mathbf{X}_{k+1}^{\top} \mathbf{X}_{k+1} - \mathbf{X}_k^{\top} \mathbf{X}_k \right\|_2 &= \left\| \sum\limits_{i=1}^{k+1} \mathbf{x}_i \mathbf{x}_i^{\top} - \sum\limits_{i=1}^{k} \mathbf{x}_i \mathbf{x}_i^{\top} \right\|_2 = \left\| \mathbf{x}_{k+1} \mathbf{x}_{k+1}^{\top} \right\|_2.
\end{align}
Матрица~$\mathbf{x}_{k+1} \mathbf{x}_{k+1}^{\top}$ является матрицей ранга 1, и ее спектральная норма равна квадрату евклидовой нормы вектора~$\mathbf{x}_{k+1}$:
\[
    \left\| \mathbf{x}_{k+1} \mathbf{x}_{k+1}^{\top} \right\|_2 = \lambda_{\max}\left( \mathbf{x}_{k+1} \mathbf{x}_{k+1}^{\top} \right) = \| \mathbf{x}_{k+1}\|_2^2.
\]
Из условия ограниченности признаков следует~$\| \mathbf{x}_{k+1}\|_2^2 \leqslant M^2$. Таким образом:
\[
    \left\| \mathbf{X}_{k+1}^{\top} \mathbf{X}_{k+1} - \mathbf{X}_k^{\top} \mathbf{X}_k \right\|_2 \leqslant M^2.
\]

Теперь рассмотрим условие на минимальное собственное значение.
Из предположения~$\lambda_{\min}\left(\mathbf{X}_k^{\top}\mathbf{X}_k \right) = \omega(\sqrt{k})$ следует, что:
\[
    \frac{1}{\lambda_{\min}\left(\mathbf{X}_k^{\top}\mathbf{X}_k \right)} = o\left(\frac{1}{\sqrt{k}}\right).
\]
Комбинируя полученные оценки, приходим к:
\[
    \|\mathbf{\Sigma}_{k+1} - \mathbf{\Sigma}_k\|_2 \leqslant \sigma^2 M^2 \cdot o\left(\frac{1}{\sqrt{k}}\right) \cdot o\left(\frac{1}{\sqrt{k}}\right) = o\left(\frac{1}{k}\right).
\]
Для перехода к норме Фробениуса воспользуемся неравенством~$\| \mathbf{A} \|_F \leqslant \sqrt{n} \| \mathbf{A} \|_2$, где~$n$~--- размерность пространства параметров:
\[
    \|\mathbf{\Sigma}_{k+1} - \mathbf{\Sigma}_k \|_F \leqslant \sqrt{n} \|\mathbf{\Sigma}_{k+1} - \mathbf{\Sigma}_k\|_2 = \sqrt{n} \cdot o\left(\frac{1}{k}\right) = o\left(\frac{1}{k}\right).
\]
Теперь перейдем к анализу сходимости математических ожиданий.
Требуется оценить:
\[
    \| \mathbf{m}_{k+1} - \mathbf{m}_k \|_2 = \left\| \left( \mathbf{X}_{k+1}^{\top} \mathbf{X}_{k+1} + \alpha \sigma^2 \mathbf{I} \right)^{-1} \mathbf{X}_{k+1}^{\top} \mathbf{y}_{k+1} - \left( \mathbf{X}_k^{\top} \mathbf{X}_k + \alpha \sigma^2 \mathbf{I} \right)^{-1} \mathbf{X}_k^{\top} \mathbf{y}_k \right\|_2.
\]
Представим расширенную матрицу признаков и вектор ответов через предыдущие значения:
\begin{align}
    \mathbf{X}_{k+1} &= \begin{bmatrix} \mathbf{X}_k \\ \mathbf{x}_{k+1}^{\top} \end{bmatrix}, &
    \mathbf{y}_{k+1} &= \begin{bmatrix} \mathbf{y}_k \\ y_{k+1} \end{bmatrix}.
\end{align}
Тогда матричные произведения принимают вид:
\begin{align}
    \mathbf{X}_{k+1}^{\top} \mathbf{X}_{k+1} &= \mathbf{X}_k^{\top} \mathbf{X}_k + \mathbf{x}_{k+1} \mathbf{x}_{k+1}^{\top}, \\
    \mathbf{X}_{k+1}^{\top} \mathbf{y}_{k+1} &= \mathbf{X}_k^{\top} \mathbf{y}_k + \mathbf{x}_{k+1} y_{k+1}.
\end{align}
Подставляя эти выражения, получаем:
\begin{align}
    \| \mathbf{m}_{k+1} - \mathbf{m}_k \|_2 &= \Bigg\| \left( \mathbf{X}_k^{\top} \mathbf{X}_k + \alpha \sigma^2 \mathbf{I} + \mathbf{x}_{k+1} \mathbf{x}_{k+1}^{\top} \right)^{-1} \left( \mathbf{X}_k^{\top} \mathbf{y}_k + \mathbf{x}_{k+1} y_{k+1} \right) \\
    &\quad - \left( \mathbf{X}_k^{\top} \mathbf{X}_k + \alpha \sigma^2 \mathbf{I} \right)^{-1} \mathbf{X}_k^{\top} \mathbf{y}_k \Bigg\|_2.
\end{align}
Для упрощения первого слагаемого применим лемму о матричном обращении:
\begin{align}
    &\left(\mathbf{X}_k^{\top} \mathbf{X}_k + \alpha\sigma^2 \mathbf{I} + \mathbf{x}_{k+1}\mathbf{x}_{k+1}^{\top}\right)^{-1} = \\
    &\quad = \left(\mathbf{X}_k^{\top} \mathbf{X}_k + \alpha\sigma^2 \mathbf{I}\right)^{-1} - \frac{\left(\mathbf{X}_k^{\top} \mathbf{X}_k + \alpha\sigma^2 \mathbf{I}\right)^{-1} \mathbf{x}_{k+1} \mathbf{x}_{k+1}^{\top} \left(\mathbf{X}_k^{\top} \mathbf{X}_k + \alpha\sigma^2 \mathbf{I}\right)^{-1}}{1 + \mathbf{x}_{k+1}^{\top} \left(\mathbf{X}_k^{\top} \mathbf{X}_k + \alpha\sigma^2 \mathbf{I}\right)^{-1} \mathbf{x}_{k+1}}.
\end{align}
После алгебраических преобразований получаем:
\begin{align}
    &\| \mathbf{m}_{k+1} - \mathbf{m}_k \|_2 = \\
    &\quad = \Bigg\| \left[ \left( \mathbf{I} + \left( \mathbf{X}_k^{\top} \mathbf{X}_k + \alpha \sigma^2 \mathbf{I} \right)^{-1} \mathbf{x}_{k+1} \mathbf{x}_{k+1}^{\top} \right)^{-1} - \mathbf{I} \right] \left( \mathbf{X}_k^{\top} \mathbf{X}_k + \alpha \sigma^2 \mathbf{I} \right)^{-1} \mathbf{X}_k^{\top} \mathbf{y}_k \\
    &\qquad + \left( \mathbf{X}_{k+1}^{\top} \mathbf{X}_{k+1} + \alpha \sigma^2 \mathbf{I} \right)^{-1} \mathbf{x}_{k+1} y_{k+1} \Bigg\|_2.
\end{align}
Применяя неравенство треугольника и свойства норм, оцениваем каждый член отдельно:
\begin{align}
    &\| \mathbf{m}_{k+1} - \mathbf{m}_k \|_2 \leqslant \\
    &\leqslant \left\| \left( \mathbf{I} + \left( \mathbf{X}_k^{\top} \mathbf{X}_k + \alpha \sigma^2 \mathbf{I} \right)^{-1} \mathbf{x}_{k+1} \mathbf{x}_{k+1}^{\top} \right)^{-1} - \mathbf{I} \right\|_2 \left\| \left( \mathbf{X}_k^{\top} \mathbf{X}_k + \alpha \sigma^2 \mathbf{I} \right)^{-1} \right\|_2 \left\| \mathbf{X}_k^{\top} \mathbf{y}_k \right\|_2 \\
    &\quad + \left\| \left( \mathbf{X}_{k+1}^{\top} \mathbf{X}_{k+1} + \alpha \sigma^2 \mathbf{I} \right)^{-1} \right\|_2 \left\| \mathbf{x}_{k+1} y_{k+1} \right\|_2.
\end{align}

Проанализируем первый множитель в первом слагаемом.
Используя матричное тождество, получаем:
\begin{align}
    &\left\| \left( \mathbf{I} + \left( \mathbf{X}_k^{\top} \mathbf{X}_k + \alpha \sigma^2 \mathbf{I} \right)^{-1} \mathbf{x}_{k+1} \mathbf{x}_{k+1}^{\top} \right)^{-1} - \mathbf{I} \right\|_2 = \\
    &\quad = \left\| - \left( \mathbf{I} + \left( \mathbf{X}_k^{\top} \mathbf{X}_k + \alpha \sigma^2 \mathbf{I} \right)^{-1} \mathbf{x}_{k+1} \mathbf{x}_{k+1}^{\top} \right)^{-1} \left( \mathbf{X}_k^{\top} \mathbf{X}_k + \alpha \sigma^2 \mathbf{I} \right)^{-1} \mathbf{x}_{k+1} \mathbf{x}_{k+1}^{\top} \right\|_2.
\end{align}
Применяя субмультипликативность нормы и учитывая, что спектральная норма произведения матриц не превышает произведения их норм, получаем оценку:
\begin{align}
    &\left\| \left( \mathbf{I} + \left( \mathbf{X}_k^{\top} \mathbf{X}_k + \alpha \sigma^2 \mathbf{I} \right)^{-1} \mathbf{x}_{k+1} \mathbf{x}_{k+1}^{\top} \right)^{-1} - \mathbf{I} \right\|_2 \leqslant \\
    &\quad \leqslant \left\| \left( \mathbf{I} + \left( \mathbf{X}_k^{\top} \mathbf{X}_k + \alpha \sigma^2 \mathbf{I} \right)^{-1} \mathbf{x}_{k+1} \mathbf{x}_{k+1}^{\top} \right)^{-1} \right\|_2 \left\| \left( \mathbf{X}_k^{\top} \mathbf{X}_k + \alpha \sigma^2 \mathbf{I} \right)^{-1} \right\|_2 \left\| \mathbf{x}_{k+1} \mathbf{x}_{k+1}^{\top} \right\|_2.
\end{align}

Оценим каждый из этих множителей.
Для первого множителя используем тот факт, что матрица~$\left( \mathbf{X}_k^{\top} \mathbf{X}_k + \alpha \sigma^2 \mathbf{I} \right)^{-1} \mathbf{x}_{k+1} \mathbf{x}_{k+1}^{\top}$ имеет единичный ранг, и ее максимальное собственное значение равно:
\[
    \lambda_{\max}\left( \left( \mathbf{X}_k^{\top} \mathbf{X}_k + \alpha \sigma^2 \mathbf{I} \right)^{-1} \mathbf{x}_{k+1} \mathbf{x}_{k+1}^{\top} \right) = \mathbf{x}_{k+1}^{\top} \left( \mathbf{X}_k^{\top} \mathbf{X}_k + \alpha \sigma^2 \mathbf{I} \right)^{-1} \mathbf{x}_{k+1}.
\]
Тогда спектральная норма обратной матрицы оценивается как:
\[
    \left\| \left( \mathbf{I} + \left( \mathbf{X}_k^{\top} \mathbf{X}_k + \alpha \sigma^2 \mathbf{I} \right)^{-1} \mathbf{x}_{k+1} \mathbf{x}_{k+1}^{\top} \right)^{-1} \right\|_2 \leqslant 1.
\]
Второй множитель оценивается через минимальное собственное значение:
\[
    \left\| \left( \mathbf{X}_k^{\top} \mathbf{X}_k + \alpha \sigma^2 \mathbf{I} \right)^{-1} \right\|_2 \leqslant \frac{1}{\lambda_{\min}\left( \mathbf{X}_k^{\top} \mathbf{X}_k \right)}.
\]
Третий множитель, как уже было установлено, равен~$\| \mathbf{x}_{k+1}\|_2^2 \leqslant M^2$.
Таким образом, получаем оценку для первого множителя:
\[
    \left\| \left( \mathbf{I} + \left( \mathbf{X}_k^{\top} \mathbf{X}_k + \alpha \sigma^2 \mathbf{I} \right)^{-1} \mathbf{x}_{k+1} \mathbf{x}_{k+1}^{\top} \right)^{-1} - \mathbf{I} \right\|_2 \leqslant \frac{M^2}{\lambda_{\min}\left( \mathbf{X}_k^{\top} \mathbf{X}_k \right)}.
\]

Теперь оценим второй множитель первого слагаемого вместе с третьим множителем:
\[
    \left\| \left( \mathbf{X}_k^{\top} \mathbf{X}_k + \alpha \sigma^2 \mathbf{I} \right)^{-1} \right\|_2 \left\| \mathbf{X}_k^{\top} \mathbf{y}_k \right\|_2 \leqslant \frac{\left\| \mathbf{X}_k^{\top} \mathbf{y}_k \right\|_2}{\lambda_{\min}\left( \mathbf{X}_k^{\top} \mathbf{X}_k \right)}.
\]
Норма~$\left\| \mathbf{X}_k^{\top} \mathbf{y}_k \right\|_2$ оценивается с использованием условия ограниченности:
\[
    \left\| \mathbf{X}_k^{\top} \mathbf{y}_k \right\|_2 = \left\| \sum\limits_{i=1}^{k} \mathbf{x}_i y_i \right\|_2 \leqslant \sum\limits_{i=1}^{k} \left\| \mathbf{x}_i y_i \right\|_2 \leqslant k M^2.
\]
Следовательно:
\[
    \left\| \left( \mathbf{X}_k^{\top} \mathbf{X}_k + \alpha \sigma^2 \mathbf{I} \right)^{-1} \right\|_2 \left\| \mathbf{X}_k^{\top} \mathbf{y}_k \right\|_2 \leqslant \frac{k M^2}{\lambda_{\min}\left( \mathbf{X}_k^{\top} \mathbf{X}_k \right)}.
\]
Теперь рассмотрим второе слагаемое:
\[
    \left\| \left( \mathbf{X}_{k+1}^{\top} \mathbf{X}_{k+1} + \alpha \sigma^2 \mathbf{I} \right)^{-1} \right\|_2 \left\| \mathbf{x}_{k+1} y_{k+1} \right\|_2 \leqslant \frac{M^2}{\lambda_{\min}\left( \mathbf{X}_{k+1}^{\top} \mathbf{X}_{k+1} \right)}.
\]

Комбинируя все полученные оценки, приходим к итоговой оценке:
\begin{align}
    \| \mathbf{m}_{k+1} - \mathbf{m}_k \|_2 &\leqslant \frac{M^2}{\lambda_{\min}\left( \mathbf{X}_k^{\top} \mathbf{X}_k \right)} \cdot \frac{k M^2}{\lambda_{\min}\left( \mathbf{X}_k^{\top} \mathbf{X}_k \right)} + \frac{M^2}{\lambda_{\min}\left( \mathbf{X}_{k+1}^{\top} \mathbf{X}_{k+1} \right)} \\
    &= \frac{k M^4}{\lambda_{\min}^2\left( \mathbf{X}_k^{\top} \mathbf{X}_k \right)} + \frac{M^2}{\lambda_{\min}\left( \mathbf{X}_{k+1}^{\top} \mathbf{X}_{k+1} \right)}.
\end{align}

Из условия~$\lambda_{\min}\left(\mathbf{X}_k^{\top}\mathbf{X}_k \right) = \omega(\sqrt{k})$ следует:
\begin{align}
    \frac{1}{\lambda_{\min}\left(\mathbf{X}_k^{\top}\mathbf{X}_k \right)} &= o\left(\frac{1}{\sqrt{k}}\right), \\
    \frac{1}{\lambda_{\min}^2\left(\mathbf{X}_k^{\top}\mathbf{X}_k \right)} &= o\left(\frac{1}{k}\right).
\end{align}
Поэтому первое слагаемое оценивается как~$k \cdot o\left(\frac{1}{k}\right) = o(1)$, а второе слагаемое как~$o\left(\frac{1}{\sqrt{k}}\right) = o(1)$. Таким образом:
\[
    \| \mathbf{m}_{k+1} - \mathbf{m}_k \|_2 = o(1) \quad \text{при} \quad k \to \infty.
\]

Это завершает доказательство сходимости как ковариационных матриц, так и математических ожиданий апостериорного распределения параметров.
\end{proof}

Теорема~\ref{chapter:samplesize:theorem-kiselev-posterior-similiarity-linear-regression} является ключевой в настоящем разделе, так как при слабых и понятных предположениях из нее следует сходимость моментов апостериорного распределения параметров.
Первое предположение в теореме~\ref{chapter:samplesize:theorem-kiselev-posterior-similiarity-linear-regression} касается ограничения на область значений признаков и целевой переменной.
Это условие обычно выполняется в практических приложениях, поэтому оно служит в первую очередь для целей теоретического анализа.
Второе условие теоремы~\ref{chapter:samplesize:theorem-kiselev-posterior-similiarity-linear-regression} представляет больший интерес, поскольку оно углубляется в поведение минимального собственного значения выборочной ковариационной матрицы признаков.
Следует отметить, что в рамках настоящей работы не приводятся строгие теоретические гарантии для данной сходимости, однако эмпирические результаты подтверждают выполнение указанного условия.

\section{Результаты вычислительных экспериментов}

В настоящем разделе представлены результаты вычислительных экспериментов для методов определения достаточного размера выборки, описанных в предыдущих разделах главы. Эксперименты направлены на валидацию теоретических результатов и сравнение эффективности различных подходов.

\subsection{Определения достаточного размера выборки на основе статистических методов}
\begin{table}[h!t]
\centering
\caption{Характеристики выборок, используемых для анализа качества методов определения достаточного размера выборки. Таблица содержит информацию о типе задачи, количестве признаков и общем размере выборки для каждого набора данных.}
\label{chapter:samplesize:experiment:static:table20}
\begin{tabular}{|l|l|c|c|}
\hline
	\centering Выборка & Задача & Число признаков & Размер выборки\\ \hline
	\hline 	Boston Housing 	&regression		&14 & 506\\
	\hline	Diabets  				& regression		&20  & 576\\
	\hline	Forest Fires 			& regression		& 13 & 517\\
  	\hline	Servo 					& regression 	& 4   & 167\\
	\hline	NBA				 		& classification	& 12 & 2235\\
\hline
\end{tabular}
\end{table} 


Проводится эксперимент для анализа свойств методов оценки достаточного размера выборки. Эксперимент состоит из трех частей.

В первой части рассматриваются оценки достаточного размера выборки для различных наборов данных с фиксированным набором гиперпараметров различных методов. В качестве данных использовались выборки, описанные в таблице~\ref{chapter:samplesize:experiment:static:table20}. Результаты представлены в таблице~\ref{chapter:samplesize:experiment:static:table2}, где показаны оценки размера выборки для соответствующих выборок.

Во второй части исследуется зависимость достаточного размера выборки от имеющегося размера выборки. В третьей части исследуется поведение методов в зависимости от изменения гиперпараметров методов.
 
\begin{table}[h!t]
\centering
\caption{Сравнение оценок достаточного размера выборки, полученных различными статистическими и байесовскими методами для пяти наборов данных. Результаты демонстрируют значительный разброс оценок между методами, что указывает на различную консервативность подходов.}
\label{chapter:samplesize:experiment:static:table2}
\begin{tabular}{|l|c|c|c|c|c|}
\hline
Методы                       & Boston & Diabetes & Forest Fires & Servo & NBA \\ \hline\hline
Lagrange Multipliers Test & 18             & 25       & 44          & 38    & 218 \\ \hline
Likelihood Ratio Test     & 17             & 25       & 43          & 18    & 110 \\ \hline
Wald Test                 & 66             & 51       & 46          & 76    & 200 \\ \hline
Cross Validation          & 178            & 441      & 172         & 120   & --   \\ \hline
Bootstrap                 & 113            & 117      & 86          & 60    & 405 \\ \hline
APVC                      & 98             & 167      & 351         & 20    & --   \\ \hline
ACC                       & 228            & 441      & 346         & 65    & --   \\ \hline
ALC                       & 98             & 267      & 516         & 25    & --   \\ \hline
Utility Function          & 148            & 172      & 206         & 105   & 925 \\ \hline
\end{tabular}
\end{table}

В данной части вычислительного эксперимента анализируется сходимость различных методов на различных выборках. В эксперименте используются выборки: Boston Housing~\cite{boston1978}, Diabetes, Forest Fires, Servo~\cite{servo1992}, NBA.
Результат анализа представлен в таблице~\ref{chapter:samplesize:experiment:static:table2}. Символ ``--'' обозначает, что исходный размер выборки недостаточный для прогноза.

Гиперпараметры каждого метода для всех выборок описаны в таблице~\ref{chapter:samplesize:experiment:static:table3}. Поскольку критерии Лагранжа, отношения правдоподобия и Вальда асимптотически эквивалентны, то параметры этих методов задавались одинаково. Параметры методов <<Average Coverage>> и <<Average Length>> также задаются одинаково.

\begin{table}[h!t]
\begin{center}
\caption{Гиперпараметры методов оценки достаточного размера выборки, установленные экспертно для экспериментов. Параметры включают уровни значимости $\alpha$, вероятности ошибки второго рода $\beta$, пороговые значения $\varepsilon$ и $l$, а также параметры обобщенных линейных моделей.}
\label{chapter:samplesize:experiment:static:table3}
\begin{tabular}{|l|l|c|c|c|c|c|c|}
\hline 
Method& GLM parameters&~$l$&~$\varepsilon$&~$\alpha$&~$\beta$\\ \hline
\hline	
Lagrange	Multipliers Test	&~$\textbf{w}_{u}^0$ & -- & 0.2& 0.05& 0.2\\
\hline	
Likelihood Ratio Test			&~$\textbf{w}_{u}^0$ & -- & 0.2& 0.05& 0.2\\
\hline	
Wald	Test								&~$\textbf{w}_{u}^0$ & -- & 0.2& 0.05& 0.2\\
\hline	
Cross Validation 					& -- & -- 	& 0.05& -- & --\\
\hline	
Bootstrap 								& -- & 0.5	& -- & 0.05& --\\
\hline	
APVC 									& -- & 0.5	& -- & -- & --\\
\hline	
ACC 									& -- & 0.25	& -- & 0.05& --\\
\hline	
ALC 										& -- & 0.5	& -- & 0.05& --\\
\hline	
Utility function 						& -- & -- 	& 0.005& -- & --\\
\hline
\end{tabular}
\end{center}
\end{table}


Вычислительный эксперимент проводился для анализа описанных методов. Выбирается некоторый размер выборки~$m$ и методом бутстрап семплируется множество подвыборок размером~$m$. Для разных значений~$m$ вычисляется~$m^*$.
    
\begin{figure}[h!t]\center
    \includegraphics[width=0.49\textwidth]{thesis/figures/chapter-3/statical/cross}
    \includegraphics[width=0.49\textwidth]{thesis/figures/chapter-3/statical/apvc}\\
    \includegraphics[width=0.49\textwidth]{thesis/figures/chapter-3/statical/acc}
    \includegraphics[width=0.49\textwidth]{thesis/figures/chapter-3/statical/alc}\\
    \includegraphics[width=0.49\textwidth]{thesis/figures/chapter-3/statical/bootstrap}
    \includegraphics[width=0.49\textwidth]{thesis/figures/chapter-3/statical/kl}\\
    \includegraphics[width=0.49\textwidth]{thesis/figures/chapter-3/statical/maxu}
    \caption{Зависимость статистических значений различных методов определения достаточного размера выборки от размера подвыборки для наборов данных Boston Housing, Diabetes, Forest Fires, Servo и NBA. Все представленные функции монотонны и асимптотически стремятся к константе, что подтверждает корректность методов.}
    \label{chapter:samplesize:experiment:static:fig1}
\end{figure}

\begin{figure}[h!t]\center
    \includegraphics[width=0.85\textwidth]{thesis/figures/chapter-3/statical/graphs}
    \caption{Зависимость оцененного достаточного размера выборки $m^*$ от доступного размера выборки $m$ для различных методов на наборе данных Boston Housing. Результаты демонстрируют сходимость методов и низкую дисперсию оценок, что указывает на вычислительную устойчивость рассмотренных подходов.}
    \label{chapter:samplesize:experiment:static:fig2}
\end{figure}

На рис.~\ref{chapter:samplesize:experiment:static:fig1} демонстрируется зависимость статистических показателей каждого метода для разных выборок с фиксированным размером выборки~$m$. Пороговые значения для каждого метода устанавливаются экспертно, что позволяет контролировать различные статистические характеристики выборки.

Представленные функции являются монотонными и асимптотически стремятся к константе, что подтверждает корректность различных методов определения достаточного размера выборки.

На рис.~\ref{chapter:samplesize:experiment:static:fig2} показаны результаты методов на выборках различного размера. Наблюдается различие методов в дисперсии вычисленного~$m^*$. Все представленные методы демонстрируют сходимость, причем результат предсказания в асимптотике не зависит от доступного размера выборки~$m$.   

Небольшое значение дисперсии интерпретируется как вычислительная устойчивость рассмотренных методов.

Показано, что некоторые методы не дают оценку достаточного размера выборки, если доступный размер выборки недостаточен для применения метода. Это означает, что указанные методы не эффективны с точки зрения прогнозирования необходимого объема данных на ранних этапах эксперимента, однако могут быть использованы для ретроспективного анализа уже проведенных экспериментов.

Анализируется оценка достаточного размера выборки в зависимости от гиперпараметров для байесовских методов, а также эвристических методов. Для анализа рассмотрена выборка Boston Housing.

Байесовские методы используют решающее правило над скалярной функцией для определения достаточного размера выборки.
На рис.~\ref{chapter:samplesize:experiment:static:fig1} показана зависимость скалярных функций от размера подвыборки.
Наблюдается, что указанные функции являются монотонными.
Характер поведения функции определяется выбранным методом. Изменение ограничений, установленных экспертно, позволяет варьировать размер выборки, соответствующий заданным ограничениям.


\subsection{Определение достаточного размера выборки на основе сэмплирования эмпирической функции ошибки}

\begin{figure}[h!t]\center
    \includegraphics[width=\textwidth]{thesis/figures/chapter-3/likelihood-bootstraping/synthetic-regression}
    \caption{Сходимость функций $D(k)$ и $M(k)$ для синтетического набора данных регрессии. Обе функции стремятся к нулю с увеличением размера выборки, что подтверждает теорему~\ref{chapter:samplesize:theorem-kiselev-likelihood-bootstraping}.}
    \label{chapter:samplesize:experiment:likelihood-bootstraping:fig:synthetic-regression}
\end{figure}

\begin{figure}[h!t]\center
    \includegraphics[width=\textwidth]{thesis/figures/chapter-3/likelihood-bootstraping/synthetic-classification}
    \caption{Сходимость функций $D(k)$ и $M(k)$ для синтетического набора данных классификации. Обе функции демонстрируют монотонное убывание к нулю с увеличением размера выборки, подтверждая применимость методов D- и M-достаточности для задач классификации.}
    \label{chapter:samplesize:experiment:likelihood-bootstraping:fig:synthetic-classification}
\end{figure}

\begin{figure}[h!t]\center
    \includegraphics[width=\textwidth]{thesis/figures/chapter-3/likelihood-bootstraping/liver-disorders}
    \caption{Сходимость функций $D(k)$ и $M(k)$ для набора данных Liver Disorders (345 объектов, 5 признаков, $B=1000$ бутстрэп-подвыборок). Обе функции демонстрируют сходимость к нулю, что подтверждает теоретические результаты и демонстрирует применимость методов на реальных данных.}
    \label{chapter:samplesize:experiment:likelihood-bootstraping:fig:liver-disorders}
\end{figure}

\begin{figure}[h!t]\center
    \includegraphics[width=\textwidth]{thesis/figures/chapter-3/likelihood-bootstraping/sufficient-vs-threshold}
    \caption{Зависимость достаточного размера выборки $m^*$ от порогового параметра $\varepsilon$ для методов D- и M-достаточности на трех наборах данных. С увеличением значения порога $\varepsilon$ достаточный размер выборки монотонно уменьшается, что позволяет выбирать меньше объектов для достижения заданного уровня стабильности функций $D(k)$ и $M(k)$.}
    \label{chapter:samplesize:experiment:likelihood-bootstraping:fig:sufficient-vs-threshold}
\end{figure}

\begin{table}[h!t]\center
    \caption{Сравнение оценок достаточного размера выборки методами D- и M-достаточности для 13 наборов данных с задачей регрессии. Методы демонстрируют сопоставимые результаты для большинства наборов данных, при этом M-достаточность иногда требует большего размера выборки.}\label{chapter:samplesize:experiment:likelihood-bootstraping:table}
    \begin{tabular}{|l|c|c|c|c|}
    \hline
    Dataset name & Objects~$m$ & Features~$n$ & D & M \\
    \hline
    Abalone & 4177 & 8 & 96 & 96  \\
    Auto MPG & 392 & 8 & 15 & 15 \\
    Automobile & 159 & 25 & 70 & 156  \\
    Liver Disorders & 345 & 6 & 12 & 19  \\
    Servo & 167 & 4 & 41 &~---  \\
    Forest fires & 517 & 12 & 208 &~--- \\
    Wine Quality & 6497 & 12 & 144 & 144  \\
    Energy Efficiency & 768 & 9 & 24 & 442  \\
    Student Performance & 649 & 32 & 129 & 177  \\
    Facebook Metrics & 495 & 18 & 31 & 388   \\
    Real Estate Valuation & 414 & 7 & 15 & 23  \\
    Heart Failure Clinical Records & 299 & 12 & 63 & 224  \\
    Bone marrow transplant: children & 142 & 36 &~--- &~--- \\
    \hline
    \end{tabular}
\end{table}

В настоящем разделе представлено эмпирическое исследование предложенных методов. Эксперименты проводились на синтетических данных и наборе данных Liver Disorders из~\cite{uci}.

Синтетические данные были сгенерированы из моделей линейной регрессии и логистической регрессии. Количество объектов составляет 1000, количество признаков~---~$20$.
Использовалось~$B=1000$ бутстрэп-подвыборок.
Вычислялись значения~$D(k)$ и~$M(k)$.
Набор данных регрессии Liver Disorders содержит 345 объектов и 5 признаков.
Также использовалось~$B=1000$ подвыборок, полученных методом бутстрэпа, для оценки математического ожидания и дисперсии функции потерь.

На рис.~\ref{chapter:samplesize:experiment:likelihood-bootstraping:fig:synthetic-regression} показаны полученные зависимости между доступным размером выборки~$k$ и предложенными функциями~$D(k)$ и~$M(k)$ для синтетического набора данных регрессии.
Результаты для синтетического набора данных классификации представлены на рис.~\ref{chapter:samplesize:experiment:likelihood-bootstraping:fig:synthetic-classification}.
На рис.~\ref{chapter:samplesize:experiment:likelihood-bootstraping:fig:liver-disorders} представлены графики для набора данных Liver Disorders. Наблюдается, что во всех случаях значения~$D(k)$ и~$M(k)$ приближаются к нулю с увеличением размера выборки.
Эти эмпирические результаты подтверждают полученные ранее теоретические выводы.

В определениях D-достаточности и M-достаточности присутствует гиперпараметр~$\varepsilon$, который соответствует порогу для достаточного размера выборки~$m^*$.
Для изучения зависимости между указанными величинами построена зависимость на рис.~\ref{chapter:samplesize:experiment:likelihood-bootstraping:fig:sufficient-vs-threshold}, которая демонстрирует возможные размеры выборки для обеспечения заданного уровня достоверности.

Для сравнения производительности предложенных методов на различных наборах данных были выбраны выборки из открытого репозитория~\cite{uci}.
Подробная информация о каждом наборе данных, количестве наблюдений и количестве признаков представлена в Таблице~\ref{chapter:samplesize:experiment:likelihood-bootstraping:table}.
В демонстрационных целях было выбрано значение гиперпараметра~$\varepsilon$, при котором значение целевой функции,~$D(k)$ или~$M(k)$, уменьшается вдвое.
Соответствующие результаты представлены в Таблице~\ref{chapter:samplesize:experiment:likelihood-bootstraping:table}.
Пропуски означают, что исходный размер выборки недостаточен.

\subsection{Определение достаточного размера выборки на основе
близости апостериорных распределений}

\begin{figure}[h!t]\center
    \includegraphics[width=\textwidth]{figures/chapter-3/posterior-distribution/eigvals}
    \caption{Зависимость минимального собственного значения $\lambda_{\min}(\mathbf{X}_k^\top \mathbf{X}_k)$ от размера выборки $k$ для синтетических данных регрессии (500 объектов, 10 признаков) и набора данных Liver Disorders (345 объектов, 5 признаков). Асимптотическое поведение соответствует условию теоремы~\ref{chapter:samplesize:theorem-kiselev-posterior-similiarity-linear-regression}: $\lambda_{\min} = \omega(\sqrt{k})$ при $k \to \infty$.}
    \label{chapter:samplesize:experiment:fig:eigvals}
\end{figure}

\begin{figure}[h!t]\center
    \includegraphics[width=\textwidth]{figures/chapter-3/posterior-distribution/synthetic-regression}
    \caption{Сходимость функций $KL(k)$ и $S(k)$ для синтетического набора данных регрессии. Функция $KL(k)$ стремится к нулю, а $S(k)$ стремится к единице, что подтверждает теоремы~\ref{chapter:samplesize:theorem-kiselev-posterior-similiarity} и~\ref{chapter:samplesize:theorem-kiselev-posterior-similiarity-s-score}.}
    \label{chapter:samplesize:experiment:fig:synthetic-regression}
\end{figure}

\begin{figure}[h!t]\center
    \includegraphics[width=\textwidth]{figures/chapter-3/posterior-distribution/liver-disorders}
    \caption{Сходимость функций $KL(k)$ и $S(k)$ для набора данных Liver Disorders (345 объектов, 5 признаков, нормальное априорное распределение, $B=100$ повторений). Результаты демонстрируют сходимость $KL(k)$ к нулю и $S(k)$ к единице, подтверждая теоретические предсказания и применимость методов KL- и S-достаточности на реальных данных.}
    \label{chapter:samplesize:experiment:fig:liver-disorders}
\end{figure}

\begin{figure}[h!t]\center
    \includegraphics[width=\textwidth]{figures/chapter-3/posterior-distribution/sufficient-vs-threshold}
    \caption{Зависимость достаточного размера выборки $m^*$ от порогового параметра $\varepsilon$ для методов KL- и S-достаточности на синтетических данных регрессии и наборе данных Liver Disorders. Метод S-достаточности требует более низких значений порога для достижения заданного уровня близости распределений, что указывает на его более строгие требования к качеству оценки.}
    \label{chapter:samplesize:experiment:fig:sufficient-vs-threshold}
\end{figure}

\begin{table}[h!t]\center
    \caption{Характеристики выборок, используемых для сравнения методов определения достаточного размера выборки на основе близости апостериорных распределений. Все наборы данных соответствуют задаче регрессии и используются для оценки методов KL- и S-достаточности.}\label{chapter:samplesize:experiment:table:descr}
    \begin{tabular}{lcc}
    \hline
        \toprule
        Выборка & Количество признаков,~$n$ & Количество объектов,~$m$ \\ 
        \midrule
        Boston Housing & 14 & 506 \\ 
        Diabetes & 10 & 576 \\ 
        Forest Fires & 13 & 517 \\ 
        Servo & 4 & 167 \\ 
        \bottomrule
    \end{tabular}
\end{table}

\begin{table}[h!t]\center
    \caption{Сравнение оценок достаточного размера выборки, полученных классическими методами и предложенными методами KL- и S-достаточности для четырех наборов данных регрессии. Метод KL-достаточности дает более консервативные оценки, требующие почти полной выборки, в то время как S-достаточность указывает на минимальные размеры выборки.}\label{chapter:samplesize:experiment:table:results}
    \begin{tabular}{lcccc}
    \hline
        \toprule
        Methods and sample sets & Boston & Diabetes & Forest Fires & Servo \\ 
        \midrule
        Lagrange Multipliers Test & 18 & 25 & 44 & 38 \\
        Likelihood Ratio Test & 17 & 25 & 43 & 18 \\
        Wald Test & 66 & 51 & 46 & 76 \\ 
        Cross Validation & 178 & 441 & 171 & 120 \\ 
        Bootstrap & 113 & 117 & 86 & 60 \\ 
        APVC & 98 & 167 & 351 & 20 \\ 
        ACC & 228 & 441 & 346 & 65 \\ 
        ALC & 98 & 267 & 516 & 25 \\ 
        Utility function & 148 & 172 & 206 & 105 \\ 
        \midrule
        KL (ours) & 493 & 437 & 86 & 165 \\ 
        S (ours) & 28 & 22 & 26 & 10 \\
        \bottomrule
    \end{tabular}
\end{table}

\begin{figure}[h!t]\center
    \includegraphics[width=\textwidth]{figures/chapter-3/posterior-distribution/dependence_on_available_sample_set.pdf}
    \caption{Зависимость оцененного достаточного размера выборки $m^*$ от доступного размера выборки $m$ для классических методов и предложенных методов KL- и S-достаточности на наборе данных Boston Housing. Критерий KL-достаточности является наиболее консервативным и требует почти полной выборки, в то время как S-достаточность указывает на минимальные размеры, что связано с высокой чувствительностью расхождения Кульбака-Лейблера к изменениям распределений.}
    \label{chapter:samplesize:experiment:fig:dependence_on_available_sample_set}
\end{figure}

В настоящем разделе представлено расширенное эмпирическое исследование предложенных методов определения достаточного размера выборки на основе близости апостериорных распределений. Эксперименты состоят из трех частей.

В первой части проверяются сходимости, полученные в ходе теоретического анализа. А именно, сначала рассматривается поведение минимального собственного значения матрицы~$\mathbf{X}_k^\top \mathbf{X}_k$ при увеличении размера выборки, что необходимо для выполнения условий теоремы~\ref{chapter:samplesize:theorem-kiselev-posterior-similiarity-linear-regression}. Затем исследуется сходимость предложенных функций~$KL(k)$ и~$S(k)$ к их предельным значениям. Наконец, изучается зависимость достаточного размера выборки от пороговых параметров~$\varepsilon$. Эксперимент проводится на двух наборах данных: синтетическая регрессия и Liver Disorders.

Во второй части оцениваются размеры выборок для различных наборов данных, используя разные подходы (KL- и S-достаточность, а также классические методы). В третьей части изучается зависимость достаточного размера выборки от объема доступных данных, что позволяет оценить стабильность методов при различных объемах выборок.

Синтетические данные генерируются из модели линейной регрессии. Количество объектов составляет 500, количество признаков~--- 10. Для генерации синтетического набора данных регрессии исходные признаки, параметры модели и шумовые остатки генерируются из стандартного нормального распределения. 

Априорное распределение параметров также задано как стандартное нормальное, как для синтетической регрессии, так и для набора данных Liver Disorders, который содержит 345 объектов и 5 признаков. Входные признаки предобрабатываются с использованием стандартного метода масштабирования данных~(англ. Standard Scaler).

Процедура эксперимента организована следующим образом. Один объект последовательно удалялся из заданной выборки до тех пор, пока количество объектов в подвыборке не становилось равным количеству признаков. Для каждого размера выборки~$k$ вычисляется минимальное собственное значение матрицы~$\mathbf{X}_k^\top \mathbf{X}_k$, а также значения~$KL(k)$ и~$S(k)$. Этот процесс повторялся~$B=100$ раз для обеспечения статистической надежности результатов.

На рис.~\ref{chapter:samplesize:experiment:fig:eigvals} показано асимптотическое поведение минимального собственного значения матрицы~$\mathbf{X}_k^\top \mathbf{X}_k$ при увеличении размера выборки.
Наблюдается, что при стремлении размера выборки к бесконечности минимальное собственное значение также стремится к бесконечности.
При этом, как и требуется для теоремы~\ref{chapter:samplesize:theorem-kiselev-posterior-similiarity-linear-regression}, график лежит выше~$\sqrt{k}$.

На рис.~\ref{chapter:samplesize:experiment:fig:synthetic-regression} представлены зависимости между доступным размером выборки~$k$ и предложенными функциями~$KL(k)$ и~$S(k)$ для синтетического набора данных регрессии.
На рис.~\ref{chapter:samplesize:experiment:fig:liver-disorders} представлены аналогичные графики для набора данных Liver Disorders.
Наблюдается, что в обоих случаях значение~$KL(k)$ приближается к нулю с увеличением размера выборки, а~$S(k)$ стремится к единице.
Эти эмпирические результаты подтверждают полученные ранее теоретические выводы.

В определениях KL-достаточности и S-достаточности присутствует гиперпараметр~$\varepsilon$, который соответствует порогу для достаточного размера выборки~$m^*$.
На рис.~\ref{chapter:samplesize:experiment:fig:sufficient-vs-threshold} показана зависимость между уровнем достоверности и размером выборки.
Для сравнения предложенных методов с базовыми использовалась следующая схема эксперимента.
Модель машинного обучения~--- линейная регрессия.
Выбрано 4 набора данных с задачей регрессии из открытых источников: Boston, Diabetes, Forestfires и Servo.
Их описательная статистика представлена в Таблице~\ref{chapter:samplesize:experiment:table:descr}.
К данным применяются 9 различных базовых методов оценки размера выборки: Тест множителей Лагранжа, Тест отношения правдоподобий, Тест Вальда, Кросс-валидация, Бутстрэп, Критерий средней апостериорной дисперсии (APVC), Критерий среднего покрытия (ACC), Критерий средней длины (ALC) и Функция полезности.

Далее в экспериментах достаточность определяется в терминах относительного изменения, а именно, размер выборки достаточным, если функция~$KL(k)$ имеет относительное отклонение от своего значения на всей выборке не более чем~$\varepsilon$.
Аналогично с функцией~$S(k)$, зафиксировав~$\varepsilon=0.05$ получаем результирующие размеры выборок.

Результаты в Таблице~\ref{chapter:samplesize:experiment:table:results} указывают на то, что критерий на основе расхождения Кульбака-Лейблера является более консервативным и требует большего размера выборки, в то время как критерий S-достаточности предполагает, что минимального размера выборки может быть достаточно.
Предполагается, что это типичный результат для функции схожести s-score, которая была разработана для сравнения различных моделей машинного обучения, особенно в случаях с неинформативными распределениями.
Если распределения имеют высокую дисперсию, функция близости приближается к единице, что приводит к тому, что критерий считает достаточным даже небольшой размер выборки.

Далее проводится комплексный анализ различных методов определения размера выборки.
Анализируется зависимость достаточного размера выборки от объема доступного набора данных.
В частности, при увеличении объема доступной выборки вычисляется достаточный размер на основе различных методов.
Результаты представлены на рис.~\ref{chapter:samplesize:experiment:fig:dependence_on_available_sample_set}, который сравнивает вышеупомянутые методы с точки зрения их консервативности.

Наблюдается, что S-достаточный размер выборки часто является минимальным.
KL-достаточный размер выборки, как правило, требует почти полной выборки.
Предполагается, что это связано с тем, что расхождение Кульбака-Лейблера чрезвычайно чувствительно к изменениям математического ожидания и дисперсии сравниваемых распределений.
Таким образом, стабилизация расстояния между ними происходит довольно поздно.

\section{Заключение по главе}

В настоящей главе решена задача разработки практических методов определения достаточного размера выборки для задач машинного обучения, что восполняет пробел между теоретическим аппаратом, разработанным в главах~\ref{chapter:complexity} и~\ref{chapter:gesian}, и практическими потребностями планирования экспериментов. В отличие от теоретических оценок, основанных на анализе матриц Гессе, предложенные методы используют наблюдаемые характеристики процесса обучения, что делает их применимыми в реальных задачах.

В главе проведен систематический обзор существующих подходов к определению достаточного размера выборки, включая статистические методы, байесовские методы и эвристические методы. Выявлены ключевые ограничения классических подходов: необходимость знания дисперсии оценки параметра или параметра нецентральности, отсутствие алгоритмических процедур для их получения, а также вычислительная сложность для моделей глубокого обучения.

Основным теоретическим вкладом главы является разработка двух новых методов определения достаточного размера выборки, основанных на анализе стабильности процесса обучения.

Первый метод основан на анализе функции правдоподобия при изменении объема данных. Введены два критерия достаточности: D-достаточность, использующая дисперсию функции правдоподобия на бутстрэп-подвыборках, и M-достаточность, анализирующая разность математических ожиданий функции правдоподобия при последовательном добавлении объектов в выборку. Теорема~\ref{chapter:samplesize:theorem-kiselev-likelihood-bootstraping} строго доказывает корректность определения M-достаточного размера выборки для модели линейной регрессии при выполнении условий сходимости математических ожиданий и ковариационных матриц оценок параметров. Следствие~\ref{chapter:samplesize:corollary-kiselev-likelihood-bootstraping} устанавливает достаточные условия сходимости к истинным значениям параметров и информационной матрице Фишера.

Второй метод основан на анализе близости апостериорных распределений параметров модели на близких подвыборках, отличающихся одним объектом. Введены критерии KL-достаточности и S-достаточности, использующие расхождение Кульбака-Лейблера и функцию схожести s-score соответственно. Для нормального апостериорного распределения получены аналитические выражения для этих мер близости, что позволило провести строгий теоретический анализ. Теорема~\ref{chapter:samplesize:theorem-kiselev-posterior-similiarity} доказывает корректность определения KL-достаточного размера выборки при сходимости математических ожиданий и ковариационных матриц апостериорных распределений. Теорема~\ref{chapter:samplesize:theorem-kiselev-posterior-similiarity-s-score} устанавливает корректность S-достаточности при более слабых условиях~--- требуется только сходимость математических ожиданий. Теорема~\ref{chapter:samplesize:theorem-kiselev-posterior-similiarity-linear-regression} для модели линейной регрессии с нормальным априорным распределением устанавливает достаточные условия сходимости моментов апостериорного распределения при условии роста минимального собственного значения матрицы $\mathbf{X}_k^\top \mathbf{X}_k$ как $\omega(\sqrt{k})$ при $k \to \infty$.

Проведены обширные вычислительные эксперименты на синтетических и реальных данных, подтвердившие эффективность предложенных методов. Эксперименты на синтетических данных регрессии и классификации, а также на наборах данных Liver Disorders, Boston Housing, Diabetes, Forest Fires, Servo и других продемонстрировали сходимость функций $D(k)$, $M(k)$ и $KL(k)$ к нулю, а функции $S(k)$~--- к единице с ростом объема выборки, что согласуется с теоретическими предсказаниями. Сравнительный анализ с классическими методами выявил особенности различных критериев: KL-дивергенция дает более консервативные оценки, требующие почти полной выборки, в то время как S-достаточность часто указывает на достаточность минимального размера выборки, что связано с высокой чувствительностью расхождения Кульбака-Лейблера к изменениям распределений.

Практические рекомендации включают использование относительных отклонений от значений на полной выборке с порогами $0.05-0.1$ для методов на основе функции правдоподобия, а также применение бутстрэпирования с $B=1000$ подвыборок для оценки математических ожиданий и дисперсий. Для методов на основе близости апостериорных распределений рекомендуется использовать порог $\varepsilon=0.05$ для S-достаточности.

Основные ограничения методов связаны с вычислительной сложностью обращения ковариационных матриц для моделей с большим количеством параметров, а также с предположением о нормальности апостериорного распределения для методов KL- и S-достаточности. Кроме того, теоретическое обоснование методов в настоящее время ограничено моделями линейной регрессии и логистической регрессии.

Перспективными направлениями дальнейших исследований являются расширение теоретического обоснования методов на более сложные модели, в том числе нейронные сети, преодоление ограничения о нормальности апостериорного распределения, разработка приближенных методов для оценки близости распределений в высокомерных пространствах параметров, а также интеграция предложенных методов с теоретическими оценками ландшафтной меры сложности из главы~\ref{chapter:complexity}.

Полученные результаты создают основу для разработки практических инструментов оценки достаточного объема данных в прикладных задачах машинного обучения и обеспечивают мост между теоретическим формализмом сложности моделей и данных, разработанным в предыдущих главах, и практическими потребностями планирования экспериментов.

% Четвертая глава
\clearpage
\chapter{Методы снижения сложности нейросетевых архитектур}
В данной главе рассмотрим методы снижения сложности параметрических моделей глубокого обучения.
Предполагается, что число параметров нейросети можно существенно снизить без значимой потери качества и значимого повышения дисперсии функции ошибки.
Предлагаются методы снижения сложности моделей на основе ковариационной матрицы градиентов функции ошибки по параметрам модели.

\section{Удаления параметров моделей глубокого обучения}

Задана выборка:
\[
    \label{ch-5:eq:st:1}
    \begin{aligned}
    \mathfrak{D} = \bigr\{\bigr(\textbf{x}_i, y_i\bigr)\bigr\}_{i=1}^{m}, \quad \textbf{x}_{i} \in \mathbb{X} = \mathbb{R}^{n}, \quad y_i \in \mathbb{Y},
    \end{aligned}
\]
где $n$~--- размерность признакового пространства, $m$~--- число объектов в выборке. Пространство ответов $\mathbb{Y} = \mathbb{R}$ в случае задачи регрессии и  $\mathbb{Y} = \{1,\ldots, R\}$ в случае задачи классификации, где $R$~--- число классов.

Задано семейство моделей параметрических функций с наперед заданной структурой:
\[
    \label{ch-5:eq:st:2}
    \begin{aligned}
    \mathfrak{F} &= \bigr\{f\bigr(\textbf{w}\bigr):\mathbb{X} \to \mathbb{Y} | \textbf{w} \in \mathbb{R}^{p}\bigr\}, \\ 
    \mathbf{h}\bigr(\textbf{w}, \textbf{x}\bigr) &= \textbf{W}_1\bm{\sigma}\bigr(\textbf{W}_2\bm{\sigma}\bigr(\ldots\bm{\sigma}\bigr(\textbf{W}_r\textbf{x}\bigr)\ldots\bigr)\bigr),\\
    f_{\text{\text{cl}}}\bigr(\textbf{w}, \textbf{x}\bigr) &= \arg \max_{j \in \bigr\{1,\ldots, R\bigr\}} \text{softmax}\bigr(\mathbf{h}\bigr(\textbf{w}, \textbf{x}\bigr)\bigr)_{j}, \\ 
    f_{\text{reg}}\bigr(\textbf{w}, \textbf{x}\bigr) & = \mathbf{h}\bigr(\textbf{w}, \textbf{x}\bigr), 
    \end{aligned}
\]
где $p$~--- размерность пространства параметров, $r$~--- число слоев нейросети, $\textbf{w} = \text{vec}[\textbf{W}_1, \textbf{W}_2, \ldots, \textbf{W}_r]$, а $\bm{\sigma}$~--- функция активации. В случае задачи регрессии структура модели имеет вид $f_{\text{\text{reg}}}$, а в случае классификации имеет вид $f_{\text{\text{cl}}}$.
Задана функция потерь:
\[
    \label{ch-5:eq:st:3}
    \begin{aligned}
    \mathcal{L}\bigr(\textbf{w}, \mathfrak{D}\bigr) &= \frac{1}{m}\sum_{i=1}^{m}l\bigr(\textbf{x}_{i}, y_i, \textbf{w}\bigr),\\
    l_{\text{\text{reg}}}\bigr(\textbf{x}, y, \textbf{w}\bigr) &= \bigr(y - f\bigr(\textbf{w}, \textbf{x}\bigr)\bigr)^{2},\\
    l_{\text{\text{cl}}}\bigr(\textbf{x}, y, \textbf{w}\bigr) &= -\sum_{j=1}^{R}\bigr([y = j]\ln\text{softmax}_j\bigr(\mathbf{h}\bigr(\textbf{w}, \textbf{x}\bigr)\bigr)\bigr),
    \end{aligned}
\]
где $l_{\text{\text{reg}}}$~--- это функция ошибки на одном элементе для задачи регрессии, $l_{\text{\text{cl}}}$~--- для задачи классификации.
Оптимальный вектор параметров $\hat{\textbf{w}}$ получим минимизацией функции потерь:
\[
    \label{ch-5:eq:st:0:1}
    \begin{aligned}
    \hat{\textbf{w}} = \arg \min_{\textbf{w}\in\mathbb{R}^{p}} \mathcal{L}\bigr(\textbf{w}, \mathfrak{D}\bigr).
    \end{aligned}
\]

Для поиска оптимальных параметров модели используется градиентный метод оптимизации:
\[
    \label{ch-5:eq:st:4}
    \begin{aligned}
    \textbf{w}_{t} = \textbf{w}_{t-1} + \Delta\textbf{w}\bigr(\textbf{g}_{S,t}, \textbf{w}_{t-1}, \textbf{w}_{t-2}, \ldots\bigr), \quad \textbf{g}_{S,t}=\frac{\partial \mathcal{L}\bigr(\textbf{w}_{t}, \textbf{X}_{S}, \textbf{Y}_{S}\bigr)}{\partial \textbf{w}},
    \end{aligned}
\]
где $t$~--- номер итерации, $\textbf{g}_{S,t}$~--- значение градиента на подвыборке размера $S$, $\Delta\textbf{w}$~--- приращение вектора параметров.
 
 
Порядок на множестве параметров модели задается при помощи ковариационной матрицы $\textbf{C}$ градиентов функции ошибки $\mathcal{L}$ по параметрам модели $\textbf{w}$. Для вычисления ковариационной матрицы $\textbf{C}$ используется итерационная формула~\cite{Chunyan2016}, которая вычисляется на каждой итерации \eqref{ch-5:eq:st:4} градиентного метода оптимизации параметров:
\[
    \label{ch-5:eq:st:5}
    \begin{aligned}
    \textbf{C}_t = \bigr(1-\kappa_t\bigr)\textbf{C}_{t-1}+\kappa_t\bigr(\textbf{g}_{1,t}-\textbf{g}_{S,t}\bigr)\bigr(\textbf{g}_{1,t}-\textbf{g}_{S,t}\bigr)^{\mathsf{T}},
    \end{aligned}
\]
 где $t$~--- номер итерации, $\textbf{g}_{S,t}$~--- значение градиента на подвыборке размера $S$, $\textbf{g}_{1,t}$~--- значение градиента на первом элементе подвыборки, $\kappa_t=\frac{1}{t}$~--- параметр сглаживания, $\textbf{C}_0$ инициализируются из равномерного распределения.
 
Пусть известно $t_0$~--- число итераций, после которого все параметры находятся в некоторой локальной окрестности минимума, тогда, как показано в работе~\cite{Chunyan2016}, матрица $\textbf{C}_{t_0}$ аппроксимирует истинную ковариационную матрицу $\textbf{C}$. Ковариационная матрица $\textbf{C}_{t_0}$ используется для упорядочения параметров модели $\textbf{w}_{t_0}$. 
 
Пусть $\mathcal{I}$~---  упорядоченный вектор индексов $[1, 2, \ldots, p]$. Обозначим $\mathcal{I}_{\textbf{w}_{t_0}}$ вектор индексов, порядок которого задан при помощи ковариационной матрицы $\textbf{C}_{t_0}$. 
 
Например, если ковариационная матрица $\textbf{C}_{t_0}$  имеет вид
$$
    \begin{bmatrix}
    0{,}3& 0 & 0\\
    0& 0{,}2 & 0\\
    0& 0 & 0{,}25\\
    \end{bmatrix},
$$
 то вектор индексов $\mathcal{I}_{\textbf{w}_{t_0}} = [3,1,2]$.
 
\paragraph{Фиксация параметров модели в процессе обучения.}
Для фиксации параметров $\textbf{w}_{t_0}$ при помощи вектора индексов $\mathcal{I}_{\textbf{w}_{t_0}}$ используется бинарный вектор $\bm{\alpha}\bigr(\zeta\bigr)$:
\[
    \label{ch-5:eq:st:6}
    \begin{aligned}
    \alpha_i\bigr(\zeta\bigr) = \begin{cases}
       1, &\text{если }\mathcal{I}_{\textbf{w}_{t_0}}[j] \leq \zeta;\\
       0 &\text{иначе},
     \end{cases}
    \end{aligned}
\]
 где $\zeta$~--- число фиксирующих параметров.
 
 Учитывая \eqref{ch-5:eq:st:6}, уравнение \eqref{ch-5:eq:st:4} приводится к виду
\[
    \label{ch-5:eq:st:7}
    \begin{aligned}
    \textbf{w}_{t} = \textbf{w}_{t-1} + \bm{\alpha}\bigr(\zeta\bigr)\cdot\Delta\textbf{w}\bigr(\textbf{g}_{S,t}, \textbf{w}_{t-1}, \textbf{w}_{t-2}, \ldots\bigr),
    \end{aligned}
\]
где $t$~--- номер итерации, $\textbf{g}_{S,t}$~--- значение градиента на подвыборке размера $S$, $\Delta\textbf{w}$~--- приращение вектора параметров. После умножения на бинарный вектор $\bm\alpha$ часть параметров не оптимизируется, что приводит к фиксации параметров.


Предлагается метод основанный на модификации метода Белсли. Пусть $\textbf{w}$~--- вектор параметров доставляющий минимум функционалу потерь $\mathcal{L}$ на  множестве $\mathbb{W_\mathcal{A}}$, а $\textbf{A}_\text{ps}$ соответствующая ему ковариационная матрица.

Выполним сингулярное разложение матрицы
\[
\textbf{A}_\text{ps} = \textbf{U}{\bf\Lambda}\textbf{V}^\mathsf{T}.
\]
Индекс обусловленности $\eta_{j}$ определим как отношение максимального элемента к $j$-му элементу матрицы ${\bf\Lambda}$. Для нахождения мультиколлинеарных признаков требуется найти индекс $\xi$ вида:
\[
\xi = \arg\max_{j\in \mathcal{A}}{\eta_j}.
\]

\begin{figure}[h!t]\center
    \subfloat[Матрица ковариации]{\includegraphics[width=0.5\textwidth]{thesis/figures/chapter-4/belsli/Cov}}
    \subfloat[Дисперсионные доли]{\includegraphics[width=0.5\textwidth]{thesis/figures/chapter-4/belsli/BelslyImage}}
    \caption{Илюстрация метода Белсли для анализа мультиколлинеарности параметров}
    \label{CovBel}
\end{figure}

\begin{table}[h!t]
    \begin{center}
    \caption{Илюстрация метода Белсли для анализа мультиколлинеарности параметров}
    \begin{tabular}{|c|cccccc|}
    \hline
    $\eta$ & $q_1$& $q_2$& $q_3$& $q_4$& $q_5$& $q_6$\\
    \hline
    $1.0$ &  $2\cdot 10^{-17}$ &  $4\cdot 10^{-17}$ &  $1\cdot 10^{-16}$ &  $2\cdot 10^{-17}$ &  $6\cdot 10^{-17}$&  $3\cdot 10^{-4}$ \\
    \hline
    $1.5$ &  $5\cdot 10^{-17}$ &  $9\cdot 10^{-17}$ &  $2\cdot 10^{-16}$ &  $5\cdot 10^{-17}$ &  $3\cdot 10^{-20}$ &  $3\cdot 10^{-2}$ \\
    \hline
    $3.3$ &  $9\cdot 10^{-18}$ &  $1\cdot 10^{-17}$ &  $2\cdot 10^{-17}$ &  $9\cdot 10^{-18}$ &  $2\cdot 10^{-19}$ &  $9\cdot 10^{-1}$ \\
    \hline
    $2\cdot 10^{15}$ &  $1\cdot 10^{-2}$ &  $1\cdot 10^{-1}$ &  $8\cdot 10^{-1}$ &  $2\cdot 10^{-3}$ &  $9\cdot 10^{-2}$ &  $1\cdot 10^{17}$ \\ 
    \hline
    $8\cdot 10^{15}$ &  $6\cdot 10^{-2}$ &  $8\cdot 10^{-1}$ &  $9\cdot 10^{-2}$ &  $8\cdot 10^{-2}$ &  $9\cdot 10^{-1}$ & $ 2\cdot 10^{17} $\\
    \hline
    $1\cdot 10^{16}$ &  $\bf9\cdot 10^{-1}$ &  $1\cdot 10^{-2}$& $ 4\cdot 10^{-2}$&  $\bf9\cdot 10^{-1}$ &  $1\cdot 10^{-3}$ & $ 5\cdot 10^{-21}$ \\
    \hline
    \end{tabular}
    \label{CovBelTable}
    \end{center}
\end{table}

Дисперсионный долевой коэффициент $q_{ij}$ определим как вклад $j$-го признака в дисперсию $i$-го элемента вектора параметра $\textbf{w}$:

\[
q_{ij} = \frac{u^2_{ij}/\lambda_{jj}}{\sum^n_{j=1}{u^2_{ij}/\lambda_{jj}}}.
\]

Большие значение дисперсионных долей указывают на наличие зависимости между параметрами. Находим долевые коэффициенты, которые вносят максимальный вклад в дисперсию параметра $w_\xi$:

\[
\zeta = \arg\max_{j\in \mathcal{A}}{q_{\xi j}}.
\]
Параметр с индексом $\zeta$ определим как наименее релевантный параметр нейросети.

Проиллюстрируем принцип работы метода Белсли на примере. Гипотеза порождения данных: 
\[
    \textbf{w} = \begin{bmatrix}
    \text{sin}(x)\\
    \text{cos}(x)\\
    \text{2+cos}(x)\\
    \text{2+sin}(x)\\
    \text{cos}(x) + \text{sin}(x)\\
    x
    \end{bmatrix}
\]
с матрицей ковариации на рис. \ref{CovBel}.a, где $x \in [0.0, 0.02, ..., 20.0]$.


В табл. \ref{CovBelTable} приведены индексы обусловленности и соответствующие им дисперсионные доли, которые также изображены на рис. \ref{CovBel}.b. Согласно этим данным, максимальный индекс обусловленности $\eta_6 = 1.2\cdot 10^{16}$. Ему соответствуют максимальные дисперсионные доли признаков с индексами 1 и 4, которые, как видно из построения выборки, являются линейно зависимые.


\section{Дистилляция моделей глубокого обучения на многодоменных данных}
\begin{figure}[h!t]\center
    {\includegraphics[width=0.4 \textwidth]{thesis/figures/chapter-4/multidomain-distillation/Distillation_only}}
    \caption{Базовая дистилляция моделей глубокого обучения}
    \label{dist_only}
\end{figure}

\begin{figure}[h!t]\center
    {\includegraphics[width=0.4 \textwidth]{thesis/figures/chapter-4/multidomain-distillation/Distillation_DA}}
    \caption{Дистилляция моделей глубокого обучения с доменной адаптацией}
    \label{dist_da}
\end{figure}

\begin{definition}
Генеральная совокупность объектов~$B$ называется близкой к генеральной совокупности~$A$, если существует инъективное отображение $\varphi: A \rightarrow B$.
\end{definition}

Предлагается использовать, помимо меток учителя на одном из доменов, связь между доменами при обучении модели ученика.
В этом случае в качестве доменов должны выступать близкие генеральные совокупности.

На Рис.~\ref{dist_only} показан процесс обучения модели ученика в базовой постановке задачи дистилляции. Модель учителя обучается на большом наборе данных из генеральной совокупности~$A$, затем ее выходы используются для обучения студенческой модели на меньшем наборе данных из того же домена.
На Рис.~\ref{dist_da} представлен предложенный метод, который задействует выходы модели учителя, обученной на другом домене, и связь между доменами.

Базовая постановка задачи дистилляции. Задан набор данных
$$\mathfrak{D}=\{(\mathbf{x_i}, \mathbf{y_i})\}_{i=1}^n,
\quad \mathbf{x_i} \in \mathbb{X},
\quad \mathbf{y_i} \in \{1,...,R\},$$
где $R$ - количество классов в задаче классификации.

Предполагается, что задана обученная модель с большим количеством параметров --- модель учителя. И требуется обучить студенческую модель с меньшим количеством параметров, учитывая ответы учителя. Модель учителя $\mathbf{f}$ и студенческая модель $\mathbf{g}$ принадлежат параметрическому семейству функций: $$\mathfrak{F}=\{\mathbf{f}|\mathbf{f}=\text{softmax}(\mathbf{v(x)} /T), \mathbf{v}:\mathbb{R}^{n}\rightarrow \mathbb{R}^{R}\}.$$

где $\mathbf{v}$ - дифференцируемая параметрическая функция заданной структуры, $T$ - параметр температуры.

Функция потерь $\mathcal{L}$, учитывающая модель учителя $\mathbf{f}$ при выборе студенческой модели $\mathbf{g}$, имеет вид:
\[
\begin{aligned}
     \mathcal{L}(\mathbf{w,X,Y,f})=&-\sum\limits_{i=1}^{m}\sum\limits_{r=1}^{R}y_{i }^{r}\log{g^{r}(x_{i})}\bigr|_{T=1}\\
     &-\sum\limits_{i=1}^{m}\sum\limits_{r=1}^{R}f^{r}(x_{i})\bigr|_{T=T_{0} }\log{g^{r}(x_{i})}\bigr|_{T=T_{0}},
\end{aligned}
\]
где $\cdot\bigr|_{T=t}$ означает, что параметр температуры $T$ в предыдущей функции равен $t$.

Получаем задачу оптимизации:
$$\hat{\mathbf{w}} = \arg\min_{\mathbf{w} \in \mathbb{W}} \mathcal{L}(\mathbf{w,X,Y,f}).$$

Постановка задачи дистилляции для многодоменной выборки. Даны две выборки:
$$\mathfrak{D}_{\text{s}}=\{(\mathbf{x_i},\mathbf{y_i})\}_{i=1}^n,
\quad \mathbf{x_i} \in \mathbb{X}_{\text{s}},
\quad \mathbf{y_i} \in \mathbb{Y}$$
$$\mathfrak{D}_{\text{t}}=\{(\mathbf{x_i},\mathbf{y_i})\}_{i=1}^m, \quad \mathbf {x_i} \in \mathbb{X}_{\text{t}},
\quad \mathbf{y_i} \in \mathbb{Y},$$
где $\mathfrak{D}_{\text{s}}, \mathfrak{D}_{\text{t}}$~--- исходный и целевой наборы данных. В базовой постановке задачи дистилляции предполагается, что
$\mathfrak{D}_{\text{t}} \subset \mathfrak{D}_{\text{s}},
\mathbb{X}_{\text{t}}=\mathbb{X}_{\text{s}}$. Предполагается, что количество объектов в наборах данных не совпадает:
$$n \gg m$$

Пусть задана модель учителя на выборке большей мощности:
$$\mathbf{f}: \mathbb{X}_{\text{s}} \rightarrow \mathbb{Y}^{\prime},$$
где $\mathbf{f}$ - модель учителя, $\mathbb{Y}^{\prime}$ - пространство вероятностей классов.

Связь между исходной и целевой выборками задается:
$$\varphi: \mathbb{X}_{\text{t}} \rightarrow \mathbb{X}_{\text{s}},$$
где $\varphi$ - инъективное отображение. Требуется получить студенческую модель для малоресурсной выборки:
$$\mathbf{g}: \mathbb{X}_{\text{t}} \rightarrow \mathbb{Y}^{\prime},$$
где $\mathbf{g}$ - студенческая модель.

В работе рассматривается функция потерь, учитывающая метки учителя и связь между доменами:
\begin{align}
    &\mathcal{L}(\mathbf{w,X,Y,f,\varphi})=\\&
    -\lambda\sum\limits_{i=1}^{m}\sum\limits_{r=1}^{ R}\mathbb{I}[y_{i}=r]\log{g^{r}(\mathbf{x}_{i},\mathbf{w})} \\&-(1-\lambda)\sum\limits_{i=1}^{m}\sum\limits_{r=1}^{R}(f\circ \varphi)^{r}(\mathbf{x }_{i})\log{g^{r}(\mathbf{x}_{i},\mathbf{w})},
\end{align}
где $\lambda$ - метапараметр, задающий вес дистилляции, $\mathbb{I}$ - индикаторная функция.

Получаем задачу оптимизации:
$$\hat{\mathbf{w}} = \arg\min_{\mathbf{w} \in \mathbb{W}} \mathcal{L}(\mathbf{w,X,Y,f,\varphi} ).$$


\section{Анти-Дистилляция моделей глубокого обучения}
В этом разделе описывается постановка задачи анти-дистилляции для задачи классификации.
Отметим, что аналогичный подход может быть применен для произвольных задач.

Даны два набора данных
\begin{align}
    \mathfrak{D}_1 = \{(\mathbf{x}_i, y_i)\}_{i=1}^{m_1},~\mathbf{x}_i \in \mathbb{R}^{n},~y_i \in C_1 = \{1, \dots, c_1\}, \\
    \mathfrak{D}_2 =  \{(\mathbf{x}_i, y_i)\}_{i=1}^{m_2},~\mathbf{x}_i \in \mathbb{R}^{n},~y_i \in C_2 = \{1, \dots, c_2\},
\end{align}
где $m_1$ и $m_2$ - количество объектов в $\mathfrak{D}_1$ и $\mathfrak{D}_2$ соответственно, $n$ - размерность входного пространства. $C_1$ и $C_2$ - множества меток классов $1, \dots, c_1, \dots, c_2$.

Предполагается, что объекты $\mathbf{x}_i$ порождены из генеральной совокупности, общей для обоих наборов данных $\mathfrak{D}_1, \mathfrak{D}_2$, и имеют схожие свойства для этих наборов.
Также предположим, что набор данных $\mathfrak{D}_2$ является более сложным для классификации и требует более сложной модели классификации.

Задана модель учителя $\mathbf{g}_\text{tr}$, обученная на первом наборе данных $\mathfrak{D}_1$:
\[
    \mathbf{g}_\text{tr}: \mathbb{R}^{n} \rightarrow \Delta^{c_1},\quad \mathbf{g}_\text{tr}(\mathbf{x}) = \mathbf{g}(\mathbf{x}, \hat{\mathbf{u}}),
\] 
где $\Delta^c$ - множество $c$-мерных вероятностных векторов,

Параметры модели учителя $\mathbf{g}_\text{tr}$ определяются следующим образом:
$$\hat{\mathbf{u}} =  \underset{\mathbf{u}}{\arg\min}~\mathcal{L}_\text{ce}(\mathbf{u}, \mathfrak{D}_1) =\underset{\mathbf{u}}{\arg\min}~\sum\limits_{i=1}^{m_1} l \left(y_i,~g(\mathbf{x}_i, \mathbf{u})\right),$$
здесь $l$ - перекрестная энтропия:
$$l(y, \hat{y}) = -\sum\limits_{k=1}^{c} [y = k] \log{\hat{y}_k},~y \in C,~\hat{y} \in \Delta^c.$$

Наша задача - построить студенческую модель
\[
    \mathbf{f}_\text{st}: \mathbb{R}^{n} \rightarrow \Delta^{c_2},\quad \mathbf{f}_\text{st}(\mathbf{x}) = \mathbf{f}(\mathbf{x}, \hat{\mathbf{w}}),
\],
которая минимизирует перекрестную энтропию на валидационной части второго набора данных $\mathfrak{D}_2$
\[
    \hat{\mathbf{w}} =  \underset{\mathbf{w}}{\arg\min}~\mathcal{L}_\text{ce}(\mathbf{w}, \mathfrak{D}^\text{val}_2),
\],
где $\mathfrak{D}_2 = \mathfrak{D}^\text{train}_2 \sqcup \mathfrak{D}^{\text{val}}_2$ и $\hat{\mathbf{w}}$~--- оптимальные параметры модели.

Поскольку валидационная функция потерь не оптимизируется напрямую, общепринятой практикой является использование градиентных методов оптимизации на обучающей части $\mathfrak{D}^\text{train}_2$ набора данных $\mathfrak{D}_2$.
Чтобы уменьшить переобучение и использовать больше информации о данных, используется информация от модели учителя $\mathbf{g}_{\text{tr}}$.
Заметим, что используется предположение, что наборы данных $\mathfrak{D}_1$ и $\mathfrak{D}_2$ имеют общие свойства.

Функция 
\[
    \boldsymbol{\varphi}: \mathbb{R}^{N_\text{tr}} \rightarrow \mathbb{R}^{N_\text{st}}
\]
отображает параметры модели учителя в начальные параметры модели ученика $\mathbf{w} = \boldsymbol{\varphi}(\hat{\mathbf{u}})$.

\begin{hypothesis}
Модели ученика, инициализированные результатом применения функции $\boldsymbol{\varphi}$ к параметрам предварительно обученной модели учителя, являются более устойчивыми и достигают более высокой точности, чем модели с параметрами по умолчанию.
\end{hypothesis}

Основная проблема предложенного метода заключается в том, что модель учителя $\mathbf{g}_\text{tr}$, обученная на простом наборе данных $\mathfrak{D}_1$, может быть намного проще, чем студенческая модель $\mathbf{f}_\text{st}$. Чтобы использовать больше информации из параметров модели учителя $\hat{\mathbf{u}}$, нам нужно расширить размерность пространства параметров модели учителя ${N_\text{tr}}$ до размерности ${N_\text{st}}$ пространства параметров студенческой модели.

Чтобы справиться с этим, оптимизируем следующую составную функцию потерь:
\begin{equation}\label{phi}
  \boldsymbol{\varphi}(\mathbf{u}) = \underset{\mathbf{w} \in \mathbb{R}^{N_\text{st}}}{\arg\min}~\mathcal{L}(\mathbf{w}),
\end{equation}
где 
\[
    \mathcal{L}(\mathbf{w}) = \lambda_1 \mathcal{L}_\text{ce}(\mathbf{w}, \mathfrak{D}_1) + \lambda_2 \mathcal{L}_2 (\mathbf{w}, \mathbf{u}) + \lambda_3 \mathcal{L}_3^\delta (\mathbf{w}, \mathfrak{D}_1) + \lambda_4 \mathcal{L}_4 (\mathbf{w}),
\]
\[
    \forall i \in \overline{1, 4} \; \lambda_i \ge 0
\]
Здесь $\mathcal{L}_\text{ce}(\mathbf{w}, \mathfrak{D}_1)$ - перекрестная энтропия, отвечающая за качество студенческой модели на $\mathfrak{D}_1$.

Второе слагаемое
\[
    \mathcal{L}_2 (\mathbf{w}, \mathbf{u}) = \|\textbf{u} - \textbf{Pr}[\textbf{w}]\|^2_2
\]
обеспечивает малую разницу между параметрами модели учителя и студенческой модели в соответствующих местах, где \textbf{Pr} берет только первые параметры, общие для обеих моделей (в случае моделей многослойного перцептрона, \textbf{Pr} берет параметры тех же нейронов для каждого слоя модели).

Компонента
\[
    \mathcal{L}_3^\delta (\mathbf{w}, \mathfrak{D}_1) = \displaystyle \sum \limits_{(\textbf{x}, y) \in \mathfrak{D}_1} \displaystyle \mathbb{E}_{\textbf{x}' \in U_\delta(\textbf{x})} \mathcal{L}_\text{ce}(\mathbf{w}, \textbf{x}', y)
\]
отвечает за устойчивость решения к шуму во входных данных, где $U_\delta(\textbf{x})$ представляет равномерное распределение в $[\delta - \textbf{x}; \delta + \textbf{x}].$

Последнее слагаемое
\[
    \mathcal{L}_4 (\mathbf{w}) = \text{tr} \left(\displaystyle \frac{\partial^2 \mathcal{L}_\text{ce}}{\partial \mathbf{w}^2}\right)
\]
выполняет регуляризацию гессиана, что также повышает устойчивость модели.

Заметим, что последнее слагаемое $\mathcal{L}_4$ включает вычисление гессиана, наивное вычисление которого может быть ресурсоемким.
Поэтому, используется метод стохастической аппроксимации~\cite{bai1996some} следа гессиана с быстрым умножением гессиан-вектор~\cite{pearlmutter1994fast}. Сложность такой процедуры линейна от количества параметров модели $\mathbf{f}_\text{st}$.

В интересующем нас случае, Анти-Дистилляции, подразумевается $\lambda_2 >0$, т.е. оптимизация, которая делает параметры модели учителя и ученика достаточно близкими.
Заметим, что важна устойчивость модели к искажению входных данных.
Для этого свойства используем слагаемые $\mathcal{L}_3$ и $\mathcal{L}_4$. Оба этих слагаемых регулируют гессиан функции перекрестной энтропии~\cite{yao2020pyhessian,chen2020stabilizing}.

\section{Результаты вычислительных экспериментов}

\subsection{Удаления параметров моделей глубокого обучения}
Для анализа свойств предложенного алгоритма и сравнения его с существующими проведен вычислительный эксперимент в котором параметры нейросети удалялись методами, которые описаны в разделах 3.1---3.3 и методом Белсли.

В качестве данных использовались три выборки. Выборки Wine~\cite{Wine} и Boston Housing~\cite{boston1978} ~--- это реальные данные. Синтетические данные сгенерированы таким образом чтобы параметры сети мультиколинеарными. Генерация данных состояла из двух этапов. 
На первом этапе генерировался вектор параметров $\mathbf{w}_{\text{synthetic}}$:
\[
    \mathbf{w}_{\text{synthetic}}  \sim \mathcal{N}(\textbf{m}_{\text{synthetic}}, \textbf{A}_{\text{synthetic}}),
\]
где 
$\textbf{m}_{\text{synthetic}} = \begin{bmatrix}
1.0\\
0.0025\\
\ldots\\
0.0025
\end{bmatrix}$,
$\textbf{A}_{\text{synthetic}} = \begin{bmatrix}
1.0& 10^{-3}& \ldots& 10^{-3}& 10^{-3}\\
10^{-3}& 1.0& \ldots& 0.95& 0.95\\
\ldots&\ldots&\ldots&\ldots&\ldots\\
10^{-3}& 0.95& \ldots& 0.95& 1.0
\end{bmatrix}$.

На втором этапе генерировалась выборка $\mathfrak{D}_{\text{synthetic}}$:
\[
    \mathfrak{D}_{\text{synthetic}} = \{(\textbf{x}_i,y_i)| \textbf{x}_i \sim  \mathcal{N}(\textbf{1}, \textbf{I}), y_i = x_{i0}, i = 1 \ldots 10000\}.
\]
В приведенном выше векторе параметров $\mathbf{w}_{\text{synthetic}}$ для выборки $\mathfrak{D}_{\text{synthetic}}$, наиболее релевантным является первый параметр, а все остальные параметры являются нерелевантными. Матрица ковариации выбрана таким образом, чтобы все нерелевантные параметры являлись зависимыми величинами, что приводит к максимальной эффективности метода Белсли.

\begin{table}[h!t]

\begin{center}
\caption{Описание выборок для анализа метода задания порядка методом Белсли}
\begin{tabular}{|c|c|c|c|}
\hline
	Выборка &Тип задачи& Размер выборки& Число признаков\\
	\hline
	
	\multicolumn{1}{|l|}{Wine}
	&
	\multicolumn{1}{|l|}{класификация}
	 & 178 & 13\\
	\hline
	
	\multicolumn{1}{|l|}{Boston Housing}
	&
	\multicolumn{1}{|l|}{регресия}
	& 506 & 13\\
	\hline
	
	\multicolumn{1}{|l|}{Synthetic data}
	&
	\multicolumn{1}{|l|}{регресия}
	& 10000 & 100\\
\hline

\end{tabular}
\end{center}
\end{table}


Для алгоритмов тренировочная и тестовая выборки составили $80\%$ и $20\%$ соответсвенно. Критерием качества прореживания служит процент параметров нейросети, удаление которого не влечет значимой потери качества прогноза. Также критерием качества служит устойчивость нейросети к зашумленности данных. 

Качеством прогноза $R_{\text{cl}}$ модели для задачи классификации является точность прогноза модели:
\[
R_{\text{cl}} = \frac{\sum_{(\textbf{x},y)\in \mathfrak{D}} [f(\textbf{x}, \textbf{w}) = y]}{\left|\mathfrak{D}\right|},
\]

Качеством прогноза $R_{\text{rg}} $ модели для задачи регрессии является среднеквадратическое отклонение результата модели от точного:

\[
R_{\text{rg}} = \frac{\sum_{(\textbf{x},y)\in \mathfrak{D}} \left(f(\textbf{x}, \textbf{w}) - y\right)^2}{\left|\mathfrak{D}\right|},
\]

\paragraph{Wine.} Рассмотрим нейроную сеть с 13 нейронами на входе, 13 нейронами в скрытом слое и 3 нейронами на выходе.

\begin{figure}[h!t]\center
    \includegraphics[width=0.8\textwidth]{thesis/figures/chapter-4/belsli/Wine/All.pdf}\\
    \caption{Качество прогноза при удаление параметров на выборке Wine}
    \label{WineAll}
\end{figure}

\begin{figure}[h!t]\center
    \subfloat[Произвольное удаление параметров]{\includegraphics[width=0.5\textwidth]{thesis/figures/chapter-4/belsli/Wine/RandomNoise3D.pdf}}
    \subfloat[Оптимальное прореживание]{\includegraphics[width=0.5\textwidth]{thesis/figures/chapter-4/belsli/Wine/OBDNoise3D.pdf}}\\
    \subfloat[Вариационный метод]{\includegraphics[width=0.5\textwidth]{thesis/figures/chapter-4/belsli/Wine/VariationalNoise3D.pdf}}
    \caption{Влияние шума в начальных данных на шум выхода нейросети на выборке Wine}
    \label{WineNoise}
\end{figure}

На рис. \ref{WineAll} показано как меняется точность прогноза $R_{\text{cl}}$ при удалении параметров указанными методами. Из графика видно, что метод оптимального прореживания, вариационный метод и метод Белсли позволяют удалить $\approx80\%$ параметров и качество всех этих методов падает при удалении $\approx90\%$ параметров нейросети. 

На рис. \ref{WineNoise} показаны поверхности изменения уровня шума ответов нейросети при изменении процента удаленных параметров и уровня шума входных данных для разных методов прореживания. На графиках показано, что при удалении параметров нейросети методом Белсли шум меньше, чем при удалении параметров другими методами, на это указывает то что поверхность которая соответствует методу Белсли ниже других поверхностей.

\paragraph{Boston Housing.} Рассмотрим нейроную сеть с 13 нейронами на входе, 39 нейронами в скрытом слое и одним нейроном на выходе.

\begin{figure}[h!t]\center
    \includegraphics[width=0.8\textwidth]{thesis/figures/chapter-4/belsli/Boston/All.pdf}\\
    \caption{Качество прогноза при удаление параметров на выборке Boston}
    \label{BostonAll}
\end{figure}

\begin{figure}[h!t]\center
    \subfloat[Произвольное удаление параметров]{\includegraphics[width=0.5\textwidth]{thesis/figures/chapter-4/belsli/Boston/RandomNoise3D.pdf}}
    \subfloat[Оптимальное прореживание]{\includegraphics[width=0.5\textwidth]{thesis/figures/chapter-4/belsli/Boston/OBDNoise3D.pdf}}\\
    \subfloat[Вариационный метод]{\includegraphics[width=0.5\textwidth]{thesis/figures/chapter-4/belsli/Boston/VariationalNoise3D.pdf}}
    \caption{Влияние шума в начальных данных на шум выхода нейросети на выборке Boston}
    \label{BostonNoise}
\end{figure}

На рис. \ref{BostonAll} показано как меняется среднеквадратическое отклонение прогноза $\mathsf{R}_{\text{rg}}$ от точного ответа  при удалении параметров указанными методами. График показывает, что метод Белсли является более эффективным, чем другие методы, так как позволяет удалить больше параметров нейросети без потери качества.

На рис. \ref{BostonNoise} показаны поверхности изменения уровня шума ответов нейросети при изменении процента удаленных параметров и уровня шума входных данных для разных методов прореживания. График показывает, что уровень шума всех методов одинаковый, так как поверхности всех методов находятся на одном уровне.


\paragraph{Синтетические данные.} Рассмотрим нейроную сеть с 100 нейронами на входе и одним нейроном на выходе.

\begin{figure}[h!t]\center
    \includegraphics[width=0.8\textwidth]{thesis/figures/chapter-4/belsli/Data1/All.pdf}\\
    \caption{Качество прогноза при удаление параметров на синтетической выборке}
    \label{Data1All}
\end{figure}

\begin{figure}[h!t]\center
    \subfloat[Произвольное удаление параметров]{\includegraphics[width=0.5\textwidth]{thesis/figures/chapter-4/belsli/Data1/RandomNoise3D.pdf}}
    \subfloat[Оптимальное прореживание]{\includegraphics[width=0.5\textwidth]{thesis/figures/chapter-4/belsli/Data1/OBDNoise3D.pdf}}\\
    \subfloat[Вариационный метод]{\includegraphics[width=0.5\textwidth]{thesis/figures/chapter-4/belsli/Data1/VariationalNoise3D.pdf}}
    \caption{Влияние шума в начальных данных на шум выхода нейросети на синтетической выборке}
    \label{Data1Noise}
\end{figure}

На рис. \ref{Data1All} показано как меняется среднеквадратическое отклонение прогноза от $\mathsf{R}_{\text{rg}}$ точного ответа при удалении параметров указанными методами. График показывает, что удаление параметров методом Белсли являеться более эффективным чем другие методы прореживания, так как качество прогноза нейросети повышается при удалении шумовых параметров.

На рис. \ref{Data1Noise} показаны поверхности изменения уровня шума ответов нейросети при изменении процента удаленных параметров и уровня шума входных данных для разных методов прореживания. На графиках показано, что при удалении параметров нейросети методом Белсли шум меньше, чем при удалении параметров другими методами, так как поверхность которая соответствует методу Белсли ниже других поверхностей.

\subsection{Дистилляция моделей глубокого обучения на многодоменных данных}
Цель вычислительного эксперимента~--- сравнить производительность моделей учителя и ученика на реальных наборах данных с использованием отображения $\varphi$ и без него для задач компьютерного зрения и обработки естественного языка. Для анализа качества дистилляции предложен интегральный критерий качества~\cite{grabovoi2022probabilistic609999418}.
\begin{itemize}
    \item Используется подмножество ImageNet~--- набор изображений, для которого необходимо решить задачу классификации на 10 классов~\cite{imagenet}. Набор данных состоит из обучающей и тестовой частей, причем обучающая часть разделена на мультиресурсную и малоресурсную части.
    \item OPUS-100 является англоцентричным, то есть все обучающие пары включают английский язык на стороне источника или цели~\cite{opus100}. Используются наборы данных fr-en и de-en.
\end{itemize}

Конфигурация алгоритма многодоменной дистилляции для задачи компьютерного зрения


\begin{table}[h!t]
\begin{center}
\caption{Структура учителя}
\label{table_1}
\begin{tabular}{|c|c|c|}
\hline
	Слой & Размер входного вектора & Количество параметров\\
	\hline
	\multicolumn{1}{|l|}{Входной слой}
	& (3, 200, 200) & 0 \\
	\hline
	\multicolumn{1}{|l|}{CONV1 (размер ядра=5)}
	& (24, 196, 196) & 1800 \\
	\hline
	\multicolumn{1}{|l|}{POOL1}
	& (24, 98, 98) & 0 \\
	\hline
	\multicolumn{1}{|l|}{CONV2 (размер ядра = 5)}
	& (48, 94, 94) & 28800 \\
	\hline
	\multicolumn{1}{|l|}{POOL2}
	& (48, 47, 47) & 0 \\
	\hline
	\multicolumn{1}{|l|}{CONV3 (размер ядра = 8)}
	& (96, 40, 40) & 294912 \\
	\hline
	\multicolumn{1}{|l|}{POOL3}
	& (96, 20, 20) & 0 \\
	\hline
    \multicolumn{1}{|l|}{CONV4 (размер ядра = 5)}
	& (192, 16, 16) & 460800 \\
	\hline
	\multicolumn{1}{|l|}{POOL4}
	& (192, 8, 8) & 0 \\
    \hline
	\multicolumn{1}{|l|}{CONV5 (размер ядра = 7)}
	& (384, 2, 2) & 3612672 \\
	\hline
	\multicolumn{1}{|l|}{POOL5}
	& (384, 1, 1) & 0 \\
	\hline
	\multicolumn{1}{|l|}{Полносвязный слой}
	& (384) & 0 \\
	\hline
	\multicolumn{1}{|l|}{Полносвязный слой}
	& (120) & 46080 \\
	\hline
	\multicolumn{1}{|l|}{Полносвязный слой}
	& (84) & 10080 \\
	\hline
	\multicolumn{1}{|l|}{Полносвязный слой}
	& (10) & 840 \\
	\hline
	\multicolumn{1}{|l|}{}
	& & $\sum=4455984$ \\
\hline
\end{tabular}
\end{center}
\end{table}


\begin{table}[h!t]
\begin{center}
\caption{Структура модели ученика}
\label{table_1_1}
\begin{tabular}{|c|c|c|}
\hline
	Слой & Размер входного вектора & Количество параметров\\
	\hline
	\multicolumn{1}{|l|}{Входный слой}
	& (3, 200, 200) & 0 \\
	\hline
	\multicolumn{1}{|l|}{CONV1 (размер ядра=5)}
	& (24, 196, 196) & 1800 \\
	\hline
	\multicolumn{1}{|l|}{POOL1}
	& (24, 98, 98) & 0 \\
	\hline
	\multicolumn{1}{|l|}{CONV2 (размер ядра = 5)}
	& (48, 94, 94) & 28800 \\
	\hline
	\multicolumn{1}{|l|}{POOL2}
	& (48, 47, 47) & 0 \\
	\hline
	\multicolumn{1}{|l|}{Полносвязный слой}
	& (106032) & 0 \\
	\hline
	\multicolumn{1}{|l|}{Полносвязный слой}
	& (120) & 12723840 \\
	\hline
	\multicolumn{1}{|l|}{Полносвязный слой}
	& (10) & 1200 \\
	\hline
	\multicolumn{1}{|l|}{}
	& & $\sum=12755640$ \\
\hline
\end{tabular}
\end{center}
\end{table}


Структуры модели учителя $\mathbf{f}$ и модели ученика $\mathbf{g}$ описаны в Таблице~\ref{table_1} и Таблице~\ref{table_1_1}. Функция активации после каждого скрытого слоя~--- ReLu. 
Используется метод градиентной оптимизации Adam \cite{adam2015} для решения задачи оптимизации.

\begin{table}[h!t]
\begin{center}
\caption{Набор данных ImageNet}
\label{table_2}
\begin{tabular}{|c|c|c|}
\hline
	Набор данных & Описание & Размер набора\\
	\hline
	\multicolumn{1}{|l|}{ImageNet-Train}
	& Обучающая часть& 9469\\
	\hline
	\multicolumn{1}{|l|}{ImageNet-Big}
	& Мультиресурсная часть& 8469\\
	\hline
	\multicolumn{1}{|l|}{ImageNet-Small}
	& Малоресурсная часть& 1000\\
	\hline
	\multicolumn{1}{|l|}{ImageNet-Test}
	& Тестовая часть& 3925\\
\hline
\end{tabular}
\end{center}
\end{table}

В Таблице~\ref{table_2} описаны наборы данных для вычислительного эксперимента по компьютерному зрению. Каждый из наборов данных состоит из обучающей и тестовой части, при этом обучающая часть разделена на мультиресурсную и малоресурсную части. Обучающая часть содержит 9 469 объектов, мультиресурсная часть содержит 8 469 объектов, малоресурсная часть содержит 1 000 объектов, а тестовая часть содержит 3 925 объектов.

Цель эксперимента~--- сравнить производительность студенческой модели, обучающейся без учителя, с учителем и с учителем на другом домене с использованием адаптации домена.
Сначала обучается модель учченика на малоресурсной части и происходит ее тестирование на тестовой части, затем используются метки модели учителя, обученной на мультиресурсной части, для обучения модели ученика.
Наконец, происходит обучение модели учителя на изображениях из мультиресурсной части и используется модель учителя и отображение для обучения модели ученика.
Предобученная модель CycleGAN~\cite{CycleGAN} используется в качестве отображения $\varphi$.

\begin{figure}[htpb]\center
    {\includegraphics[width=0.5\textwidth]{thesis/figures/chapter-4/multidomain-distillation/Mapping}}
    \caption{Сравнение примера объекта до и после преобразования.}
    \label{mapping}
\end{figure}

Рис.~\ref{mapping} показывает одно из изображений в наборе данных ImageNet~\cite{imagenet} и то же изображение после преобразования с помощью модели CycleGAN~\cite{CycleGAN}. Результаты усредняются по 5 запускам и для вычисления среднего значение и дисперсии метрик.

Конфигурация алгоритма многодоменной дистилляции для задачи обработки естественного языка.
Набор данных OPUS100 был разделен на обучающую часть для учителя, состоящую из немецко-английских предложений, и обучающую часть и тестовую часть для модели ученика, состоящую из французско-английских предложений.
Обучающая часть учителя содержала 5 000 предложений, обучающая часть модели ученика содержала 2 000 предложений, а тестовая часть содержала 500 предложений.

\begin{table}[h!t]
\begin{center}
\caption{Набор данных OPUS100}
\label{table_3}
\begin{tabular}{|c|c|c|c|}
\hline
	Набор данных & Описание & Язык & Размер набора\\
	\hline
	\multicolumn{1}{|l|}{Teacher-Train}
	& Обучающая часть модели учителя & de-en & 5000\\
	\hline
	\multicolumn{1}{|l|}{Student-Train}
	& Обучающая часть модели ученика& fr-en & 2000\\
	\hline
	\multicolumn{1}{|l|}{Student-Test}
	& Тестовая часть & fr-en & 500\\
\hline
\end{tabular}
\end{center}
\end{table}

Таблица~\ref{table_3} описывает наборы данных для вычислительного эксперимента по обработке естественного языка.

Использовалась модель ученика модель~\textbf{g} и модель учителя~\textbf{f} в качестве трансформерной модели на основе статьи~\cite{attention} и метод градиентной оптимизации Adam \cite{adam2015} для решения задачи оптимизации. Модель NLLB\cite{nllb} использовалась в качестве отображения $\varphi$.
Эта модель переводила французские предложения в немецкие.

Аналогично эксперименту по компьютерному зрению, сравнивается производительности модели ученика без учителя, с учителем и с учителем и адаптацией домена.
Результаты усреднялись по 5 запускам для вычисления среднего значение и дисперсии метрик.
\begin{figure}[htpb]
    \centerline{\includegraphics[width=0.5\textwidth]{thesis/figures/chapter-4/multidomain-distillation/Dist_acc.png}}
    \caption{Точность аппроксимации на тестовой выборке. Все результаты усреднены по 5 запускам.}
    \label{cv_acc}
\end{figure}

\begin{figure}[htpb]
    \centerline{\includegraphics[width=0.5\textwidth]{thesis/figures/chapter-4/multidomain-distillation/Dist_loss.png}}
    \caption{Ошибка перекрестной энтропии между истинными и предсказанными студенческими метками на тестовой выборке. Все результаты усреднены по 5 запускам.}
    \label{cv_loss}
    \end{figure}

Как видно из Рис.~\ref{cv_acc} и Рис.~\ref{cv_loss}, модели, обученные с использованием учителя, достигают лучшего качества и точности. Можно заметить, что студенческая модель, обученная с использованием меток учителя на том же домене (зеленые линии), достигает наивысшей точности и наименьших потерь. Студенческая модель, обученная с использованием меток учителя и адаптации домена (красные линии), показывает лучшее качество аппроксимации, чем модель без использования учителя.

Таким образом, экспериментально показано, что дистилляция с использованием адаптации домена приводит к более эффективным нейронным сетям с меньшим количеством параметров.


\begin{table}[h!t]
\begin{center}
\caption{Качество моделей для компьютерного зрения}
\label{table_4}
\resizebox{\linewidth}{!}{
\begin{tabular}{|c|c|c|c|c|c|}
\hline
	Ученика & Учитель & Отображение $\varphi$ & Точность & \begin{tabular}[c]{@{}c@{}}Потери перекрестной\\ энтропии\end{tabular} & \begin{tabular}[c]{@{}c@{}}Интегральный\\ критерий\end{tabular}\\
	\hline
	\multicolumn{1}{|l|}{ImageNet-Small}
	& --- & --- & $0{,}34 \pm 0{,}01$ & $4{,}41 \pm 1{,}10$ & $53{,}89 \pm 14{,}99$ \\
    \hline

    \multicolumn{1}{|l|}{ImageNet-Small}
	& ImageNet-Big & StyleTransfer & $0{,}37 \pm 0{,}01$ & $\textbf{2{,}01} \pm \textbf{0{,}03}$ & $28{,}30 \pm 0{,}79$ \\
\hline

    
    \hline
    \multicolumn{1}{|l|}{ImageNet-Small}
	& ImageNet-Big & --- & $\textbf{0{,}44} \pm \textbf{0{,}01}$ & $2{,}03 \pm 0{,}02$ & $\textbf{28{,}08} \pm \textbf{1{,}22}$ \\
    \hline
	
\end{tabular}
}
\end{center}
\end{table}

Результаты также представлены в табличной форме. Таблица~\ref{table_4} содержит данные о валидационной точности, потерях и интегральном критерии моделей, обученных с дистилляцией и адаптацией домена и без них, в эксперименте по компьютерному зрению.


\begin{figure}[h!t]\center
    {\includegraphics[width=0.5 \textwidth]{thesis/figures/chapter-4/multidomain-distillation/NLP_loss.png}}
    \caption{Ошибка перекрестной энтропии на тестовом наборе данных. Все результаты усреднены по 3 запускам.}
    \label{nlp_loss}
\end{figure}

Как видно из Рис.~\ref{nlp_loss}, модели, обученные с использованием учителя, достигают лучшего качества.

Таблица~\ref{table_5} показывает результаты сравнения студенческих моделей, полученных с использованием дистилляции и без.

\begin{table}[h!t]
\begin{center}
\caption{Качество моделей для NLP}
\label{table_5}
\resizebox{\linewidth}{!}{
\begin{tabular}{|c|c|c|c|c|}
\hline
	Ученик & Учитель & Отображение $\varphi$ & Потери перекрестной энтропии & \begin{tabular}[c]{@{}c@{}}BLEU\end{tabular} \\
\hline
	\multicolumn{1}{|l|}{Student-Train}
	& --- & ---& $5{,}367 \pm 0{,}015$ & $0{,}0282$ \\
    \hline
	\multicolumn{1}{|l|}{Student-Train}
	& Teacher-Train & NLLB & $\textbf{5{,}233} \pm \textbf{0{,}007}$ & $\textbf{0{,}0572}$\\
\hline
\end{tabular}
}
\end{center}
\end{table}



\subsection{Анти-Дистилляция моделей глубокого обучения}
Цель вычислительного эксперимента~--- сравнить производительность моделей в зависимости от инициализации параметров. 

Производится сравнение различных подходов к инициализации:
\begin{enumerate}
    \item Xavier~--- заполнение всех параметров модели $U[-\frac1{\sqrt{n}}, \frac1{\sqrt{n}}]$, где $n$~--- количество нейронов входного слоя \cite{glorot2010understanding}, т.е. инициализация параметров модели по умолчанию.
    \item Zero pad~--- заполнение расширенных параметров нулями.
    \item Uniform pad~--- заполнение расширенных параметров равномерно распределенными случайными величинами $U[-\frac1{\sqrt{n}}, \frac1{\sqrt{n}}]$, где $n$~--- количество нейронов входного слоя.
    \item Transfer learning~--- взятие предобученной модели и изменение только классификационного слоя для новой задачи классификации. Сначала модель обучалась с замороженными параметрами на всех слоях, кроме классификационного. После 3 эпох обучения все параметры размораживались. Начиная с четвертой эпохи, оптимизировались все параметры нейронной сети.
    \item Net2Net~--- инкрементальный алгоритм расширения пространства параметров модели\cite{net2net}.
    \item With Data Noise~--- получение инициализации студенческой модели путем решения задачи оптимизации \ref{phi} с $\lambda_1, \lambda_3 = 1$ и $\lambda_2, \lambda_4 = 0$.
    \item Anti-Distillation, $\lambda_4 = 0$~--- инициализация методом Анти-Дистилляции с оптимизацией гиперпараметров $\lambda_1, \lambda_2, \lambda_3$ с помощью байесовской оптимизации ($\lambda_4 = 0$) \cite{akiba2019optuna}.
    \item Anti-Distillation~--- оптимизация всех $\lambda_i$.
\end{enumerate}

Критериями качества являются: точность на валидационном наборе, точность на валидационном наборе, искаженном атакой FSGM \cite{goodfellow2014explaining}, точность на валидационном наборе при условии, что параметры модели искажены шумом: ${\mathbf{w}_\varepsilon} = \mathbf{w} + \varepsilon \boldsymbol{\xi}$, где $\boldsymbol{\xi} \sim \mathcal{N} (\mathbf{0}, \mathbf{I})$.


Fashion-MNIST~--- это набор данных изображений статей Zalando, состоящий из обучающего набора из 60 000 примеров и тестового набора из 10 000 примеров. Каждый пример представляет собой полутоновое изображение 28x28, связанное с меткой из 10 классов \cite{fashionmnist}.

Проведение экспериментов осуществляется следующим образом: обучается модель учителя, увеличивается ее сложность и сравниваем различные способы инициализации параметров модели.
Рассматривается модель полносвязной сети. Модель учителя имеет следующие размеры скрытых слоев: $[128, 64, 32]$. Модели ученика имеет $[256, 128, 64]$ нейронов в скрытых слоях.

Модель учителя обучался в течение 30 эпох с начальной скоростью обучения 1e-2, которая затем уменьшалась до 1e-3 после 10 эпох.
Модель ученика сравнивались при обучении в течение 10 эпох со скоростью обучения 1e-3.
Оптимизация проводится с использованием алгоритма оптимизации Adam \cite{adam2015}.
Методы инициализации сравниваются, измеряя точность предсказаний, значение функции потерь перекрестной энтропии на валидационной выборке и дисперсию предсказаний.
Также исследуется случай зашумленных входных данных, рассматривая указанные критерии качества в зависимости от процента искаженных изображений.

\begin{figure}[!t]
\centering
  \includegraphics[width=0.5\textwidth]{thesis/figures/chapter-4/anti-distillation/file}
 \caption{Сравнение валидационной точности для различных методов инициализации}
  \label{fig:1}
\end{figure}
\begin{figure}[!t]
\centering
  \includegraphics[width=0.5\textwidth]{thesis/figures/chapter-4/anti-distillation/fsgm}
 \caption{Зависимость валидационной точности от адверсарного шума в данных}
  \label{fig:2}
\end{figure}
\begin{figure}[!t]
\centering
  \includegraphics[width=0.5\textwidth]{thesis/figures/chapter-4/anti-distillation/noise}
 \caption{Зависимость валидационной точности от параметра интенсивности шума $\varepsilon$}
  \label{fig:3}
\end{figure}

Набор данных $\mathfrak{D}_2$ состоит из Fasion-MNIST, а $\mathfrak{D}_1 = \{(\textbf{x}, y) \;|\; (\textbf{x}, y) \in \mathfrak{D}_2, y \in C_1\}$, $C_1 \subset C_2$, $C_1 = \{0, \dots 4\}, C_2 = \{0, \dots 9\}$. 


Как видно на Рисунке \ref{fig:1}, модели, использующие Анти-Дистилляцию, в среднем имеют меньшую дисперсию и более высокую точность, чем модели с различной инициализацией параметров.
Обучение модели с нуля оказалось не лучшим решением.
Предложенный метод дает нам лучшие результаты с меньшим количеством итераций для сходимости.
Отметим, что не учитывалось количество итераций, необходимых для расширения модели учителя, которое также требует процедуры оптимизации.
Предполагается, что во многих реальных случаях этим временем можно пренебречь, поскольку предложенный метод позволяет нам расширить модель учителя один раз, используя только базовый набор данных $\mathfrak{D}_1$, для последующего использования в множественных задачах обучения модели ученика~\cite{sun2019meta}.

Рис.~\ref{fig:2} показывает, что Анти-Дистилляция является наиболее устойчивым к адверсарным атакам методом инициализации параметров модели, поскольку она имеет наивысшую валидационную точность с большим отрывом при высоких уровнях шума.

На Рисунке \ref{fig:3} видно, что метод Анти-Дистилляции без регуляризации гессиана ($\lambda_4=0$) является наиболее устойчивым к нормальному шуму в параметрах модели, поскольку сохраняет наивысшую точность при максимальном рассматриваемом уровне шума.

\begin{table}[!h]
\caption{\label{acc_tab} Точность на валидационном наборе.}

 \begin{tabular}{|c|c|c|c|}
        \hline
        Метод инициализации & Точность &  Атака FSGM & 
Шум в параметрах \\
        \hline
        Xavier & 0.68 $\pm$ 0.08 & 0.42 $\pm$ 0.04 & 0.58 $\pm$ 0.06\\
    \hline
Zero Pad & \textbf{0.86 $\pm$ 0.02} & 0.50 $\pm$ 0.01 & 0.71 $\pm$ 0.03\\
\hline
Uniform Pad & 0.85 $\pm$ 0.04 & 0.52 $\pm$ 0.03 & \textbf{0.73 $\pm$ 0.03}\\
\hline
Transfer Learning & 0.74 $\pm$ 0.09 & 0.50 $\pm$ 0.06 & 0.53 $\pm$ 0.05\\
\hline
Net2Net & 0.85 $\pm$ 0.04 & 0.51 $\pm$ 0.02 & 0.70 $\pm$ 0.03\\
\hline
With Data Noise & 0.81 $\pm$ 0.07 & 0.51 $\pm$ 0.03 & 0.70 $\pm$ 0.05\\
\hline
Anti-Distillation, $\lambda_4$=0 & \textbf{0.86 $\pm$ 0.05} & 0.53 $\pm$ 0.03 & \textbf{0.73 $\pm$ 0.04}\\
\hline
Anti-Distillation & \textbf{0.86 $\pm$ 0.05} & \textbf{0.57 $\pm$ 0.03} & 0.67 $\pm$ 0.03\\
\hline
\end{tabular}

\end{table}

Результаты также представлены в табличной форме. Таблица \ref{acc_tab} содержит данные о валидационной точности для различных методов инициализации после последней эпохи обучения, значения валидационной точности при наивысшем уровне шума изображения от адверсарной атаки и информацию о значениях точности для наивысшего уровня шума в параметрах модели.

\section{Заключение по главе}

В главе были рассмотрены методы передачи знаний между нейронными сетями в условиях различных доменов и сложности данных.

Мультидоменная дистилляция продемонстрировала эффективность передачи знаний от более сложной модели, обученной на большом наборе данных, к менее сложной модели, обучаемой на малом наборе данных того же или другого домена. Эксперименты в областях компьютерного зрения и обработки естественного языка подтвердили улучшение качества аппроксимации студенческой модели. Ожидаемо, обучение в рамках одного домена показало лучшие результаты по сравнению с использованием адаптации домена, однако последняя также обеспечила значительное улучшение по сравнению с базовыми подходами.

Антидистилляция решила задачу расширения модели для работы с более сложными наборами данных. Предложенный метод передачи знаний от простой модели к более сложной не только повысил точность на сложных данных, но и увеличил устойчивость модели к шуму во входных данных и нормальному шуму в параметрах модели. Эксперименты на наборе данных Fashion-MNIST подтвердила эффективность подхода.

Оба метода открывают перспективы для практического применения в условиях ограниченных данных и необходимости адаптации моделей к новым доменам. Дальнейшие исследования будут направлены на интеграцию байесовских методов дистилляции, учитывающих распределения параметров, а также на применение разработанных подходов к другим архитектурам нейронных сетей и наборам данных.


% Шестая глава
\clearpage
\chapter{Применение теоретических оценок в прикладных задачах}
В предыдущих главах был разработан единый теоретический аппарат для формализации соотношения между сложностью модели и сложностью данных.
В главе~\ref{chapter:complexity} введены формальные определения меры сложности выборки $\mu_D(D)$ и меры сложности модели $\mu_f(f)$, а также установлен критерий обучаемости $\mu_f(f) \leq \mu_D(D)$.
Кроме того, разработана ландшафтная мера сложности модели $\mu_f(f|D)$, определяемая через спектральные свойства матриц Гессе функции потерь.
В главе~\ref{chapter:gesian} получены конкретные оценки спектральных норм матриц Гессе для различных архитектур нейронных сетей, что обеспечивает вычислимо осуществимые методы оценки ландшафтной меры сложности.
В главе~\ref{chapter:samplesize} разработаны практические методы определения достаточного размера выборки на основе анализа стабильности процесса обучения.
В главе~\ref{chapter:pruning} предложены методы снижения сложности моделей через удаление параметров, дистилляцию и анти-дистилляцию.

Однако для полной валидации разработанного теоретического аппарата и демонстрации его практической значимости необходимо применить полученные результаты к реальным прикладным задачам машинного обучения.
В отличие от предыдущих глав, где основной акцент делался на разработке строгих математических оценок и алгоритмических процедур, в настоящей главе демонстрируется адаптация теоретических подходов к конкретным практическим проблемам.
Это позволяет оценить эффективность предложенного формализма в условиях реальных данных и ограниченных вычислительных ресурсов.

Настоящая глава охватывает три ключевых направления практического применения теоретического аппарата сложности моделей и данных.
Во-первых, рассматривается применение меры сложности Радемахера для анализа многозадачного обучения с адаптерами LoRA, что позволяет количественно оценить преимущества совместного использования параметров энкодера и эффективность низкоранговой параметризации.
Во-вторых, демонстрируется снижение сложности данных в задаче декодирования фМРТ-изображений из видеопоследовательностей, где предварительное сжатие данных обеспечивает существенное сокращение времени обучения без потери качества реконструкции.
В-третьих, разрабатываются методы оценки качества данных в задаче детекции машинно-генерированного текста, основанные на топологической статистике и анализе устойчивости к адверсарным возмущениям, что позволяет выявить смещения в обучающих наборах данных и улучшить надежность детекторов.

Полученные результаты создают основу для практического применения теоретического аппарата сложности моделей и данных в реальных задачах машинного обучения, демонстрируя эффективность предложенных подходов и открывая перспективы для дальнейшего развития методов управления сложностью в прикладных задачах.

\section{Сложность моделей в многозадачном обучении}

\subsection{Радемахеровская сложность моделей глубокого обучения}
Пусть $\mathcal{X}$ обозначает входное пространство, а $\mathcal{Y} = \{1, \dots, C\}$~--- общее пространство меток для всех доменов.
Домен определяется как пара $\mathcal{D} = (\mathcal{X}, P(X))$, где $P(X)$~--- распределение над $\mathcal{X}$.

Предполагается наличие $K$ размеченных исходных доменов
\begin{equation}
    \mathcal{D}_{S_k} = (\mathcal{X}, P_{S_k}(X)), 
    \quad k = 1, \dots, K,
\end{equation}
с размеченными выборками
\begin{equation}
    \mathcal{S}_k = \{(x_i^{S_k}, y_i^{S_k})\}_{i=1}^{n_{S_k}}, 
    \quad x_i^{S_k} \sim P_{S_k}(X),  y_i^{S_k} \in \mathcal{Y}.
\end{equation}

Также рассматривается целевой домен
\begin{equation}
    \mathcal{D}_T = (\mathcal{X}, P_T(X)),
\end{equation}
с выборками
\begin{equation}
    \mathcal{T} = \{(x_j^T, y_j^T)\}_{j=1}^{n_T^{\text{lab}}} \cup \{x_j^T\}_{j=1}^{n_T^{\text{unlab}}}.
\end{equation}

Цель кросс-доменной классификации~--- обучить классификатор
\begin{equation}
    f_\theta : \mathcal{X} \to \mathcal{Y}, \quad f_\theta(x) = \arg\max_{y \in \mathcal{Y}} p_\theta(y \mid x),
\end{equation}
такой, чтобы ожидаемая целевая ошибка
\begin{equation}
    \epsilon_T(f_\theta) = \mathbb{E}_{(x,y) \sim P_T(X,Y)} [\mathbf{1}\{f_\theta(x) \neq y\}]
\end{equation}
минимизировалась за счет использования знаний из всех источников $\{\mathcal{S}_k\}_{k=1}^K$.
Цель состоит в обучении классификатора, который способен достигать высоких метрик качества на примерах из доменов, не встречавшихся во время обучения или представленных лишь ограниченными данными.

Современные задачи классификации часто решаются путем адаптации больших предобученных моделей к конкретному домену или набору данных~\cite{devlin2019bert}.
Распространенная стратегия~--- тонкая настройка (fine-tuning), когда параметры предобученной модели обновляются с использованием размеченных данных из целевой задачи.
Хотя этот подход обычно дает высокую производительность, он может быть вычислительно дорогим, поскольку количество обучаемых параметров очень велико.

Основная идея LoRA заключается в том, что для адаптации к задаче не обязательно обновлять всю матрицу параметров.
Вместо этого обновления могут быть эффективно представлены в низкоразмерном подпространстве, которое захватывает существенные вариации, необходимые для адаптации.
Это значительно сокращает количество обучаемых параметров, в значительной степени сохраняя выразительную способность модели.

Формально, пусть $\Delta W \in \mathbb{R}^{d \times k}$ обозначает обновление параметров для данного слоя.
При стандартной тонкой настройке $\Delta W$ изучается напрямую.
LoRA ограничивает $\Delta W$ условием низкого ранга путем факторизации:
\begin{equation}
    \Delta W \approx A B,
\end{equation}
где $A \in \mathbb{R}^{d \times r}$ и $B \in \mathbb{R}^{r \times k}$ с $r \ll \min(d,k)$. 
Обновленная матрица параметров тогда имеет вид:
\begin{equation}
    W_{\text{upd}} = W + \Delta W = W + AB,
\end{equation}
где $W$ обозначает замороженные предобученные веса, а $AB$~--- изучаемое низкоранговое обновление. 
Ранг $r$ выступает в качестве гиперпараметра, контролирующего размер изучаемого подпространства.

\begin{table}[t]
    \caption{Сравнение формул обновления параметров для полной тонкой настройки и метода LoRA. В полной тонкой настройке обновление $\Delta W$ изучается напрямую, тогда как в LoRA обновление параметризуется в виде низкоранговой факторизации $AB$, где $r \ll \min(d,k)$, что значительно сокращает количество обучаемых параметров.} \label{table:lora}
    \begin{center}
    \begin{tabular}{cc}
    \toprule
    \textbf{Полная тонкая настройка} & \textbf{Тонкая настройка с LoRA} \\
    \midrule
    $W_{\text{upd}} = W + \Delta W$ & $W_{\text{upd}} = W + AB$ \\
    $\hat{y} = x(W + \Delta W)$ & $\hat{y} = x(W + AB)$ \\
    \bottomrule
    \end{tabular}
    \end{center}
\end{table}

В задачах классификации предсказание обычно имеет вид
\begin{equation}
    p(c \mid \mathbf{x}) = \operatorname{softmax}(W^{\top} \mathbf{x}),
\end{equation}
где $\mathbf{x}$~--- это вектор признаков входного объекта, а $W$~--- обучаемая матрица весов.
LoRA может быть применена здесь путем замены $W$ на ее низкоранговую адаптированную форму $W + AB$.

Многозадачное обучение~--- это подход, в котором несколько связанных задач изучаются совместно с целью достижения лучшей обобщающей способности по сравнению с обучением каждой задачи независимо.
Ключевое предположение заключается в том, что задачи разделяют некоторую базовую структуру, так что обмен информацией между ними уменьшает переобучение и улучшает прогнозную производительность. 

Формально, пусть $\{(x_{t,i}, y_{t,i})\}_{i=1}^{n_t}$ обозначает обучающие данные для задачи $t \in \{1,\dots,T\}$, взятые из распределения $P_t$ над $\mathcal{X} \times \mathcal{Y}$.
Цель~--- обучить дискриминативные функции $f_t : \mathcal{X} \to \mathcal{Y}$ для всех задач.
Распространенный подход формулирует это как задачу регуляризованной минимизации эмпирического риска:
\begin{equation}
    \min_{f = (f_1,\ldots,f_T)} 
        \frac{1}{T} \sum_{t=1}^T \frac{1}{n_t} \sum_{i=1}^{n_t} 
            \ell(f_t(x_{t,i}), y_{t,i}) 
        + \lambda \, \Omega(f),
\end{equation}
где первое слагаемое~--- это средняя эмпирическая ошибка по задачам, а $\Omega(f)$~--- регуляризующее слагаемое, поощряющее обмен информацией. 

С теоретической точки зрения, преимущество MTL может быть интерпретировано как эффективное сокращение пространства гипотез за счет использования связанности задач, что приводит к более узким обобщающим границам.

Традиционные методы машинного обучения часто обучаются в однозадачной постановке.
В отличие от этого, многозадачное обучение (MTL) совместно обучается на нескольких задачах с целью улучшения общей производительности за счет использования общих представлений~\cite{anonymous2024}. 

Такая постановка особенно эффективна, когда задачи связаны, так как передача информации может уменьшить переобучение и улучшить обобщение.
Кроме того, даже для одной целевой задачи вспомогательные задачи могут служить формой индуктивного смещения, помогая модели изучать более устойчивые представления.

Центральный вопрос в многозадачном обучении, как и в стандартном обучении с учителем, заключается в том, насколько хорошо изученные функции обобщаются на новые данные.
Границы обобщающей ошибки предоставляют теоретические гарантии, связывая истинный риск с его эмпирическим аналогом.
В постановке MTL границы часто выводятся в рамках упомянутого выше подхода регуляризованной минимизации эмпирического риска. 

Пусть даны обучающие примеры $\{(X_1, Y_1), \ldots, (X_n, Y_n)\}$, взятые i.i.d. из неизвестного распределения $P$.
Цель~--- обучить функцию $f \in \mathcal{F}$ с малой ожидаемой потерей $\mathbb{E}_{(X,Y) \sim P} [\ell(f(X), Y)]$.
Поскольку $P$ неизвестно, обычно связывают этот риск со средним эмпирическим значением функции потерь плюс слагаемое сложности, которое зависит от богатства гипотезного класса $\mathcal{F}$:
\begin{align}
    \mathbb{E}_{(X,Y) \sim P}[\ell(f(X), Y)] &\leq \frac{1}{n} \sum_{i=1}^{n} \ell(f(X_i), Y_i) \\
    &\quad + h\!\left(\text{complexity of }\mathcal{F}, n \right).
\end{align}

Среди различных мер сложности широко используется сложность Радемахера.
Эта мера количественно определяет, насколько хорошо гипотезный класс может аппроксимировать случайный шум, что напрямую связано с обобщающей способностью модели.
Формально:

\begin{definition}[Сложность Радемахера]
    Пусть $\mathcal{G} := \{g : \mathcal{Z} \to \mathbb{R}\}$~--- гипотезный класс, а $S := \{z_1, \ldots, z_n\}$~--- i.i.d. выборка из $P$.
    Эмпирическая сложность Радемахера для $\mathcal{G}$ определяется как
    \[
    \widehat{R}(\mathcal{G}) := \mathbb{E}_{\sigma} \left[ 
        \sup_{g \in \mathcal{G}} \frac{1}{n} \sum_{i=1}^n \sigma_i g(z_i) \right],
    \]
    где $\sigma_i$~--- независимые случайные величины Радемахера, равномерно распределенные в $\{\pm 1\}$.
    (Ожидаемая) сложность Радемахера определяется как
    \[
    R(\mathcal{G}) := \mathbb{E}_{S \sim P^n} \left[ \widehat{R}(\mathcal{G}) \right].
    \]
\end{definition}

Интуитивно, меньшая сложность Радемахера соответствует менее выразительному предсказательному классу, что, в свою очередь, приводит к более узким границам обобщения.
В настоящем исследовании выполняется сопоставление сложности модели в условиях многозадачного и однозадачного обучения.
В отличие от других работ, мы не сравниваем различные типы границ обобщения, а фокусируемся на сравнении сложности моделей.

Для эмпирической проверки нашего теоретического анализа сфокусируемся на задаче обнаружения машинно-генерированного текста.
Эта задача требует высокой точности классификации с низкой частотой ложных срабатываний, чтобы избежать некорректного помечания контента, написанного человеком.
Стремительное улучшение больших языковых моделей (LLMs) сделало тексты, сгенерированные ИИ, все более неотличимыми от написанных человеком ~\cite{zellers2019grover, solaiman2019gpt2}, создавая насущную потребность в устойчивых методах обнаружения для сохранения академической честности и борьбы с дезинформацией. 

Большинство подходов рассматривают это как задачу бинарной классификации текста~\cite{jawahar2020automatic, uchendu2021authorship}.
Тонкая настройка архитектур на основе Transformer стала доминирующим решением ~\cite{valiaiev2024detectionmachinegeneratedtextliterature}, демонстрируя высокую производительность, когда тестовые данные поступают из того же или близкородственного домена.
Однако производительность часто резко падает при появлении новых доменов, делая модели менее надежными~\cite{anonymous2025}.
Чтобы решить эту проблему, можно либо использовать более мощные модели, либо интегрировать дополнительную информацию о тексте, такую как его источник, домен или стилистические особенности.
Эти сигналы могут выступать в качестве вспомогательных регуляризаторов, улучшая обобщение.
Многозадачное обучение предоставляет естественную основу для объединения нескольких задач, включая интеграцию такой вспомогательной информации, при этом сохраняя основную цель классификации.

\subsection{Радемахеровская сложность многозадачного обучения в LoRA-адаптерах}

Предполагается, что решается задача классификации с использованием моделей с архитектурой Transformer~\cite{attention}.
Тонкая настройка таких моделей в режиме обучения с учителем является современным методом для задач классификации.

Анализ LoRA показывает, что минимизация эмпирического риска при параметризации LoRA остается согласованной с минимизацией истинного риска, гарантируя, что низкоранговая адаптация сохраняет асимптотические статистические гарантии полной тонкой настройки.

\begin{theorem}[Состоятельность]\label{theorem:consistency}
Пусть $\mathcal{D} = \{(X_i, c_i)\}_{i=1}^n$~--- независимые одинаково распределенные выборки из истинного распределения $P_{\mathrm{true}}$, где $c_i \in [N_c]$ обозначает метки классов. 
Предположим следующее:
\begin{enumerate}
    \item Существует параметр $\Theta^\star$ такой, что модель распределения $P_{\mathrm{model}}(\cdot \mid \Theta)$ аппроксимирует истинное распределение с минимальным расхождением Кульбака-Лейблера:
    \begin{equation}
    \label{eq:kl-opt}
    \Theta^\star \in \arg\min_{\Theta} D_{\mathrm{KL}}\big(P_{\mathrm{true}} \,\|\, P_{\mathrm{model}}(\cdot \mid \Theta)\big).
    \end{equation}

    \item При $n \to \infty$ эмпирическое распределение сходится по вероятности к $P_{\mathrm{true}}$.
    
    \item Функция потерь задается как отрицательное логарифмическое правдоподобие
    \begin{equation}
    \label{eq:loss}
        \mathscr{L}_n(\Theta) = -\frac{1}{n} \sum_{i=1}^n \log \Big(P_{\Phi_0 + \Theta}\big(c_i \mid X_i\big)\Big),
    \end{equation}
    где $\Phi_0$~--- замороженные предобученные веса, а $\Theta$ соответствует обучаемым низкоранговым параметрам в LoRA.
    Предполагается, что $\mathcal{L}_n(\Theta)$ непрерывна и дифференцируема.
\end{enumerate}

Тогда минимизация эмпирического риска является состоятельной:
\begin{equation}
\label{eq:consistency}
\lim_{n \to \infty} \arg\min_{\Theta} \, \mathscr{L}_n(\Theta) = \Theta^\star.
\end{equation}
\end{theorem}
\begin{proof}
Пусть истинный риск и его эмпирический аналог определены как
\[
    L(\Theta) = \mathbb{E}_{(X,c)\sim P_{\mathrm{true}}}[\mathscr{L}(X,c;\Theta)], \quad \widehat{L}_n(\Theta) = \frac{1}{n}\sum_{i=1}^n \mathscr{L}(X_i,c_i;\Theta),
\]
где $\mathscr{L}(X_i,c_i;\Theta) = -\log P_{\Phi_0+\Theta}(c_i\mid X_i)$~--- отрицательное логарифмическое правдоподобие для отдельного примера.

В соответствии с равномерным законом больших чисел,
для непрерывной и ограниченной функции $\mathscr{L}$ имеем
\begin{equation}
    \sup_{\Theta}\big|\,L(\Theta)-\widehat{L}_n(\Theta)\,\big|\xrightarrow[n\to\infty]{p}0.
\label{eq:unif-conv}
\end{equation}
Следовательно, последовательность эмпирических рисков $\widehat{L}_n(\Theta)$ сходится равномерно к истинному риску $L(\Theta)$.
Равномерная сходимость влечет состоятельность минимизации эмпирического риска, т.е.
\[
    \arg\min_{\Theta}\widehat{L}_n(\Theta)\xrightarrow[n\to\infty]{}\arg\min_{\Theta}L(\Theta).
\]

Из определения $L(\Theta)$ и с использованием тождества
\[
D_{\mathrm{KL}}\!\big(P_{\mathrm{true}}\|P_{\mathrm{model}}(\cdot\mid\Theta)\big)
= \mathbb{E}_{(X,c)\sim P_{\mathrm{true}}}
   \!\left[\log\frac{P_{\mathrm{true}}(c\mid X)}{P_{\mathrm{model}}(c\mid X;\Theta)}\right],
\]
минимизация ожидаемых потерь $L(\Theta)$ эквивалентна минимизации расхождения Кульбака-Лейблера между $P_{\mathrm{true}}$ и $P_{\mathrm{model}}(\cdot\mid\Theta)$.
Согласно Предположению (1), минимизатором этого расхождения является $\Theta^\star$.
Следовательно,
\[
\lim_{n\to\infty}\arg\min_{\Theta}\widehat{L}_n(\Theta)
=\arg\min_{\Theta}L(\Theta)=\Theta^\star,
\]
что завершает доказательство.
\end{proof}

\begin{theorem}[Корректность при низкоранговых обновлениях]
\label{theorem:lowrank}
Предположим следующее:
\begin{enumerate}
    \item Выходной слой задается в виде
    \begin{equation}
    \label{eq:softmax}
        \hat{\mathbf{y}} = \operatorname{softmax}\!\left(W_{\mathrm{upd}}^{\top} \mathbf{x}\right),
    \end{equation}
    где $\mathbf{x} \in \mathbb{R}^d$~--- это вектор признаков BERT, $W \in \mathbb{R}^{d \times k}$~--- замороженные предобученные веса, а 
    \begin{equation}
    \label{eq:update}
        W_{\mathrm{upd}} = W + \Delta W.
    \end{equation}

    \item Вместо непосредственного обучения $\Delta W \in \mathbb{R}^{d \times k}$, обновление параметризуется в низкоранговой форме:
    \begin{equation}
    \label{eq:lora}
        \Delta W = A B, 
        \quad A \in \mathbb{R}^{d \times r}, 
        B \in \mathbb{R}^{r \times k}, 
        r \ll \min(d,k).
    \end{equation}

    \item Выполнены условия Теоремы~\ref{theorem:consistency}, т.е. модель остается статистически состоятельной при минимизации эмпирического риска.
\end{enumerate}

Тогда при параметризации~\eqref{eq:lora} выход модели $\hat{\mathbf{y}}$ сохраняется в том смысле, что низкоранговое обновление не искажает корректность классификационного слоя.
\end{theorem}
\begin{proof}
Пусть выходной слой определен как
\[
    \hat{\mathbf{y}} = \operatorname{softmax}\!\big(W_{\mathrm{upd}}^{\top}\mathbf{x}\big), \quad W_{\mathrm{upd}} = W + \Delta W,
\]
где $W \in \mathbb{R}^{d\times k}$~--- замороженные предобученные веса, а $\mathbf{x}\in\mathbb{R}^d$~--- вектор признаков энкодера. 
В силу дистрибутивности матричного сложения,
\[
    W_{\mathrm{upd}}^{\top}\mathbf{x} = W^{\top}\mathbf{x} + \Delta W^{\top}\mathbf{x}.
\]
Следовательно, предсказанные вероятности могут быть записаны как
\begin{equation}\label{eq:softmax-expanded}
    \hat{\mathbf{y}} = \frac{\exp\!\big(W^{\top}\mathbf{x} + \Delta W^{\top}\mathbf{x}\big)}      {\sum_{i=1}^{k}\exp\!\big((W^{\top}\mathbf{x} + \Delta W^{\top}\mathbf{x})_i\big)}.
\end{equation}

При параметризации LoRA $\Delta W = AB$ с $A\in\mathbb{R}^{d\times r}$ и $B\in\mathbb{R}^{r\times k}$, член обновления принимает вид
\[
    \Delta W^{\top}\mathbf{x} = (AB)^{\top}\mathbf{x} = B^{\top}(A^{\top}\mathbf{x}) \in \mathbb{R}^{k}.
\]
Это показывает, что низкоранговая форма лишь ограничивает $\Delta W$ рангом $r$ в подпространстве $\mathbb{R}^{d\times k}$, но не изменяет выходную размерность или отображение $\mathbf{x}\mapsto\hat{\mathbf{y}}$.
Логиты в~\eqref{eq:softmax-expanded} остаются хорошо определенными и дифференцируемыми для всех $\mathbf{x}$.

Более того, поскольку обновление $\Delta W$ изучается через минимизацию эмпирического риска, согласованную с Теоремой 1, результирующие параметры $(A,B)$ дают тот же минимизатор истинного ожидаемого риска, что и неограниченный $\Delta W$.
Следовательно, низкоранговая параметризация сохраняет корректность выходного слоя в том смысле, что~$\hat{\mathbf{y}}_{\mathrm{LoRA}} = \operatorname{softmax}\!\big((W+AB)^{\top}\mathbf{x}\big)$ порождает то же решающее правило, что и~$\operatorname{softmax}\!\big((W+\Delta W)^{\top}\mathbf{x}\big)$.
Таким образом, введение модулей LoRA не искажает выходное распределение или классификационные границы модели.
\end{proof}

\begin{corollary}[LoRA с теоретическими гарантиями]
\label{corollary:lora}
Объединяя Теорему~\ref{theorem:consistency} и Теорему~\ref{theorem:lowrank}, мы получаем, что тонкая настройка на основе LoRA одновременно сохраняет:
\begin{enumerate}
    \item Статистическую состоятельность: минимизация эмпирического риска сходится к минимизатору истинного риска с ростом объема выборки;
    \item Корректность выходного слоя: низкоранговые обновления $AB$ не искажают выход классификации.
\end{enumerate}
Таким образом, адаптация с помощью LoRA наследует те же асимптотические гарантии, что и полная тонкая настройка, при этом требуя значительно меньшего количества обучаемых параметров.
\end{corollary}

Приведенные выше результаты показывают, что LoRA не ослабляет теоретические основы тонкой настройки.
Со статистической точки зрения, метод сохраняет состоятельность: с увеличением количества обучающих примеров адаптированная модель сходится к тому же оптимальному решению, что и при полной тонкой настройке.
С функциональной точки зрения, ограничение обновлений низкоранговым подпространством не искажает выход классификационного слоя.
В совокупности эти свойства объясняют, почему LoRA достигает сопоставимой производительности с полной тонкой настройкой на практике, будучи при этом существенно более эффективной.

В настоящем разделе анализируется влияние многозадачного обучения на сложность обобщения для каждой задачи.
Для этого выполняется сравнение однозадачного обучения модели на основе Transformer с MTL-подходом, использующим общий энкодер.
Для обеспечения справедливого сравнения предполагается, что архитектуры энкодера и головы идентичны и состоят из линейных слоев.
Это позволяет изолировать эффект многозадачного обучения от влияния различий в архитектуре.
В случае STL присутствует единственная голова для целевой задачи, тогда как в настройке MTL вводятся дополнительные головы для связанных задач, обеспечивая неявную регуляризацию для целевой задачи.

Формально, в случаях STL и MTL гипотезные классы имеют вид:
\begin{align}
\mathcal{F}_{\mathrm{STL}}
&= \Big\{\, x \mapsto w_{\mathrm{head}}^\top \phi(x; w_{\mathrm{enc}}) \\
&\qquad \big|\, w_{\mathrm{enc}}\in\mathcal{W}_{\mathrm{enc}},
               w_{\mathrm{head}}\in\mathcal{W}_{\mathrm{head}} \Big\},
\\[4pt]
\mathcal{F}_{\mathrm{MTL}}
&= \Big\{\, (\,x \mapsto w_t^\top \phi(x; w_{\mathrm{shared}})\,)_{t=1}^T \\
&\qquad \big|\, w_{\mathrm{shared}}\in\mathcal{W}_{\mathrm{shared}},
               w_t\in\mathcal{W}_{\mathrm{head}} \Big\}.
\end{align}


\begin{theorem}[Сложность Радемахера на задачу при MTL]\label{thm:per-task-rc-fair-scaling}
Пусть $S_t=\{x_i\}_{i=1}^n$~--- выборка фиксированной целевой задачи $t$ с $\|x_i\|_2 \le R$.
Пусть $\phi(\cdot;w)$~--- энкодер, и рассмотрим линейные головы
$f_{w,h}(x)=h^\top \phi(x;w)$ с $\|h\|_2 \le B_{\mathrm{head}}$.
Предположим:
\begin{enumerate}
\item Ограничение на признаки: для всех $x,w$ 
$\|\phi(x;w)\|_2 \le L\,\|w\|\,\|x\|_2$.
\item Энкодер STL удовлетворяет $\|w_{\mathrm{enc}}\|\le B_{\mathrm{enc}}$,
общий энкодер MTL удовлетворяет $\|w_{\mathrm{shared}}\|\le B_{\mathrm{shared}}$.
\item Многозадачное масштабирование: $B_{\mathrm{shared}} \le B_{\mathrm{enc}}/\sqrt{T}$.
\end{enumerate}

Обозначим через $\widehat{\mathfrak R}_n(\cdot;S_t)$ эмпирическую сложность Радемахера на $S_t$.
Тогда
\begin{equation}\label{eq:mtl-vs-stl-rc}
    \widehat{\mathfrak R}_n\!\big(\mathcal{F}_{\mathrm{MTL}}^{(t)};S_t\big)
    \le \frac{1}{\sqrt{T}}\,
    \widehat{\mathfrak R}_n\!\big(\mathcal{F}_{\mathrm{STL}}^{(t)};S_t\big).
\end{equation}
\end{theorem}
\begin{proof}
Пусть $S_t = \{x_i\}_{i=1}^n$~--- выборка для целевой задачи $t$ с 
$\|x_i\|_2 \le R$. 
Для $B > 0$ определим гипотезный класс
\[
    \mathcal{F}(B)
    = \big\{\,x \mapsto h^\top \phi(x;w) 
    \big|
    \|h\|_2 \le B_{\mathrm{head}},\, \|w\|_2 \le B
    \big\}.
\]
Эмпирическая сложность Радемахера на $S_t$ равна
\[
    \widehat{\mathfrak R}_n(\mathcal{F}(B);S_t)
    = \frac{1}{n}\,\mathbb{E}_{\sigma}
      \sup_{\|h\|\le B_{\mathrm{head}},\,\|w\|\le B}
      \sum_{i=1}^n \sigma_i\, h^\top \phi(x_i;w),
\]
где $\sigma_i \in \{\pm1\}$~--- независимые одинаково распределенные величины Радемахера.

Для фиксированных $w$ и $\sigma$, по дуальности Коши-Буняковского,
\[
    \sup_{\|h\|\le B_{\mathrm{head}}}
    \sum_{i=1}^n \sigma_i\, h^\top \phi(x_i;w)
    = B_{\mathrm{head}}
      \Big\|\sum_{i=1}^n \sigma_i\, \phi(x_i;w)\Big\|_2.
\]
Следовательно,
\[
    \widehat{\mathfrak R}_n(\mathcal{F}(B);S_t)
    \le
    \frac{B_{\mathrm{head}}}{n}\,
    \mathbb{E}_{\sigma}
    \sup_{\|w\|\le B}
      \Big\|\sum_{i=1}^n \sigma_i\, \phi(x_i;w)\Big\|_2.
\]

По неравенствам Йенсена и Хинчина,
\[
    \mathbb{E}_{\sigma}
      \Big\|\sum_{i=1}^n \sigma_i\,a_i\Big\|_2
    \le
    \Big(\sum_{i=1}^n \|a_i\|_2^2\Big)^{1/2},
\]
что дает
\[
    \widehat{\mathfrak R}_n(\mathcal{F}(B);S_t)
    \le
    \frac{B_{\mathrm{head}}}{n}\,
    \sup_{\|w\|\le B}
      \Big(\sum_{i=1}^n \|\phi(x_i;w)\|_2^2\Big)^{1/2}.
\]

Используя ограничение на признаки $\|\phi(x;w)\|_2 \le L\,\|w\|_2\,\|x\|_2$ и $\|x_i\|_2 \le R$, получаем
\[
    \widehat{\mathfrak R}_n(\mathcal{F}(B);S_t)
    \le
    \frac{B_{\mathrm{head}}\,L\,R\,B}{\sqrt{n}}.
\]
Применяя это с $B=B_{\mathrm{enc}}$ для STL и $B=B_{\mathrm{shared}}$ для MTL, находим
\[
    \widehat{\mathfrak R}_n(\mathcal{F}(B_{\mathrm{shared}});S_t)
    \le
    \frac{B_{\mathrm{head}}\,L\,R\,B_{\mathrm{shared}}}{\sqrt{n}}
    \le
    \frac{1}{\sqrt{T}}\,
    \frac{B_{\mathrm{head}}\,L\,R\,B_{\mathrm{enc}}}{\sqrt{n}}
    =
    \frac{1}{\sqrt{T}}\,
    \widehat{\mathfrak R}_n(\mathcal{F}(B_{\mathrm{enc}});S_t),
\]
где последнее неравенство следует из предположения многозадачного масштабирования $B_{\mathrm{shared}}\le B_{\mathrm{enc}}/\sqrt{T}$.
\end{proof}

Полученный результат указывает на снижение эмпирической сложности Радемахера на задачу в $1/\sqrt{T}$ раз при справедливом масштабировании.
Это согласуется с современными анализами многозадачного обучения, основанными на средних Радемахера, как показано в~\cite{maurer2006}.

Данный результат имеет важное практическое значение: при совместном использовании параметров энкодера между $T$ задачами эффективная сложность модели на каждую задачу снижается пропорционально $1/\sqrt{T}$, что количественно объясняет преимущества многозадачного обучения в терминах меры сложности Радемахера.
Однако более ранняя работа~\cite{baxter2000} установила более сильное улучшение типа $1/T$ в другой постановке.
В постановке Бакстера, количество задач $T$ само по себе способствует оценке общего индуктивного смещения: с ростом $T$ сложность данных на задачу уменьшается пропорционально $1/T$.
В отличие от этого, наш анализ сохраняет размер выборки целевой задачи $n$ фиксированным и сравнивает STL и MTL на одном и том же $n$, поэтому улучшение проявляется как множитель $1/\sqrt{T}$ в сложности Радемахера на задачу.

Обе сложности на задачу оцениваются на одной и той же выборке $S_t$ размера $n$, поэтому общий множитель $1/\sqrt{n}$ сокращается; разница обусловлена только бюджетом энкодера. 

Более того, в анализе обеспечивается, чтобы обе модели STL и MTL обучались на одинаковом общем количестве примеров ($nT$).
Для STL энкодер обучается на $nT$ примерах только из целевой задачи.
Для MTL каждая из $T$ голов получает $n$ примеров из своей соответствующей задачи, так что общий энкодер видит те же самые $nT$ примеров, причем одна из голов соответствует целевой задаче.
Это гарантирует, что наблюдаемое снижение сложности на задачу обусловлено совместным использованием параметров, а не неравными бюджетами данных.

Наш теоретический анализ установил, что LoRA сохраняет статистическую состоятельность полной тонкой настройки и поддерживает корректность выходного слоя, в то время как многозадачное обучение при соответствующем масштабировании снижает сложность Радемахера на задачу в $1/\sqrt{T}$ раз.
Вместе эти результаты показывают, что оба подхода сохраняют фундаментальные гарантии классической тонкой настройки, но достигают большей эффективности с точки зрения параметров (LoRA) или обобщения (MTL).

Полученные теоретические результаты создают основу для практического применения LoRA и многозадачного обучения в реальных задачах, обеспечивая теоретическое обоснование их эффективности и количественную оценку преимуществ по сравнению с полной тонкой настройкой.


\section{Снижение сложности данных в задаче декодирования фМРТ-снимков}

В предыдущем разделе рассматривалось снижение сложности моделей через низкоранговую параметризацию и многозадачное обучение.
В настоящем разделе демонстрируется альтернативный подход~--- снижение сложности данных без изменения архитектуры модели.
Такой подход особенно актуален в задачах, где данные имеют высокую размерность, но могут быть эффективно сжаты без существенной потери информации.
В задаче декодирования фМРТ-изображений из видеопоследовательностей предварительное сжатие томографических данных позволяет существенно сократить время обучения при сохранении качества реконструкции, что иллюстрирует практическую значимость управления сложностью данных в прикладных задачах нейровизуализации.

Частота кадров $\nu \in \mathbb{R}$ и продолжительность $t \in \mathbb{R}$ видеопоследовательности задаются.
Видеопоследовательность задается как
\begin{equation}
	\label{eq1}
	\mathbf{P} = [\mathbf{p}_1, \ldots, \mathbf{p}_{\nu t}], \quad \mathbf{p}_{\ell} \in \mathbb{R}^{W \times H \times C},
\end{equation}
где $W$, $H$ и $C$~--- ширина, высота и количество каналов изображения соответственно.

Обозначим частоту фМРТ-изображений через $\mu \in \mathbb{R}$.
Зададим последовательность изображений
\begin{equation}
	\label{eq2}
	\mathbf{S} = [\mathbf{s}_1, \ldots, \mathbf{s}_{\mu t}], \quad \mathbf{s}_{\ell} \in \mathbb{R}^{X \times Y \times Z},
\end{equation}
где $X$, $Y$ и $Z$~--- размеры воксельного изображения.

Задача состоит в построении отображения, учитывающего задержку $\Delta t$ между фМРТ-изображением и видеопоследовательностью, а также предыдущие томографические данные.
Формально необходимо найти такое отображение $\mathbf{g}$, что
\begin{equation}
	\label{eq3}
	\mathbf{g}(\mathbf{p}_1, \ldots, \mathbf{p}_{k_{\ell} - \nu \Delta t}; \mathbf{s}_1, \ldots, \mathbf{s}_{\ell-1}) = \mathbf{s}_{\ell}, \quad \ell = 1, \ldots, \mu t,
\end{equation}
где для $\ell$-го фМРТ-изображения номер соответствующего кадра $k_{\ell}$ определяется по формуле
\begin{equation}
	\label{eq4}
	k_{\ell} = \dfrac{\ell \cdot \nu}{\mu}.
\end{equation}

Схема предложенного метода реконструкции фМРТ-изображений показана на рис.~\ref{fig:scheme}.

\begin{figure}[h!t]
	\centering
	\includegraphics[width=\textwidth]{thesis/figures/chapter-5/fmri/scheme}
	\caption{Схема метода декодирования фМРТ-изображений из видеопоследовательностей. Метод использует архитектуру ResNet152 для векторизации видеокадров и линейную регрессию с $L_2$-регуляризацией для предсказания разностей между последовательными фМРТ-снимками с учетом временной задержки $\Delta t$.}
	\label{fig:scheme}
\end{figure}

Обозначим фМРТ-изображение как $\mathbf{s}_{\ell} = [v^{\ell}_{ijk}] \in \mathbb{R}^{X \times Y \times Z}$, где $v^{\ell}_{ijk} \in \mathbb{R}_+$ — значение соответствующего вокселя.
Для сокращения времени работы метода предлагается использовать сжатие фМРТ-изображений путем снижения размерности.
Сжатие в 2 раза представляется в виде отображения
\[
    \boldsymbol{\chi}: \mathbb{R}^{X \times Y \times Z} \to \mathbb{R}^{X/2 \times Y/2 \times Z/2}.
\]
Сжатие в $2^k$ раз получается последовательным применением $\boldsymbol{\chi}$ $k$ раз.
В дальнейшем для простоты сохраняем обозначения размерностей изображения $X \times Y \times Z$.

Предположим, что для последовательности снимков выполняется свойство Маркова, т.е. каждый снимок зависит только от одного изображения и предыдущего снимка.
Тогда соответствующее отображение записывается в виде
\begin{equation}
	\label{eq5}
	\mathbf{g}(\mathbf{p}_{k_{\ell} - \nu \Delta t}) = \mathbf{s}_{\ell} - \mathbf{s}_{\ell-1} = \boldsymbol{\delta}_{\ell}, \quad \ell = 2, \ldots, \mu t.
\end{equation}
где $\boldsymbol{\delta}_{\ell} = [v^{\ell}_{ijk} - v^{\ell-1}_{ijk}] = [\delta^{\ell}_{ijk}] \in \mathbb{R}^{X \times Y \times Z}$~--- разность двух последовательных снимков.

Отображение $\mathbf{g}: \mathbf{P} \to \mathbf{S}$ представляется как композиция двух других:
\[
    \mathbf{g} = \boldsymbol{\varphi} \circ \boldsymbol{\psi},
\]
где
\begin{align}
	 & \boldsymbol{\psi}: \mathbf{P} \to \mathbb{R}^d \text{~--- векторизация изображения,}        \\
	 & \boldsymbol{\varphi}: \mathbb{R}^d \to \mathbf{S} \text{~--- целевое отображение.}
\end{align}

Для каждого изображения из видеопоследовательности имеем вектор внедрения размерности $d$:
\[
    \mathbf{x}_{\ell} = [x^{\ell}_1, \ldots, x^{\ell}_{d}]^\top \in \mathbb{R}^{d}, \quad \ell = 1, \ldots, \nu t.
\]
В качестве архитектуры используется нейронная сеть ResNet152 без последнего линейного слоя.

При заданном $k_{\ell} = \ell \cdot \nu / \mu$ общее количество пар составляет $N = \mu (t - \Delta t)$.
Таким образом, для каждого вокселя задана выборка
\[
    \mathfrak{D}_{ijk} = \{(\mathbf{x}_{\ell}, \delta^{\ell}_{ijk}) \ | \ \ell = 2, \ldots, N \}.
\]

Задача формулируется как регрессия
\begin{equation}\label{eq6}
	y_{ijk}: \mathbb{R}^{d} \to \mathbb{R}.
\end{equation}

В качестве модели используется линейная регрессия с вектором параметров
\[
    \mathbf{w}_{ijk} = [w^{ijk}_1, \ldots, w^{ijk}_{d}]^\top \in \mathbb{R}^{d}:
\]
\begin{equation}
	\label{eq7}
	f_{ijk}(\mathbf{x}, \mathbf{w}_{ijk}) = \langle \mathbf{x}, \mathbf{w}_{ijk} \rangle.
\end{equation}

Для модели $f_{ijk}$ с соответствующим вектором параметров $\mathbf{w}_{ijk} \in \mathbb{R}^{d}$ определим квадратичную функцию потерь с $L_2$-регуляризацией:
\begin{equation}
	\label{eq8}
	\mathcal{L}_{ijk}(\mathbf{w}_{ijk}) = \sum\limits_{\ell = 2}^{N} \big(f_{ijk}(\mathbf{x}_{\ell}, \mathbf{w}_{ijk}) - \delta^{\ell}_{ijk}\big)^2 + \alpha \| \mathbf{w}_{ijk} \|_2^2,
\end{equation}
где $\alpha \in \mathbb{R}$~--- коэффициент регуляризации.

Требуется найти параметры, доставляющие минимум функционалу потерь $\mathcal{L}_{ijk}(\mathbf{w}_{ijk})$ при заданных гиперпараметрах $\Delta t$ и $\alpha$.
Задача оптимизации формулируется следующим образом:
\begin{equation}
	\label{eq9}
	\hat{\mathbf{w}}_{ijk} = \arg\min_{\mathbf{w}_{ijk}} \mathcal{L}_{ijk}(\mathbf{w}_{ijk}).
\end{equation}

Минимум функции потерь находится методом наименьших квадратов.
Введем матрицу объектов-признаков
\begin{equation}
	\label{eq10}
	\mathbf{X} = [\mathbf{x}_2, \ldots, \mathbf{x}_N]^\top = [x^i_j] \in \mathbb{R}^{(N-1) \times d}
\end{equation}
и вектор, компонентами которого являются разности значений одного и того же вокселя на разных изображениях:
\begin{equation}
	\label{eq11}
	\mathbf{\Delta}_{ijk} = [\delta^2_{ijk}, \ldots, \delta^N_{ijk}]^\top \in \mathbb{R}^{N-1}.
\end{equation}

Решение задачи оптимизации записывается в виде
\begin{equation}
	\label{eq12}
	\hat{\mathbf{w}}_{ijk} = (\mathbf{X}^\top \mathbf{X} + \alpha \mathbf{I})^{-1} \mathbf{X}^\top \mathbf{\Delta}_{ijk}.
\end{equation}

Выведем формулу для восстановленных фМРТ-изображений.
Введем матрицу весов
\begin{equation}
	\label{eq13}
	\hat{\mathbf{W}} = [\hat{\mathbf{w}}_1, \ldots, \hat{\mathbf{w}}_{XYZ}]^\top = [\hat{w}^i_j] \in \mathbb{R}^{XYZ \times d}.
\end{equation}

Введем для тензоров $\mathbf{s}_{\ell}, \boldsymbol{\delta}_{\ell} \in \mathbb{R}^{X \times Y \times Z}$ векторы
\[ \mathbf{s}_{\ell}^{R} = [ v^{\ell}_1, \ldots, v^{\ell}_{XYZ} ]^\top, \boldsymbol{\delta}_{\ell}^{R} = [ \delta^{\ell}_1, \ldots, \delta^{\ell}_{XYZ} ]^\top \in \mathbb{R}^{XYZ}. \]

Тогда вектор прогнозируемого изображения находится по формуле
\begin{equation}
	\label{eq14}
	\hat{\mathbf{s}}_{\ell}^{R} = \mathbf{s}_{\ell-1}^{R} + \hat{\boldsymbol{\delta}}_{\ell}^{R} = \mathbf{s}_{\ell-1}^{R} + \hat{\mathbf{W}} \mathbf{x}_{\ell}.
\end{equation}

\section{Качество данных в задаче детекции машинно-генерированного контента}

В предыдущих разделах рассматривались методы управления сложностью моделей и данных в контексте обучения.
В настоящем разделе акцент смещается на оценку качества самих данных, что является критически важным аспектом для обеспечения надежности моделей машинного обучения.
Низкое качество обучающих данных может приводить к завышенным оценкам производительности моделей и их неспособности к обобщению на реальные данные.
Это особенно актуально для задачи детекции машинно-генерированного контента, где качество обучающих наборов напрямую влияет на надежность детекторов.

В последние годы появилось значительное количество AI-детекторов и коллекций с AI-фрагментами.
Несколько методов детекции продемонстрировали качество распознавания до 99,9\% согласно целевым метрикам в таких коллекциях.
Однако качество таких детекторов часто резко падает в реальных условиях, что ставит вопрос о надежности детекторов: действительно ли они высоконадежны, или их высокие бенчмарк-показатели связаны с низким качеством оценочных наборов данных?
Подчеркнем необходимость создания надежных и качественных методов оценки сгенерированных данных, которые были бы устойчивы к смещениям и низкой способности к обобщению будущих моделей.
В настоящем разделе рассматриваются работы, посвященные детекции AI-генерированного контента, и предлагаются методы оценки качества наборов данных, содержащих AI-фрагменты.

Предлагается оценить различные наборы данных с едиными настройками, чтобы увидеть, насколько хорошо стандартные подходы работают на них.
Целью исследования является не достичь максимальных показателей, а сравнить производительность одного и того же метода на разных наборах данных.

\paragraph{Базовые методы.} Для методов, основанных на возмущениях, использовался фреймворк DetectGPT с GPT-2~\cite{radford2019language} в качестве базовой модели и T5-Large \cite{t5} в качестве генератора возмущений.
Однако из-за высоких вычислительных затрат DetectGPT использовался Fast-DetectGPT~\cite{fast-detectgpt}, который заменяет этап возмущения в DetectGPT более эффективным этапом сэмплирования.
Для методов zero-shot использовался Binoculars~\cite{hans2024spotting} с улучшенной оценкой перплексии.
Эти два базовых метода не требуют тонкой настройки, что является важным аспектом для задачи детекции, поскольку обучать детектор для каждого домена и генератора непрактично.
Наконец, в качестве метода на основе энкодера использовалась модель
mDeBERTa~\cite{he2021deberta}, которая является современной моделью для детекции машинно-генерированного текста в мультиязычном контексте~\cite{macko-etal-2023-multitude}.
Эти три детектора, охватывают все основные категории детекторов.

\paragraph{Топологическая статистика.} В работе \cite{Tulchinskii_phd} было показано, что если рассмотреть внутреннюю размерность многообразия на множестве эмбеддингов, то можно отделить тексты, написанные человеком, от машинно-сгенерированных.
Авторы использовали размерность персистентной гомологии (англ. PHD) и показали, что статистически тексты, сгенерированные человеком, имеют более высокую PHD, чем машинно-сгенерированные тексты, предложив таким образом новый детектор.
Дополнительно, в ~\cite{kushnareva2024boundary} было предложено вычислять PHD внутри скользящего окна.
Эти внутренние размерности текста в пределах скользящего окна могут быть использованы в качестве признака для детекторов.
Авторы демонстрируют, что метрика устойчива к изменению домена и генераторов.
Для возможности сравнения наборов данных между собой разработана симметричная оценка, использующая KL-дивергенцию.
Пусть $h_d$, $m_d$~--- распределения внутренних размерностей для двух типов текстов из одного набора данных, человеческого и машинного происхождения, тогда наша оценка $\text{KL}_{\text{TTS}}$ выглядит следующим образом:
\[
    \text{KL}_{\text{TTS}} (h_d, m_d) = | D_{\text{KL}}(h_d || m_d) - D_{\text{KL}}(m_d || h_d) |
\]
Чем ниже эта оценка, тем ближе друг к другу $h_d$ и $m_d$, что означает почти неразличимые тексты, и наоборот.

\paragraph{Возмущения и перемешивание.} Основываясь на результатах исследований модификации текста~\cite{sadasivan2024can, mitchell2023detectgpt}, которые показывают, как небольшие возмущения влияют на системы машинного понимания текста, рассмотривается этот способ как возможный метод оценки качества набора данных.
Ключевая идея заключается в том, что ИИ-модели чувствительны к таким адверсарным изменениям, в отличие от людей.
Рассматриваются две модификации: Адверсарное возмущение токенов и перемешивание предложений.

В данном подходе текст разбивается на токены, и каждый токен случайным образом заменяется на синоним из коллекции WordNet~\cite{wordnet} с вероятностью 70\%.
Далее данная техника применяется к каждому представленному классу, и с использованием модели-энкодера получаются эмбеддинги для каждого из текстов в текущем наборе данных.
Наконец, измеряются средние сдвиги эмбеддингов для классов человеческих и сгенерированных текстов, используя косинусное расстояние между эмбеддингами исходных текстов и модифицированных.
В итоге, после модификаций получаем $\Delta_{\text{shift}}$~--- логарифм разности средних сдвигов эмбеддингов.
\[
    \Delta_{\text{shift}} = \log \frac{{\frac{1}{n} \sum_{i=1}^{n} \text{cos}_d(h_{h_i}^o, h_{h_i}^p)}}{{\frac{1}{m}\sum_{j=1}^{m} \text{cos}_d(h_{m_j}^o, h_{m_j}^p)}},
\]
где~$n$ и~$m$~--- количество примеров в человеческой и сгенерированной частях набора данных соответственно, $h_{h_i}^o$~--- эмбеддинг $i$-го фрагмента человеческой части данных, $h_{h_i}^p$~--- тот же эмбеддинг после пертурбации.
Аналогично, $h_{m_i}^o$ и $h_{m_i}^p$~--- эмбеддинги для машинно-сгенерированных текстов.
Функция $\text{cos}_d$ измеряет косинусное расстояние между двумя векторами.

В данном подходе предложения случайным образом меняются местами, что влияет на связность текста.
Фрагмент разделяется на предложения, и случайным образом изменяется порядок 70\% выбранных предложений.
Данная техника применяется к каждому представленному классу.
Затем, используя модель кодирования текста, получаются эмбеддинги для каждого из текстов текущего набора данных.
Наконец, выполняется измерение сдвига эмбеддингов для класса человеческих и сгенерированных текстов, после чего сдвиги преобразуются в распределения, подобные вероятностным.
Это в итоге приводит к~$\text{KL}_{\text{shuffle}}(H,M)$~--- дивергенции Кульбака-Лейблера между сдвигами человеческих и сгенерированных текстов.
\[
    \text{KL}_{\text{shuffle}}(H,M)= \sum_{i} H(i) \log \frac{H(i)}{M(i)},
\]
\[
    H(i) = \frac{\text{cos}_d(h_{h_i}^o, h_{h_i}^p) + \epsilon}{\sum_j \left(\text{cos}_d(h_{h_j}^o, h_{h_j}^p) + \epsilon\right)},
\]
где $M(i)$ имеет ту же структуру, что и $H(i)$, за исключением того, что вместо текстов человеческого класса используются тексты сгенерированного класса, $\epsilon$~--- малая константа, добавляемая для избежания деления на ноль.

\section{Результаты вычислительных экспериментов}

\subsection{Сложность моделей в многозадачном обучении}

\begin{figure}[ht]
  \centering
    \includegraphics[width=0.75\linewidth]{thesis/figures/chapter-5/rademacher/lora_rank_vs_f1-1.pdf}
    \caption{Зависимость метрики $F_1$ от ранга $r$ адаптеров LoRA для модели DeBERTa-v3-base на задаче детекции машинно-генерированного текста. Наилучшая производительность достигается при $r=8$, что указывает на снижение отдачи при увеличении ранга и подтверждает эффективность низкоранговой параметризации.}
    \label{fig:r_vs_f1}
\label{fig:lora_rank}
\end{figure}

Для эмпирической проверки теоретических свойств, установленных выше, рассматривается задача обнаружения машинно-генерированного текста.
Данная задача формулируется как проблема бинарной классификации, что служит компактным, но информативным эталоном для оценки как прогнозной производительности, так и эффективности обучения. 

Выбор данной задачи мотивирован ее практической значимостью, разнообразием доменов в наборе данных и необходимостью эффективных стратегий адаптации для достижения сильного обобщения.

В качестве набора данных используется GenAI Detection Challenge (COLING~2025, Task~1), представленный в~\cite{wang2025genaicontentdetectiontask}.
Из исходных 600 000 примеров выбирается 60 000 для обучения и 5 000 для тестирования, сохраняя распределение меток. 

Набор данных охватывает широкий спектр доменов.
Внутрираспределительные категории включают финансы, право, психологию, новости и медицину.
Внераспределительные категории включают эссе IELTS, научные статьи, биомедицинские аннотации и юридические статьи. 

Результаты предыдущих соревнований указывают на то, что стандартные методы тонкой настройки испытывают трудности при достижении сильного обобщения на этом гетерогенном тестовом наборе, что подчеркивает необходимость более эффективных стратегий адаптации.

В экспериментах фиксируется энкодер как DeBERTa-base~\cite{he2023debertav3improvingdebertausing}, который продемонстрировал высокую производительность в недавних задачах классификации текста. 
Для анализа качества рассматриваются следующие метрики: точность (accuracy), точность (precision), полнота (recall), F1-мера, потеря на валидации и общее время обучения.

Для эмпирической проверки гарантии корректности для моделей Transformer с добавлением LoRA (теорема~\ref{theorem:lowrank}) выполняется сравнение DeBERTa-v3-base в двух режимах тонкой настройки: полная тонкая настройка всех параметров и адаптация с помощью LoRA.
Цели эксперимента состоят в следующем: (i) подтвердить, что LoRA сохраняет прогнозную корректность, существенно сокращая сложность параметров, и 
(ii) проверить, что теоретические выгоды в эффективности транслируются в сокращение времени обучения. 
Все условия обучения сохраняются идентичными, а полные детали гиперпараметров приведены в дополнительных материалах.

При полной тонкой настройке обновляются все параметры DeBERTa-v3-base.
Для LoRA исходные веса замораживаются, а низкоранговые адаптеры вставляются в каждый блок Transformer с рангом $r=8$, коэффициентом масштабирования $\alpha=32$ и долей отсева (dropout) для адаптеров $0.1$. 
Эта конфигурация была выбрана после дополнительного анализа чувствительности к рангу, представленного на рис.~\ref{fig:lora_rank}.

Результаты суммированы в таблице~\ref{tab:combined}.
По сравнению с полной тонкой настройкой, LoRA демонстрирует лишь незначительное снижение прогнозных метрик (1.4–3.2\%), при этом обеспечивая существенно более низкие потери на валидации (снижение на 0.657, или 36.3\%). 

Дополнительная диагностика, представленная в дополнительных материалах, показывает более гладкие нормы градиентов и траектории потерь при обучении для LoRA.
Это свидетельствует о более стабильной сходимости и большей эмпирической устойчивости LoRA по сравнению с полной тонкой настройкой.
Полученные результаты согласуются с теоретической гарантией о том, что низкоранговые обновления сохраняют корректность выходного слоя (теорема~\ref{theorem:lowrank}).
Для оценки вычислительной эффективности также рассматривается общее время обучения.
LoRA обеспечивает ускорение в 12.6\%, что указывает на то, что низкоранговая адаптация снижает эффективную сложность и ускоряет оптимизацию при незначительной потере в метриках, основанных на точности.

\begin{table}[h!t]
    \caption{Сравнение производительности модели DeBERTa-v3-base при полной тонкой настройке и адаптации с LoRA на задаче детекции машинно-генерированного текста. LoRA демонстрирует незначительное снижение метрик точности (1.4--3.2\%), но обеспечивает существенное снижение потерь на валидации (36.3\%) и ускорение обучения (12.6\%), что подтверждает теоретические гарантии корректности выходного слоя при низкоранговых обновлениях.}
    \label{tab:combined}
    \centering
    \begin{tabular}{l c c c} 
    \toprule
    Metric & DeBERTa & DeBERTa \& LoRA & Change (\%) \\
    \midrule
    Accuracy $\uparrow$        & 0.7104 & 0.6876 & $-3.2$ \\
    Precision $\uparrow$      & 0.6573 & 0.6413 & $-2.4$ \\
    Recall $\uparrow$         & 0.9608 & 0.9470 & $-1.4$ \\
    $F_{1}$-score $\uparrow$  & 0.7806 & 0.7648 & $-2.0$ \\
    \midrule
    Validation Loss $\downarrow$ & 1.8094 & \textbf{1.1522} & $+36.3$ \\
    Training Time (s) $\downarrow$ & 5570 & \textbf{4867} & $+12.6$ \\
    \bottomrule
    \end{tabular}
\end{table}

Задача GenAI Detection предоставляет, помимо бинарной метки, вспомогательные метаданные, такие как поддомен и источник benchmark.
Эта информация используется для создания многозадачной постановки с тремя выходными головками: исходная задача бинарной классификации и две вспомогательные многоклассовые задачи.
Такая конфигурация побуждает модель изучать более богатые текстовые представления и отражает теоретическое преимущество MTL в снижении сложности на задачу.

На основе аналитического результата об эффективности Радемахера (теорема~\ref{thm:per-task-rc-fair-scaling}) выполняется адаптация~DeBERTa-base для выполнения необходимых предположений.
Обеспечивается: (i) ограничения нормы весов через проецирование параметров на $L_2$-шары после каждого обновления; (ii) липшицева непрерывность с помощью спектральной нормализации всех линейных слоев; и (iii) ограниченность входов через нормализацию токенных эмбеддингов. 

Эти модификации гарантируют, что эмпирическая установка соответствует теоретическим условиям, позволяя проводить валидную оценку преимуществ MTL в рамках установленной схемы.
Подробный процесс подготовки модели описан в дополнительных материалах.

Сначала оценивается прогнозная производительность в условиях STL и MTL.
Выполняется обучение двух вариантов: (i) базовой STL-модели с частичной тонкой настройкой энкодера и классификационной головой, и (ii) MTL-модели с общим энкодером, обновляемым совместно по всем трем задачам. 

На этапе тестирования для MTL-модели используется только общий энкодер и бинарная голова, отбрасывая вспомогательные головы.
Результаты в таблице~\ref{tab:stl-mtl-performance-erc} показывают, что включение вспомогательных задач в процесс обучения улучшает точность на исходной бинарной задаче, подтверждая, что многозадачные сигналы действуют как полезное индуктивное смещение.

\begin{table}[t]
    \caption{Сравнение производительности и эмпирической сложности Радемахера (ERC) модели DeBERTa-v3-base в условиях однозадачного (STL) и многозадачного (MTL) обучения на задаче бинарной классификации GenAI Detection. MTL демонстрирует улучшение всех метрик качества (F1: 0.781→0.826, ROC-AUC: 0.788→0.834) и снижение ERC с $0.0159 \pm 0.0009$ до $0.0111 \pm 0.0010$, что согласуется с теоретическим предсказанием снижения сложности на задачу в $1/\sqrt{T}$ раз.}
    \label{tab:stl-mtl-performance-erc}
    \begin{center}
    \begin{tabular}{@{}lccccc@{}}
    \toprule
    Model & Mode & F1 $\uparrow$ & ROC--AUC $\uparrow$ & Acc. $\uparrow$ & ERC $\downarrow$ \\
    \midrule
    DeBERTa & STL & 0.781 & 0.788 & 0.710 & $0.0159 \pm 0.0009$ \\
    DeBERTa & MTL & \textbf{0.826} & \textbf{0.834} & \textbf{0.781} & $\mathbf{0.0111} \pm \mathbf{0.0010}$ \\
    \bottomrule
    \end{tabular}
    \end{center}
\end{table}

Далее, для оценки не только прогнозных улучшений, но и сложности модели на целевой задаче, выполняется прямое сравнение эмпирической сложности Радемахера (ERC).
В соответствии с определением, ERC измеряется на тех же данных, что использовались для обучения, после замены истинных меток случайным шумом. 

Это гарантирует, что оценка отражает только емкость гипотезного класса, а не информацию о реальных метках.
После обучения ERC оценивается исключительно для целевой задачи как для STL, так и для MTL.
Результаты представлены в колонке ERC в таблице~\ref{tab:stl-mtl-performance-erc}. 

Более подробное описание процесса проведения обоих экспериментов приведено в дополнительных материалах.

Анализ показывает, что MTL не только улучшает прогнозную производительность, но и достигает более низкой эмпирической сложности Радемахера по сравнению с STL.
Это демонстрирует, что вспомогательные задачи обеспечивают полезное индуктивное смещение, которое как улучшает обобщающую способность на целевой задаче, так и эффективно снижает сложность на задачу, что согласуется с предсказанным снижением на $1/\sqrt{T}$ из нашего теоретического анализа.

\begin{table}[h!t]
  \centering
  \caption{Сравнение моделей DeBERTa-v3-base при полной тонкой настройке и адаптации с LoRA ($r=8$) после 5 эпох обучения. LoRA обеспечивает более низкие потери (0.2972 vs 0.3058), более гладкие нормы градиентов (17.41 vs 24.65) и требует только 0.16\% параметров от полной модели (296{,}450 vs 184M), демонстрируя эффективность низкоранговой адаптации.}
  \label{tab:models_comparison}
  \begin{tabular}{lccccc}
    \toprule
    Model & Loss & Grad. Norm & Runtime (s) & Params & Epochs \\
    \midrule
    Full Tuning & 0.3058 & 24.65 & 679{,}660 & 184M (100\%) & 5 \\
    LoRA ($r=8$) & 0.2972 & 17.41 & 700{,}284 & 296{,}450 ($\approx$0.16\%) & 5 \\
    \bottomrule
  \end{tabular}
\end{table}

\begin{table}[h!t]
  \centering
  \caption{Сравнение конфигураций LoRA с различными рангами $r \in \{8, 16, 32\}$ для модели DeBERTa-v3-base. Анализ показывает, что более высокие ранги ($r=32$) обеспечивают более низкие потери при обучении, но $r=8$ обеспечивает лучшее обобщение, что подтверждает оптимальность низкоранговой параметризации для данной задачи.}
  \label{tab:lora_rank}
  \begin{tabular}{lccc}
    \toprule
    \textbf{Rank ($r$)} & Loss & Grad. Norm & Runtime (s) \\
    \midrule
    8  & 0.3413 & 2.95 & 1681 \\
    16 & 0.3358 & 5.18 & 1684 \\
    32 & 0.3194 & 2.55 & 1688 \\
    \bottomrule
  \end{tabular}
\end{table}

Все исходные параметры DeBERTa были заморожены, а низкоранговые адаптерные модули были вставлены в каждый блок трансформера.
Ранг адаптера выбран~$r=8$, коэффициент масштабирования $\alpha=32$ и использовался dropout $0.1$ в слоях адаптера.
Данная конфигурация выбрана как наиболее эффективная после проведения поиска по сетке для $r=\{8, 16, 32\}$.
Это оставляет обучаемыми только 296 450 параметров из 184 млн общих ($\approx0.16\%$), что согласуется с теоретической мотивацией снижения размерности.
Все конфигурации экспериментов указаны в таблице~\ref{tab:models_comparison}.

Кроме того, на рис.~\ref{fig:combined} представлена производительность модели: модели DeBERTa и DeBERTa~\&~LoRA обучались в течение 5 эпох с размороженными 6 слоями из 12.
Анализ показывает, что норма градиента является значительно более гладкой для модели, использующей LoRA, по сравнению с моделью без нее.
Тот же эффект наблюдается на графике функции потерь обучения: потери при обучении модели с LoRA являются более гладкими и сходятся быстрее, чем у модели без нее.

Для определения оптимального ранга адаптера сравнивается~$r=8, 16, 32$ с точки зрения потерь при обучении, представленных на рис.~\ref{fig:r_vs_loss}.
Хотя более высокие ранги, в частности $r=32$, демонстрировали более быструю сходимость и более низкие финальные потери при обучении, $r=8$ обеспечивает лучшее обобщение, несмотря на более медленную сходимость.
В целом, $r=8$ был выбран в качестве наиболее эффективной конфигурации в текущей настройке.
Конфигурации экспериментов показаны в таблице~\ref{tab:lora_rank}.

На основе аналитических результатов об эффективности Радемахера выполняется адаптация~\texttt{DeBERTa-base} для выполнения необходимых предположений.
Введенные модификации гарантируют, что эмпирическая установка соответствует теоретическим условиям, позволяя провести валидную оценку преимуществ MTL в рамках установленной схемы.
Подробный процесс подготовки описан в Алгоритме~\ref{alg:theorem-constraints}.

\begin{algorithm}[h!t]
    \caption{Алгоритм обеспечения условий теоремы~\ref{thm:per-task-rc-fair-scaling} для оценки эмпирической сложности Радемахера (ERC) с использованием модели DeBERTa. Алгоритм применяет спектральную нормализацию к линейным слоям энкодера, нормализует входные последовательности и проецирует параметры на $L_2$-шары для выполнения ограничений на нормы весов, необходимых для теоретического анализа.}\label{alg:theorem-constraints}
    \KwIn{Модель $f_t(x) = w_t^\top \phi(x; w_{\text{shared}})$, с ограничениями~$B_{\text{shared}}, B_{\text{head}}, R$}
    \KwOut{Модель удовлетворяющая условиям теоремы~\ref{thm:per-task-rc-fair-scaling}.}
    \ForEach{linear layer $W$ in encoder $\phi(\cdot; w_{\text{shared}})$}{
        $W \leftarrow \text{SpectralNorm}(W)$ \tcp*{$\|\phi(x;w)\|_2 \leq L\|w\|\|x\|_2$}
    }
    \ForEach{input sequence $x = [x_1 \dots x_m]$}{
        $\widetilde{x}_i \leftarrow x_i / \max(1, \|x_i\|_2) \quad \forall i$
        $x \leftarrow \left[\widetilde{x}_1 \dots \widetilde{x}_m\right] \cdot \min\left(1, \frac{R}{\|\widetilde{x}\|_F}\right)$ \tcp*{$\|x\| \leq R$}
    }
    \ForEach{training step}{
        \ForEach{parameter group $(w, B)$ in $\{(w_{\text{shared}}, B_{\text{shared}})\} \cup \{(w_t, B_{\text{head}}) \mid t \in [T]\}$}{
            $w \leftarrow w \cdot \min\left(1, \dfrac{B}{\|w\|_2}\right)$ \tcp*{$\|w\|_2 \leq B$}
        }
    }
\end{algorithm}

\begin{figure}[ht]
    \centering
    \includegraphics[width=0.7\linewidth]{thesis/figures/chapter-5/rademacher/r_vs_loss.png}
    \caption{Динамика потерь при обучении модели DeBERTa-v3-base с адаптерами LoRA различных рангов ($r \in \{8, 16, 32\}$) на задаче детекции машинно-генерированного текста. Анализ показывает, что более высокие ранги обеспечивают более быструю сходимость и более низкие финальные потери, но $r=8$ обеспечивает лучшее обобщение, что подтверждает оптимальность низкоранговой параметризации.}
    \label{fig:r_vs_loss}
\end{figure}

\begin{figure}[htbp]
    \centering
    \subfloat[Норма градиента]{\includegraphics[width=0.75\textwidth]{thesis/figures/chapter-5/rademacher/Gradient_norm.png}}\\
    \subfloat[Ошибка на обучающей выборке]{\includegraphics[width=0.75\textwidth]{thesis/figures/chapter-5/rademacher/Loss.png}}\\
    \caption{Динамика нормы градиента и потерь при обучении моделей DeBERTa-v3-base с полной тонкой настройкой и адаптацией LoRA ($r=8$) в течение 5 эпох на задаче детекции машинно-генерированного текста. LoRA демонстрирует более гладкие траектории нормы градиента и потерь, что указывает на более стабильную сходимость и большую эмпирическую устойчивость по сравнению с полной тонкой настройкой.}
    \label{fig:combined}
\end{figure}

Для оценки ERC обучается как STL, так и MTL модели в контролируемых и сопоставимых условиях.
В обоих случаях использовалось одинаковое общее количество обучающих примеров (\(nT\)), что гарантирует, что любые наблюдаемые различия в ERC вызваны совместным использованием параметров, а не дисбалансом данных.

В эксперименте STL использовался энкодер \texttt{DeBERTa-v3-base} с одной бинарной классификационной головой, соответствующей целевой задаче.
Модель обучалась на \(nT = 90{,}000\) примерах (\(n = 30{,}000\), \(T = 3\)), выбранных исключительно из набора данных целевой бинарной классификации.
Чтобы гарантировать, что ERC отражает репрезентационную способность модели, а не ее соответствие истинному распределению меток, все целевые метки были заменены случайным шумом из \(\{-1, 1\}\).
Энкодер и классификационная головка подвергались тонкой настройке на этих зашумленных метках в течение 3 эпох.
После обучения ERC вычислялась на том же наборе данных со случайными метками.

В настройке MTL использовался тот же энкодер DeBERTa-v3-base, разделяемый между \(T=3\) классификационными головками.
Одна головка соответствовала целевой бинарной задаче, в то время как две другие обучались на вспомогательных многоклассовых задачах: предсказание поддомена с 5 классами и предсказание поддомена с 6 классами.
Каждая задача предоставляла \(n = 30{,}000\) примеров, так что общий энкодер обрабатывал в сумме \(nT = 90{,}000\) обучающих примеров.
Только вспомогательные задачи использовали свои исходные метки; целевая задача использовала зашумленные метки для обеспечения независимости от емкости модели.
После обучения ERC оценивалась только на подмножестве целевой задачи.

Данная процедура гарантирует, что как STL, так и MTL модели обучались в идентичных условиях по объему данных и вычислительным ресурсам.
Следовательно, наблюдаемое снижение ERC типа \(1/\sqrt{T}\) непосредственно количественно оценивает эффект совместного использования параметров энкодера на эффективную емкость модели.

\subsection{Снижение сложности данных в задаче декодирования фМРТ-снимков}

Для анализа производительности предложенного метода и проверки гипотез был проведен вычислительный эксперимент.
В качестве данных использовалась выборка, представленная в~\cite{Berezutskaya2022}. 

Набор данных содержит результаты обследования 63 испытуемых.
Для тридцати из них известны данные фМРТ.
В выборке 16 мужчин и 14 женщин в возрасте от 7 до 47 лет.
Средний возраст испытуемых составляет 22 года.
Характеристики выборки: продолжительность обследования, частоты кадров видеопоследовательностей фМРТ и изображений, а также их размеры суммированы в таблице~\ref{table:sample}.

\begin{table}[h!t]\center
    \caption{Характеристики выборки для эксперимента по декодированию фМРТ-изображений из видеопоследовательностей. Выборка содержит данные 30 испытуемых с продолжительностью обследования 390 с, частотой кадров видео 25 Гц и частотой фМРТ-изображений 1.64 Гц, что обеспечивает временное соответствие между видеокадрами и томографическими данными.}\label{table:sample}
    \begin{tabular}{@{}ccc@{}}
    \toprule
    Name & Notation & Value \\ 
    \midrule
    Duration of examination & $t$ & 390 s \\
    Video frame rate & $\nu$ & 25 Hz \\
    fMRI frame rate & $\mu$ & 1.64 Hz \\
    Video dimensions & $W, H, C$ & 640, 480, 3 \\
    fMRI dimensions & $X, Y, Z$ & 40, 64, 64 \\
    \bottomrule
    \end{tabular}
\end{table}

Выборка разделена на обучающую и тестовую части в соотношении 70\% и 30\% соответственно.
Критерием качества реконструкции фМРТ-изображений является MSE~--- сумма квадратов отклонений между истинными и реконструированными изображениями, усредненная по всем вокселям каждого изображения из тестовой выборки.

Для сокращения времени работы алгоритма фМРТ-изображение предварительно сжимается с использованием слоя MaxPool3D.
Рассматриваются коэффициенты сжатия 1, 2, 4 и 8.
Значения вокселей нормализованы к диапазону $[0; 1]$ с помощью процедуры MinMaxScale.

Была проанализирована зависимость MSE от параметра регуляризации $\alpha$.
Рассматривались коэффициенты сжатия 1, 2, 4 и 8.
Соответствующие графики показаны на рис.~\ref{fig:mse-alpha}.
Для построения графика было выполнено усреднение по испытуемым.
Показаны границы стандартного отклонения.
Графики демонстрируют, что оптимальное значение коэффициента $\alpha \approx 1000$.
Форма кривой сохраняется независимо от коэффициента сжатия фМРТ-изображений.
\begin{figure}[h!t]\centering
	\includegraphics[width=0.65\textwidth]{thesis/figures/chapter-5/fmri/subs_MSE_alpha}
	\caption{Зависимость метрики MSE от параметра регуляризации $\alpha$ для метода декодирования фМРТ-изображений на тестовой выборке при различных коэффициентах сжатия (1, 2, 4, 8). Оптимальное значение коэффициента $\alpha \approx 1000$ сохраняется независимо от коэффициента сжатия, что подтверждает стабильность метода при снижении размерности данных.}
	\label{fig:mse-alpha}
\end{figure}

Выполняется сравнение времени обучения модели при использовании различных коэффициентов сжатия фМРТ-изображений.
Рассматриваются коэффициенты 1, 2, 4 и 8.
Для каждого значения коэффициента сжатия вычисляется среднее значение времени обучения модели для всех испытуемых. 

Результаты экспериментов представлены в таблице~\ref{table:coeffs}.
Время работы метода существенно сокращается при использовании предварительного сжатия фМРТ-изображений.
Эксперимент с подбором оптимального коэффициента регуляризации подтверждает, что сжатие изображений не изменяет характер зависимостей между параметрами и качеством реконструкции.

\begin{table}[h!t]\center
    \caption{Зависимость среднего времени обучения модели декодирования фМРТ-изображений от коэффициента сжатия данных. Использование предварительного сжатия фМРТ-изображений с коэффициентами 2, 4 и 8 обеспечивает существенное сокращение времени обучения (с 36.3 с до 6.7 с, 1.6 с и 1.4 с соответственно), что демонстрирует эффективность снижения сложности данных без потери качества реконструкции.}\label{table:coeffs}
    \begin{tabular}{@{}ccc@{}}
    \toprule
    Compression coefficient & Mean time, s & Std, s \\
    \midrule
    1 & 36.3 & 6.1 \\
    2 & 6.7 & 0.5 \\
    4 & 1.6 & 0.1 \\
    8 & 1.4 & 0.3 \\
    \bottomrule
    \end{tabular}
\end{table}

\subsection{Качество данных в задаче детекции машинно-генерированного контента}

\begin{table}[h!t]\centering
    \begin{tabular}{c|c|c|c}
    \toprule    
        Dataset & DeBERTa & Binoculars & DetectGPT \\
    \midrule
        GPT-2 &  0.972 & 0.495 & 0.412\\ 
        HC3 & 0.998 &  0.931 & 0.972 \\
        GhostBuster & 0.910 &   0.683  &  0.711 \\
        MGTBench & 0.961 & 0.364 & 0.447 \\
        MAGE &  0.835 &  0.632 &  0.654 \\
        M4 & 0.987 &  0.871 &  0.881   \\
        OutFox & 0.901 & 0.692 & 0.707 \\
        TweepFake & 0.941 & 0.845 & 0.864 \\ 
    \midrule
        SemEval24 Mono & 0.991 & 0.913 & 0.924 \\
        SemEval24 Multi & 0.994 & -- & --\\
        RuATD & 0.765 & -- & -- \\
        DAGPap22 &  0.968 & 0.333 & 0.562 \\
        PAN24  & 0.826 &  0.411 & 0.890 \\
        AuTex23en & 0.941 & 0.783 & 0.911\\
        AuTex23es & 0.933 & -- & --\\
        IberAuTex & 0.964 & -- & --\\
        MGT-1 Mono & 0.904 & 0.665 & 0.683 \\
        MGT-1 Multi & 0.934 & -- & --\\
    \bottomrule    
    \end{tabular}
    \caption{Результаты классификации различных детекторов машинно-генерированного текста (DeBERTa, Binoculars, DetectGPT) на множественных наборах данных, оцененные с помощью метрики $F_1$-score. Детектор на основе DeBERTa демонстрирует наиболее стабильную производительность на различных наборах данных, тогда как Binoculars и DetectGPT показывают значительные вариации, что указывает на проблемы с устойчивостью этих методов к различным доменам.}
    \label{tab:classifier}
\end{table}

\begin{table}[h!t]
    \centering
    \begin{tabular}{l|c|c|c|c|c}
    \toprule
        Dataset & $\text{KL}_{\text{TTS}}$ $\downarrow$ & $\text{PHD}_{\text{human}}$ & $\text{PHD}_{\text{machine}}$ &  $\Delta_{\text{shift}}$ $\downarrow$ &  $\text{KL}_\text{shuffle}$ $\downarrow$  \\
    \midrule
        GPT-2 & \textbf{0.014} & 9.23 $\pm$ 1.98 & 10.27 $\pm$ 1.84 &  0.084 & 1.255  \\
        HC3 & 0.053  & 8.76 $\pm$ 1.83 & 7.38 $\pm$ 1.05 & 0.264 &   1.167  \\
        GhostBuster & 0.053 & 9.84 $\pm$ 1.18 & 9.76 $\pm$ 1.15 & \textbf{0.024} &  \textbf{0.359}  \\
        MGTBench & 0.043  & 8.77 $\pm$ 1.31 & 9.97 $\pm$ 1.02 &  \textbf{0.031} &   \textbf{0.421} \\
        MAGE & \textbf{0.011}  & 9.8 $\pm$ 2.14 & 9.38 $\pm$ 3.04 & 0.094  & \textbf{0.310}  \\
        M4 & 0.036 & 7.26 $\pm$ 1.99 & 8.59 $\pm$ 1.4 & 0.107 &  \textbf{0.483}  \\
        OutFox & 0.025  & 8.96 $\pm$ 1.21 & 11.48 $\pm$ 1.13  & 0.095 & \textbf{0.237} \\
        TweepFake & \textbf{-} & 9.02 $\pm$ 3.19 & 8.12 $\pm$ 4.02 & 0.116 & 1.001 \\
    \midrule
        SemEval24 Mono & \textbf{0.012} & 9.11 $\pm$ 1.19 & 9.41 $\pm$ 1.2 &    0.191 &  2.576  \\
        SemEval24 Multi & \textbf{0.001}  & 9.65 $\pm$ 1.81 & 9.42 $\pm$ 1.44 &  0.059 &  2.046  \\
        RuATD & \underline{0.007} & 7.33 $\pm$ 1.4 & 7.46 $\pm$ 1.41 &  0.315 &  14.028  \\
        DAGPap22 & 0.083 & 8.35 $\pm$ 1.33 & 7.48 $\pm$ 2.01 &  \textbf{0.039} &   \textbf{0.472}  \\
        PAN24 & 0.053 & 9.4 $\pm$ 1.05 & 8.52 $\pm$ 1.59 &  \textbf{0.050} &  \textbf{0.331}  \\
        AuTex23 Eng & \underline{0.021} & 8.07 $\pm$ 2.26 & 8.1 $\pm$ 2.68 &  0.110 &  4.331 \\
        AuTex23 Esp &  \underline{0.001}& 9.16 $\pm$ 3.49 & 9.25 $\pm$ 3.26  &   0.105  &   1.306 \\
        IberAuTex & \textbf{0.012} & 9.33 $\pm$ 2.45 & 8.47 $\pm$ 2.73 &  0.223 & 5.516 \\
        MGT-1 Mono & \textbf{0.019} & 9.19 $\pm$ 1.75 & 8.96 $\pm$ 2.24 & \textbf{0.031} & 0.587\\
        MGT-1 Multi & \textbf{0.006} & 
8.76 $\pm$ 1.85 & 8.6 $\pm$ 2.29 &  \textbf{0.027} & 0.522\\
    \bottomrule    
    \end{tabular}
    \caption{Статистика качества данных для выбранных наборов данных детекции машинно-генерированного текста, включающая метрики $\text{KL}_{\text{TTS}}$, PHD, $\Delta_{\text{shift}}$ и $\text{KL}_\text{shuffle}$. Высокие значения метрик указывают на различимость текстов разного происхождения, тогда как низкие значения отражают схожую устойчивость к модификациям, что является признаком качественных данных.}
    \label{tab:results}
\end{table}

\begin{figure}[h!t]\centering
    \includegraphics[width=\textwidth]{thesis/figures/chapter-5/ai-datasets/violins_PHD}
    \caption{Распределения размерности персистентной гомологии (PHD) для человеческих и машинно-сгенерированных текстов в различных наборах данных детекции. Качественные наборы данных (SemEval, PAN24, MGT-1) демонстрируют схожие распределения PHD для обоих типов текстов, что указывает на высокое качество сгенерированных данных и их близость к человеческим текстам по топологическим характеристикам.}
    \label{fig:phd_values}
\end{figure}

\begin{figure}[t]
\centering
\includegraphics[width=1.00\textwidth]{thesis/figures/chapter-5/ai-datasets/4_datasets_TTS}
\caption{Топологические временные ряды (TTS) для четырех наборов данных с высокими значениями метрики $\text{KL}_{\text{TTS}}$: GhostBuster, PAN24, MGTBench и DAGPap22. Высокие значения $\text{KL}_{\text{TTS}}$ для GhostBuster и PAN24 обусловлены расхождением в текстах с более высокими размерностями, тогда как для MGTBench и DAGPap22~--- разницей в самих распределениях PHD, что указывает на различные типы различий между человеческими и машинно-сгенерированными текстами.} 
\label{tts}
\end{figure}

Из каждого набора данных выбрано 1000 документов из тестовой выборки, сбалансированных между двумя классами.
Для базовых методов выполняется тонкая настройка mdeberta-v3-base для каждого набора данных, после чего модель оценивается.
Для оценки качества Binoculars и Fast-DetectGPT использовался falcon-rw-1b~\cite{almazrouei2023falconseriesopenlanguage} и gpt-neo-2.7B~\cite{gpt-neo} соответственно.
Следует отметить, что для последних двух методов качество измерялось только на англоязычных выборках.
В эксперименте с топологическими признаками использовалась модель roberta-base, как и авторы оригинальной работы.
В эксперименте с пертурбациями и перемешиванием энкодер multilingual-e5-large использовался для построения эмбеддингов текстов, который показывает высокие метрики для кодирования высокоресурсных языков~\cite{e5}.
Результаты сравнения разработанных признаков в выбранных наборах данных представлены в таблице~\ref{tab:results}. 

Относительно PHD и оценки TTS, в предыдущих работах было показано, что тексты языковых моделей имеют меньшие значения PHD, чем написанные человеком.
Однако данный результат был получен для моделей GPT-2, GPT-3.5 и OPT.
Для более современных языковых моделей, которые генерируют более похожие на человеческие тексты, данная тенденция могла измениться. 

Высокое значение $\text{KL}_{\text{TTS}}$ для текстов разного происхождения означает, что детектору легче разделить такие тексты.
Следует отметить, что $\text{KL}_{\text{TTS}}$ также ограничена для более коротких текстов.
Относительно PHD предполагается, что сгенерированные тексты хорошего качества должны иметь PHD, аналогичный написанным человеком. 

Дополнительно сравниваются распределения PHD для всех наборов данных на рис.~\ref{fig:phd_values}.
Распределения для текстов обоих типов происхождения должны быть схожими, что в основном соблюдается для текстов из SemEval, PAN24 и MGT-1. 

В следующих столбцах приводится статистика, наблюдаемая на модифицированных текстах.
Для обеих метрик ($\Delta_{\text{shift}}$ и $\text{KL}_\text{shuffle}$) чем ниже значение, тем лучше, так как это отражает схожую степень устойчивости сгенерированных и человеческих текстов к адверсарным атакам.
Качественно сгенерированные данные без смещений должны принимать значения, близкие к человеческим.

В таблице~\ref{tab:classifier} показаны результаты применения современных детекторов к выбранным тестовым наборам данных.
На наборах данных с низкими значениями метрик качества в таблице~\ref{tab:results} может быть достигнуто качество, близкое к 1, что указывает на явное наличие смещения детектора в их сторону или структурной особенности, которая слишком очевидна для модели детекции. 

Невозможно судить о качестве данных только по достижению значений $F_1$, близких к 1.
Однако комбинирование значений двух таблиц позволяет оценить, какой набор имеет данные лучшего качества, а какой~--- худшего.

Относительно $\text{KL}_{\text{TTS}}$, на рис.~\ref{tts} показаны 4 набора данных с высоким значением этого показателя.
GhostBuster и PAN24 получили такой высокий балл из-за расхождения в текстах с более высокими размерностями, тогда как MGTBench и DAGPap22~--- из-за разницы в самих распределениях. 

Следует отметить, что $\text{KL}_{\text{TTS}}$ может плохо работать с очень короткими текстами, поскольку внутренний метод вычисления PHD требует достаточно длинных текстов для стабильного вычисления.
По этой причине $\text{KL}_{\text{TTS}}$ для RuATD, AuTex23-es и Tweepfake отбрасываются, так как они не соответствуют критериям.
Кроме того, показано, что тексты должны быть достаточной длины~\cite{gritsay2022automatic526652068} для построения надежных детекторов.

Анализ значений в таблице~\ref{tab:results} показывает наличие данных достаточно высокого качества в выбранных наборах данных.
Разработанные атрибуты в совокупности способны отражать качество сгенерированного набора данных с различных точек зрения. 

Предлагается использовать эти атрибуты в сочетании с другими статистическими инструментами для оценки качества данных, например, с законом Ципфа~\cite{zipf}.

Представленная статистика может быть использована для оценки качества коллекций и их улучшения.
Кроме того, наборы данных, собирающие машинно-генерированный контент, могут быть полезны для двух более общих целей.
Во-первых, высококачественные сгенерированные данные могут быть использованы для оценки качества каузальной модели во время обучения, что служит одной из целей обучения для улучшения ответов модели и приближения их к человеческим.
Во-вторых, хорошие детекторы могут помочь очистить обучающие наборы, поскольку большая доля низкокачественных сгенерированных текстов в этих наборах может привести к возникновению смещений в сторону некорректной структуры и артефактов в выходных данных модели в будущем.

Вопрос о том, означает ли низкая производительность детекторов низкое качество набора данных, не имеет однозначного ответа.
Например, в~\cite{hans2024spotting} метод Binoculars достигает оценки $F_1$, близкой к 1.0, в то время как в наших экспериментах был получен широкий диапазон оценок: от 0.33 на DAGPap22 до 0.93 на HC3.
Для HC3 все три детектора показали схожие результаты, что позволяет предположить, что тексты HC3 относительно легко обнаружить.
Однако данная согласованность не распространяется на DAGPap22.
Детектор на основе DeBERTa достиг оценки $F_1$ 0.96, в то время как DetectGPT показал только 0.562.
Данная закономерность, когда детектор на основе DeBERTa демонстрирует заметно более высокие результаты, чем два других метода, наблюдалась на значительной части проанализированных наборов данных.

Низкие оценки для Binoculars заслуживают дополнительного изучения.
Даже при фокусировке на доменах, специально протестированных его авторами, таких как PAN24 и Outfox, оценки оказываются значительно ниже почти идеальных результатов, представленных в~\cite{hans2024spotting}.
Данное расхождение позволяет предположить, что детектор Binoculars может быть нерепрезентативным.
Аналогично, в наших экспериментах оценки DetectGPT сопоставимы с оценками Binoculars, что указывает на схожие базовые проблемы с устойчивостью этих детекторов.

\section{Заключение по главе}

В настоящей главе продемонстрировано практическое применение теоретического аппарата оценки сложности моделей и данных, разработанного в главах~\ref{chapter:complexity} и~\ref{chapter:gesian}, к решению прикладных задач машинного обучения. 

В отличие от предыдущих глав, где основной акцент делался на разработке строгих математических оценок и методов анализа, здесь представлена адаптация теоретических подходов к реальным проблемам.
Глава охватывает три ключевых направления: (1) управление сложностью моделей в многозадачном обучении, (2) снижение сложности данных в задачах нейровизуализации и (3) оценка качества данных в задачах детекции машинно-генерированного контента.

В разделе о сложности моделей в многозадачном обучении получены фундаментальные теоретические результаты для адаптеров LoRA и многозадачного обучения.
Теорема~\ref{theorem:consistency} строго доказывает статистическую состоятельность минимизации эмпирического риска при параметризации LoRA, устанавливая, что низкоранговая адаптация сохраняет асимптотические статистические гарантии полной тонкой настройки.
Теорема~\ref{theorem:lowrank} доказывает корректность выходного слоя при низкоранговых обновлениях, показывая, что параметризация $\Delta W = AB$ с $r \ll \min(d,k)$ не искажает выходное распределение или классификационные границы модели.
Теорема~\ref{thm:per-task-rc-fair-scaling} устанавливает снижение эмпирической сложности Радемахера на задачу в $1/\sqrt{T}$ раз при многозадачном обучении с соответствующим масштабированием, что количественно характеризует преимущества совместного использования параметров энкодера.

Эмпирические эксперименты на задаче детекции машинно-генерированного текста (GenAI Detection Challenge) подтвердили теоретические предсказания.
LoRA-адаптеры с рангом $r=8$ демонстрируют незначительное снижение метрик точности (1.4--3.2\%) по сравнению с полной тонкой настройкой, но обеспечивают существенное снижение потерь на валидации (36.3\%) и ускорение обучения (12.6\%), требуя при этом только 0.16\% параметров от полной модели.
Многозадачное обучение с тремя выходными головками улучшает метрики качества (F1: 0.781→0.826, ROC-AUC: 0.788→0.834) и снижает эмпирическую сложность Радемахера с $0.0159 \pm 0.0009$ до $0.0111 \pm 0.0010$, что количественно подтверждает теоретическое предсказание снижения сложности на задачу в $1/\sqrt{T}$ раз.

В разделе о снижении сложности данных в задаче декодирования фМРТ-изображений разработан метод реконструкции томографических данных из видеопоследовательностей, использующий архитектуру ResNet152 для векторизации видеокадров и линейную регрессию с $L_2$-регуляризацией для предсказания разностей между последовательными фМРТ-снимками.
Предложенный метод предварительного сжатия фМРТ-изображений с коэффициентами 2, 4 и 8 обеспечивает существенное сокращение времени обучения (с 36.3 с до 6.7 с, 1.6 с и 1.4 с соответственно) без потери качества реконструкции.
Экспериментальные результаты на выборке из 30 испытуемых показали, что оптимальное значение коэффициента регуляризации $\alpha \approx 1000$ сохраняется независимо от коэффициента сжатия, что подтверждает стабильность метода при снижении размерности данных и демонстрирует практическую эффективность управления сложностью данных в задачах нейровизуализации.

В разделе о качестве данных в задаче детекции машинной генерации разработаны методы оценки качества наборов данных, содержащих AI-фрагменты, основанные на топологической статистике, анализе устойчивости к адверсарным возмущениям и перемешиванию предложений.
Предложены четыре метрики качества: $\text{KL}_{\text{TTS}}$, PHD, $\Delta_{\text{shift}}$ и $\text{KL}_\text{shuffle}$.
Экспериментальный анализ множественных наборов данных с использованием трех детекторов выявил значительные вариации в качестве данных: детектор на основе DeBERTa демонстрирует наиболее стабильную производительность на различных наборах данных, тогда как другие детекторы показывают значительные вариации, что указывает на проблемы с устойчивостью этих методов к различным доменам.
Качественные наборы данных (SemEval, PAN24, MGT-1) демонстрируют схожие распределения PHD для человеческих и машинно-сгенерированных текстов, что указывает на высокое качество сгенерированных данных и их близость к человеческим текстам по топологическим характеристикам.

Полученные результаты создают основу для практического применения теоретического аппарата сложности моделей и данных в реальных задачах машинного обучения.
Теоретические гарантии для LoRA и многозадачного обучения обеспечивают обоснованный выбор архитектурных решений при разработке эффективных и компактных моделей.
Методы снижения сложности данных в задачах нейровизуализации демонстрируют практическую значимость управления размерностью данных без потери качества.
Метрики оценки качества данных открывают возможности для создания надежных бенчмарков и улучшения качества обучающих наборов данных.

Основные ограничения представленных подходов связаны с вычислительной сложностью анализа топологических характеристик для очень коротких текстов, а также с необходимостью адаптации методов оценки качества данных к различным доменам и языкам.
Теоретическое обоснование методов снижения сложности данных в настоящее время ограничено задачами регрессии с линейными моделями.

Перспективными направлениями дальнейших исследований являются разработка упрощенных практических метрик сложности, которые могли бы служить мостом между строгим теоретическим аппаратом предыдущих глав и потребностями прикладных задач, расширение методов оценки качества данных на другие модальности, интеграция предложенных метрик в процесс создания и валидации обучающих наборов данных, а также разработка адаптивных методов выбора ранга LoRA-адаптеров на основе теоретических оценок сложности Радемахера.


% Выводы
\clearpage
\chapter*{Заключение}
\addcontentsline{toc}{section}{Заключение}
В диссертации рассмотрена научная проблема, имеющая теоретическое и прикладное значение в области математических методов моделей глубокого обучения, связанная с отсутствием системного подхода к оценке и управлению сложностью моделей и данных.
В рамках диссертационной работы предложено решение проблемы разработки единого математического аппарата для оценки и управления сложностью как моделей глубокого обучения, так и данных, что позволяет перейти к системному проектированию архитектур нейронных сетей и оптимизации процессов их обучения.
Предложенный подход основан на введении сложности в рамках теории мер, анализе матриц Гессе и ландшафта оптимизационной задачи, что обеспечивает теоретическую основу для предсказания поведения моделей при масштабировании и количественной оценки соответствия между сложностью модели и сложностью данных.

Основные результаты диссертационной работы перечислены в следующих пунктах:
\begin{enumerate}
    \item Разработан единый формальный аппарат для оценки сложности моделей и данных на основе теории мер и анализа ландшафта оптимизационной задачи.
    Введены определения меры сложности выборки и меры сложности модели, установлен критерий обучаемости модели на выборке, определяющий необходимое условие предотвращения переобучения.
    \item Введена ландшафтная мера сложности модели, определяемая через спектральные свойства матриц Гессе функции потерь, и установлена ее связь с условной сложностью выборки.
    \item Получены теоретические оценки ландшафтных мер на основе спектральных норм матриц Гессе для разных архитектур глубокого обучения: полносвязных, сверточных и трансформерных сетей. Установлены зависимости ландшафтных мер от глубины сети, размеров слоев и других структурных параметров архитектур.
    \item Разработаны практические методы оценки достаточного объема выборки. Проведен систематический сравнительный анализ методов на основе сэмплирования эмпирической функции ошибки и анализа близости апостериорных распределений параметров с классическими статистическими и байесовскими методами, подтверждена эффективность предложенных подходов.
    \item Предложены методы мультидоменной дистилляции и анти-дистилляции для передачи знаний между моделями различной сложности и между различными доменами данных.
    Для многозадачного обучения получены теоретические результаты о статистической состоятельности низкоранговых адаптеров и снижении эмпирической сложности при совместном использовании параметров энкодера.
\end{enumerate}

В целом совокупность полученных в диссертации теоретических и практических результатов позволяет сделать вывод о том, что цель исследований достигнута, сформулированная научная проблема решена.
Разработан единый математический аппарат, связывающий сложность моделей и данных в рамках строгой теоретической основы, применимой к широкому классу архитектур нейронных сетей.
Получены вычислимо осуществимые методы оценки сложности, не требующие прямого вычисления матриц Гессе для параметрических моделей глубокого обучения.
Разработаны практические инструменты для оценки достаточного объема выборки и снижения сложности моделей, подтвержденные экспериментально на реальных задачах.
Перечисленные результаты получили высокую оценку научного сообщества при апробации на ведущих международных конференциях и в научных публикациях.

\clearpage
\chapter*{Общие свойства и определения}
\addcontentsline{toc}{section}{Общие свойства и определения}
\begin{property}[Норма матричного произведения]
\label{prop:matrix_product_norm}
    Пусть матрицы $\mathbf{A} \in \mathbb{R}^{m \times n}$ и $\mathbf{B} \in \mathbb{R}^{n \times q}$, тогда справедливо неравенство
    \begin{equation}
        \| \mathbf{A} \mathbf{B}\|_2 \leq \| \mathbf{A}\|_2 \| \mathbf{B}\|_2
    \end{equation}
\end{property}

\begin{property}[Норма матричного произведения Кронекера]\label{prop:kronecker_product_norm}
    Пусть матрицы $\mathbf{A} \in \mathbb{R}^{m \times n}$ и $\mathbf{B} \in \mathbb{R}^{p \times q}$, тогда справедливо равенство
    \begin{equation}
        \| \mathbf{A} \otimes \mathbf{B}\|_2 = \| \mathbf{A}\|_2 \| \mathbf{B}\|_2
    \end{equation}
\end{property}

\begin{property}[Норма матричного транспонирования]\label{prop:transposed_matrix_norm}
    Пусть матрица $\mathbf{A} \in \mathbb{R}^{m \times n}$, тогда справедливо равенство
    \begin{equation}
        \| \mathbf{A}\|_2 = \| \mathbf{A}^\top\|_2
    \end{equation}
\end{property}

\begin{property}[Соотношения между матричными нормами] \label{prop:matrix_norm_inequalities}
    Пусть матрица $\mathbf{A} \in \mathbb{R}^{m \times n}$, тогда следующие неравенства между матричными нормами справедливы:
    
    \begin{center}
        \begin{tabular}{c|ccccc}
            \diagbox[height=2em, width=4em]{X}{Y} & $\|\mathbf{A}\|_{\max}$ & $\|\mathbf{A}\|_1$ & $\|\mathbf{A}\|_\infty$ & $\|\mathbf{A}\|_2$ & $\|\mathbf{A}\|_F$ \\
            \hline
            $\|\mathbf{A}\|_{\max}$ &  & 1 & 1 & 1 & 1 \\
            $\|\mathbf{A}\|_1$ & $m$ &  & $m$ & $\sqrt{m}$ & $\sqrt{m}$ \\
            $\|\mathbf{A}\|_\infty$ & $n$ & $n$ &  & $\sqrt{n}$ & $\sqrt{n}$ \\
            $\|\mathbf{A}\|_2$ & $\sqrt{mn}$ & $\sqrt{n}$ & $\sqrt{m}$ &  & 1 \\
            $\|\mathbf{A}\|_F$ & $\sqrt{mn}$ & $\sqrt{n}$ & $\sqrt{m}$ & $\sqrt{d}$ & 1 \\
        \end{tabular}
    \end{center}
    где $d = \operatorname{rank}(\mathbf{A})$.
    Таблица читается следующим образом: для каждой пары норм $\|\cdot\|_X$ и $\|\cdot\|_Y$,
    \begin{equation}
        \|\mathbf{A}\|_X \leq c \cdot \|\mathbf{A}\|_Y
    \end{equation}
    где $c \geq 0$~--- неотрицательная константа на пересечении строки $X$ и столбца $Y$.
\end{property}

\begin{property}[Связь между $\mathrm{vec}$ и $\mathrm{vec}_r$]\label{prop:vec_relation}
    Пусть задана матрица~$\mathbf{A} \in \mathbb{R}^{m \times n}$. Тогда оператор построчной векторизации~$\mathrm{vec}_r$ и стандартный оператор покоординатной векторизации~$\mathrm{vec}$ связаны через операцию транспонирования:
    \begin{equation}
        \mathrm{vec}_r(\mathbf{A}) = \mathrm{vec}(\mathbf{A}^\top)
    \end{equation}
\end{property}

\begin{property}[Норма матричной суммы]\label{prop:matrix_sum_norm}
    Пусть матрицы $\mathbf{A}$ и $\mathbf{B}$ принадлежат пространству матриц $\mathbb{R}^{m \times n}$, тогда
    \begin{equation}
        \| \mathbf{A} + \mathbf{B}\|_2 \leq \| \mathbf{A}\|_2 + \| \mathbf{B}\|_2
    \end{equation}
\end{property}

\begin{property}[Неравенство для нормы блочной матрицы]\label{prop:block_matrix_norm}
    Пусть матрица~$\mathbf{A} \in \mathbb{R}^{m \times n}$ представляет собой блочную матрицу, разбитую на блоки $\mathbf{B}_{i,j}$ размеров $m_i \times n_j$, где $\sum_i m_i = m$ и $\sum_j n_j = n$. Тогда справедливо неравенство:
    \begin{equation}
        \| \mathbf{A} \|_2 \leq \sqrt{m n} \max\limits_{i,j}\| \mathbf{B}_{i,j}\|_2
    \end{equation}
    
    Отметим, что если матрица~$\mathbf{A}$ является блочно-диагональной, то имеет место строгое равенство~$\| \mathbf{A} \|_2 = \max\limits_{i}\| \mathbf{B}_{i,i}\|_2$, что следует из того, что спектральная норма блочно-диагональной матрицы равна максимальной спектральной норме диагональных блоков.
\end{property}

\begin{property}[Поэлементное деление]\label{prop:elem_wise_division}
    Пусть задана матрица~$\mathbf{A} \in \mathbb{R}^{m\times n}$ и вектор~$\mathbf{b} \in \mathbb{R}^{m \times 1}$ такой, что $b_i \neq 0$ для всех $i = 1, \ldots, m$. Тогда для матрицы $\mathbf{C} \in \mathbb{R}^{m \times n}$, где $c_{i,j} = \frac{a_{i,j}}{b_i}$, справедливо равенство
    \begin{equation}
        \mathbf{C} = \mathrm{diag}^{-1}(\mathbf{b})\mathbf{A}
    \end{equation}
\end{property}

\begin{property}[Производная матричного произведения]\label{prop:matrix_product_derivative}
    Пусть заданы матрицы~$\mathbf{A} \in \mathbb{R}^{m \times n}$, $\mathbf{X} \in \mathbb{R}^{n \times q}$ и $\mathbf{B} \in \mathbb{R}^{q \times p}$, тогда
    \begin{equation}
        \frac{\partial\mathbf{A}\mathbf{X}\mathbf{B}}{\partial\mathbf{X}} = \mathbf{A} \otimes \mathbf{B}^\top
    \end{equation}
    где~$\mathbf{A}$ и~$\mathbf{B}$ не зависят от $\mathbf{X}$.
\end{property}

\begin{property}[Построчная векторизация произведения Адамара] \label{prop:vec_r_hadamard_product}
    Пусть заданы матрицы~$\mathbf{A}, \mathbf{B} \in \mathbb{R}^{m \times n}$. Тогда
    \begin{equation}
        \mathrm{vec}_r(\mathbf{A} \circ \mathbf{B}) = \mathrm{diag}(\mathrm{vec}_r(\mathbf{A})) \mathrm{vec}_r(\mathbf{B})
    \end{equation}
    где $\circ$ обозначает произведение Адамара.

    Утверждение непосредственно следует из \cite{magnus1988matrix}, где был получен аналогичный результат для покоординатной векторизации.
\end{property}

\begin{property}[Построчная векторизация матричного произведения]\label{prop:vec_r_matrix_product}
    Пусть заданы матрицы~$\mathbf{A} \in \mathbb{R}^{m \times n}$, $\mathbf{X} \in \mathbb{R}^{n \times q}$ и $\mathbf{B} \in \mathbb{R}^{q \times p}$, тогда
    \begin{equation}
        \mathrm{vec}_r(\mathbf{A} \mathbf{X}\mathbf{B}) = (\mathbf{A} \otimes \mathbf{B}^\top) \mathrm{vec}_r(\mathbf{X})
    \end{equation}
\end{property}

\begin{property}[Производная произведения Кронекера]\label{prop:kronecker_product_derivative}
    Пусть заданы матрица~$\mathbf{X} \in \mathbb{R}^{n \times q}$ и матрица~$ \mathbf{Y} \in \mathbb{R}^{p \times r}$. Тогда
    \begin{equation}
        \frac{\partial (\mathbf{X} \otimes \mathbf{Y})}{\partial \mathbf{X}} = (\mathbf{I}_n \otimes \mathbf{K}_{p,q} \otimes \mathbf{I}_r) \left( \mathbf{I}_{nq} \otimes \mathrm{vec}_r(\mathbf{Y}) \right),
    \end{equation}
    и аналогично
    \begin{equation}
        \frac{\partial (\mathbf{X} \otimes \mathbf{Y})}{\partial \mathbf{Y}} = (\mathbf{I}_n \otimes \mathbf{K}_{p,q} \otimes \mathbf{I}_r) \left( \mathrm{vec}_r(\mathbf{X}) \otimes \mathbf{I}_{pr} \right).
    \end{equation}
\end{property}

\begin{definition}[Матрица перестановки]\label{def:commutation_matrix}
    Матрица перестановки~$\mathbf{K}_{m,n} \in \mathbb{R}^{mn \times mn}$~--- это единственная матрица такая, что для любой матрицы~$\mathbf{A} \in \mathbb{R}^{m \times n}$ выполняется:
    \begin{equation}
        \mathbf{K}_{m,n} \mathrm{vec}(\mathbf{A}) = \mathrm{vec}(\mathbf{A}^\top)
    \end{equation}
    В силу свойства~\ref{prop:vec_relation} справедливо соотношение:
    \begin{equation}
        \mathrm{vec}_r(\mathbf{A}) = \mathbf{K}_{m,n} \mathrm{vec}(\mathbf{A}) \quad \text{и} \quad \mathrm{vec}(\mathbf{A}) = \mathbf{K}_{n,m} \mathrm{vec}_r(\mathbf{A})
    \end{equation}
    поскольку $\mathbf{K}_{n,m}\mathbf{K}_{m,n} = \mathbf{I}_{mn}$.
\end{definition}

\begin{definition}[Векторизация и поэлементные операции]\label{def:vec_elem_ops}
    Пусть задана матрица~$\mathbf{A} \in \mathbb{R}^{m \times n}$ и вектор~$\mathbf{v} \in \mathbb{R}^{n}$. Тогда
    \begin{itemize}
        \item $\mathrm{vec}_r(\mathbf{A})$ обозначает построчную векторизацию матрицы $\mathbf{A}$, отображающую матрицу $\mathbf{A}$ в вектор размерности $mn$ путем последовательной записи строк матрицы.
        \item $\mathbf{A}^{\circ \alpha}$ обозначает поэлементное возведение матрицы $\mathbf{A}$ в степень $\alpha$, то есть $(\mathbf{A}^{\circ \alpha})_{ij} = (\mathbf{A}_{ij})^\alpha$ для всех $i = 1, \ldots, m$ и $j = 1, \ldots, n$.
        \item $\mathrm{diag}(\mathbf{v})$ создает диагональную матрицу размерности $n \times n$ с вектором $\mathbf{v}$ на главной диагонали и нулями вне главной диагонали.
    \end{itemize}
\end{definition}

\begin{definition}[Матричные нормы]\label{def:matrix_norms}
    Для матрицы $\mathbf{A} \in \mathbb{R}^{m \times n}$, где $r = \operatorname{rank}(\mathbf{A})$ и $\sigma_1 \geq \sigma_2 \geq \ldots \geq \sigma_r > 0$~--- упорядоченные по убыванию сингулярные числа матрицы $\mathbf{A}$ (с учетом кратности), определены следующие нормы:
    \begin{align}
        \|\mathbf{A}\|_2 &= \sigma_1, \\
        \|\mathbf{A}\|_F &= \sqrt{\sum_{i=1}^m \sum_{j=1}^n |a_{ij}|^2} = \sqrt{\sum_{i=1}^r \sigma_i^2}, \\
        \|\mathbf{A}\|_1 &= \max_{1 \leq j \leq n} \sum_{i=1}^m |a_{ij}|,  \\
        \|\mathbf{A}\|_\infty &= \max_{1 \leq i \leq m} \sum_{j=1}^n |a_{ij}|, \\
        \|\mathbf{A}\|_{\max} &= \max_{i,j} |a_{ij}|.
    \end{align}
\end{definition}


\clearpage 
\addcontentsline{toc}{section}{Список иллюстраций}
\listoffigures

\clearpage
\addcontentsline{toc}{section}{Список таблиц}
\listoftables

\clearpage
\addcontentsline{toc}{section}{Список литературы}
\renewcommand{\bibname}{Список литературы}

\nocite{*}
\bibliography{dis_literature}

\appendix
\clearpage
\chapter{Дополнительные Леммы и утверждения}
\begin{lemma}[Производная умножения матричнозначных функций]\label{lemma:matrix_funcs_product_derivative}
    Пусть заданы~$\mathbf{A}(\mathbf{X}) \in \mathbb{R}^{p \times r}$ и~$\mathbf{B}(\mathbf{X}) \in \mathbb{R}^{r \times q}$ матрицезначные функции переменной~$\mathbf{X}$, тогда
    \begin{equation}
        \frac{\partial\mathbf{A}(\mathbf{X})\mathbf{B}(\mathbf{X})}{\partial \mathbf{X}} = \left(\mathbf{A} \otimes \mathbf{I}_q 
     \right) \frac{\partial\mathbf{B}}{\partial\mathbf{X}} + \left( \mathbf{I}_p \otimes \mathbf{B}^\top\right) \frac{\partial\mathbf{A}}{\partial\mathbf{X}}
    \end{equation}
\end{lemma}
\begin{proof}
Применить цепное правило для вычисления производной сложной функции, а затем объединим его со свойством~\ref{prop:matrix_product_derivative}
    \begin{align}
        \frac{\partial\mathbf{A}(\mathbf{X}) \mathbf{B}(\mathbf{X})}{\partial \mathbf{X}} &= \frac{\partial \mathbf{A}\mathbf{B}}{\partial\mathbf{B}}\frac{\partial\mathbf{B}}{\partial \mathbf{X}} + \frac{\partial \mathbf{A}\mathbf{B}}{\partial\mathbf{A}}\frac{\partial\mathbf{A}}{\partial \mathbf{X}} =\\
        &=\frac{\partial \mathbf{A}\mathbf{B}\mathbf{I}_q}{\partial\mathbf{B}}\frac{\partial\mathbf{B}}{\partial \mathbf{X}} + \frac{\partial \mathbf{I}_p\mathbf{A}\mathbf{B}}{\partial\mathbf{A}}\frac{\partial\mathbf{A}}{\partial \mathbf{X}} = \\
        &= \left( \mathbf{A} \otimes \mathbf{I}_q \right) \frac{\partial \mathbf{B}}{\partial \mathbf{X}} + \left( \mathbf{I}_p \otimes \mathbf{B}^\top \right) \frac{\partial \mathbf{A}}{\partial \mathbf{X}}
    \end{align}
\end{proof}

\begin{lemma}[Теорема идентификации для построчной векторизации] \label{lemma:identification_theorem_vec_r}

Пусть отображение~$\mathbf{F}: \mathbb{R}^{m\times n} \rightarrow \mathbb{R}^{p, q}$ является дифференциальной матричнозначной функцией от~$\mathbf{X} \in \mathbb{R}^{m \times n}$.
Если дифференциал функции~$\mathbf{F}$ может быть записан в виде:
\begin{equation}
    d \mathrm{vec}_r(\mathbf{F}(\mathbf{X})) = \mathbf{J} \cdot d \mathrm{vec}_r(\mathbf{X}),
\end{equation}
где матрица~$\mathbf{J} \in \mathbb{R}^{pq \times mn}$ является, некоторой константной матрицей относительно переменной~$d\mathbf{X}$, тогда матрица~$\mathbf{J}$ является матрицей Якоби преобразования $\mathbf{F}(\mathbf{X})$ относительно построчной векторизации. Обозначим это в следующем виде:
\begin{equation}
    \frac{\partial \mathbf{F}(\mathbf{X})}{\partial \mathbf{X}} := \frac{\partial \mathrm{vec}_r(\mathbf{F}(\mathbf{X}))}{\partial (\mathrm{vec}_r(\mathbf{X}))^\top} = \mathbf{J}
\end{equation}
    
\end{lemma}
\begin{proof}
Это построчный $\mathrm{vec}_r$-аналог первой теоремы об идентификации (англ. first identification theorem) в главе 12~\cite{magnus1988matrix} для векторизации по столбцам.
\end{proof}

\begin{lemma}[Производная квадрата Адамара]\label{lemma:hadamard_square_derivative}
Для матрицы~$\mathbf{A} \in \mathbb{R}^{m \times n}$, производная поэлементного квадрата равна
\[
    \frac{\partial \mathbf{A}^{\circ 2}}{\partial \mathbf{A}}
    = 2 \cdot \mathrm{diag}\!\big(\mathrm{vec}_r(\mathbf{A})\big).
\]
\end{lemma}
\begin{proof}
Используя определение~\ref{def:vec_elem_ops}, а именно $(\mathbf{A}^{\circ 2}){ij} = (\mathbf{A}{ij})^2$.
Выполняя поэлеметное взятие дифференциала, получаем, что:
\[
    d(\mathbf{A}^{\circ 2}) = 2\mathbf{A} \circ d\mathbf{A}.
\]
Далее, применяя оператор~$\mathrm{vec}_r$ и используя свойство~\ref{prop:vec_r_hadamard_product} получаем выражение:
\[
    \mathrm{vec}_r(d(\mathbf{A}^{\circ 2})) = 2 \textit{diag}(\mathrm{vec}_r(\mathbf{A})) \mathrm{vec}_r (d\mathbf{A}),
\]
причем, используя построчный аналог первой теоремы об идентификации~\ref{lemma:identification_theorem_vec_r} получаем следующий вид:
\[
    \frac{\partial \mathbf{A}^{\circ 2}}{\partial \mathbf{A}}
    = \frac{\partial \mathrm{vec}_r(\mathbf{A}^{\circ 2})}{\partial \mathrm{vec}_r(\mathbf{A})} = 2 \cdot \mathrm{diag}\!\big(\mathrm{vec}_r(\mathbf{A})\big).
\]
\end{proof}

\begin{lemma}[Производная корня Адамара]\label{lemma:hadamard_root_derivative}
Для матричнозначной функции~$\mathbf{A} \in \mathbb{R_+}^{m \times n}$ с положительными элементами, производная поэлементного корня равна
\[
    \frac{\partial \mathbf{A}^{\circ \frac{1}{2}}}{\partial \mathbf{A}}
    = \tfrac{1}{2} \mathrm{diag}^{-1}\!\big(\mathrm{vec}_r^{\circ \frac{1}{2}}(\mathbf{A})\big).
\]
\end{lemma}
\begin{proof}
Аналогично, доказательству леммы~\ref{lemma:hadamard_square_derivative} получаем~$d(\mathbf{A}^{\circ 1/2}) = \frac{1}{2}\mathbf{A}^{\circ -1/2} \circ d\mathbf{A},$  откуда в векторном виде получаем, следующее выражение:
\[
    \frac{\partial \mathbf{A}^{\circ \frac{1}{2}}}{\partial \mathbf{A}}
    = \frac{\partial \mathrm{vec}_r(\mathbf{A}^{\circ \frac{1}{2}})}{\partial \mathrm{vec}_r(\mathbf{A})} = \tfrac{1}{2} \mathrm{diag}^{-1}\!\big(\mathrm{vec}_r^{\circ \frac{1}{2}}(\mathbf{A})\big).
\]
\end{proof}

\begin{lemma}[Производная обратной матрицы]\label{lemma:invert_derivative}
Пусть задана обратимая квадратная матрица~$\mathbf{D} \in \mathbb{R}^{n \times n}$, тогда производная операции обращения равно:
\[
    \frac{\partial \mathbf{D}^{-1}}{\partial \mathbf{D}} 
    = -\mathbf{D}^{-1} \otimes \mathbf{D}^{-\top}.
\]
\end{lemma}
\begin{proof}
По определению из~\cite{petersen2012matrix} и~\cite{magnus1988matrix}:
\[
    d(\mathbf{D}^{-1}) = -\mathbf{D}^{-1} (d\mathbf{D}) \mathbf{D}^{-1}.
\] 
Используя оператор~$\mathrm{vec}_r$ и свойство~\ref{prop:vec_r_matrix_product}, получаем 
\[
    \mathrm{vec}_r(-\mathbf{D}^{-1} (d\mathbf{D}) \mathbf{D}^{-1}) = (-\mathbf{D}^{-1} \otimes \mathbf{D}^{-\top}) \mathrm{vec}_r(d\mathbf{D}),
\]
причем, используя лемму~\ref{lemma:identification_theorem_vec_r} получаем:
\[
    \mathrm{vec}_r(d\mathbf{D}^{-1}) = \frac{\partial \mathrm{vec}_r \mathbf{D}^{-1}}{\partial\mathrm{vec}_r\mathbf{D}} \mathrm{vec}_r(d\mathbf{D}).
\]
Следовательно получаем выражение, которое заканчивает доказательство леммы~ $\frac{\partial \mathrm{vec}_r \mathbf{D}^{-1}}{\partial\mathrm{vec}_r\mathbf{D}} = (-\mathbf{D}^{-1} \otimes \mathbf{D}^{-\top})$
\end{proof}

\begin{lemma}[Производная $\mathrm{diag}(\cdot)$]\label{lemma:diag_derivative}
Для вектора~$\mathbf{v} \in \mathbb{R}^{L \times 1}$, производная оператора диагонализации является:
\[
    \frac{\partial \mathrm{diag}(\mathbf{v})}{\partial \mathbf{v}}
    = \big(\mathbf{e}_1 \otimes \mathbf{e}_1 \quad \dots \quad \mathbf{e}_L \otimes \mathbf{e}_L\big),
\]
где вектора~$\mathbf{e}_i$ являются базисными в пространстве~$\mathbb{R}^L$.
\end{lemma}

\begin{proof}
По определению~\ref{def:vec_elem_ops} оператор~$\mathrm{diag}(\mathbf{v})$ отображает элемент~$v_i,$ в позицию~$(i,i)$ результирующей диагональной матрицы.
Тогда производная оператора~$\mathrm{diag}(\mathbf{v})$ является матрицей~$\mathbf{E}_{ii} = \mathbf{e}_i \mathbf{e}_i^\top,$ в которой~$1$ в позиции~$(i,i)$~и $0$ иначе. Причем используя свойство~\ref{prop:vec_r_matrix_product}, применяя оператор построчной векторизации, получаем:
\[
    \mathrm{vec}_r(\mathbf{E}_{i,i}) = \mathbf{e}_i \otimes \mathbf{e}_i.
\]

Итого, применяя для всех~$i=1,\dots,L$, матрица Якоби принимает вид:
\[
    \frac{\partial \mathrm{diag}(\mathbf{v})}{\partial \mathbf{v}}
    = \big(\mathbf{e}_1 \otimes \mathbf{e}_1 \quad \dots \quad \mathbf{e}_L \otimes \mathbf{e}_L\big).
\]
\end{proof}

\begin{lemma}[Производная транспонированной матрицы]\label{lemma:transposed_matrix_derivative}
    Пусть задана матрица~$\mathbf{A} \in \mathbb{R}^{m \times n}$, тогда справедливо следующее равенство:
    \begin{equation}
        \frac{\partial \mathbf{A}^\top}{\partial \mathbf{A}} = \mathbf{K}_{n, m},
    \end{equation}
    где матрица~$\mathbf{K}_{n, m}$ является коммутационной матрицей (англ. commutation matrix) описанной в определении~\ref{def:commutation_matrix}.
\end{lemma} 
\begin{proof}
    Объединяя аналогичное свойство из~\cite{magnus1988matrix} для постолбцовой векторизации с правилом соединения столбцов и строк~\ref{prop:vec_relation} и \ref{def:commutation_matrix}, получаем утверждение теоремы.
\end{proof}


\begin{lemma}[Производная произведения Кронекера матричнозначных функций]\label{lemma:matrix_funcs_kronecker_product_derivative}
    Пусть заданы матричнозначные функции~$\mathbf{A}(\mathbf{X}) \in \mathbb{R}^{n \times q}$ и~$\mathbf{B}(\mathbf{X}) \in \mathbb{R}^{p \times r}$ матрицы $\mathbf{X}$, тогда
    \begin{equation}
        \frac{\partial\mathbf{A}(\mathbf{X}) \otimes \mathbf{B}(\mathbf{X})}{\partial \mathbf{X}} = (\mathbf{I}_n \otimes \mathbf{K}_{p,q} \otimes \mathbf{I}_r) \left( \left( \mathrm{vec}_r \mathbf{A} \otimes \mathbf{I}_{pr} \right) \frac{\partial\mathbf{B}}{\partial \mathbf{X}} + \left( \mathbf{I}_{nq} \otimes \mathrm{vec}_r \mathbf{B} \right) \frac{\partial\mathbf{A}}{\partial \mathbf{X}}\right).
    \end{equation}
\end{lemma}
\begin{proof}
    Применить цепное правило для вычисления производной сложной функции, а затем объединим его со свойством~\ref{prop:kronecker_product_derivative}
    \begin{align}
        \frac{\partial\mathbf{A}(\mathbf{X}) \otimes \mathbf{B}(\mathbf{X})}{\partial \mathbf{X}} &= \frac{\partial \mathbf{A}\otimes\mathbf{B}}{\partial\mathbf{B}}\frac{\partial\mathbf{B}}{\partial \mathbf{X}} + \frac{\partial \mathbf{A}\otimes\mathbf{B}}{\partial\mathbf{A}}\frac{\partial\mathbf{A}}{\partial \mathbf{X}} = \\
        &= (\mathbf{I}_n \otimes \mathbf{K}_{p,q} \otimes \mathbf{I}_r) \left( \mathrm{vec}_r \mathbf{A} \otimes \mathbf{I}_{pr} \right) \frac{\partial\mathbf{B}}{\partial \mathbf{X}} +\\
        &\quad+(\mathbf{I}_n \otimes \mathbf{K}_{p,q} \otimes \mathbf{I}_r) \left( \mathbf{I}_{nq} \otimes \mathrm{vec}_r \mathbf{B} \right) \frac{\partial\mathbf{A}}{\partial \mathbf{X}} = \\
        &= (\mathbf{I}_n \otimes \mathbf{K}_{p,q} \otimes \mathbf{I}_r) \left( \left( \mathrm{vec}_r \mathbf{A} \otimes \mathbf{I}_{pr} \right) \frac{\partial\mathbf{B}}{\partial \mathbf{X}} + \left( \mathbf{I}_{nq} \otimes \mathrm{vec}_r \mathbf{B} \right) \frac{\partial\mathbf{A}}{\partial \mathbf{X}}\right).
    \end{align}
\end{proof}

\begin{lemma}\label{lemma:1_spectral_norm}
    Пусть задана единичная матрица~$\mathbf{A} = \mathbf{1}_{L \times L}$. Тогда ее спектральная норма принимает следующее значение:
    \begin{equation}
        \|\mathbf{A}\|_2 = L
    \end{equation}
\end{lemma}
\begin{proof}
Используя основные свойства линейной алгебры, получаем~$\mathrm{tr}(\mathbf{A}) = L$ и~$\mathrm{rank}(\mathbf{A}) = 1 = \mathrm{dim}(\mathrm{Im}(\mathbf{X}))$.
Следовательно, используя~$\mathrm{dim}(\mathrm{Im}(\mathbf{X})) + \mathrm{dim}(\mathrm{Ker}(\mathbf{X})) = L$, получаем~$\mathrm{dim}(\mathrm{Ker}(\mathbf{X})) = L - 1$.
Таким образом, для~$i \in \{2, \dots L\}$ имеем~$\lambda_i = 0$, а~$\lambda_1 = L$.
Тогда единственное ненулевое сингулярное число матрицы~$\mathbf{A}$ равно $\sqrt{L^2} = L$.
Следовательно, получаем, что~$\|\mathbf{A}\|_2 = L$, в соответствии с определением~\ref{def:matrix_norms}.
\end{proof}

\end{document}