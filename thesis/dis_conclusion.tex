В диссертации рассмотрена крупная научная проблема, имеющая важное теоретическое и прикладное значение в области математических методов моделей глубокого обучения, связанная с отсутствием системного подхода к оценке и управлению сложностью моделей и данных.
В рамках диссертационной работы предложено решение фундаментальной проблемы разработки единого математического аппарата для оценки и управления сложностью как моделей глубокого обучения, так и данных, что позволяет перейти к системному проектированию архитектур нейронных сетей и оптимизации процессов их обучения.
Предложенный подход основан на введении формальных мер сложности в рамках теории мер, анализе матриц Гессе и ландшафта оптимизационной задачи, что обеспечивает теоретическую основу для предсказания поведения моделей при масштабировании и количественной оценки соответствия между сложностью модели и сложностью данных.

Основные результаты диссертационной работы перечислены в следующих пунктах:
\begin{enumerate}
    \item Разработан единый формальный аппарат для оценки сложности моделей и данных на основе теории мер и анализа ландшафта оптимизационной задачи.
    Введены формальные определения меры сложности выборки и меры сложности модели, установлен критерий обучаемости модели на выборке, определяющий необходимое условие предотвращения переобучения.
    Введены понятия условной сложности выборки и ландшафтной меры сложности модели, определяемой через спектральные свойства матриц Гессе функции потерь.
    \item Получены строгие теоретические оценки спектральных норм матриц Гессе для основных архитектур глубокого обучения: полносвязных, сверточных и трансформерных сетей.
    Установлены зависимости спектральных норм от глубины сети, размеров слоев и других структурных параметров архитектур.
    Для трансформерных архитектур впервые получены явные выражения для матриц Якоби и Гессе ключевых компонентов и установлены верхние оценки для полной матрицы Гессе.
    Разработан унифицированный подход к анализу матриц Гессе на основе матричной факторизации, обеспечивающий вычислимо осуществимые методы оценки ландшафтной меры сложности без прямого вычисления полных матриц Гессе.
    \item Разработаны практические методы оценки достаточного объема выборки, восполняющие пробел между теоретическим аппаратом оценки сложности и практическими потребностями планирования экспериментов.
    Предложены методы на основе сэмплирования эмпирической функции ошибки и анализа близости апостериорных распределений параметров, для которых получены теоретические оценки сходимости.
    Проведен систематический сравнительный анализ с классическими статистическими и байесовскими методами, подтверждена эффективность предложенных подходов.
    \item Предложены и исследованы методы снижения сложности моделей глубокого обучения, опирающиеся на разработанный теоретический аппарат.
    Разработан метод удаления параметров на основе анализа ковариационной матрицы градиентов функции ошибки, позволяющий существенно сократить число параметров без потери качества.
    Предложены методы мультидоменной дистилляции и анти-дистилляции для передачи знаний между моделями различной сложности и между различными доменами данных.
    Все методы экспериментально подтверждены на реальных задачах компьютерного зрения, обработки естественного языка и классификации.
    \item Продемонстрировано практическое применение разработанного теоретического аппарата в прикладных задачах многозадачного обучения, декодирования фМРТ-изображений и детекции машинно-генерированного контента.
    Для многозадачного обучения получены фундаментальные теоретические результаты о статистической состоятельности низкоранговых адаптеров и снижении эмпирической сложности при совместном использовании параметров энкодера.
    В задаче декодирования фМРТ-изображений предложен метод предварительного сжатия данных, обеспечивающий существенное сокращение времени обучения без потери качества реконструкции.
    В задаче детекции машинно-генерированного контента разработаны метрики оценки качества данных на основе топологической статистики и анализа устойчивости, проведен комплексный анализ множественных наборов данных.
\end{enumerate}

В целом совокупность полученных в диссертации теоретических и практических результатов позволяет сделать вывод о том, что цель исследований достигнута, сформулированная научная проблема решена.
Разработан единый математический аппарат, связывающий сложность моделей и данных в рамках строгой теоретической основы, применимой к широкому классу архитектур нейронных сетей.
Получены вычислимо осуществимые методы оценки сложности, не требующие прямого вычисления матриц Гессе для крупных моделей.
Созданы практические инструменты для оценки достаточного объема выборки и снижения сложности моделей, подтвержденные экспериментально на реальных задачах.
Перечисленные результаты получили высокую оценку научного сообщества при апробации на ведущих международных конференциях и в научных публикациях.