В диссертации рассмотрена научная проблема, имеющая теоретическое и прикладное значение в области математических методов моделей глубокого обучения, связанная с отсутствием системного подхода к оценке и управлению сложностью моделей и данных.
В рамках диссертационной работы предложено решение проблемы разработки единого математического аппарата для оценки и управления сложностью как моделей глубокого обучения, так и данных, что позволяет перейти к системному проектированию архитектур нейронных сетей и оптимизации процессов их обучения.
Предложенный подход основан на введении сложности в рамках теории мер, анализе матриц Гессе и ландшафта оптимизационной задачи, что обеспечивает теоретическую основу для предсказания поведения моделей при масштабировании и количественной оценки соответствия между сложностью модели и сложностью данных.

Основные результаты диссертационной работы перечислены в следующих пунктах:
\begin{enumerate}
    \item Разработан единый формальный аппарат для оценки сложности моделей и данных на основе теории мер и анализа ландшафта оптимизационной задачи.
    Введены определения меры сложности выборки и меры сложности модели, установлен критерий обучаемости модели на выборке, определяющий необходимое условие предотвращения переобучения.
    \item Введена ландшафтная мера сложности модели, определяемая через спектральные свойства матриц Гессе функции потерь, и установлена ее связь с условной сложностью выборки.
    \item Получены теоретические оценки ландшафтных мер на основе спектральных норм матриц Гессе для разных архитектур глубокого обучения: полносвязных, сверточных и трансформерных сетей. Установлены зависимости ландшафтных мер от глубины сети, размеров слоев и других структурных параметров архитектур.
    \item Разработаны практические методы оценки достаточного объема выборки. Проведен систематический сравнительный анализ методов на основе сэмплирования эмпирической функции ошибки и анализа близости апостериорных распределений параметров с классическими статистическими и байесовскими методами, подтверждена эффективность предложенных подходов.
    \item Предложены методы мультидоменной дистилляции и анти-дистилляции для передачи знаний между моделями различной сложности и между различными доменами данных.
    Для многозадачного обучения получены теоретические результаты о статистической состоятельности низкоранговых адаптеров и снижении эмпирической сложности при совместном использовании параметров энкодера.
\end{enumerate}

В целом совокупность полученных в диссертации теоретических и практических результатов позволяет сделать вывод о том, что цель исследований достигнута, сформулированная научная проблема решена.
Разработан единый математический аппарат, связывающий сложность моделей и данных в рамках строгой теоретической основы, применимой к широкому классу архитектур нейронных сетей.
Получены вычислимо осуществимые методы оценки сложности, не требующие прямого вычисления матриц Гессе для параметрических моделей глубокого обучения.
Разработаны практические инструменты для оценки достаточного объема выборки и снижения сложности моделей, подтвержденные экспериментально на реальных задачах.
Перечисленные результаты получили высокую оценку научного сообщества при апробации на ведущих международных конференциях и в научных публикациях.