В диссертации рассмотрена крупная научная проблема, имеющая важное теоретическое и прикладное применение в области математических методов моделей глубокого обучения, связанная с отсутствием системного подхода к оценке и управлению сложностью моделей и данных. В частности в рамках диссертационной работе предложено решение фундаментальной проблемы разработки математического аппарата для оценки и управления сложностью как моделей глубокого обучения, так и данных, что позволяет перейти к системному проектированию архитектур нейронных сетей и оптимизации процессов их обучения. Предложенный подход основан на анализе матриц Гессе и ландшафта оптимизационной задачи, что обеспечивает теоретическую основу для предсказания поведения моделей при масштабировании.

Основные результаты диссертационной работы перечислены в следующих пунктах:
\begin{enumerate}
    \item Разработан формальный аппарат для оценки сложности моделей и данных на основе анализа ландшафта оптимизационной задачи, введены меры сложности и установлены соотношения между сложностью модели и сложностью данных, определяющие условия обучаемости.
    \item Получены теоретические оценки матриц Гессе для основных архитектур глубокого обучения, установлены зависимости спектральных норм от глубины сети, размеров слоев и других гиперпараметров, что позволило количественно характеризовать кривизну функции потерь.
    \item Разработаны методы оценки достаточного объема выборки на основе статистических, байесовских и оригинальных подходов, использующих сэмплирование эмпирической функции ошибки и близость апостериорных распределений, установлена связь между объемом выборки и сложностью моделей.
    \item Предложены и исследованы методы снижения сложности моделей глубокого обучения, включая удаление параметров, дистилляцию и анти-дистилляцию на многодоменных данных, показана их эффективность для уменьшения вычислительной нагрузки при сохранении качества моделей.
    \item Продемонстрировано практическое применение разработанного теоретического аппарата в прикладных задачах многозадачного обучения, декодирования фМРТ снимков и детекции машинной генерации, подтверждена эффективность предложенных подходов.
\end{enumerate}

В целом совокупность полученных в диссертации теоретических и практических результатов позволяет сделать вывод о том, что цель исследований достигнута, сформулированная научная проблема решена.
Перечисленные результаты получили высокую оценку научного сообщества при апробации на ведущих международных конференциях и в научных публикациях.